% !TeX program = lualatex
\documentclass[12pt, a5paper]{article}
\usepackage{fullpage}
\usepackage{subfiles}
\usepackage{fontspec}
\usepackage{libertine}
\usepackage{xcolor}
\usepackage{GotIn}
\usepackage{geometry}
\usepackage{multicol}
\usepackage{multicolrule}
\usepackage{graphicx}
\usepackage{enumitem}
\usepackage[autocompile]{gregoriotex}

\geometry{top=1cm, bottom=1cm, right=1cm, left=1cm}
\pagestyle{empty}

\definecolor{red}{HTML}{C70039}
% \input GoudyIn.fd
% \newcommand*\initfamily{\usefont{U}{GoudyIn}{xl}{n}}

\input Acorn.fd
\newcommand*\initfamily{\usefont{U}{Acorn}{xl}{n}}
% cette ligne ajoute de l'espace entre les portées
% \grechangedim{baselineskip}{60pt}{scalable}

\begin{document}
\gresetlinecolor{gregoriocolor}
  \small

  \begin{center}
    \begin{large}
      I\textsuperscript{er} Dimanche de Carême
    \end{large}
  \end{center}

  \begin{center}
    \textcolor{red}{\normalsize{Capitule}}\\
    \small\textit{
      II\textsuperscript{e} Épître aux Corinthiens. 6, 1-2.
    }
  \end{center}

  \begin{multicols}{2}
    \parindent=0pt
    \small
    Fratres, Exhortámur vos, ne in vácuum grátiam Dei recipiátis.  \textcolor{red}{†} Ait enim : Témpore
    accépto exaudívi te, \textcolor{red}{*} et un die salútis adjúvi te.\\
    \textcolor{red}{\Rbar.} Deo grátias.

    \columnbreak
  
    \textit{ Frères, nous vous exhortons de ne pas recevoir en
    vain la grâce de Dieu ; car il dit lui-même : Je t’ai
    exaucé au temps favorable, et je t’ai aidé au jour du salut.\\
    \textcolor{red}{\Rbar.} Rendons grâce à Dieu.
    }
  \end{multicols}
  
  \begin{center}
    \rule{2cm}{0.4pt}
  \end{center}

  \begin{center}
    \textcolor{red}{\normalsize{Antienne à Magnificat}}\\
  \end{center}

  \gresetinitiallines{1}
  \greillumination{\initfamily\fontsize{11mm}{11mm}\selectfont E}
  \gregorioscore{careme/an--ecce_nunc_tempus--solesmes_2000s}
  \smallskip
  \begin{footnotesize}
    \textit{
      Voici maintenant le temps favorable, voici maintenant le jour du salut :
      montrons-nous donc en ces jours, comme les serviteurs de Dieu, avec une grande patience, dans les les jeûnes, les veilles, et une charité sincère.
    }
  \end{footnotesize}

  \gresetinitiallines{0}
  \gregorioscore{magnificat/magnificat-8g}
  
  \begin{enumerate}
    \setcounter{enumi}{2}
    \item Quia respéxit humilitátem ancíllæ \textbf{su}æ:\textcolor{red}{~*} \\ \-\hspace{2cm} ecce enim ex hoc beátam me dicent omnes gene\textit{ra}\textit{ti}\textbf{ó}nes.

    \item Quia fecit mihi magna qui \textbf{pot}ens est:\textcolor{red}{~*} \\ \-\hspace{2cm} et sanctum \textit{no}\textit{men} \textbf{e}jus.

    \item Et misericórdia ejus a progénie in pro\textbf{gé}nies\textcolor{red}{~*} \\ \-\hspace{2cm} timén\textit{ti}\textit{bus} \textbf{e}um.

    \item Fecit poténtiam in bráchio \textbf{su}o:\textcolor{red}{~*} \\ \-\hspace{2cm} dispérsit supérbos mente \textit{cor}\textit{dis} \textbf{su}i.

    \item Depósuit poténtes de \textbf{se}de,\textcolor{red}{~*} \\ \-\hspace{2cm} et exal\textit{tá}\textit{vit} \textbf{hú}miles.

    \item Esuriéntes implévit \textbf{bo}nis:\textcolor{red}{~*} \\ \-\hspace{2cm} et dívites dimí\textit{sit} \textit{in}\textbf{á}nes.

    \item Suscépit Israël púerum \textbf{su}um,\textcolor{red}{~*} \\ \-\hspace{2cm} recordátus misericór\textit{di}\textit{æ} \textbf{su}æ.

    \item Sicut locútus est ad patres \textbf{nos}tros,\textcolor{red}{~*} \\ \-\hspace{2cm} Abraham et sémini e\textit{jus} \textit{in} \textbf{sǽ}cula.

    \item Glória Patri, et \textbf{Fí}lio,\textcolor{red}{~*} \\ \-\hspace{2cm} et Spirí\textit{tu}\textit{i} \textbf{Sanc}to.

    \item Sicut erat in princípio, et nunc, et \textbf{sem}per,\textcolor{red}{~*} \\ \-\hspace{2cm} et in sǽcula sæcu\textit{ló}\textit{rum}. \textbf{A}men.
  \end{enumerate}

  \begin{center}
    \rule{2cm}{0.4pt}
  \end{center}

  \begin{center}
    \textcolor{red}{\normalsize{Oraison}}\\
  \end{center}

  \begin{multicols}{2}
    \parindent=0pt
    \begin{flushright}
      \textcolor{red}{\Vbar.} Dominus vobiscum.\\
      \textcolor{red}{\Rbar.} Et cum spiritu tuo.\\
    \end{flushright}
  
    \columnbreak
    
    \textit{\textcolor{red}{\Vbar.} Le Seigneur soit avec vous.\\
    \textcolor{red}{\Rbar.} Et avec votre esprit.}\\
  \end{multicols}

  \begin{multicols}{2}
    \parindent=0pt
    Deus, qui Ecclésiam tuam ánnua quadragesimáli observatióne puríficas : \textcolor{red}{†} præsta famíliæ tuæ ; ut, quod a te obtinére abstinéndo nítitur, \textcolor{red}{*} hoc bonis opéribus
    exsequátur.\\ Per Dóminum nostrum...
    \textcolor{red}{\Rbar.} Amen.

    \columnbreak
  
    \textit{ Dieu, qui accordez chaque année à votre Eglise le
    temps du Carême pour la purifier ; faites-lui
    pratiquer les vertus qu’elle s’efforce d’obtenir de
    vous par son abstinence. Par Notre Seigneur...
    Amen.
    }
  \end{multicols}


  % ============ Deuxième dimanche de Carême ====================

  \begin{center}
    \begin{large}
      II\textsuperscript{e} Dimanche de Carême
    \end{large}
  \end{center}

  \begin{center}
    \textcolor{red}{\normalsize{Capitule}}\\
    \small\textit{
      I\textsuperscript{e} Épître aux Thessaloniciens. 4, 1.
    }
  \end{center}

  \begin{multicols}{2}
    \parindent=0pt
    \small
    Fratres, Rogámus vos et obsecrámus in Dómino Jesu :  \textcolor{red}{†} ut, quemádmodum accepístis a nobis, quómodo opórteat vos ambuláre et placére Deo,  \textcolor{red}{*} sic et ambulétis, ut abundétis magis.\\
    \textcolor{red}{\Rbar.} Deo grátias.

    \columnbreak
  
    \textit{ Frères, puisque vous avez appris de nous comment vous devez vous conduire et plaire à Dieu, et que c’est là ce que vous faites, nous vous prions et nous vous conjurons au nom du Seigneur Jésus de marcher à cet égard de progrès en progrès.\\
    \textcolor{red}{\Rbar.} Rendons grâce à Dieu.
    }
  \end{multicols}
  
  \begin{center}
    \rule{2cm}{0.4pt}
  \end{center}

  \begin{center}
    \textcolor{red}{\normalsize{Antienne à Magnificat}}\\
  \end{center}

  \gresetinitiallines{1}
  \greillumination{\initfamily\fontsize{11mm}{11mm}\selectfont V}
  \gregorioscore{careme/an--visionem_quam_vidistis--solesmes_2000s}
  \smallskip
  \begin{footnotesize}
    \textit{
      La vision que vous avez eue, n’en parlez à personne avant que le Fils de l’homme ne ressuscite d’entre les morts.
    }
  \end{footnotesize}

  \gresetinitiallines{0}
  \gregorioscore{magnificat/magnificat-1f}
  
  \begin{enumerate}
    \setcounter{enumi}{2}
    \item Quia respéxit humilitátem an\textbf{cíl}læ \textbf{su}æ:\textcolor{red}{~*} \\ \-\hspace{2cm} ecce enim ex hoc beátam me dicent omnes gene\textit{ra}\textit{ti}\textbf{ó}nes.

    \item Quia fecit mihi \textbf{ma}gna qui \textbf{pot}ens est:\textcolor{red}{~*} \\ \-\hspace{2cm} et sanctum \textit{no}\textit{men} \textbf{e}jus.

    \item Et misericórdia ejus a progénie \textbf{in} pro\textbf{gé}nies\textcolor{red}{~*} \\ \-\hspace{2cm} timén\textit{ti}\textit{bus} \textbf{e}um.

    \item Fecit poténtiam in \textbf{brá}chio \textbf{su}o:\textcolor{red}{~*} \\ \-\hspace{2cm} dispérsit supérbos mente \textit{cor}\textit{dis} \textbf{su}i.

    \item Depósuit pot\textbf{én}tes de \textbf{se}de,\textcolor{red}{~*} \\ \-\hspace{2cm} et exal\textit{tá}\textit{vit} \textbf{hú}miles.

    \item Esuriéntes im\textbf{plé}vit \textbf{bo}nis:\textcolor{red}{~*} \\ \-\hspace{2cm} et dívites dimí\textit{sit} \textit{in}\textbf{á}nes.

    \item Suscépit Israël \textbf{pú}erum \textbf{su}um,\textcolor{red}{~*} \\ \-\hspace{2cm} recordátus misericór\textit{di}\textit{æ} \textbf{su}æ.

    \item Sicut locútus est ad \textbf{pa}tres \textbf{nos}tros,\textcolor{red}{~*} \\ \-\hspace{2cm} Abraham et sémini e\textit{jus} \textit{in} \textbf{sǽ}cula.

    \item Glória \textbf{Pa}tri, et \textbf{Fí}lio,\textcolor{red}{~*} \\ \-\hspace{2cm} et Spirí\textit{tu}\textit{i} \textbf{Sanc}to.

    \item Sicut erat in princípio, et \textbf{nunc}, et \textbf{sem}per,\textcolor{red}{~*} \\ \-\hspace{2cm} et in sǽcula sæcu\textit{ló}\textit{rum}. \textbf{A}men.
  \end{enumerate}

  \begin{center}
    \rule{2cm}{0.4pt}
  \end{center}

  \begin{center}
    \textcolor{red}{\normalsize{Oraison}}\\
  \end{center}

  \begin{multicols}{2}
    \parindent=0pt
    \begin{flushright}
      \textcolor{red}{\Vbar.} Dominus vobiscum.\\
      \textcolor{red}{\Rbar.} Et cum spiritu tuo.\\
    \end{flushright}
  
    \columnbreak
    
    \textit{\textcolor{red}{\Vbar.} Le Seigneur soit avec vous.\\
    \textcolor{red}{\Rbar.} Et avec votre esprit.}\\
  \end{multicols}

  \begin{multicols}{2}
    \parindent=0pt
    Deus, qui cónspicis omni nos virtúte destítui : \textcolor{red}{†} intérius exteriúsque custódi ; ut ab ómnibus adversitátibus muniámur in córpore, \textcolor{red}{*} et a pravis cogitatiónibus mundémur in mente.\\ Per Dóminum nostrum...
    \textcolor{red}{\Rbar.} Amen.

    \columnbreak
  
    \textit{ Dieu qui voyez combien nous sommes dépourvu de
    force, gardez-nous donc intérieurement comme
    extérieurement, afin que notre corps soit préservé de toute adversité et notre âme délivrée de toute pensée mauvaise. Par Notre Seigneur...
    Amen.
    }
  \end{multicols}

  \newpage


  % ============ Troisième dimanche de Carême ====================

  \begin{center}
    \begin{large}
      III\textsuperscript{e} Dimanche de Carême
    \end{large}
  \end{center}

  \begin{center}
    \textcolor{red}{\normalsize{Capitule}}\\
    \small\textit{
      Épître aux Éphésiens. 5, 1-2.
    }
  \end{center}

  \begin{multicols}{2}
    \parindent=0pt
    \small
    Fratres, estote imitatores Dei, sicut fílii caríssimi : \textcolor{red}{†} et ambuláte in dilectióne, sicut et Christus dilexit nos, et tradidit semetipsum
    pro nobis, \textcolor{red}{*} oblatiónem, et hostiam Deo in odorem suavitátis.\\
    \textcolor{red}{\Rbar.} Deo grátias.

    \columnbreak
  
    \textit{ Frères, tachez de ressembler à Dieu comme des fils bien-aimés. Marchez dans la voie de la charité, à l'exemple du Christ qui nous a aimés jusqu'à se livrer pour nous en oblation et en sacrifice d’agréable odeur, offert à Dieu.\\
    \textcolor{red}{\Rbar.} Rendons grâce à Dieu.
    }
  \end{multicols}
  
  \begin{center}
    \rule{2cm}{0.4pt}
  \end{center}

  \begin{center}
    \textcolor{red}{\normalsize{Antienne à Magnificat}}\\
  \end{center}

  \gresetinitiallines{1}
  \greillumination{\initfamily\fontsize{11mm}{11mm}\selectfont E}
  \gregorioscore{careme/an--extollens_quaedam_mulier--solesmes}
  \smallskip
  \begin{footnotesize}
    \textit{
      Une femme élevant la voix du milieu du peuple dit à Jésus : Heureuses sont les entrailles qui vous ont porté, et les mamelles qui vous ont nourri. Jésus lui dit : Mais plutôt heureux sont ceux qui entendent la parole de Dieu, et qui la pratiquent.
    }
  \end{footnotesize}

  \gresetinitiallines{0}
  \gregorioscore{magnificat/magnificat-8g}
  
  \begin{enumerate}
    \setcounter{enumi}{2}
    \item Quia respéxit humilitátem ancíllæ \textbf{su}æ:\textcolor{red}{~*} \\ \-\hspace{2cm} ecce enim ex hoc beátam me dicent omnes gene\textit{ra}\textit{ti}\textbf{ó}nes.

    \item Quia fecit mihi magna qui \textbf{pot}ens est:\textcolor{red}{~*} \\ \-\hspace{2cm} et sanctum \textit{no}\textit{men} \textbf{e}jus.

    \item Et misericórdia ejus a progénie in pro\textbf{gé}nies\textcolor{red}{~*} \\ \-\hspace{2cm} timén\textit{ti}\textit{bus} \textbf{e}um.

    \item Fecit poténtiam in bráchio \textbf{su}o:\textcolor{red}{~*} \\ \-\hspace{2cm} dispérsit supérbos mente \textit{cor}\textit{dis} \textbf{su}i.

    \item Depósuit poténtes de \textbf{se}de,\textcolor{red}{~*} \\ \-\hspace{2cm} et exal\textit{tá}\textit{vit} \textbf{hú}miles.

    \item Esuriéntes implévit \textbf{bo}nis:\textcolor{red}{~*} \\ \-\hspace{2cm} et dívites dimí\textit{sit} \textit{in}\textbf{á}nes.

    \item Suscépit Israël púerum \textbf{su}um,\textcolor{red}{~*} \\ \-\hspace{2cm} recordátus misericór\textit{di}\textit{æ} \textbf{su}æ.

    \item Sicut locútus est ad patres \textbf{nos}tros,\textcolor{red}{~*} \\ \-\hspace{2cm} Abraham et sémini e\textit{jus} \textit{in} \textbf{sǽ}cula.

    \item Glória Patri, et \textbf{Fí}lio,\textcolor{red}{~*} \\ \-\hspace{2cm} et Spirí\textit{tu}\textit{i} \textbf{Sanc}to.

    \item Sicut erat in princípio, et nunc, et \textbf{sem}per,\textcolor{red}{~*} \\ \-\hspace{2cm} et in sǽcula sæcu\textit{ló}\textit{rum}. \textbf{A}men.
  \end{enumerate}

  \begin{center}
    \rule{2cm}{0.4pt}
  \end{center}

  \begin{center}
    \textcolor{red}{\normalsize{Oraison}}\\
  \end{center}

  \begin{multicols}{2}
    \parindent=0pt
    \begin{flushright}
      \textcolor{red}{\Vbar.} Dominus vobiscum.\\
      \textcolor{red}{\Rbar.} Et cum spiritu tuo.\\
    \end{flushright}
  
    \columnbreak
    
    \textit{\textcolor{red}{\Vbar.} Le Seigneur soit avec vous.\\
    \textcolor{red}{\Rbar.} Et avec votre esprit.}\\
  \end{multicols}

  \begin{multicols}{2}
    \parindent=0pt
    Quaesumus, omnípotens Deus, vota humílium réspice : \textcolor{red}{†} atque ad defensiónem nostram, \textcolor{red}{*} déxteram tuæ majestátis exténde.\\ Per Dóminum nostrum...
    \textcolor{red}{\Rbar.} Amen.

    \columnbreak
  
    \textit{ Dieu tout-puissant, recevez, s’il vous plaît, les vœux et les prières de nos cœurs humiliés, et daignez étendre pour notre défense, le bras invincible de
    votre majesté. Par Notre Seigneur...
    Amen.
    }
  \end{multicols}

  \newpage

  % ============ Quatrième dimanche de Carême ====================

  \begin{center}
    \begin{large}
      IV\textsuperscript{e} Dimanche de Carême\\
    \end{large}
    \begin{footnotesize}
      \textit{
        Dimanche de Laetare
      }
    \end{footnotesize}
  \end{center}

  \begin{center}
    \textcolor{red}{\normalsize{Capitule}}\\
    \small\textit{
      Épître aux Galates. 4, 22-24.
    }
  \end{center}

  \begin{multicols}{2}
    \parindent=0pt
    \small
    Fratres, scriptum est : Quóniam Abraham duos fílios habuit : unum de ancílla, et unum de líbera. \textcolor{red}{†} Sed qui de ancílla, secúndum
    carnem natus est : qui autem de líbera, per repromissiónem : \textcolor{red}{*} quæ sunt per allegóriam dicta.\\
    \textcolor{red}{\Rbar.} Deo grátias.

    \columnbreak
  
    \textit{ Frères, il est écrit qu’Abraham eut deux fils, l’un d'une femme esclave (Agar), l’autre de sa femme libre (Sara). Or celui de
    l’esclave naquit selon la chair, mais celui de la
    femme libre naquit à la suite de la promesse de Dieu. Ces faits ont valeur de symboles.\\
    \textcolor{red}{\Rbar.} Rendons grâce à Dieu.
    }
  \end{multicols}
  
  \begin{center}
    \rule{2cm}{0.4pt}
  \end{center}

  \begin{center}
    \textcolor{red}{\normalsize{Antienne à Magnificat}}\\
  \end{center}

  \gresetinitiallines{1}
  \greillumination{\initfamily\fontsize{11mm}{11mm}\selectfont S}
  \gregorioscore{careme/an--subiit_ergo_in_montem--solesmes}
  \smallskip
  \begin{footnotesize}
    \textit{
      Jésus monta sur la montagne, et là il s'assit avec ses disciples.
    }
  \end{footnotesize}

  \gresetinitiallines{0}
  \gregorioscore{magnificat/magnificat-1G}
  
  \begin{enumerate}
    \setcounter{enumi}{2}
    \item Quia respéxit humilitátem an\textbf{cíl}læ \textbf{su}æ:\textcolor{red}{~*} \\ \-\hspace{2cm} ecce enim ex hoc beátam me dicent omnes gene\textit{ra}\textit{ti}\textbf{ó}nes.

    \item Quia fecit mihi \textbf{ma}gna qui \textbf{pot}ens est:\textcolor{red}{~*} \\ \-\hspace{2cm} et sanctum \textit{no}\textit{men} \textbf{e}jus.

    \item Et misericórdia ejus a progénie \textbf{in} pro\textbf{gé}nies\textcolor{red}{~*} \\ \-\hspace{2cm} timén\textit{ti}\textit{bus} \textbf{e}um.

    \item Fecit poténtiam in \textbf{brá}chio \textbf{su}o:\textcolor{red}{~*} \\ \-\hspace{2cm} dispérsit supérbos mente \textit{cor}\textit{dis} \textbf{su}i.

    \item Depósuit pot\textbf{én}tes de \textbf{se}de,\textcolor{red}{~*} \\ \-\hspace{2cm} et exal\textit{tá}\textit{vit} \textbf{hú}miles.

    \item Esuriéntes im\textbf{plé}vit \textbf{bo}nis:\textcolor{red}{~*} \\ \-\hspace{2cm} et dívites dimí\textit{sit} \textit{in}\textbf{á}nes.

    \item Suscépit Israël \textbf{pú}erum \textbf{su}um,\textcolor{red}{~*} \\ \-\hspace{2cm} recordátus misericór\textit{di}\textit{æ} \textbf{su}æ.

    \item Sicut locútus est ad \textbf{pa}tres \textbf{nos}tros,\textcolor{red}{~*} \\ \-\hspace{2cm} Abraham et sémini e\textit{jus} \textit{in} \textbf{sǽ}cula.

    \item Glória \textbf{Pa}tri, et \textbf{Fí}lio,\textcolor{red}{~*} \\ \-\hspace{2cm} et Spirí\textit{tu}\textit{i} \textbf{Sanc}to.

    \item Sicut erat in princípio, et \textbf{nunc}, et \textbf{sem}per,\textcolor{red}{~*} \\ \-\hspace{2cm} et in sǽcula sæcu\textit{ló}\textit{rum}. \textbf{A}men.
  \end{enumerate}

  \begin{center}
    \rule{2cm}{0.4pt}
  \end{center}

  \begin{center}
    \textcolor{red}{\normalsize{Oraison}}\\
  \end{center}

  \begin{multicols}{2}
    \parindent=0pt
    \begin{flushright}
      \textcolor{red}{\Vbar.} Dominus vobiscum.\\
      \textcolor{red}{\Rbar.} Et cum spiritu tuo.\\
    \end{flushright}
  
    \columnbreak
    
    \textit{\textcolor{red}{\Vbar.} Le Seigneur soit avec vous.\\
    \textcolor{red}{\Rbar.} Et avec votre esprit.}\\
  \end{multicols}

  \begin{multicols}{2}
    \parindent=0pt
    Concede, quæsumus, omnípotens Deus :  \textcolor{red}{†} ut,
    qui ex mérito nostræ actiónis afflígimur, \textcolor{red}{*} tuæ
    grátiæ consolatióne respirémus.\\ Per Dóminum nostrum...
    \textcolor{red}{\Rbar.} Amen.

    \columnbreak
  
    \textit{ Faites, s’il vous plaît, Dieu tout-puissant, que nous qui méritons d’être affligés en raison de nos œuvres, nous respirions par la consolation de votre grâce. Par Notre Seigneur...
    Amen.
    }
  \end{multicols}

  \newpage

  % ============ Dimanche de la Passion ====================

  \begin{center}
    \begin{large}
      Dimanche de la Passion
    \end{large}
  \end{center}

  \begin{center}
    \textcolor{red}{\normalsize{Capitule}}\\
    \small\textit{
      Épître aux Hébreux. 9, 11.
    }
  \end{center}

  \begin{multicols}{2}
    \parindent=0pt
    \small
    Fratres, Christus assístens Pontifex futurórum bonórum, per ámplius et perféctius
    tabernáculum non manufáctum, id est, non hujus creatiónis : \textcolor{red}{†} neque per sánguinem hircórum aut vitulórum, sed per próprium sánguinem introívit semel in Sancta, \textcolor{red}{*} ætérna redemptióne invénta.\\
    \textcolor{red}{\Rbar.} Deo grátias.

    \columnbreak
  
    \textit{ Frères, quand le Christ est venu comme grand
    prêtre des biens à venir, c’est par une tente plus
    grande et plus parfaite, une tente qui n’est pas
    l’œuvre des hommes, - c’est-à-dire qui n’appartient
    pas à cette création, - et ce n’est point par le sang
    des boucs et des taureaux, mais par son propre
    sang, qu’il est entré une fois pour toutes dans le
    sanctuaire, ayant acquis une rédemption éternelle.\\
    \textcolor{red}{\Rbar.} Rendons grâce à Dieu.
    }
  \end{multicols}

  \begin{center}
    \rule{2cm}{0.4pt}
  \end{center}

  \begin{center}
    \textcolor{red}{\normalsize{Hymne}}\\
  \end{center}

  \grechangedim{baselineskip}{65pt}{scalable}
  \gresetinitiallines{1}
  \greillumination{\initfamily\fontsize{11mm}{11mm}\selectfont V}
  \gregorioscore{hymnes/hy--vexilla_regis_prodeunt--solesmes}
  \grechangedim{baselineskip}{50pt}{scalable}

  \newpage

  \begin{center}
    \begin{footnotesize}
      \textit{
        \textcolor{red}{1. } L'étendard du Roi s'avance : voici que brille le mystère de la croix ; sur la croix, la vie a subi la mort, et, par la mort, à fait naître la vie.
      }
    \end{footnotesize}
  \end{center}

  \begin{multicols}{2}
    \begin{footnotesize}
      \begin{enumerate}[label=\textcolor{red}{\emph{\arabic*}}]
        \setcounter{enumi}{1}
        \item \textit{Là, du côté qu'a blessé la pointe cruelle de la lance, on vit couler, pour nous purifier de souillures de nos crimes, le sang et l'eau.}
        \item \textit{Voici donc accompli l'oracle que fit entendre le prophète David, disant aux nations : par le bois, Dieu a régné.}
        \item \textit{Arbre splendide, éblouissant, paré de la pourpre du Roi, tige choisie entre mille pour toucher des membres si saints !}
      \end{enumerate}
    \end{footnotesize}

    \columnbreak
    \begin{footnotesize}
      \begin{enumerate}[label=\textcolor{red}{\emph{\arabic*}}]
        \setcounter{enumi}{4}
        \item \textit{Ô heureux bois donc les bras ont portés le prix du monde, balance qui pesat ce corps et ravit sa proie aux enfers !}
        \item \textit{Ô croix, unique espoir, salut : en ces temps de la Passion, comblez de grâce les justes, pardonnez leurs crimes aux pécheurs.}
        \item \textit{Ô Trinité, source de salut, que toute âme vous glorifie ; par la croix, vous nous donnez la victoire ; ajoutez-y la récompense ! Amen.}
      \end{enumerate}
    \end{footnotesize}
  \end{multicols}

  \medskip

  \begin{center}
    \rule{2cm}{0.4pt}
  \end{center}

  \gresetinitiallines{0}
  \gabcsnippet{(c3)<c><v>\Vbar</v>.</c> E(h)ri(h)pe(h) me,(h) Dó(h)mi(h)ne(h) ab(h) hò(h)mi(h)ne(h) mà(h)lo.(g'_) (hvGF'Efgf.) (::) (Z) <c><v>\Rbar</v>.</c> A(h) vì(h)ro(h) in(h)ìquo(h) é(h)ri(h)pe(h) me.(g'_) (hvGF'Efgf.) (::)}
  \smallskip
  \begin{center}
    \textit{\textcolor{red}{\Vbar.} Arrachez-moi , Seigneur, à l’homme mauvais.\\
    \textcolor{red}{\Rbar.} A l’homme inique, arrachez-moi.}
  \end{center}
  
  \begin{center}
    \rule{2cm}{0.4pt}
  \end{center}

  \newpage

  \begin{center}
    \textcolor{red}{\normalsize{Antienne à Magnificat}}\\
  \end{center}

  \gresetinitiallines{1}
  \greillumination{\initfamily\fontsize{11mm}{11mm}\selectfont A}
  \gregorioscore{careme/an--abraham_pater--solesmes}
  \smallskip
  \begin{footnotesize}
    \textit{
      Abraham votre père a exulté à la pensée de voir mon jour ; et il l’a vu, et il s’est réjoui.
    }
  \end{footnotesize}

  \gresetinitiallines{0}
  \gregorioscore{magnificat/magnificat-2D}
  
  \begin{enumerate}
    \setcounter{enumi}{2}
    \item Quia respéxit humilitátem ancíllæ \textbf{su}æ:\textcolor{red}{~*} \\ \-\hspace{2cm} ecce enim ex hoc beátam me dicent omnes genera\textit{ti}\textbf{ó}nes.

    \item Quia fecit mihi magna qui \textbf{pot}ens est:\textcolor{red}{~*} \\ \-\hspace{2cm} et sanctum no\textit{men} \textbf{e}jus.

    \item Et misericórdia ejus a progénie in pro\textbf{gé}nies\textcolor{red}{~*} \\ \-\hspace{2cm} timénti\textit{bus} \textbf{e}um.

    \item Fecit poténtiam in bráchio \textbf{su}o:\textcolor{red}{~*} \\ \-\hspace{2cm} dispérsit supérbos mente cor\textit{dis} \textbf{su}i.

    \item Depósuit poténtes de \textbf{se}de,\textcolor{red}{~*} \\ \-\hspace{2cm} et exaltá\textit{vit} \textbf{hú}miles.

    \item Esuriéntes implévit \textbf{bo}nis:\textcolor{red}{~*} \\ \-\hspace{2cm} et dívites dimísit \textit{in}\textbf{á}nes.

    \item Suscépit Israël púerum \textbf{su}um,\textcolor{red}{~*} \\ \-\hspace{2cm} recordátus misericórdi\textit{æ} \textbf{su}æ.

    \item Sicut locútus est ad patres \textbf{nos}tros,\textcolor{red}{~*} \\ \-\hspace{2cm} Abraham et sémini ejus \textit{in} \textbf{sǽ}cula.

    \item Glória Patri, et \textbf{Fí}lio,\textcolor{red}{~*} \\ \-\hspace{2cm} et Spirítu\textit{i} \textbf{Sanc}to.

    \item Sicut erat in princípio, et nunc, et \textbf{sem}per,\textcolor{red}{~*} \\ \-\hspace{2cm} et in sǽcula sæculó\textit{rum}. \textbf{A}men.
  \end{enumerate}

  \begin{center}
    \rule{2cm}{0.4pt}
  \end{center}

  \begin{center}
    \textcolor{red}{\normalsize{Oraison}}\\
  \end{center}

  \begin{multicols}{2}
    \parindent=0pt
    \begin{flushright}
      \textcolor{red}{\Vbar.} Dominus vobiscum.\\
      \textcolor{red}{\Rbar.} Et cum spiritu tuo.\\
    \end{flushright}
  
    \columnbreak
    
    \textit{\textcolor{red}{\Vbar.} Le Seigneur soit avec vous.\\
    \textcolor{red}{\Rbar.} Et avec votre esprit.}\\
  \end{multicols}

  \begin{multicols}{2}
    \parindent=0pt
    Quæsumus, omnípotens Deus, famíliam tuam propítius réspice :  \textcolor{red}{†} ut, te largiénte, regátur in córpore ; \textcolor{red}{*} et, te servánte, custodiátur in mente.\\ Per Dóminum nostrum...
    \textcolor{red}{\Rbar.} Amen.

    \columnbreak
  
    \textit{ Faites, s’il vous plaît, Dieu tout-puissant, que nous qui méritons d’être affligés en raison de nos œuvres, nous respirions par la consolation de votre grâce. Par Notre Seigneur...
    Amen.
    }
  \end{multicols}

  \newpage

  % ============ Dimanche des Rameaux ====================

  \begin{center}
    \begin{large}
      Dimanche de la Passion
    \end{large}
  \end{center}

  \begin{center}
    \textcolor{red}{\normalsize{Capitule}}\\
    \small\textit{
      Épître aux Philippiens. 2, 5-7.
    }
  \end{center}

  \begin{multicols}{2}
    \parindent=0pt
    \small
    Fratres, hoc enim sentíte in vobis, quod et in Christo Jesu: qui, cum in forma Dei esset, non rapínam arbitrátus est esse se æqualem Deo : \textcolor{red}{†} sed semetípsum exinanívit, formam servi accípiens, in similitúdinem hóminum factus, \textcolor{red}{*} et hábitu invéntus ut homo.\\
    \textcolor{red}{\Rbar.} Deo grátias.

    \columnbreak
  
    \textit{ Frères, ayez en vous les sentiments qui furent dans
    le Christ Jésus. Etant de condition divine, il ne
    retint pas avidement le rang qui l’égalait à Dieu ;
    mais il se dépouilla lui-même en prenant la
    condition d’esclave, se faisant semblable aux
    hommes ; et reconnu à son aspect pour un homme.\\
    \textcolor{red}{\Rbar.} Rendons grâce à Dieu.
    }
  \end{multicols}

  \begin{center}
    \rule{2cm}{0.4pt}
  \end{center}

  \begin{center}
    \textcolor{red}{\normalsize{Hymne}}\\
  \end{center}

  \grechangedim{baselineskip}{65pt}{scalable}
  \gresetinitiallines{1}
  \greillumination{\initfamily\fontsize{11mm}{11mm}\selectfont V}
  \gregorioscore{hymnes/hy--vexilla_regis_prodeunt--solesmes}

  \grechangedim{baselineskip}{50pt}{scalable}

  \newpage

  \begin{center}
    \begin{footnotesize}
      \textit{
        \textcolor{red}{1. } L'étendard du Roi s'avance : voici que brille le mystère de la croix ; sur la croix, la vie a subi la mort, et, par la mort, à fait naître la vie.
      }
    \end{footnotesize}
  \end{center}

  \begin{multicols}{2}
    \begin{footnotesize}
      \begin{enumerate}[label=\textcolor{red}{\emph{\arabic*}}]
        \setcounter{enumi}{1}
        \item \textit{Là, du côté qu'a blessé la pointe cruelle de la lance, on vit couler, pour nous purifier de souillures de nos crimes, le sang et l'eau.}
        \item \textit{Voici donc accompli l'oracle que fit entendre le prophète David, disant aux nations : par le bois, Dieu a régné.}
        \item \textit{Arbre splendide, éblouissant, paré de la pourpre du Roi, tige choisie entre mille pour toucher des membres si saints !}
      \end{enumerate}
    \end{footnotesize}

    \columnbreak
    \begin{footnotesize}
      \begin{enumerate}[label=\textcolor{red}{\emph{\arabic*}}]
        \setcounter{enumi}{4}
        \item \textit{Ô heureux bois donc les bras ont portés le prix du monde, balance qui pesat ce corps et ravit sa proie aux enfers !}
        \item \textit{Ô croix, unique espoir, salut : en ces temps de la Passion, comblez de grâce les justes, pardonnez leurs crimes aux pécheurs.}
        \item \textit{Ô Trinité, source de salut, que toute âme vous glorifie ; par la croix, vous nous donnez la victoire ; ajoutez-y la récompense ! Amen.}
      \end{enumerate}
    \end{footnotesize}
  \end{multicols}

  \begin{center}
    \rule{2cm}{0.4pt}
  \end{center}

  \medskip

  \gresetinitiallines{0}
  \gabcsnippet{(c3)<c><v>\Vbar</v>.</c> E(h)ri(h)pe(h) me,(h) Dó(h)mi(h)ne(h) ab(h) hò(h)mi(h)ne(h) mà(h)lo.(g'_) (hvGF'Efgf.) (::) (Z) <c><v>\Rbar</v>.</c> A(h) vì(h)ro(h) in(h)ìquo(h) é(h)ri(h)pe(h) me.(g'_) (hvGF'Efgf.) (::)}
  \smallskip
  \begin{center}
    \textit{\textcolor{red}{\Vbar.} Arrachez-moi , Seigneur, à l’homme mauvais.\\
    \textcolor{red}{\Rbar.} A l’homme inique, arrachez-moi.}
  \end{center}
  
  \begin{center}
    \rule{2cm}{0.4pt}
  \end{center}

  \newpage

  \begin{center}
    \textcolor{red}{\normalsize{Antienne à Magnificat}}\\
  \end{center}

  \gresetinitiallines{1}
  \greillumination{\initfamily\fontsize{11mm}{11mm}\selectfont A}
  \gregorioscore{careme/an--scriptum_est_enim_percutiam--solesmes_2000s}
  \smallskip
  \begin{footnotesize}
    \textit{
      Il est écrit : Je frapperai le pasteur et les brebis du troupeau seront dispersées ; mais après que je serai ressuscité, je vous précéderai en Galilée, c’est là que vous me verrez, dit le Seigneur.
    }
  \end{footnotesize}

  \gresetinitiallines{0}
  \gregorioscore{magnificat/magnificat-7a}
  
  \begin{enumerate}
    \setcounter{enumi}{2}
    \item Quia respéxit humilitátem an\textbf{cíl}læ \textbf{su}æ:\textcolor{red}{~*} \\ \-\hspace{2cm}  ecce enim ex hoc beátam me dicent omnes gene\textbf{ra}ti\textbf{ó}nes.

    \item Quia fecit mihi \textbf{ma}gna qui \textbf{pot}ens est:\textcolor{red}{~*} \\ \-\hspace{2cm}  et sanctum \textbf{no}men \textbf{e}jus.

    \item Et misericórdia ejus a progénie \textbf{in} pro\textbf{gé}nies\textcolor{red}{~*} \\ \-\hspace{2cm}  ti\textbf{mén}tibus \textbf{e}um.

    \item Fecit poténtiam in \textbf{brá}chio \textbf{su}o:\textcolor{red}{~*} \\ \-\hspace{2cm}  dispérsit supérbos mente \textbf{cor}dis \textbf{su}i.

    \item Depósuit pot\textbf{én}tes de \textbf{se}de,\textcolor{red}{~*} \\ \-\hspace{2cm}  et exal\textbf{tá}vit \textbf{hú}miles.

    \item Esuriéntes im\textbf{plé}vit \textbf{bo}nis:\textcolor{red}{~*} \\ \-\hspace{2cm}  et dívites di\textbf{mí}sit in\textbf{á}nes.

    \item Suscépit Israël \textbf{pú}erum \textbf{su}um,\textcolor{red}{~*} \\ \-\hspace{2cm}  recordátus miseri\textbf{cór}diæ \textbf{su}æ.

    \item Sicut locútus est ad \textbf{pa}tres \textbf{nos}tros,\textcolor{red}{~*} \\ \-\hspace{2cm}  Abraham et sémini \textbf{e}jus in \textbf{sǽ}cula.

    \item Glória \textbf{Pa}tri, et \textbf{Fí}lio,\textcolor{red}{~*} \\ \-\hspace{2cm}  et Spi\textbf{rí}tui \textbf{Sanc}to.

    \item Sicut erat in princípio, et \textbf{nunc}, et \textbf{sem}per,\textcolor{red}{~*} \\ \-\hspace{2cm}  et in sǽcula sæcu\textbf{ló}rum. \textbf{A}men.
  \end{enumerate}

  \begin{center}
    \rule{2cm}{0.4pt}
  \end{center}

  \begin{center}
    \textcolor{red}{\normalsize{Oraison}}\\
  \end{center}

  \begin{multicols}{2}
    \parindent=0pt
    \begin{flushright}
      \textcolor{red}{\Vbar.} Dominus vobiscum.\\
      \textcolor{red}{\Rbar.} Et cum spiritu tuo.\\
    \end{flushright}
  
    \columnbreak
    
    \textit{\textcolor{red}{\Vbar.} Le Seigneur soit avec vous.\\
    \textcolor{red}{\Rbar.} Et avec votre esprit.}\\
  \end{multicols}

  \begin{multicols}{2}
    \parindent=0pt
    Omnípotens sempitérne Deus, qui humáno géneri ad imitándum humilitátis exémplum,
    Salvatórem nostrum carnem súmere, et crucem subíre fecísti : \textcolor{red}{†} concéde propítius ; ut et patiéntiæ ipsíus habére documénta, \textcolor{red}{*} et
    resurrectiónis consórtia mereámur. Per Dóminum nostrum...
    \textcolor{red}{\Rbar.} Amen.

    \columnbreak
  
    \textit{ Dieu tout puissant et éternel, pour donner au genre
    humain un exemple d’humilité, tu as voulu que
    notre Sauveur prît chair et endurât le supplice de la
    croix : dans ta bonté, que nous puissions recevoir
    l’enseignement de sa passion et méritions d’avoir
    part à sa résurrection. Par Notre Seigneur...
    Amen.
    }
  \end{multicols}

\end{document}
