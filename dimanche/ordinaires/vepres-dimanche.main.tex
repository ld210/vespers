% !TeX program = lualatex
\documentclass[12pt, a5paper]{article}
\usepackage{fullpage}
\usepackage{subfiles}
\usepackage{fontspec}
\usepackage{libertine}
\usepackage{xcolor}
\usepackage{GotIn}
\usepackage{geometry}
\usepackage{multicol}
\usepackage{multicolrule}
\usepackage{graphicx}
\usepackage{enumitem}
\usepackage[autocompile]{gregoriotex}

\geometry{top=1cm, bottom=1cm, right=1cm, left=1cm}
\pagestyle{empty}

\definecolor{red}{HTML}{C70039}
% \input GoudyIn.fd
% \newcommand*\initfamily{\usefont{U}{GoudyIn}{xl}{n}}

\input Acorn.fd
\newcommand*\initfamily{\usefont{U}{Acorn}{xl}{n}}
% cette ligne ajoute de l'espace entre les portées
% \grechangedim{baselineskip}{60pt}{scalable}

\begin{document}
\gresetlinecolor{gregoriocolor}
\small

\begin{titlepage}\centering
  \vspace*{\fill}\
  \huge Secondes Vêpres\\
  \large des dimanches ordinaires\\
  \smallskip
  \begin{footnotesize}
    \textit{
      Depuis le 2 Dimanche après l'Épiphanie,\\ au Dimanche de la Quinquagesime inclusivement.\\
    }
  \end{footnotesize}
  \medskip
  \large et\\
  \medskip
  \LARGE Salut du Saint-Sacrement
  \bigskip
  \begin{figure}[h!]
    \centering
    \includegraphics[width=7cm]{logo.png}
  \end{figure}

  \vspace*{\fill}
  \large\textit{
    Livret latin-français
  }
\end{titlepage}

\begin{center}
  \rule{2cm}{0.4pt}
\end{center}

\vspace{5mm}
\begin{center}
  \textcolor{red}{\normalsize{Ouverture.}}
\end{center}

% greillumination: remplace la première lettre, ici par une font ornementale
\greillumination{\initfamily\fontsize{11mm}{11mm}\selectfont D}
\gregorioscore{vepres-deus_in_adjutorium}
\medskip

\begin{center}
  \small{
  \emph{
    Dieu, venez à mon aide ; Seigneur, hatez-vous de me secourir.\\
    Gloire au Père, au Fils et au Saint Esprit, comme il était au commencement, maintenant et toujour et dans les siècles des siècles.\\
    Ainsi soit-il. Alleluia.
  }
}
\end{center}


% \gregorioscore{vepres-deus_in_adjutorium_septuagesime}

% vfill : prends l'espace vertical disponible 
% \vfill

\begin{center}
  \rule{2cm}{0.4pt}
\end{center}
% greseparator: ornement. Le premier parametre est le type (de 1 à 5), le second la taille en points
% \greseparator{4}{30}

\newpage

\subfile{psaumes-dimanches-ordinaires.tex}

% \newpage

% \begin{center}
%   \rule{2cm}{0.4pt}
% \end{center}

% \vspace{5mm}
\begin{center}
  \textcolor{red}{\normalsize{Capitule.}}\\
  \small\textit{
      Épître aux Galates. 4, 22-24.\\
    }
  \footnotesize{
    \emph{Sauf dimanches de la Septuagesime, Sexagesime et Quinquagesime.}
  }
\end{center}

\begin{multicols}{2}
  \textcolor{red}{\Vbar.} Benedíctus Deus, et Pater Dómini nostri Iesu Chris\textit{ti}, \textcolor{red}{~†} Pater misericordiárum, et Deus totíus conso\textit{la}tiónis, \textcolor{red}{~*} qui consolátur nos in omni tribulatióne nostra.
  \textcolor{red}{\Rbar.} Deo grátias

  \columnbreak

  \textit{Béni soit le Dieu et Père de Notre-Seigneur Jésus-Christ, le Père des miséricordes et le Dieu de toute consolation, qui nous console dans nos tribulations.}
\end{multicols}

\begin{center}
  \rule{2cm}{0.4pt}
\end{center}

% \newpage

\begin{center}
  \textcolor{red}{\normalsize{Hymne.}}\\
  \footnotesize{
    \emph{En célébrant la création de la lumière, oeuvre du premier jour, c'est à dire du Dimanche, elle nous exhorte à fuir les ténèbres du péché. On l'attribut à St. Grégoire le Grand, Pape du VI\textsuperscript{e} s.}
  }
\end{center}

% \grechangedim{baselineskip}{70pt}{scalable}

\gresetinitiallines{1}
\greillumination{\initfamily\fontsize{11mm}{11mm}\selectfont L}
\gregorioscore{hymnes/hy--lucis_creator_optime--solesmes}
\bigskip
\begin{center}
  \begin{footnotesize}
    \textit{
      \textcolor{red}{1. } O Très bon Créateur de la
      lumière, qui faites naître la
      clarté des jours ; aux premiers
      rayons de la lumière nouvelle,
      vous préparez l'origine du
      monde.
    }
  \end{footnotesize}
\end{center}

\begin{multicols}{2}
  \begin{footnotesize}
    \begin{enumerate}[label=\textcolor{red}{\emph{\arabic*}}]
      \setcounter{enumi}{1}
      \item \textit{Vous, qui faites appeler jour Le temps qui s'écoule du matin au soir ; Voici l'approche de la nuit, Ecoutez nos prières mêlées de larmes ;}
      \item \textit{Ne permettez pas que notre âme, Chargée de crimes soit privée du bienfait de la vie,
      Tandis que sans penser à l'éternité, Elle s'embarrasse dans les liens du péché.}
    \end{enumerate}
  \end{footnotesize}

  \columnbreak
  \begin{footnotesize}
    \begin{enumerate}[label=\textcolor{red}{\emph{\arabic*}}]
      \setcounter{enumi}{3}
      \item \textit{Qu'elle frappe enfin à la porte du ciel ; Qu'elle remporte la récompense de la vie ;
      Qu'elle évite tout mal Et se purifie de toute iniquité.}
      \item \textit{Accordez-nous cette grâce, Père très miséricordieux, Ainsi que vous, Fils unique, égal au Père, Qui, avec l'Esprit Consolateur, Régnez à jamais. Ainsi soit-il.}
    \end{enumerate}
  \end{footnotesize}
\end{multicols}

% \grechangedim{baselineskip}{50pt}{scalable}

\begin{center}
  \rule{2cm}{0.4pt}
\end{center}

\newpage


\gresetinitiallines{0}
\gabcsnippet{(c3)<c><v>\Vbar</v>.</c> Di(h)ri(h)gá(h)tur(h) Dó(h)mi(h)ne(h) o(h)rá(h)ti(h)o(h) mé(h)a.(g'_) (hvGF'Efgf.) (::) (Z) <c><v>\Rbar</v>.</c> Si(h)cut(h) in(h)cén(h)sum(h) in(h) cons(h)pé(h)ctu(h) tú(h)o.(g'_) (hvGF'Efgf.) (::)}
\begin{center}
  \textit{\textcolor{red}{\Vbar.} Que ma prière s’élève,
  \textcolor{red}{\Rbar.} Seigneur, comme l’encens devant votre face.}
\end{center}

\begin{center}
  \rule{2cm}{0.4pt}
\end{center}

\begin{center}
  \textcolor{red}{\normalsize{Cantique de la Bienheureuse Vierge Marie}}\\
  \footnotesize{
    \emph{Au propre du jour}
  }
\end{center}

\begin{center}
  \textcolor{red}{\normalsize{Oraison}}\\
  \footnotesize{
    \emph{Au propre du jour}
  }
\end{center}

\begin{center}
  \rule{2cm}{0.4pt}
\end{center}

\begin{center}
  \textcolor{red}{\normalsize{Conclusion de l'office}}
\end{center}


\begin{multicols}{2}
  \parindent=0pt
  \begin{flushright}
    \textcolor{red}{\Vbar.} Dominus vobiscum.\\
    \textcolor{red}{\Rbar.} Et cum spiritu tuo.\\
  \end{flushright}

  \columnbreak
  
  \textit{\textcolor{red}{\Vbar.} Le Seigneur soit avec vous.\\
  \textcolor{red}{\Rbar.} Et avec votre esprit.}\\
\end{multicols}

\gresetinitiallines{1}
\greillumination{\initfamily\fontsize{11mm}{11mm}\selectfont B}
\gregorioscore{benedicamus/ky--benedicamus_xi--solesmes}
\begin{center}
  \begin{footnotesize}
    \textcolor{red}{\textit{Sur un ton très grave : }}
  \end{footnotesize}
\end{center}
\begin{multicols}{2}
  \parindent=0pt
  \textcolor{red}{\Vbar.} Fidélium ánimæ per misericórdiam Dei requiéscant in pace.\\
  \textcolor{red}{\Rbar.} Amen.\\

  \columnbreak
  
  \textit{\textcolor{red}{\Vbar.} Que les âmes des fidèles défunts, par la
  miséricorde de Dieu, reposent en paix.\\
  \textcolor{red}{\Rbar.} Amen.}\\
\end{multicols}

\newpage

\begin{center}
  \textcolor{red}{\normalsize{Salut du Très Saint Sacrement}}\\
  \textit{Chant d'exposition}
\end{center}

\smallskip
\begin{figure}[h!]
  \centering
  \includegraphics[width=\linewidth]{o-salutaris.jpg}
\end{figure}

\begin{center}
  \begin{footnotesize}
    \textit{
      Ô réconfortante
      Hostie, Qui nous
      ouvres les portes du
      ciel, les armées ennemies
      nous poursuivent,
      Donne-nous la force,
      porte-nous secours.
    }
  \end{footnotesize}
\end{center}

\begin{multicols}{2}
  \parindent=0pt
  \begin{flushright}
    O vere digna Hostia,\\
    Spes unica fidelium,\\
    In te confidit Francia,\\
    Da pacem, serva lilium.\\
  \end{flushright}
  \columnbreak
  \textit{
    Ô vraiment digne Hostie,\\
    Unique espoir des fidèles,\\
    en toi se confie la France,\\
    Donne-lui la paix, conserve le lys.\\
  }
\end{multicols}
\begin{multicols}{2}
  \begin{flushright}
    Uni trinoque Domino\\
    Sit sempiterna gloria :\\
    Qui vitam sine termino,\\
    Nobis donet in patria. Amen.\\
  \end{flushright}
  \columnbreak
  \textit{
    Au Seigneur unique en trois personnes,\\
    La gloire éternelle;\\
    qu'il nous donne en son Royaume\\
    La vie qui n'aura pas de fin. Amen\\
  }
\end{multicols}

\begin{center}
  \rule{2cm}{0.4pt}
\end{center}


\begin{center}
  \textcolor{red}{\normalsize{Antienne à la Sainte Vierge.}}\\
  \textit{Voir au propre du jour}
\end{center}

\begin{center}
  \rule{2cm}{0.4pt}
\end{center}

\begin{center}
  \textcolor{red}{\normalsize{En l'honneur Du Saint Sacrement}}
\end{center}

\gresetinitiallines{1}
\greillumination{\initfamily\fontsize{11mm}{11mm}\selectfont T}
\gregorioscore{hymnes/hy--tantum_ergo--solesmes}
\begin{center}
  \begin{footnotesize}
    \begin{enumerate}[label=\textcolor{red}{\emph{\arabic*}}]
      \item \textit{Devant un sacrement si grand, prosternons-nous, adorons ; et que les symboles anciens s'effacent devant le rite nouveau ; que la foi vienne suppléer à la faiblesse de nos sens.}
      \item \textit{Au Père et au Fils louanges et acclamations, gloire honneur et puissance ainsi que bénédictions. A Celui qui de tous deux procède offrons une égale louange.}
    \end{enumerate}
  \end{footnotesize}
\end{center}

\begin{multicols}{2}
  \parindent=0pt
  \textcolor{red}{\Vbar.} Panem de cœlo præstitísti eis.\\
  \textcolor{red}{\Rbar.} Omne delectaméntum in se habéntem.\\
  
  \textit{\textcolor{red}{\Vbar.} Tu leur a donné le pain du ciel.\\
  \textcolor{red}{\Rbar.} Toute saveur se trouve en lui.}\\
  
\end{multicols}

\newpage

\begin{center}
  \textcolor{red}{\normalsize{Oraison}}
\end{center}

\begin{multicols}{2}
  \parindent=0pt
  Deus, qui nobis sub sacramento mirabili
  passionis tuæ memoriam reliquisti : \textcolor{red}{~†}
  tribue, quæsumus, ita nos Corporis et
  Sanguinis tui sacra mysteria venerari, \textcolor{red}{~*} ut
  redemptionis tuæ fructum in nobis
  jugiter sentiamus.\\
  Qui vivis et regnas
  cum Deo Patre in unitate Spiritus Sancti,
  Deus, per omnia sæcula sæculorum.
  Amen.
  \columnbreak

  \textit{
    Seigneur Jésus Christ, dans cet admirable
    sacrement tu nous a laissé le mémorial de
    ta passion ; donne-nous de vénérer d’un si
    grand amour le mystère de ton Corps et de
    ton Sang, que nous puissions recueillir
    sans cesse le fruit de ta rédemption. Toi
    qui règnes avec le Père et le Saint Esprit
    pour les siècles des siècles.
    Amen. 
  }
\end{multicols}

\begin{center}
  \rule{2cm}{0.4pt}
\end{center}


\begin{center}
  \textcolor{red}{\normalsize{Louanges divines}}
\end{center}


\begin{normalsize}
  \parindent=0pt
  Dieu soit béni.\\
  Béni soit son Saint Nom.\\
  Béni soit Jésus-Christ, vrai Dieu et vrai homme.\\
  Béni soit le Nom de Jésus.\\
  Béni soit son Sacré Cœur.\\
  Béni soit son précieux Sang.\\
  Béni soit Jésus dans le très Saint Sacrement de l’autel.\\
  Béni soit l’Esprit Saint Consolateur.\\
  Bénie soit l’auguste Mère de Dieu, la très Sainte Vierge Marie.\\
  Bénie soit sa Sainte et Immaculée Conception.\\
  Bénie soit sa glorieuse Assomption.\\
  Béni soit le nom de Marie, Vierge et Mère.\\
  Béni soit Saint Joseph, son très chaste époux.\\
  Béni soit Dieu dans ses anges et dans ses saints.\\
  Seigneur, donnez-nous des prêtres.\\
  Seigneur, donnez-nous de saints prêtres.\\
  Seigneur, donnez-nous beaucoup de saints prêtres.\\
  Seigneur, donnez-nous beaucoup de saintes vocations religieuses.\\
\end{normalsize}

\begin{center}
  \rule{2cm}{0.4pt}
\end{center}

\newpage

\begin{center}
  \textcolor{red}{\normalsize{Déposition}}\\
  \textit{Psaume 116}
\end{center}

\gresetinitiallines{1}
\greillumination{\initfamily\fontsize{11mm}{11mm}\selectfont L}
\gregorioscore{ps--laudate_dominum_omnes_gentes_(psalmus_116)--solesmes}
\bigskip
\begin{footnotesize}
  \textit{
    Louez le Seigneur, tous les
    peuples ;
    Fêtez-Le, tous les pays !
    Son Amour envers nous
    S'est montré le plus fort ;
    Eternelle est la Fidélité du
    Seigneur !
    Gloire au Père, au Fils
    Et au Saint-Esprit,
    Comme il était au
    commencement,
    Maintenant et toujours,
    Pour les siècles des siècles,
    amen.
  }
\end{footnotesize}

\begin{center}
  \rule{2cm}{0.4pt}
\end{center}

\newpage

\begin{titlepage}\centering
  \vspace*{\fill}
  \LARGE Propre du temps.
  \vspace*{\fill}
\end{titlepage}

\newpage

\grechangedim{baselineskip}{50pt}{scalable}

\subfile{propre-epiphanie.tex}
\newpage
\subfile{septuagesime.tex}
\newpage
\subfile{vierge-marie.tex}

\end{document}
