% !TeX program = lualatex
\documentclass[12pt, a4paper]{article}
\usepackage{fullpage}
\usepackage{subfiles}
\usepackage{fontspec}
\usepackage{libertine}
\usepackage{xcolor}
\usepackage{GotIn}
\usepackage{geometry}
\usepackage{multicol}
\usepackage{multicolrule}
\usepackage{graphicx}
\usepackage{enumitem}
\usepackage[autocompile]{gregoriotex}

\geometry{top=2cm, bottom=2cm}
\pagestyle{empty}

\definecolor{red}{HTML}{C70039}
% \input GoudyIn.fd
% \newcommand*\initfamily{\usefont{U}{GoudyIn}{xl}{n}}

\input Acorn.fd
\newcommand*\initfamily{\usefont{U}{Acorn}{xl}{n}}
% cette ligne ajoute de l'espace entre les portées
% \grechangedim{baselineskip}{60pt}{scalable}

\begin{document}
\gresetlinecolor{gregoriocolor}

\begin{titlepage}\centering
  \vspace*{\fill}\
  \huge Secondes Vêpres\\
  \smallskip
  \begin{normalsize}
    \textit{
      Depuis les Dimanches de Carême,\\
      au Dimanche des Rameaux inclusivement.\\
    }
  \end{normalsize}
  \medskip
  \large et\\
  \medskip
  \LARGE Salut du Saint-Sacrement
  \bigskip
  \begin{figure}[h!]
    \centering
    \includegraphics[width=7cm]{logo.png}
  \end{figure}

  \vspace*{\fill}
  \normalsize\textit{
    Livret latin-français
  }
\end{titlepage}

\newpage

\vspace{5mm}
\begin{center}
  \textcolor{red}{\large{Ouverture.}}
\end{center}

% greillumination: remplace la première lettre, ici par une font ornementale
\greillumination{\initfamily\fontsize{11mm}{11mm}\selectfont D}
% \gregorioscore{vepres-deus_in_adjutorium}
% \begin{footnotesize}
%   \textcolor{red}{\textit{Depuis la Septuagesime jusqu'à Pâques, à la place de l'Allelúia, on dit :}}
% \end{footnotesize}
\gregorioscore{vepres-deus_in_adjutorium_septuagesime}
\medskip

\begin{center}
  \small{
  \emph{
    Dieu, venez à mon aide ; Seigneur, hatez-vous de me secourir.\\
    Gloire au Père, au Fils et au Saint Esprit, comme il était au commencement, maintenant et toujour et dans les siècles des siècles.\\
    Ainsi soit-il. Louange à vous, Seigneur, Roi d’éternelle gloire !
  }
}
\end{center}


% \gregorioscore{vepres-deus_in_adjutorium_septuagesime}

% vfill : prends l'espace vertical disponible 
% \vfill

\begin{center}
  \rule{2cm}{0.4pt}
\end{center}
% greseparator: ornement. Le premier parametre est le type (de 1 à 5), le second la taille en points
% \greseparator{4}{30}

\newpage

\subfile{psaumes-dimanches-ordinaires.tex}

% \newpage

% \begin{center}
%   \rule{2cm}{0.4pt}
% \end{center}

% \vspace{5mm}
\begin{center}
  \textcolor{red}{\large{Capitule.}}\\
  \footnotesize{
    \emph{Au propre du jour}
  }
\end{center}
\medskip

\begin{center}
  \textcolor{red}{\large{Hymne.}}\\
  \begin{footnotesize}
    \textit{
      Les Dimanches de Carême.
    }
  \end{footnotesize}
\end{center}

% \grechangedim{baselineskip}{70pt}{scalable}

\gresetinitiallines{1}
\greillumination{\initfamily\fontsize{11mm}{11mm}\selectfont A}
\gregorioscore{hymnes/hy--audi_benigne_conditor--solesmes}
\medskip

\begin{footnotesize}
  \parindent=0pt
  \begin{enumerate}[label=\textcolor{red}{\emph{\arabic*}}]
    \item \textit{Créateur plein de bonté, écoutez les
    prières, et regardez les larmes dont
    nous accompagnons le jeûne sacré de
    cette sainte quarantaine.}
    \item \textit{Il est vrai que nous avons beaucoup
    péché; mais pardonnez-nous, en
    considération de l'humble aveu que nous
    vous en faisons; et pour la gloire de
    votre nom, guérissez nos âmes malades.}
    \item \textit{Père des miséricordes, scrutateur des
    cœurs, vous connaissez notre faiblesse;
    pardonnez à des enfants qui reviennent
    sincèrement à vous.}
    \item \textit{Faites que, pendant que nos corps seront
    mortifiés par l'abstinence, nos âmes par
    un jeûne plus saint, s'abstiennent de tout
    péché.}
    \item \textit{O bienheureuse Trinité,
    qui êtes un seul Dieu,
    que votre grâce rende utile à vos serviteurs
    l'offrande qu'ils vous font de leurs jeûnes. Amen.}
  \end{enumerate}
\end{footnotesize}
% \grechangedim{baselineskip}{50pt}{scalable}

  \medskip
  \gresetinitiallines{0}
  \gabcsnippet{(c3)<c><v>\Vbar</v>.</c> An(h)ge(h)lis(h) sù(h)is(h) Dé(h)us(h) man(h)dá(h)vit(h) de(h) te.(g'_) (hvGF'Efgf.) (::) (Z) <c><v>\Rbar</v>.</c> Ut(h) cu(h)stó(h)di(h)ant(h) te(h) in(h) óm(h)ni(h)bus(h) vi(h)is(h) tu(h)is.(g'_) (hvGF'Efgf.) (::)}
  \smallskip
  \begin{center}
    \textit{\textcolor{red}{\Vbar.} Dieu a commandé à ses Anges,\\
    \textcolor{red}{\Rbar.} De vous garder dans toutes vos voies.}
  \end{center}

  \begin{center}
    \rule{2cm}{0.4pt}
  \end{center}

  \begin{center}
    \textcolor{red}{\large{Hymne}}\\
    \begin{footnotesize}
      \textit{
        du Dimanche de la Passion, et du Dimanche des Rameaux.
      }
    \end{footnotesize}
  \end{center}

  \grechangedim{baselineskip}{65pt}{scalable}
  \gresetinitiallines{1}
  \greillumination{\initfamily\fontsize{11mm}{11mm}\selectfont V}
  \gregorioscore{hymnes/hy--vexilla_regis_prodeunt--solesmes}
  \grechangedim{baselineskip}{50pt}{scalable}

  \begin{center}
    \begin{footnotesize}
      \textit{
        \textcolor{red}{1. } L'étendard du Roi s'avance : voici que brille le mystère de la croix ; sur la croix, la vie a subi la mort, et, par la mort, à fait naître la vie.
      }
    \end{footnotesize}
  \end{center}

  \begin{multicols}{2}
    \begin{footnotesize}
      \begin{enumerate}[label=\textcolor{red}{\emph{\arabic*}}]
        \setcounter{enumi}{1}
        \item \textit{Là, du côté qu'a blessé la pointe cruelle de la lance, on vit couler, pour nous purifier de souillures de nos crimes, le sang et l'eau.}
        \item \textit{Voici donc accompli l'oracle que fit entendre le prophète David, disant aux nations : par le bois, Dieu a régné.}
        \item \textit{Arbre splendide, éblouissant, paré de la pourpre du Roi, tige choisie entre mille pour toucher des membres si saints !}
        \item \textit{Ô heureux bois donc les bras ont portés le prix du monde, balance qui pesat ce corps et ravit sa proie aux enfers !}
        \item \textit{Ô croix, unique espoir, salut : en ces temps de la Passion, comblez de grâce les justes, pardonnez leurs crimes aux pécheurs.}
        \item \textit{Ô Trinité, source de salut, que toute âme vous glorifie ; par la croix, vous nous donnez la victoire ; ajoutez-y la récompense ! Amen.}
      \end{enumerate}
    \end{footnotesize}
  \end{multicols}

  \medskip
  \gresetinitiallines{0}
  \gabcsnippet{(c3)<c><v>\Vbar</v>.</c> E(h)ri(h)pe(h) me,(h) Dó(h)mi(h)ne(h) ab(h) hò(h)mi(h)ne(h) mà(h)lo.(g'_) (hvGF'Efgf.) (::) (Z) <c><v>\Rbar</v>.</c> A(h) vì(h)ro(h) in(h)ìquo(h) é(h)ri(h)pe(h) me.(g'_) (hvGF'Efgf.) (::)}
  \smallskip
  \begin{center}
    \textit{\textcolor{red}{\Vbar.} Arrachez-moi , Seigneur, à l’homme mauvais.\\
    \textcolor{red}{\Rbar.} A l’homme inique, arrachez-moi.}
  \end{center}

  \medskip
  \begin{center}
    \rule{2cm}{0.4pt}
  \end{center}
  \medskip


  \newpage

  \begin{center}
    \textcolor{red}{\large{Cantique de la Bienheureuse Vierge Marie}}\\
    \footnotesize{
      \emph{Au propre du jour}
    }
  \end{center}


  \bigskip

  \begin{center}
    \textcolor{red}{\large{Oraison}}\\
    \footnotesize{
      \emph{Au propre du jour}
    }
  \end{center}

  \bigskip

  \begin{center}
    \textcolor{red}{\large{Conclusion de l'office}}
  \end{center}


  \begin{multicols}{2}
    \parindent=0pt
    \begin{flushright}
      \textcolor{red}{\Vbar.} Dominus vobiscum.\\
      \textcolor{red}{\Rbar.} Et cum spiritu tuo.\\
    \end{flushright}

    \columnbreak
    
    \textit{\textcolor{red}{\Vbar.} Le Seigneur soit avec vous.\\
    \textcolor{red}{\Rbar.} Et avec votre esprit.}\\
  \end{multicols}

  \vspace{10pt}

  \gresetinitiallines{1}
  \greillumination{\initfamily\fontsize{11mm}{11mm}\selectfont B}
  \gregorioscore{benedicamus/or--benedicamus_domino_(sundays_of_advent_and_lent)--solesmes_1961}
  \begin{center}
    \begin{footnotesize}
      \textcolor{red}{\textit{Sur un ton très grave : }}
    \end{footnotesize}
  \end{center}
  \begin{multicols}{2}
    \parindent=0pt
    \textcolor{red}{\Vbar.} Fidélium ánimæ per misericórdiam Dei requiéscant in pace.\\
    \textcolor{red}{\Rbar.} Amen.\\

    \columnbreak
    
    \textit{\textcolor{red}{\Vbar.} Que les âmes des fidèles défunts, par la
    miséricorde de Dieu, reposent en paix.\\
    \textcolor{red}{\Rbar.} Amen.}\\
  \end{multicols}

  \newpage

  \begin{center}
    \textcolor{red}{\large{Salut du Très Saint Sacrement}}\\
    \textit{Chant d'exposition}
  \end{center}

  \smallskip
  \begin{figure}[h!]
    \centering
    \includegraphics[width=\linewidth]{o-salutaris.jpg}
  \end{figure}

  \begin{center}
    \begin{footnotesize}
      \textit{
        Ô réconfortante
        Hostie, Qui nous
        ouvres les portes du
        ciel, les armées ennemies
        nous poursuivent,
        Donne-nous la force,
        porte-nous secours.
      }
    \end{footnotesize}
  \end{center}

  \begin{multicols}{2}
    \parindent=0pt
    \begin{flushright}
      O vere digna Hostia,\\
      Spes unica fidelium,\\
      In te confidit Francia,\\
      Da pacem, serva lilium.\\
    \end{flushright}
    \columnbreak
    \textit{
      Ô vraiment digne Hostie,\\
      Unique espoir des fidèles,\\
      en toi se confie la France,\\
      Donne-lui la paix, conserve le lys.\\
    }
  \end{multicols}
  \begin{multicols}{2}
    \begin{flushright}
      Uni trinoque Domino\\
      Sit sempiterna gloria :\\
      Qui vitam sine termino,\\
      Nobis donet in patria. Amen.\\
    \end{flushright}
    \columnbreak
    \textit{
      Au Seigneur unique en trois personnes,\\
      La gloire éternelle;\\
      qu'il nous donne en son Royaume\\
      La vie qui n'aura pas de fin. Amen\\
    }
  \end{multicols}

  \begin{center}
    \rule{2cm}{0.4pt}
  \end{center}

  \newpage

  % \begin{center}
  %   \textcolor{red}{\normalsize{Antienne à la Sainte Vierge.}}\\
  % \end{center}
  \begin{center}
    \textcolor{red}{\large{Ave Regina caelorum}}\\
    \begin{footnotesize}
      \textit{
      Depuis les Complies du 2 Février inclusivement, jusqu'aux Complies du Mercredi Saint inclusivement.
      }
    \end{footnotesize}
  \end{center}

  \gresetinitiallines{1}
  \greillumination{\initfamily\fontsize{11mm}{11mm}\selectfont A}
  \gregorioscore{antiennes-mariales/an--ave_regina_caelorum_(simple_tone)--solesmes}
  \medskip
  \begin{footnotesize}
    \textit{
      Salut, Reine des cieux ! Salut, Souveraine des Anges ! Salut, Tige, salut, Ô Porte par qui la lumière s'est levée sur le monde. Réjouissez-vous, Vierge glorieuse, qui l'emportez sur toutes en beauté ! Adieu, Ô toute belle, et priez le Christ pour nous.
    }
  \end{footnotesize}

  \begin{multicols}{2}
    \parindent=0pt
    \begin{flushright}
      \textcolor{red}{\Vbar.} Dignáre me laudáre te, Virgo sacráta.\\
      \textcolor{red}{\Rbar.} Da mihi virtútem contra hostes tuos.\\
    \end{flushright}

    \columnbreak
    
    \textit{\textcolor{red}{\Vbar.} Agréez que j’annonce vos louanges, Vierge
    sainte.\\
    \textcolor{red}{\Rbar.} Obtenez-moi la force contre vos ennemis.}\\
  \end{multicols}

  \begin{multicols}{2}
    \parindent=0pt
    Concede, miséricors Deus, fragilitáti nostræ præsídium :   \textcolor{red}{†} ut, qui sanctæ Dei Genitrícis memóriam ágimus ; \textcolor{red}{*} intercessiónis ejus
    auxílio, a nostris iniquitátibus resurgámus. Per Christum Dóminum nostrum.
    \textcolor{red}{\Rbar.} Amen.

    \columnbreak

    \textit{ Accordez, Dieu miséricordieux, à notre faiblesse les
      secours de votre grâce et comme nous célébrons la
      mémoire de la sainte Mère de Dieu, faites qu’étant
      aidés auprès de vous de son intercession, nous nous
      relevions de nos péchés. Par le Christ notre Seigneur.
      Amen.
    }
  \end{multicols}

  \begin{center}
    \rule{2cm}{0.4pt}
  \end{center}

  \begin{center}
    \textcolor{red}{\large{En l'honneur Du Saint Sacrement}}
  \end{center}

  \gresetinitiallines{1}
  \greillumination{\initfamily\fontsize{11mm}{11mm}\selectfont T}
  \gregorioscore{hy--tantum_ergo--solesmes}

  \begin{center}
    \begin{footnotesize}
      \begin{enumerate}[label=\textcolor{red}{\emph{\arabic*}}]
        \item \textit{Devant un sacrement si grand, prosternons-nous, adorons ; et que les symboles anciens s'effacent devant le rite nouveau ; que la foi vienne suppléer à la faiblesse de nos sens.}
        \item \textit{Au Père et au Fils louanges et acclamations, gloire honneur et puissance ainsi que bénédictions. A Celui qui de tous deux procède offrons une égale louange.}
      \end{enumerate}
    \end{footnotesize}
  \end{center}

  \medskip

  \begin{multicols}{2}
    \parindent=0pt
    \textcolor{red}{\Vbar.} Panem de caelo praestitisti eis.\\
    \textcolor{red}{\Rbar.} Omne delectamentum in se habentem.\\
    
    \textit{\textcolor{red}{\Vbar.} Tu leur a donné le pain du ciel.\\
    \textcolor{red}{\Rbar.} Toute saveur se trouve en lui.}\\
    
  \end{multicols}

  \begin{center}
    \rule{2cm}{0.4pt}
  \end{center}

  \begin{center}
    \textcolor{red}{\large{Oraison}}
  \end{center}

  \begin{multicols}{2}
    \parindent=0pt
    Deus, qui nobis sub sacramento mirabili
    passionis tuæ memoriam reliquisti : \textcolor{red}{~†}
    tribue, quæsumus, ita nos Corporis et
    Sanguinis tui sacra mysteria venerari, \textcolor{red}{~*} ut
    redemptionis tuæ fructum in nobis
    jugiter sentiamus.\\
    Qui vivis et regnas
    cum Deo Patre in unitate Spiritus Sancti,
    Deus, per omnia sæcula sæculorum.
    Amen.
    \columnbreak

    \textit{
      Seigneur Jésus Christ, dans cet admirable
      sacrement tu nous a laissé le mémorial de
      ta passion ; donne-nous de vénérer d’un si
      grand amour le mystère de ton Corps et de
      ton Sang, que nous puissions recueillir
      sans cesse le fruit de ta rédemption. Toi
      qui règnes avec le Père et le Saint Esprit
      pour les siècles des siècles.
      Amen. 
    }
  \end{multicols}

  \begin{center}
    \rule{2cm}{0.4pt}
  \end{center}


  \begin{center}
    \textcolor{red}{\large{Louanges divines}}
  \end{center}


  \begin{normalsize}
    \parindent=0pt
    Dieu soit béni.\\
    Béni soit son Saint Nom.\\
    Béni soit Jésus-Christ, vrai Dieu et vrai homme.\\
    Béni soit le Nom de Jésus.\\
    Béni soit son Sacré Cœur.\\
    Béni soit son précieux Sang.\\
    Béni soit Jésus dans le très Saint Sacrement de l’autel.\\
    Béni soit l’Esprit Saint Consolateur.\\
    Bénie soit l’auguste Mère de Dieu, la très Sainte Vierge Marie.\\
    Bénie soit sa Sainte et Immaculée Conception.\\
    Bénie soit sa glorieuse Assomption.\\
    Béni soit le nom de Marie, Vierge et Mère.\\
    Béni soit Saint Joseph, son très chaste époux.\\
    Béni soit Dieu dans ses anges et dans ses saints.\\
    Seigneur, donnez-nous des prêtres.\\
    Seigneur, donnez-nous de saints prêtres.\\
    Seigneur, donnez-nous beaucoup de saints prêtres.\\
    Seigneur, donnez-nous beaucoup de saintes vocations religieuses.\\
  \end{normalsize}


  \newpage

  \begin{center}
    \textcolor{red}{\large{Déposition}}\\
    \textit{Psaume 116}
  \end{center}

  \gresetinitiallines{1}
  \greillumination{\initfamily\fontsize{11mm}{11mm}\selectfont L}
  \gregorioscore{ps--laudate_dominum_omnes_gentes_(psalmus_116)--solesmes}
  \bigskip
  \begin{footnotesize}
    \textit{
      Louez le Seigneur, tous les
      peuples ;
      Fêtez-Le, tous les pays !
      Son Amour envers nous
      S'est montré le plus fort ;
      Eternelle est la Fidélité du
      Seigneur !
      Gloire au Père, au Fils
      Et au Saint-Esprit,
      Comme il était au
      commencement,
      Maintenant et toujours,
      Pour les siècles des siècles,
      amen.
    }
  \end{footnotesize}

  \newpage

  \begin{titlepage}\centering
    \vspace*{\fill}
    \LARGE Propre du temps.
    \vspace*{\fill}
  \end{titlepage}

  \newpage

  \grechangedim{baselineskip}{50pt}{scalable}

  \subfile{propre-careme.tex}
  % \newpage
  % \subfile{vierge-marie.tex}

\end{document}
