% !TeX program = lualatex
\documentclass[12pt, a5paper]{article}
\usepackage{fullpage}
\usepackage{subfiles}
\usepackage{fontspec}
\usepackage{libertine}
\usepackage{xcolor}
\usepackage{GotIn}
\usepackage{geometry}
\usepackage{multicol}
\usepackage{multicolrule}
\usepackage{graphicx}
\usepackage[autocompile]{gregoriotex}

\geometry{top=1cm, bottom=1cm, right=1cm, left=1cm}
\pagestyle{empty}

\definecolor{red}{HTML}{C70039}
% \input GoudyIn.fd
% \newcommand*\initfamily{\usefont{U}{GoudyIn}{xl}{n}}

\input Acorn.fd
\newcommand*\initfamily{\usefont{U}{Acorn}{xl}{n}}
% cette ligne ajoute de l'espace entre les portées
% \grechangedim{baselineskip}{60pt}{scalable}

\begin{document}
  \gresetlinecolor{gregoriocolor}
  \small

  \begin{titlepage}
    \centering
    \vspace*{\fill}
    \large Antiennes à la Sainte Vierge.
    \vspace*{\fill}
  \end{titlepage}

  \newpage

  \begin{center}
    \textcolor{red}{\normalsize{Alma Redemptoris Mater}}\\
    \begin{footnotesize}
      \textit{
        Depuis les Vêpres du samedi avant le 1 Dimanche de l'Avent jusqu'aux Vêpres du 2 Février inclusivement.
      }
    \end{footnotesize}
  \end{center}

  \gresetinitiallines{1}
  \greillumination{\initfamily\fontsize{11mm}{11mm}\selectfont A}
  \gregorioscore{antiennes-mariales/an--alma_redemptoris--solesmes}
  \smallskip
  \begin{footnotesize}
    \textit{
      Mère féconde du Rédempteur, vous qui demeurez la Porte toujours ouverte du ciel et l'Étoile de la mer, venez au secours de ce peuple déchu qui voudrait se relever. Vous qui, au grand étonnement de la nature, avez donné le jour à votre divin Auteur, et qui êtes restée Vierge après comme auparavant, en recevant cet Ave de la bouche de Gabriel, ayez pitié des pécheurs.
    }
  \end{footnotesize}

  \newpage

  \begin{multicols}{2}
    \parindent=0pt
    \textcolor{red}{\Vbar.} Post partum, Virgo, invioláta\\ permansísti.\\
    \textcolor{red}{\Rbar.} Dei Génitrix, intercéde pro nobis.
  
    \columnbreak
    
    \begin{footnotesize}
      \textit{\textcolor{red}{\Vbar.} Après votre enfantement, ô Vierge, vous êtes
      demeurée sans tache.\\
      \textcolor{red}{\Rbar.} Mère de Dieu, intercédez pour nous.}
    \end{footnotesize}
  \end{multicols}

  \begin{multicols}{2}
    \parindent=0pt
    Deus, qui salútis ætérnæ, beátæ Maríæ virginitáte fecúnda, humáno géneri præmia præstitísti :   \textcolor{red}{†} tríbue, quæsumus ; ut ipsam pro nobis intercédere sentiámus, per quam merúimus auctórem vitæ suscípere, \textcolor{red}{*} Dóminum nostrum Iesum Christum, Fílium tuum.
      \textcolor{red}{\Rbar.} Amen.

    \columnbreak
  
    \textit{ Dieu, qui par la virginité féconde de la
      bienheureuse Marie, avez procuré au genre
      humain les avantages du salut éternel ;
      accordez-nous, s’il vous plaît, de ressentir les
      effets de l’intercession de celle par qui nous
      avons eu la grâce de recevoir l’auteur de la vie,
      notre Seigneur Jésus-Christ, votre Fils. Amen.
      }
  \end{multicols}

  \begin{center}
    \rule{2cm}{0.4pt}
  \end{center}

  % \newpage

  \begin{center}
    \textcolor{red}{\normalsize{Ave Regina caelorum}}\\
    \begin{footnotesize}
      \textit{
      Depuis les Complies du 2 Février inclusivement, jusqu'aux Complies du Mercredi Saint inclusivement.
      }
    \end{footnotesize}
  \end{center}

  \gresetinitiallines{1}
  \greillumination{\initfamily\fontsize{11mm}{11mm}\selectfont A}
  \gregorioscore{antiennes-mariales/an--ave_regina_caelorum_(simple_tone)--solesmes}
  \medskip
  \begin{footnotesize}
    \textit{
      Salut, Reine des cieux ! Salut, Souveraine des Anges ! Salut, Tige, salut, Ô Porte par qui la lumière s'est levée sur le monde. Réjouissez-vous, Vierge glorieuse, qui l'emportez sur toutes en beauté ! Adieu, Ô toute belle, et priez le Christ pour nous.
    }
  \end{footnotesize}

  \begin{multicols}{2}
    \parindent=0pt
    \begin{flushright}
      \textcolor{red}{\Vbar.} Dignáre me laudáre te, Virgo sacráta.\\
      \textcolor{red}{\Rbar.} Da mihi virtútem contra hostes tuos.\\
    \end{flushright}
  
    \columnbreak
    
    \textit{\textcolor{red}{\Vbar.} Agréez que j’annonce vos louanges, Vierge
    sainte.\\
    \textcolor{red}{\Rbar.} Obtenez-moi la force contre vos ennemis.}\\
  \end{multicols}

  \begin{multicols}{2}
    \parindent=0pt
    Concede, miséricors Deus, fragilitáti nostræ præsídium :   \textcolor{red}{†} ut, qui sanctæ Dei Genitrícis memóriam ágimus ; \textcolor{red}{*} intercessiónis ejus
    auxílio, a nostris iniquitátibus resurgámus. Per Christum Dóminum nostrum.
    \textcolor{red}{\Rbar.} Amen.

    \columnbreak
  
    \textit{ Accordez, Dieu miséricordieux, à notre faiblesse les
      secours de votre grâce et comme nous célébrons la
      mémoire de la sainte Mère de Dieu, faites qu’étant
      aidés auprès de vous de son intercession, nous nous
      relevions de nos péchés. Par le Christ notre Seigneur.
      Amen.
    }
  \end{multicols}

  \begin{center}
    \rule{2cm}{0.4pt}
  \end{center}

\end{document}
