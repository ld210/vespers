% !TeX program = lualatex
\documentclass[12pt, a5paper]{article}
\usepackage{fullpage}
\usepackage{fontspec}
\usepackage{libertine}
\usepackage{xcolor}
\usepackage{GotIn}
\usepackage{geometry}
\usepackage{multicol}
\usepackage{multicolrule}
\usepackage[autocompile]{gregoriotex}

\geometry{top=1cm, bottom=1cm, right=1cm, left=1cm}
\pagestyle{empty}

\definecolor{red}{HTML}{C70039}
% \input GoudyIn.fd
% \newcommand*\initfamily{\usefont{U}{GoudyIn}{xl}{n}}

\input Acorn.fd
\newcommand*\initfamily{\usefont{U}{Acorn}{xl}{n}}

\begin{document}
  \gresetlinecolor{gregoriocolor}

  \begin{center}
    \large{Dimanche de la Septuagesime}
  \end{center}

  \begin{center}
    \textcolor{red}{\normalsize{Capitule}}\\
    \footnotesize{
    \textit{
      I. Épître aux Corinthiens. IX, 2-4.
    }
  }
  \end{center}

  \begin{multicols}{2}
    \parindent=0pt
    Fratres, nescítis, quod ii, qui in stádio currunt, \textcolor{red}{†} omnes quidem currunt, sed unus áccipit bra-ví-um ? \textcolor{red}{*} Sic cúrrite, ut comprehen-dátis.\\
    \textcolor{red}{\Rbar.} Deo gratias.

    \columnbreak
  
    \textit{ Frères : ne savez-vous pas que quand on court dans
    la carrière, tous courent, mais un seul remporte le
    prix ? courez-donc de telle sorte que vous
    remportiez le prix.\\
    \textcolor{red}{\Rbar.} Rendons grâce à Dieu.
    }
  \end{multicols}
  
  \begin{center}
    \rule{2cm}{0.4pt}
  \end{center}

  \begin{center}
    \textcolor{red}{\normalsize{Antienne à Magnificat}}\\
  \end{center}

  \gresetinitiallines{1}
  \greillumination{\initfamily\fontsize{11mm}{11mm}\selectfont D}
  \gregorioscore{septuagesime/antiennes/an--dixit_paterfamilias--solesmes_2000s}
  \smallskip
  \begin{footnotesize}
    \textit{
      Le père de famille dit à ses ouvriers : Pourquoi demeurez-vous là tout le jour sans travailler : C’est, dirent-ils, que personne ne nous as embauchés.
      Allez-vous en aussi vous autres à ma vigne ; et je vous donnerai un juste salaire.
    }
  \end{footnotesize}
  \medskip

  \gresetinitiallines{0}
  \gregorioscore{magnificat/magnificat-7a}
  
  \begin{enumerate}
    \setcounter{enumi}{2}
    \item Quia respéxit humilitátem an\textbf{cíl}læ \textbf{su}æ:\textcolor{red}{~*} \\ \-\hspace{2cm} ecce enim ex hoc beátam me dicent omnes gene\textbf{ra}ti\textbf{ó}nes.

    \item Quia fecit mihi \textbf{ma}gna qui \textbf{pot}ens est:\textcolor{red}{~*} \\ \-\hspace{2cm} et sanctum \textbf{no}men \textbf{e}jus.

    \item Et misericórdia ejus a progénie \textbf{in} pro\textbf{gé}nies\textcolor{red}{~*} \\ \-\hspace{2cm} ti\textbf{mén}tibus \textbf{e}um.

    \item Fecit poténtiam in \textbf{brá}chio \textbf{su}o:\textcolor{red}{~*} \\ \-\hspace{2cm} dispérsit supérbos mente \textbf{cor}dis \textbf{su}i.

    \item Depósuit pot\textbf{én}tes de \textbf{se}de,\textcolor{red}{~*} \\ \-\hspace{2cm} et exal\textbf{tá}vit \textbf{hú}miles.

    \item Esuriéntes im\textbf{plé}vit \textbf{bo}nis:\textcolor{red}{~*} \\ \-\hspace{2cm} et dívites di\textbf{mí}sit in\textbf{á}nes.

    \item Suscépit Israël \textbf{pú}erum \textbf{su}um,\textcolor{red}{~*} \\ \-\hspace{2cm} recordátus miseri\textbf{cór}diæ \textbf{su}æ.

    \item Sicut locútus est ad \textbf{pa}tres \textbf{nos}tros,\textcolor{red}{~*} \\ \-\hspace{2cm} Abraham et sémini \textbf{e}jus in \textbf{sǽ}cula.

    \item Glória \textbf{Pa}tri, et \textbf{Fí}lio,\textcolor{red}{~*} \\ \-\hspace{2cm} et Spi\textbf{rí}tui \textbf{Sanc}to.

    \item Sicut erat in princípio, et \textbf{nunc}, et \textbf{sem}per,\textcolor{red}{~*} \\ \-\hspace{2cm} et in sǽcula sæcu\textbf{ló}rum. \textbf{A}men.
  \end{enumerate}

  \newpage

  \begin{center}
    \rule{2cm}{0.4pt}
  \end{center}

  \begin{center}
    \textcolor{red}{\normalsize{Oraison}}\\
  \end{center}

  \begin{multicols}{2}
    \parindent=0pt
    \begin{flushright}
      \textcolor{red}{\Vbar.} Dominus vobiscum.\\
      \textcolor{red}{\Rbar.} Et cum spiritu tuo.\\
    \end{flushright}
  
    \columnbreak
    
    \textit{\textcolor{red}{\Vbar.} Le Seigneur soit avec vous.\\
    \textcolor{red}{\Rbar.} Et avec votre esprit.}\\
  \end{multicols}

  \begin{multicols}{2}
    \parindent=0pt
    Preces pópuli tui, quæsumus, Dómine, cleménter exáudi : \textcolor{red}{†} ut, qui juste pro peccátis nostris afflígimur, \textcolor{red}{*} pro tui nóminis glória misericórditer liberémur.

    \columnbreak
  
    \textit{ Aux prières de votre peuple, Seigneur, montrezvous favorable ; et, pour votre gloire, faites que nous soyons libérés, par miséricorde, de ce que, en justice, nous souffrons pour nos péchés.
    }
  \end{multicols}
  \begin{multicols*}{2}
    \parindent=0pt
    Per Dóminum nostrum Jesum Christum Fílium tuum, \textcolor{red}{†} qui tecum vivit et regnat in unitáte Spiritus sancti Deus : \textcolor{red}{*} per ómnia sæcula sæculórum.\\
    \textcolor{red}{\Rbar.} Amen.

    \columnbreak
    \textit{
      Par Notre Seigneur Jésus-Christ, votre Fils, qui avec vous vit et règne en l’unité du Saint Esprit, Dieu pour les siècles des siècles.\\
      Amen.
    }
  \end{multicols*}

  % ======================== SEXAGESIME ========================

  \begin{center}
    \large{Dimanche de la Sexagesime}
  \end{center}

  \begin{center}
    \textcolor{red}{\normalsize{Capitule}}\\
    \footnotesize{
    \textit{
      II. Épître aux Corinthiens. XI, 19-20.
    }
  }
  \end{center}

  \begin{multicols}{2}
    \parindent=0pt
    Fratres :  Libénter suffértis insipiéntes, cum sitis ipsi sapiéntes : \textcolor{red}{†} sustinétis enim si quis vos in servitútem rédigit, si quis dévorat, si quis áccipit, si quis extóllitur, \textcolor{red}{*} si quis in fáciem vos cædit.\\
    \textcolor{red}{\Rbar.} Deo gratias.

    \columnbreak
  
    \textit{ Frères : vous supportez si bien les fous, vous qui
    êtes des hommes sensés ! Vous supportez qu’on
    vous tyrannise, qu’on vous dévore, qu’on vous pille,
    qu’on soit arrogant, qu’on vous frappe au visage.\\
    \textcolor{red}{\Rbar.} Rendons grâce à Dieu.
    }
  \end{multicols}
  
  \begin{center}
    \rule{2cm}{0.4pt}
  \end{center}

  \begin{center}
    \textcolor{red}{\normalsize{Antienne à Magnificat}}\\
  \end{center}

  \gresetinitiallines{1}
  \greillumination{\initfamily\fontsize{11mm}{11mm}\selectfont V}
  \gregorioscore{septuagesime/antiennes/an--vobis_datum_est_nosse--solesmes}
  \smallskip
  \begin{footnotesize}
    \textit{
      Pour vous, il vous est donné de connaître le mystère du Royaume de Dieu ;
      mais aux autres en paraboles, dit Jésus à ses disciples.
    }
  \end{footnotesize}
  \medskip

  \gresetinitiallines{0}
  \gregorioscore{magnificat/magnificat-6}
  
  \begin{enumerate}
    \setcounter{enumi}{2}
    \item Quia respéxit humilitátem an\textbf{cíl}læ \textbf{su}æ:\textcolor{red}{~*} \\ \-\hspace{2cm} ecce enim ex hoc beátam me dicent omnes gene\textit{ra}\textit{ti}\textbf{ó}nes.

    \item Quia fecit mihi \textbf{ma}gna qui \textbf{pot}ens est:\textcolor{red}{~*} \\ \-\hspace{2cm} et sanctum \textit{no}\textit{men} \textbf{e}jus.

    \item Et misericórdia ejus a progénie \textbf{in} pro\textbf{gé}nies\textcolor{red}{~*} \\ \-\hspace{2cm} timén\textit{ti}\textit{bus} \textbf{e}um.

    \item Fecit poténtiam in \textbf{brá}chio \textbf{su}o:\textcolor{red}{~*} \\ \-\hspace{2cm} dispérsit supérbos mente \textit{cor}\textit{dis} \textbf{su}i.

    \item Depósuit pot\textbf{én}tes de \textbf{se}de,\textcolor{red}{~*} \\ \-\hspace{2cm} et exal\textit{tá}\textit{vit} \textbf{hú}miles.

    \item Esuriéntes im\textbf{plé}vit \textbf{bo}nis:\textcolor{red}{~*} \\ \-\hspace{2cm} et dívites dimí\textit{sit} \textit{in}\textbf{á}nes.

    \item Suscépit Israël \textbf{pú}erum \textbf{su}um,\textcolor{red}{~*} \\ \-\hspace{2cm} recordátus misericór\textit{di}\textit{æ} \textbf{su}æ.

    \item Sicut locútus est ad \textbf{pa}tres \textbf{nos}tros,\textcolor{red}{~*} \\ \-\hspace{2cm} Abraham et sémini e\textit{jus} \textit{in} \textbf{sǽ}cula.

    \item Glória \textbf{Pa}tri, et \textbf{Fí}lio,~* et Spirí\textit{tu}\textit{i} \textbf{Sanc}to.

    \item Sicut erat in princípio, et \textbf{nunc}, et \textbf{sem}per,\textcolor{red}{~*} \\ \-\hspace{2cm} et in sǽcula sæcu\textit{ló}\textit{rum}. \textbf{A}men.
  \end{enumerate}

  \begin{center}
    \rule{2cm}{0.4pt}
  \end{center}

  \newpage

  \begin{center}
    \textcolor{red}{\normalsize{Oraison}}\\
  \end{center}

  \begin{multicols}{2}
    \parindent=0pt
    \begin{flushright}
      \textcolor{red}{\Vbar.} Dominus vobiscum.\\
      \textcolor{red}{\Rbar.} Et cum spiritu tuo.\\
    \end{flushright}
  
    \columnbreak
    
    \textit{\textcolor{red}{\Vbar.} Le Seigneur soit avec vous.\\
    \textcolor{red}{\Rbar.} Et avec votre esprit.}\\
  \end{multicols}

  \begin{multicols}{2}
    \parindent=0pt
    Deus, qui cónspicis, quia ex nulla nostraactióne confídimus : \textcolor{red}{†} concéde propítius : ut contra advérsa ómnia, \textcolor{red}{*} Doctóris géntium protectióne muniámur.

    \columnbreak
  
    \textit{ Dieu qui voyez que nous mettons notre confiance dans aucune de nos œuvres, accordez-nous par votre bonté, que l’assistance du Docteur des Gentils, nous fortifie contre les maux qui nous environnent.
    }
  \end{multicols}

  \begin{multicols*}{2}
    \parindent=0pt
    Per Dóminum nostrum Jesum Christum Fílium tuum, \textcolor{red}{†} qui tecum vivit et regnat in unitáte Spiritus sancti Deus : \textcolor{red}{*} per ómnia sæcula sæculórum.\\
    \textcolor{red}{\Rbar.} Amen.

    \columnbreak
    \textit{
      Par Notre Seigneur Jésus-Christ, votre Fils, qui avec vous vit et règne en l’unité du Saint Esprit, Dieu pour les siècles des siècles.\\
      Amen.
    }
  \end{multicols*}

  % ======================== QUINQUAGESIME ========================

  \begin{center}
    \large{Dimanche de la Quinquagesime}
  \end{center}

  \begin{center}
    \textcolor{red}{\normalsize{Capitule}}\\
    \footnotesize{
    \textit{
      I. Épître aux Corinthiens. XIII, 13.
    }
  }
  \end{center}

  \begin{multicols}{2}
    \parindent=0pt
    Fratres :  Si linguis hóminum loquar et Angelórum, \textcolor{red}{†} caritátem au-tem non hábeam, \textcolor{red}{*} factus sum velut æs sonans aut cýmbalum tínniens.\\
    \textcolor{red}{\Rbar.} Deo gratias.

    \columnbreak
  
    \textit{ Frères, si je parle les langues des hommes et des
    anges, mais que je n’aie pas la charité, je suis un
    bronze sonore ou une cymbale retentissante.\\
    \textcolor{red}{\Rbar.} Rendons grâce à Dieu.
    }
  \end{multicols}
  
  \begin{center}
    \rule{2cm}{0.4pt}
  \end{center}

  \begin{center}
    \textcolor{red}{\normalsize{Antienne à Magnificat}}\\
  \end{center}

  \gresetinitiallines{1}
  \greillumination{\initfamily\fontsize{11mm}{11mm}\selectfont S}
  \gregorioscore{septuagesime/antiennes/an--stans_autem_jesus--solesmes}
  \smallskip
  \begin{footnotesize}
    \textit{
      Jésus, s’arrêtant, ordonna qu’on lui amène l’aveugle. Quand il se fut approché, Jésus lui demanda : « Que veux-tu que je fasse pour toi ? » Il dit : « Seigneur, que je voie ! » Jésus lui dit : « Vois. Ta foi t’a sauvé. » A l’instant même, il vit. Et il le suivait en glorifiant Dieu.
    }
  \end{footnotesize}
  \medskip

  \gresetinitiallines{0}
  \gregorioscore{magnificat/magnificat-1d}
  
  \begin{enumerate}
    \setcounter{enumi}{2}
    \item Quia respéxit humilitátem an\textbf{cíl}læ \textbf{su}æ:\textcolor{red}{~*} \\ \-\hspace{2cm} ecce enim ex hoc beátam me dicent omnes gene\textit{ra}\textit{ti}\textbf{ó}nes.

    \item Quia fecit mihi \textbf{ma}gna qui \textbf{pot}ens est:\textcolor{red}{~*} \\ \-\hspace{2cm} et sanctum \textit{no}\textit{men} \textbf{e}jus.

    \item Et misericórdia ejus a progénie \textbf{in} pro\textbf{gé}nies\textcolor{red}{~*} \\ \-\hspace{2cm} timén\textit{ti}\textit{bus} \textbf{e}um.

    \item Fecit poténtiam in \textbf{brá}chio \textbf{su}o:\textcolor{red}{~*} \\ \-\hspace{2cm} dispérsit supérbos mente \textit{cor}\textit{dis} \textbf{su}i.

    \item Depósuit pot\textbf{én}tes de \textbf{se}de,\textcolor{red}{~*} \\ \-\hspace{2cm} et exal\textit{tá}\textit{vit} \textbf{hú}miles.

    \item Esuriéntes im\textbf{plé}vit \textbf{bo}nis:\textcolor{red}{~*} \\ \-\hspace{2cm} et dívites dimí\textit{sit} \textit{in}\textbf{á}nes.

    \item Suscépit Israël \textbf{pú}erum \textbf{su}um,\textcolor{red}{~*} \\ \-\hspace{2cm} recordátus misericór\textit{di}\textit{æ} \textbf{su}æ.

    \item Sicut locútus est ad \textbf{pa}tres \textbf{nos}tros,\textcolor{red}{~*} \\ \-\hspace{2cm} Abraham et sémini e\textit{jus} \textit{in} \textbf{sǽ}cula.

    \item Glória \textbf{Pa}tri, et \textbf{Fí}lio,\textcolor{red}{~*} \\ \-\hspace{2cm} et Spirí\textit{tu}\textit{i} \textbf{Sanc}to.

    \item Sicut erat in princípio, et \textbf{nunc}, et \textbf{sem}per,\textcolor{red}{~*} \\ \-\hspace{2cm} et in sǽcula sæcu\textit{ló}\textit{rum}. \textbf{A}men.
  \end{enumerate}

  \begin{center}
    \rule{2cm}{0.4pt}
  \end{center}

  \newpage

  \begin{center}
    \textcolor{red}{\normalsize{Oraison}}\\
  \end{center}

  \begin{multicols}{2}
    \parindent=0pt
    \begin{flushright}
      \textcolor{red}{\Vbar.} Dominus vobiscum.\\
      \textcolor{red}{\Rbar.} Et cum spiritu tuo.\\
    \end{flushright}
  
    \columnbreak
    
    \textit{\textcolor{red}{\Vbar.} Le Seigneur soit avec vous.\\
    \textcolor{red}{\Rbar.} Et avec votre esprit.}\\
  \end{multicols}

  \begin{multicols}{2}
    \parindent=0pt
    Preces nostras, quæsumus, Dómine, cleménter exáudi :  \textcolor{red}{†} atque a peccatórum vínculis absolútos, \textcolor{red}{*} ab omni nos adversitáte custódi.

    \columnbreak
  
    \textit{ Exaucez avec bienveillance, Seigneur, nos prières,
    et, après nous avoir dégagés des liens du péché,
    gardez-nous de toute adversité.
    }
  \end{multicols}
  \begin{multicols*}{2}
    \parindent=0pt
    Per Dóminum nostrum Jesum Christum Fílium tuum, \textcolor{red}{†} qui tecum vivit et regnat in unitáte Spiritus sancti Deus : \textcolor{red}{*} per ómnia sæcula sæculórum.\\
    \textcolor{red}{\Rbar.} Amen.

    \columnbreak
    \textit{
      Par Notre Seigneur Jésus-Christ, votre Fils, qui avec vous vit et règne en l’unité du Saint Esprit, Dieu pour les siècles des siècles.\\
      Amen.
    }
  \end{multicols*}

\end{document}