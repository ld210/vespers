% !TeX program = lualatex
\documentclass[12pt, a5paper]{article}
\usepackage{fullpage}
\usepackage{subfiles}
\usepackage{fontspec}
\usepackage{libertine}
\usepackage{xcolor}
\usepackage{GotIn}
\usepackage{geometry}
\usepackage{multicol}
\usepackage{multicolrule}
\usepackage{graphicx}
\usepackage[autocompile]{gregoriotex}

\geometry{top=1cm, bottom=1cm, right=1cm, left=1cm}
\pagestyle{empty}

\definecolor{red}{HTML}{C70039}
% \input GoudyIn.fd
% \newcommand*\initfamily{\usefont{U}{GoudyIn}{xl}{n}}

\input Acorn.fd
\newcommand*\initfamily{\usefont{U}{Acorn}{xl}{n}}

\begin{document}
  \gresetlinecolor{gregoriocolor}

  \begin{center}
    \textcolor{red}{\normalsize{Capitule.}}
  \end{center}
  
  \begin{multicols}{2}
    \textcolor{red}{\Vbar.} Benedíctus Deus, et Pater Dómini nostri Iesu Chris\textit{ti}, \textcolor{red}{~†} Pater misericordiárum, et Deus totíus conso\textit{la}tiónis, \textcolor{red}{~*} qui consolátur nos in omni tribulatióne nostra.
    \textcolor{red}{\Rbar.} Deo grátias
  
    \columnbreak
  
    \textit{Béni soit le Dieu et Père de Notre-Seigneur Jésus-Christ, le Père des miséricordes et le Dieu de toute consolation, qui nous console dans nos tribulations.}
  \end{multicols}
  
  \begin{center}
    \textcolor{red}{\normalsize{Hymne.}}\\
    \footnotesize{
      \emph{En célébrant la création de la lumière, oeuvre du premier jour, c'est à dire du Dimanche, elle nous exhorte à fuir les ténèbres du péché. On l'attribut à St. Grégoire le Grand, Pape du VI\textsuperscript{e} s.}
    }
  \end{center}
  
  \gresetinitiallines{1}
  \greillumination{\initfamily\fontsize{11mm}{11mm}\selectfont L}
  \gregorioscore{hymnes/hy--lucis_creator_optime--solesmes}
  \bigskip
  
  \begin{footnotesize}
    \textit{
      O Très bon Créateur de la
      lumière, qui faites naître la
      clarté des jours ; aux premiers
      rayons de la lumière nouvelle,
      vous préparez l'origine du
      monde.
      Vous, qui faites appeler jour
      Le temps qui s'écoule du matin
      au soir ; Voici l'approche de la
      nuit, Ecoutez nos prières mêlées
      de larmes ;
      Ne permettez pas que notre
      âme, Chargée de crimes soit
      privée du bienfait de la vie,
      Tandis que sans penser à
      l'éternité, Elle s'embarrasse
      dans les liens du péché.
      Qu'elle frappe enfin à la porte
      du ciel ; Qu'elle remporte la
      récompense de la vie ;
      Qu'elle évite tout mal
      Et se purifie de toute iniquité.
      Accordez-nous cette grâce, Père
      très miséricordieux, Ainsi que
      vous, Fils unique, égal au Père,
      Qui, avec l'Esprit Consolateur,
      Régnez à jamais.
      Ainsi soit-il.
    }
  \end{footnotesize}
  
  
  \gresetinitiallines{0}\
  \begin{center}
    \begin{minipage}[t]{9cm}%
      \gabcsnippet{(c3)<c><v>\Vbar</v>.</c> Di(h)ri(h)gá(h)tur(h) Dó(h)mi(h)ne(h) o(h)rá(h)ti(h)o(h) mé(h)a.(g'_) (hvGF'Efgf.) (::)(z)<c><v>\Rbar</v>.</c> Si(h)cut(h) in(h)cén(h)sum(h) in(h) cons(h)pé(h)ctu(h) tú(h)o.(g'_) (hvGF'Efgf.) (::)}%
    \end{minipage}%  
  \end{center}
  \begin{center}
    \textit{\textcolor{red}{\Vbar.} Que ma prière s’élève,
    \textcolor{red}{\Rbar.} Seigneur, comme l’encens devant votre face.}\\
  \end{center}
  
  
  \begin{center}
    \rule{2cm}{0.4pt}
  \end{center}

  \newpage
  
  \begin{center}
    \textcolor{red}{\normalsize{Cantique de la Bienheureuse Vierge Marie}}\\
    \footnotesize{
      \emph{Au propre dur jour}
    }
  \end{center}
  
  \begin{center}
    \rule{2cm}{0.4pt}
  \end{center}
  
  \begin{center}
    \textcolor{red}{\normalsize{Oraison}}\\
    \footnotesize{
      \emph{Au propre du jour}
    }
  \end{center}
  
  \begin{center}
    \rule{2cm}{0.4pt}
  \end{center}
  
  \begin{center}
    \textcolor{red}{\normalsize{Conclusion de l'office}}
  \end{center}
  
  
  \begin{multicols}{2}
    \parindent=0pt
    \begin{flushright}
      \textcolor{red}{\Vbar.} Dominus vobiscum.\\
      \textcolor{red}{\Rbar.} Et cum spiritu tuo.\\
    \end{flushright}
  
    \columnbreak
    
    \textit{\textcolor{red}{\Vbar.} Le Seigneur soit avec vous.\\
    \textcolor{red}{\Rbar.} Et avec votre esprit.}\\
  \end{multicols}
  
  \vspace{10pt}
  
  \begin{center}
    \begin{minipage}[t]{8cm}%
      \gresetinitiallines{1}%
      \greillumination{\initfamily\fontsize{11mm}{11mm}\selectfont B}%
      \gregorioscore{ky--benedicamus_xi--solesmes}%
    \end{minipage}%
  \end{center}
  
  \newpage
  
  \begin{center}
    \textcolor{red}{\normalsize{Salut du Très Saint Sacrement}}\\
    \textit{Chant d'exposition}
  \end{center}
  
  % \gresetinitiallines{1}
  % \greillumination{\initfamily\fontsize{11mm}{11mm}\selectfont O}
  % \gregorioscore{hy--o_salutaris_hostia--abbe_du_gue}
  \medskip
  \begin{figure}[h!]
    \centering
    \includegraphics[width=\linewidth]{o-salutaris.jpg}
  \end{figure}
  
  \begin{center}
    \begin{footnotesize}
      \textit{
        Ô réconfortante
        Hostie, Qui nous
        ouvres les portes du
        ciel, les armées ennemies
        nous poursuivent,
        Donne-nous la force,
        porte-nous secours.
      }
    \end{footnotesize}
  \end{center}
  
  \begin{multicols}{2}
    \parindent=0pt
    \begin{flushright}
      O vere digna Hostia,\\
      Spes unica fidelium,\\
      In te confidit Francia,\\
      Da pacem, serva lilium.\\
    \end{flushright}
    \columnbreak
    \textit{
      Ô vraiment digne Hostie,\\
      Unique espoir des fidèles,\\
      en toi se confie la France,\\
      Donne-lui la paix, conserve le lys.\\
    }
  \end{multicols}
  \begin{multicols}{2}
    \begin{flushright}
      Uni trinoque Domino\\
      Sit sempiterna gloria :\\
      Qui vitam sine termino,\\
      Nobis donet in patria. Amen.\\
    \end{flushright}
    \columnbreak
    \textit{
      Au Seigneur unique en trois personnes,\\
      La gloire éternelle;\\
      qu'il nous donne en son Royaume\\
      La vie qui n'aura pas de fin. Amen\\
    }
  \end{multicols}
  
  \begin{center}
    \rule{2cm}{0.4pt}
  \end{center}
  
  \newpage
  
  \begin{center}
    \textcolor{red}{\normalsize{En l'honneur de la Vierge Marie}}\\
    \textit{Voir au propre du jour}
  \end{center}
  
  \begin{center}
    \rule{2cm}{0.4pt}
  \end{center}
  
  \begin{center}
    \textcolor{red}{\normalsize{En l'honneur Du Saint Sacrement}}
  \end{center}
  
  \gresetinitiallines{1}
  \greillumination{\initfamily\fontsize{11mm}{11mm}\selectfont T}
  \gregorioscore{hy--tantum_ergo--solesmes}
  
  \begin{multicols}{2}
    \parindent=0pt
    \textcolor{red}{\Vbar.} Panem de caelo praestitisti eis.\\
    \textcolor{red}{\Rbar.} Omne delectamentum in se habentem.\\
    
    \textit{\textcolor{red}{\Vbar.} Tu leur a donné le pain du ciel.\\
    \textcolor{red}{\Rbar.} Toute saveur se trouve en lui.}\\
    
  \end{multicols}
  
  \newpage
  
  \begin{center}
    \textcolor{red}{\normalsize{Oraison}}
  \end{center}
  
  \begin{multicols}{2}
    \parindent=0pt
    Deus, qui nobis sub sacramento mirabili
    passionis tuæ memoriam reliquisti : \textcolor{red}{~†}
    tribue, quæsumus, ita nos Corporis et
    Sanguinis tui sacra mysteria venerari, \textcolor{red}{~*} ut
    redemptionis tuæ fructum in nobis
    jugiter sentiamus.\\
    Qui vivis et regnas
    cum Deo Patre in unitate Spiritus Sancti,
    Deus, per omnia sæcula sæculorum.
    Amen.
    \smallskip
    \columnbreak
  
    \textit{
      Seigneur Jésus Christ, dans cet admirable
      sacrement tu nous a laissé le mémorial de
      ta passion ; donne-nous de vénérer d’un si
      grand amour le mystère de ton Corps et de
      ton Sang, que nous puissions recueillir
      sans cesse le fruit de ta rédemption. Toi
      qui règnes avec le Père et le Saint Esprit
      pour les siècles des siècles.
      Amen. 
    }
  \end{multicols}
  
  \begin{center}
    \rule{2cm}{0.4pt}
  \end{center}
  
  \begin{center}
    \textcolor{red}{\normalsize{Louanges divines}}
  \end{center}
  
  \parindent=0pt
  Dieu soit béni.\\
  Béni soit son Saint Nom.\\
  Béni soit Jésus-Christ, vrai Dieu et vrai homme.\\
  Béni soit le Nom de Jésus.\\
  Béni soit son Sacré Cœur.\\
  Béni soit son précieux Sang.\\
  Béni soit Jésus dans le très Saint Sacrement de l’autel.\\
  Béni soit l’Esprit Saint Consolateur.\\
  Bénie soit l’auguste Mère de Dieu, la très Sainte Vierge Marie.\\
  Bénie soit sa Sainte et Immaculée Conception.\\
  Bénie soit sa glorieuse Assomption.\\
  Béni soit le nom de Marie, Vierge et Mère.\\
  Béni soit Saint Joseph, son très chaste époux.\\
  Béni soit Dieu dans ses anges et dans ses saints.\\
  Seigneur, donnez-nous des prêtres.\\
  Seigneur, donnez-nous de saints prêtres.\\
  Seigneur, donnez-nous beaucoup de saints prêtres.\\
  Seigneur, donnez-nous beaucoup de saintes vocations religieuses.\\
  
  \begin{center}
    \rule{2cm}{0.4pt}
  \end{center}
  
  \newpage
  
  \begin{center}
    \textcolor{red}{\normalsize{Déposition}}\\
    \textit{Psaume 116}
  \end{center}
  
  \gresetinitiallines{1}
  \greillumination{\initfamily\fontsize{11mm}{11mm}\selectfont L}
  \gregorioscore{ps--laudate_dominum_omnes_gentes_(psalmus_116)--solesmes}
  \bigskip
  \begin{footnotesize}
    \textit{
      Louez le Seigneur, tous les
      peuples ;
      Fêtez-Le, tous les pays !
      Son Amour envers nous
      S'est montré le plus fort ;
      Eternelle est la Fidélité du
      Seigneur !
      Gloire au Père, au Fils
      Et au Saint-Esprit,
      Comme il était au
      commencement,
      Maintenant et toujours,
      Pour les siècles des siècles,
      amen.
    }
  \end{footnotesize}
\end{document}