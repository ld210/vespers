% !TeX program = lualatex
\documentclass[12pt, a5paper]{article}
\usepackage{fullpage}
\usepackage{fontspec}
\usepackage{libertine}
\usepackage{xcolor}
\usepackage{GotIn}
\usepackage{geometry}
\usepackage{multicol}
\usepackage{multicolrule}
\usepackage[autocompile]{gregoriotex}

\geometry{top=1cm, bottom=1cm, right=1cm, left=1cm}
\pagestyle{empty}

\definecolor{red}{HTML}{C70039}
\input GoudyIn.fd
\newcommand*\initfamily{\usefont{U}{GoudyIn}{xl}{n}}

\begin{document}
\gresetlinecolor{gregoriocolor}

\begin{titlepage}\centering
  \vspace*{\fill}
  \LARGE Vêpres des dimanches mobiles\\
  \large Temps après l'Épiphanie\\
  \large Temps après la Pentecôte
  \vspace*{\fill}
\end{titlepage}


Ton ordinaire

% cette ligne ajoute de l'espace entre les portées
\grechangedim{baselineskip}{70pt}{scalable}

% greillumination: remplace la première lettre, ici par une font ornementale
\greillumination{\initfamily\fontsize{11mm}{11mm}\selectfont D}

\gregorioscore{vepres-deus_in_adjutorium}
\vspace{15mm}

\small{
  \textcolor{red}{
    \emph{
      Depuis la Septuagesime jusqu'à Pâques, au lieu de l'Alleluia.
    }
  }
}
\gregorioscore{vepres-deus_in_adjutorium_septuagesime}

% vfill : prends l'espace vertical disponible 
\vfill

% greseparator: ornement. Le premier parametre est le type (de 1 à 5), le second la taille en points
\greseparator{4}{30}

\newpage

% ===== DEBUT Antienne =========
\greillumination{\initfamily\fontsize{11mm}{11mm}\selectfont D}
\gregorioscore{antiennes/dixit-dominus-domino-meo}
\vspace{5mm}
% ===== FIN Antienne ===========

% ===== DEBUT psaume ===========
% gresetinitiallines : avec le parametre à 0, supprime l'ornement
\gresetinitiallines{0}

\begin{center}
  \normalsize{Psaume 109.}\\
  \footnotesize{
    \emph{Génération éternelle du Christ, Prêtre, Roi et Juge.}
  }
\end{center}

\gregorioscore{psaumes/psaume109-VIIC2}
\begin{enumerate}
  \setcounter{enumi}{2}
  \item Virgam virtútis tuæ emíttet Dómi\textbf{nus} ex \textbf{Si}on:\textcolor{red}{~*} \\ \-\hspace{2cm} domináre in médio inimi\textbf{có}rum tu\textbf{ó}rum.

  \item Tecum princípium in die virtútis tuæ in splendóri\textbf{bus} sanc\textbf{tó}rum:\textcolor{red}{~*}\\ \-\hspace{2cm}  ex útero ante lucíferum \textbf{gé}nu\textbf{i} te.

  \item Jurávit Dóminus, et non pœni\textbf{té}bit \textbf{e}um:\textcolor{red}{~*}\\ \-\hspace{2cm}  Tu es sacérdos in ætérnum secúndum órdi\textbf{nem} Mel\textbf{chí}sedech.

  \item Dóminus a \textbf{dex}tris \textbf{tu}is,\textcolor{red}{~*} confrégit in die iræ \textbf{su}æ \textbf{re}ges.

  \item Judicábit in natiónibus, im\textbf{plé}bit ru\textbf{í}nas:\textcolor{red}{~*} 
  conquassábit cápita in \textbf{ter}ra mul\textbf{tó}rum.

  \item De torrénte in \textbf{vi}a \textbf{bi}bet:\textcolor{red}{~*} proptérea exal\textbf{tá}bit \textbf{ca}put.

  \item Glória \textbf{Pa}tri, et \textbf{Fí}lio,\textcolor{red}{~*} et Spi\textbf{rí}tui \textbf{Sanc}to.

  \item Sicut erat in princípio, et \textbf{nunc}, et \textbf{sem}per,\textcolor{red}{~*} et in sǽcula sæcu\textbf{ló}rum. \textbf{A}men.
\end{enumerate}

% ===== FIN psaume =========

\vfill
\greseparator{4}{30}

\newpage

% ===== DEBUT Antienne =========
\gresetinitiallines{1}
\greillumination{\initfamily\fontsize{11mm}{11mm}\selectfont M}
\gregorioscore{antiennes/an--magna_opera_domini--solesmes}
\vspace{5mm}
% ===== FIN Antienne ===========

% ===== DEBUT psaume ===========
% gresetinitiallines : avec le parametre à 0, supprime l'ornement
\gresetinitiallines{0}

\begin{center}
  \normalsize{Psaume 110.}\\
  \footnotesize{
    \emph{Bienfaits accordés par Dieu à son peuple.}
  }
\end{center}
\grechangedim{baselineskip}{70pt}{scalable}
\gregorioscore{psaumes/psaume110-IIIb}
\begin{enumerate}
  \setcounter{enumi}{1}
  \item Magna \textbf{ó}pera \textbf{Dó}\textbf{mi}ni:\textcolor{red}{~*} exquisíta in omnes voluntá\textit{tes} \textbf{e}jus.

  \item Conféssio et magnificéntia \textbf{o}pus \textbf{e}jus:\textcolor{red}{~*}\\ \-\hspace{2cm} et justítia ejus manet in sǽcu\textit{lum} \textbf{sǽ}culi.

  \item Memóriam fecit mirabílium suórum,\textcolor{red}{~†} miséricors et mise\textbf{rá}tor \textbf{Dó}\textbf{mi}nus:\textcolor{red}{~*} \\ \-\hspace{2cm} escam dedit timén\textit{ti}\textbf{bus} se.

  \item Memor erit in sǽculum testa\textbf{mén}ti \textbf{su}i:\textcolor{red}{~*} \\ \-\hspace{2cm} virtútem óperum suórum annuntiábit pópu\textit{lo} \textbf{su}o:

  \item Ut det illis heredi\textbf{tá}tem \textbf{gén}\textbf{ti}um:\textcolor{red}{~*} ópera mánuum ejus véritas, et \textit{ju}\textbf{dí}cium.

  \item Fidélia ómnia mandáta ejus:\textcolor{red}{~†} confirmáta in \textbf{sǽ}culum \textbf{sǽ}\textbf{cu}li,\textcolor{red}{~*} \\ \-\hspace{2cm} facta in veritáte et æ\textit{qui}\textbf{tá}te.

  \item Redemptiónem misit \textbf{pó}pulo \textbf{su}o:\textcolor{red}{~*} mandávit in ætérnum testamén\textit{tum} \textbf{su}um.

  \item Sanctum, et terríbile \textbf{no}men \textbf{e}jus:\textcolor{red}{~*} inítium sapiéntiæ ti\textit{mor} \textbf{Dó}mini.

  \item Intelléctus bonus ómnibus faci\textbf{én}tibus \textbf{e}um:\textcolor{red}{~*} \\ \-\hspace{2cm} laudátio ejus manet in sǽcu\textit{lum} \textbf{sǽ}culi.

  \item Glória \textbf{Pa}tri, et \textbf{Fí}\textbf{li}o,\textcolor{red}{~*} et Spirítu\textit{i} \textbf{Sanc}to.

  \item Sicut erat in princípio, et \textbf{nunc}, et \textbf{sem}per,\textcolor{red}{~*} et in sǽcula sæculó\textit{rum}. \textbf{A}men.
\end{enumerate}

% ===== FIN psaume =========
\greseparator{4}{30} 
% ===== DEBUT Antienne =========
\gresetinitiallines{1}
\greillumination{\initfamily\fontsize{11mm}{11mm}\selectfont Q}
\gregorioscore{antiennes/an--qui_timet_dominum--solesmes}
\vspace{5mm}
% ===== FIN Antienne ===========

% ===== DEBUT psaume ===========
% gresetinitiallines : avec le parametre à 0, supprime l'ornement
\gresetinitiallines{0}

\begin{center}
  \normalsize{Psaume 111.}\\
  \footnotesize{
    \emph{Portrait du juste, et tableau de son bonheur.}
  }
\end{center}

\gregorioscore{psaumes/psaume111-IVG}
\begin{enumerate}
  \setcounter{enumi}{1}
  \item Potens in terra erit \textit{se}\textit{men} \textbf{e}jus:\textcolor{red}{~*}  generátio rectórum benedi\textbf{cé}tur.

  \item Glória, et divítiæ in \textit{do}\textit{mo} \textbf{e}jus:\textcolor{red}{~*}  et justítia ejus manet in sǽculum \textbf{sǽ}culi.

  \item Exórtum est in ténebris \textit{lu}\textit{men} \textbf{rec}tis:\textcolor{red}{~*}  miséricors, et miserátor, et \textbf{jus}tus.
  
  \item Jucúndus homo qui miserétur et cómmodat,\textcolor{red}{~†} dispónet sermónes suos \textit{in} \textit{ju}\textbf{dí}cio:\textcolor{red}{~*}  quia in ætérnum non commo\textbf{vé}bitur.

  \item In memória ætérna \textit{e}\textit{rit} \textbf{jus}tus:\textcolor{red}{~*}  ab auditióne mala non ti\textbf{mé}bit.

  \item Parátum cor ejus speráre in Dómino,\textcolor{red}{~†} confirmátum \textit{est} \textit{cor} \textbf{e}jus:\textcolor{red}{~*} \\ \-\hspace{2cm} non commovébitur donec despíciat inimícos \textbf{su}os.

  \item Dispérsit, dedit paupéribus:\textcolor{red}{~†} justítia ejus manet in sǽ\textit{cu}\textit{lum} \textbf{sǽ}culi,\textcolor{red}{~*} \\ \-\hspace{2cm} cornu ejus exaltábitur in \textbf{gló}ria.

  \item Peccátor vidébit, et irascétur,\textcolor{red}{~†} déntibus suis fremet \textit{et} \textit{ta}\textbf{bé}scet:\textcolor{red}{~*} \\ \-\hspace{2cm} desidérium peccatórum per\textbf{í}bit.

  \item Glória Pa\textit{tri}, \textit{et} \textbf{Fí}lio,\textcolor{red}{~*}  et Spirítui \textbf{Sanc}to.

  \item Sicut erat in princípio, et \textit{nunc}, \textit{et} \textbf{sem}per,\textcolor{red}{~*}  et in sǽcula sæculórum. \textbf{A}men.
\end{enumerate}

% ===== FIN psaume =========
\greseparator{4}{30} 
% ===== DEBUT Antienne =========
\gresetinitiallines{1}
\greillumination{\initfamily\fontsize{11mm}{11mm}\selectfont S}
\gregorioscore{antiennes/an--sit_nomen_domini--solesmes}
\vspace{5mm}
% ===== FIN Antienne ===========

% ===== DEBUT psaume ===========
% gresetinitiallines : avec le parametre à 0, supprime l'ornement
\gresetinitiallines{0}

\begin{center}
  \normalsize{Psaume 112.}\\
  \footnotesize{
    \emph{Invitation à louer Dieu et sa Providence souveraine.}
  }
\end{center}

\gregorioscore{psaumes/psaume112-VIIC}
\begin{enumerate}
  \setcounter{enumi}{1}
  \item Sit nomen Dómini \textbf{be}ne\textbf{díc}tum,\textcolor{red}{~*}  ex hoc nunc, et \textbf{us}que in \textbf{sǽ}culum.

  \item A solis ortu usque \textbf{ad} oc\textbf{cá}sum,\textcolor{red}{~*}  laudábile \textbf{no}men \textbf{Dó}mini.

  \item Excélsus super omnes \textbf{gen}tes \textbf{Dó}minus,\textcolor{red}{~*}  et super cælos \textbf{gló}ria \textbf{e}jus.

  \item Quis sicut Dóminus, Deus noster, qui in \textbf{al}tis \textbf{há}bitat,\textcolor{red}{~*} \\ \-\hspace{2cm} et humília réspicit in cælo \textbf{et} in \textbf{ter}ra?

  \item Súscitans a \textbf{ter}ra \textbf{ín}opem,\textcolor{red}{~*}  et de stércore \textbf{é}rigens \textbf{páu}perem:
  
  \item Ut cóllocet eum \textbf{cum} prin\textbf{cí}pibus,\textcolor{red}{~*}  cum princípibus \textbf{pó}puli \textbf{su}i.

  \item Qui habitáre facit stéri\textbf{lem} in \textbf{do}mo,\textcolor{red}{~*}  matrem fili\textbf{ó}rum læ\textbf{tán}tem.

  \item Glória \textbf{Pa}tri, et \textbf{Fí}lio,\textcolor{red}{~*}  et Spi\textbf{rí}tui \textbf{Sanc}to.

  \item Sicut erat in princípio, et \textbf{nunc}, et \textbf{sem}per,\textcolor{red}{~*}  et in sǽcula sæcu\textbf{ló}rum. \textbf{A}men.
\end{enumerate}

% ===== FIN psaume =========

\vfill
\greseparator{4}{30}

% ===== DEBUT Antienne =========
\gresetinitiallines{1}
\greillumination{\initfamily\fontsize{11mm}{11mm}\selectfont D}
\gregorioscore{antiennes/an--deus_autem_noster--solesmes}
\vspace{5mm}
% ===== FIN Antienne ===========

% ===== DEBUT psaume ===========
% gresetinitiallines : avec le parametre à 0, supprime l'ornement
\gresetinitiallines{0}

\begin{center}
  \normalsize{Psaume 113.}\\
  \footnotesize{
    \emph{Le peuple délivré d'Egypte}\\
    \emph{chante son libérateur et le proclame seul vrai Dieu.}
  }
\end{center}

\gregorioscore{psaumes/psaume113-tPer}
\begin{enumerate}
  \setcounter{enumi}{1}
  \item Facta est Judǽa sanctifi\textit{cá}\textit{ti}\textit{o} \textbf{e}jus,\textcolor{red}{~*}  Israël potés\textit{tas} \textbf{e}jus.

  \item Mare \textit{vi}\textit{dit}, \textit{et} \textbf{fu}git:\textcolor{red}{~*}  Jordánis convérsus est \textit{re}\textbf{trór}sum.

  \item Montes exsultavé\textit{runt} \textit{ut} \textit{a}\textbf{rí}etes,\textcolor{red}{~*}  et colles sicut a\textit{gni} \textbf{ó}vium.

  \item Quid est tibi, ma\textit{re}, \textit{quod} \textit{fu}\textbf{gís}ti:\textcolor{red}{~*}  et tu, Jordánis, quia convérsus es \textit{re}\textbf{trór}sum?

  \item Montes, exsultástis \textit{sic}\textit{ut} \textit{a}\textbf{rí}etes,\textcolor{red}{~*}  et colles, sicut a\textit{gni} \textbf{ó}vium.

  \item A fácie Dómini \textit{mo}\textit{ta} \textit{est} \textbf{ter}ra,\textcolor{red}{~*}  a fácie De\textit{i} \textbf{Ja}cob.

  \item Qui convértit petram in \textit{sta}\textit{gna} \textit{a}\textbf{quá}rum,\textcolor{red}{~*}  et rupem in fontes \textit{a}\textbf{quá}rum.

  \item Non nobis, Dó\textit{mi}\textit{ne}, \textit{non} \textbf{no}bis:\textcolor{red}{~*}  sed nómini tuo \textit{da} \textbf{gló}riam.

  \item Super misericórdia tua, et ve\textit{ri}\textit{tá}\textit{te} \textbf{tu}a:\textcolor{red}{~*} \\ \-\hspace{2cm} nequándo dicant gentes: Ubi est Deus \textit{e}\textbf{ó}rum?

  \item Deus autem \textit{nos}\textit{ter} \textit{in} \textbf{cæ}lo:\textcolor{red}{~*}  ómnia quæcúmque vólu\textit{it}, \textbf{fe}cit.

  \item Simulácra géntium ar\textit{gén}\textit{tum}, \textit{et} \textbf{au}rum,\textcolor{red}{~*}  ópera mánu\textit{um} \textbf{hó}minum.

  \item Os habent, \textit{et} \textit{non} \textit{lo}\textbf{quén}tur:\textcolor{red}{~*}  óculos habent, et non \textit{vi}\textbf{dé}bunt.

  \item Aures ha\textit{bent}, \textit{et} \textit{non} \textbf{áu}dient:\textcolor{red}{~*}  nares habent, et non o\textit{do}\textbf{rá}bunt.

  \item Manus habent, et non palpábunt:\textcolor{red}{~†} pedes habent, et \textit{non} \textit{am}\textit{bu}\textbf{lá}bunt:\textcolor{red}{~*} \\ \-\hspace{2cm} non clamábunt in gúttu\textit{re} \textbf{su}o.

  \item Símiles illis fiant qui \textit{fá}\textit{ci}\textit{unt} \textbf{e}a:\textcolor{red}{~*}  et omnes qui confídunt \textit{in} \textbf{e}is.

  \item Domus Israël spe\textit{rá}\textit{vit} \textit{in} \textbf{Dó}mino:\textcolor{red}{~*}  adjútor eórum et protéctor \textit{e}\textbf{ó}rum est,

  \item Domus Aaron spe\textit{rá}\textit{vit} \textit{in} \textbf{Dó}mino:\textcolor{red}{~*}  adjútor eórum et protéctor \textit{e}\textbf{ó}rum est,

  \item Qui timent Dóminum, spera\textit{vé}\textit{runt} \textit{in} \textbf{Dó}mino:\textcolor{red}{~*} \\ \-\hspace{2cm} adjútor eórum et protéctor \textit{e}\textbf{ó}rum est.

  \item Dóminus me\textit{mor} \textit{fu}\textit{it} \textbf{nos}tri:\textcolor{red}{~*}  et benedí\textit{xit} \textbf{no}bis:

  \item Benedíxit \textit{dó}\textit{mu}\textit{i} \textbf{Is}raël:\textcolor{red}{~*}  benedíxit dómu\textit{i} \textbf{A}aron.

  \item Benedíxit ómnibus, \textit{qui} \textit{ti}\textit{ment} \textbf{Dó}minum,\textcolor{red}{~*}  pusíllis cum \textit{ma}\textbf{jó}ribus.

  \item Adjíciat \textit{Dó}\textit{mi}\textit{nus} \textbf{su}per vos:\textcolor{red}{~*}  super vos, et super fíli\textit{os} \textbf{ves}tros.

  \item Benedíc\textit{ti} \textit{vos} \textit{a} \textbf{Dó}mino,\textcolor{red}{~*}  qui fecit cælum, \textit{et} \textbf{ter}ram.

  \item Cæ\textit{lum} \textit{cæ}\textit{li} \textbf{Dó}mino:\textcolor{red}{~*}  terram autem dedit fíli\textit{is} \textbf{hó}minum.

  \item Non mórtui lau\textit{dá}\textit{bunt} \textit{te}, \textbf{Dó}mine:\textcolor{red}{~*}  neque omnes, qui descéndunt in \textit{in}\textbf{fér}num.

  \item Sed nos qui vívimus, bene\textit{dí}\textit{ci}\textit{mus} \textbf{Dó}mino,\textcolor{red}{~*}  ex hoc nunc et usque \textit{in} \textbf{sǽ}culum.

  \item Glória \textit{Pa}\textit{tri}, \textit{et} \textbf{Fí}lio,\textcolor{red}{~*}  et Spirítu\textit{i} \textbf{Sanc}to.

  \item Sicut erat in princípio, \textit{et} \textit{nunc}, \textit{et} \textbf{sem}per,\textcolor{red}{~*}  et in sǽcula sæculó\textit{rum}. \textbf{A}men.
\end{enumerate}

% ===== FIN psaume =========

\gresetinitiallines{0}
% \grechangestyle{commentary}{\scriptsize\textit}
\gresetheadercapture{commentary}{grecommentary}{}
\gregorioscore{antiennes/an--deus_autem_noster--solesmes}
\vspace{10mm}
\greseparator{4}{30}
\vspace{5mm}
\begin{center}
  \textcolor{red}{\normalsize{Capitule.}}
\end{center}

\begin{multicols}{2}
  \textcolor{red}{\Vbar.} Benedíctus Deus, et Pater Dómini nostri Iesu Chris\textit{ti}, \textcolor{red}{~†} Pater misericordiárum, et Deus totíus conso\textit{la}tiónis, \textcolor{red}{~*} qui consolátur nos in omni tribulatióne nostra.
  \textcolor{red}{\Rbar.} Deo grátias

  \columnbreak

  \textit{Béni soit le Dieu et Père de Notre-Seigneur Jésus-Christ, le Père des miséricordes et le Dieu de toute consolation, qui nous console dans nos tribulations.}
\end{multicols}

\begin{center}
  \textcolor{red}{\normalsize{Hymne.}}\\
  \footnotesize{
    \emph{En célébrant la création de la lumière, oeuvre du premier jour, c'est à dire du Dimanche, elle nous exhorte à fuir les ténèbres du péché. On l'attribut à St. Grégoire le Grand, Pape du VI\textsuperscript{e} s.}
  }
\end{center}

\gresetinitiallines{1}
\greillumination{\initfamily\fontsize{11mm}{11mm}\selectfont L}
\gregorioscore{hymnes/hy--lucis_creator_optime--solesmes}

\vspace{5mm}

\greseparator{4}{30}

\vspace{5mm}

\gresetinitiallines{0}\
\begin{center}
  \begin{minipage}[t]{9cm}%
    \gabcsnippet{(c3)<c><v>\Vbar</v>.</c> Di(h)ri(h)gá(h)tur(h) Dó(h)mi(h)ne(h) o(h)rá(h)ti(h)o(h) mé(h)a.(g'_) (hvGF'Efgf.) (::)(z)<c><v>\Rbar</v>.</c> Si(h)cut(h) in(h)cén(h)sum(h) in(h) cons(h)pé(h)ctu(h) tú(h)o.(g'_) (hvGF'Efgf.) (::)}%
  \end{minipage}%
\end{center}
\vfill

\newpage

\begin{center}
  \textcolor{red}{\normalsize{Cantique de la Bienheureuse Vierge Marie}}\\
  \footnotesize{
    \emph{Au propre dur jour}
  }
\end{center}

\begin{center}
  \rule{2cm}{0.4pt}
\end{center}

\begin{center}
  \textcolor{red}{\normalsize{Oraison}}\\
  \footnotesize{
    \emph{Au propre du jour}
  }
\end{center}

\begin{center}
  \rule{2cm}{0.4pt}
\end{center}

\begin{center}
  \textcolor{red}{\normalsize{Conclusion de l'office}}
\end{center}


\begin{multicols}{2}
  \parindent=0pt
  \begin{flushright}
    \textcolor{red}{\Vbar.} Dominus vobiscum.\\
    \textcolor{red}{\Rbar.} Et cum spiritu tuo.\\
  \end{flushright}

  \columnbreak
  
  \textit{\textcolor{red}{\Vbar.} Le Seigneur soit avec vous.\\
  \textcolor{red}{\Rbar.} Et avec votre esprit.}\\
\end{multicols}

\vspace{10pt}

\begin{center}
  \begin{minipage}[t]{8cm}%
    \gresetinitiallines{1}%
    \greillumination{\initfamily\fontsize{11mm}{11mm}\selectfont B}%
    \gregorioscore{ky--benedicamus_xi--solesmes}%
  \end{minipage}%
\end{center}

\newpage

\begin{center}
  \textcolor{red}{\normalsize{Salut du Très Saint Sacrement}}\\
  \textit{Chant d'exposition}
\end{center}

\gresetinitiallines{1}
\greillumination{\initfamily\fontsize{11mm}{11mm}\selectfont O}
\gregorioscore{hy--o_salutaris_hostia--abbe_du_gue}

\begin{multicols}{2}
  \parindent=0pt
  \begin{flushright}
    O vere digna Hostia,\\
    Spes unica fidelium,\\
    In te confidit Francia,\\
    Da pacem, serva lilium.\\
  \end{flushright}
  \columnbreak
  \textit{
    Ô vraiment digne Hostie,\\
    Unique espoir des fidèles,\\
    en toi se confie la France,\\
    Donne-lui la paix, conserve le lys.\\
  }
\end{multicols}
\smallskip
\begin{multicols}{2}
  \begin{flushright}
    Uni trinoque Domino\\
    Sit sempiterna gloria :\\
    Qui vitam sine termino,\\
    Nobis donet in patria. Amen.\\
  \end{flushright}
  \columnbreak
  \textit{
    Au Seigneur unique en trois personnes,\\
    La gloire éternelle;\\
    qu'il nous donne en son Royaume\\
    La vie qui n'aura pas de fin. Amen\\
  }
\end{multicols}

\begin{center}
  \rule{2cm}{0.4pt}
\end{center}

\begin{center}
  \textcolor{red}{\normalsize{En l'honneur de la Vierge Marie}}\\
  \textit{Voir au propre du jour}
\end{center}

\newpage


\begin{center}
  \textcolor{red}{\normalsize{En l'honneur Du Saint Sacrement}}
\end{center}

\gresetinitiallines{1}
\greillumination{\initfamily\fontsize{11mm}{11mm}\selectfont T}
\gregorioscore{hy--tantum_ergo--solesmes}

\begin{multicols}{2}
  \parindent=0pt
  \textcolor{red}{\Vbar.} Panem de caelo praestitisti eis.\\
  \textcolor{red}{\Rbar.} Omne delectamentum in se habentem.\\
  
  \textit{\textcolor{red}{\Vbar.} Tu leur a donné le pain du ciel.\\
  \textcolor{red}{\Rbar.} Toute saveur se trouve en lui.}\\
  
\end{multicols}

\newpage

\begin{center}
  \textcolor{red}{\normalsize{Oraison}}
\end{center}

\begin{multicols}{2}
  \parindent=0pt
  Deus, qui nobis sub sacramento mirabili
  passionis tuæ memoriam reliquisti :
  tribue, quæsumus, ita nos Corporis et
  Sanguinis tui sacra mysteria venerari, ut
  redemptionis tuæ fructum in nobis
  jugiter sentiamus. Qui vivis et regnas
  cum Deo Patre in unitate Spiritus Sancti,
  Deus, per omnia sæcula sæculorum.
  Amen.
  \smallskip
  \columnbreak

  \textit{
    Seigneur Jésus Christ, dans cet admirable
    sacrement tu nous a laissé le mémorial de
    ta passion ; donne-nous de vénérer d’un si
    grand amour le mystère de ton Corps et de
    ton Sang, que nous puissions recueillir
    sans cesse le fruit de ta rédemption. Toi
    qui règnes avec le Père et le Saint Esprit
    pour les siècles des siècles.
    Amen. 
  }
\end{multicols}

\begin{center}
  \rule{2cm}{0.4pt}
\end{center}

\begin{center}
  \textcolor{red}{\normalsize{Louanges divines}}
\end{center}

\parindent=0pt
Dieu soit béni.\\
Béni soit son Saint Nom.\\
Béni soit Jésus-Christ, vrai Dieu et vrai homme.\\
Béni soit le Nom de Jésus.\\
Béni soit son Sacré Cœur.\\
Béni soit son précieux Sang.\\
Béni soit Jésus dans le très Saint Sacrement de l’autel.\\
Béni soit l’Esprit Saint Consolateur.\\
Bénie soit l’auguste Mère de Dieu, la très Sainte Vierge Marie.\\
Bénie soit sa Sainte et Immaculée Conception.\\
Bénie soit sa glorieuse Assomption.\\
Béni soit le nom de Marie, Vierge et Mère.\\
Béni soit Saint Joseph, son très chaste époux.\\
Béni soit Dieu dans ses anges et dans ses saints.\\
Seigneur, donnez-nous des prêtres.\\
Seigneur, donnez-nous de saints prêtres.\\
Seigneur, donnez-nous beaucoup de saints prêtres.\\
Seigneur, donnez-nous beaucoup de saintes vocations religieuses.\\

\begin{center}
  \rule{2cm}{0.4pt}
\end{center}

\newpage

\begin{center}
  \textcolor{red}{\normalsize{Déposition}}\\
  \textit{Psaume 116}
\end{center}

\gresetinitiallines{1}
\greillumination{\initfamily\fontsize{11mm}{11mm}\selectfont L}
\gregorioscore{ps--laudate_dominum_omnes_gentes_(psalmus_116)--solesmes}

\end{document}
