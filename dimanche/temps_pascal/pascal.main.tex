% !TeX program = lualatex
\documentclass[12pt, a4paper]{article}
\usepackage{fullpage}
\usepackage{subfiles}
\usepackage{fontspec}
\usepackage{libertine}
\usepackage{xcolor}
\usepackage{GotIn}
\usepackage{geometry}
\usepackage{multicol}
\usepackage{multicolrule}
\usepackage{graphicx}
\usepackage{enumitem}
\usepackage[autocompile]{gregoriotex}

\geometry{top=2cm, bottom=2cm}
\pagestyle{empty}

\definecolor{red}{HTML}{C70039}
% \input GoudyIn.fd
% \newcommand*\initfamily{\usefont{U}{GoudyIn}{xl}{n}}

\input Acorn.fd
\newcommand*\initfamily{\usefont{U}{Acorn}{xl}{n}}
% cette ligne ajoute de l'espace entre les portées
% \grechangedim{baselineskip}{60pt}{scalable}

\begin{document} 
  \gresetlinecolor{gregoriocolor}

  \begin{titlepage}\centering
    \vspace*{\fill}\
    \huge Secondes Vêpres\\
    \smallskip
    \begin{normalsize}
      \textit{
        du Temps Pascal.\\
        Depuis le Dimanche de Pâques à la Pentecôte.\\
      }
    \end{normalsize}
    \medskip
    \large et\\
    \medskip
    \LARGE Salut du Saint-Sacrement
    \bigskip
    \begin{figure}[h!]
      \centering
      \includegraphics[width=7cm]{../ordinaires/logo.png}
    \end{figure}

    \vspace*{\fill}
    \normalsize\textit{
      Livret latin-français
    }
  \end{titlepage}

  \newpage

  \vspace{5mm}
  \begin{center}
    \textcolor{red}{\large{Ouverture.}}
  \end{center}

  % greillumination: remplace la première lettre, ici par une font ornementale
  \greillumination{\initfamily\fontsize{11mm}{11mm}\selectfont D}
  \gregorioscore{vepres-deus_in_adjutorium}
  \medskip
  \par \textcolor{red}{\textit{Depuis la Septuagesime jusqu'à Pâques, au lieu de l'Allelúia}}
  \gresetinitiallines{0}
  \gabcsnippet{(c4)  L{au}s(j) tí(j)bi(j) Dó(j)mi(j)ne(j'_) Rex(j) æ(j)tér(j)næ(k) gló(j')ri(j)æ.(i.) (::)}
  \gresetinitiallines{1}
  \medskip

  \begin{center}
    \small{
    \emph{
      Dieu, venez à mon aide ; Seigneur, hatez-vous de me secourir.\\
      Gloire au Père, au Fils et au Saint Esprit, comme il était au commencement, maintenant et toujour et dans les siècles des siècles. Allelúia\\
      Ainsi soit-il. Louange à vous, Seigneur, Roi d’éternelle gloire !
    }
  }
  \end{center}

  \newpage

  

  \subfile{apres-paques.tex}

  \newpage

  \begin{center}
    \textcolor{red}{\large{Salut du Très Saint Sacrement}}\\
    \textit{Chant d'exposition}
  \end{center}

  \smallskip
  \begin{figure}[h!]
    \centering
    \includegraphics[width=\linewidth]{../ordinaires/o-salutaris.jpg}
  \end{figure}

  \begin{center}
    \begin{footnotesize}
      \textit{
        Ô réconfortante
        Hostie, Qui nous
        ouvres les portes du
        ciel, les armées ennemies
        nous poursuivent,
        Donne-nous la force,
        porte-nous secours.
      }
    \end{footnotesize}
  \end{center}

  \begin{multicols}{2}
    \parindent=0pt
    \begin{flushright}
      O vere digna Hostia,\\
      Spes unica fidelium,\\
      In te confidit Francia,\\
      Da pacem, serva lilium.\\
    \end{flushright}
    \columnbreak
    \textit{
      Ô vraiment digne Hostie,\\
      Unique espoir des fidèles,\\
      en toi se confie la France,\\
      Donne-lui la paix, conserve le lys.\\
    }
  \end{multicols}
  \begin{multicols}{2}
    \begin{flushright}
      Uni trinoque Domino\\
      Sit sempiterna gloria :\\
      Qui vitam sine termino,\\
      Nobis donet in patria. Amen.\\
    \end{flushright}
    \columnbreak
    \textit{
      Au Seigneur unique en trois personnes,\\
      La gloire éternelle;\\
      qu'il nous donne en son Royaume\\
      La vie qui n'aura pas de fin. Amen\\
    }
  \end{multicols}

  \begin{center}
    \rule{2cm}{0.4pt}
  \end{center}

  \newpage

  % \begin{center}
  %   \textcolor{red}{\normalsize{Antienne à la Sainte Vierge.}}\\
  % \end{center}
  \begin{center}
    \textcolor{red}{\large{Regína Caéli}}\\
    \begin{footnotesize}
      \textit{
      Du Dimanche de Pâques jusqu'au Vendredi après la Pentecôte inclusivement.
      }
    \end{footnotesize}
  \end{center}

  \gresetinitiallines{1}
  \greillumination{\initfamily\fontsize{11mm}{11mm}\selectfont A}
  \gregorioscore{antiennes/an--regina_caeli_laetare--solesmes}
  \medskip
  \begin{footnotesize}
    \textit{
      Reine du ciel, réjouissez-vous, alléluia, Car celui que vous avez mérité de porter, alléluia, Est ressuscité, comme il l’avait dit, alléluia, Priez pour nous Dieu, alléluia. 
    }
  \end{footnotesize}

  \begin{multicols}{2}
    \parindent=0pt
    \begin{flushright}
      \textcolor{red}{\Vbar.} Gaude et lætáre Virgo María, alleluia.\\
      \textcolor{red}{\Rbar.} Quia surrexit Dóminus vere, alleluia.\\
    \end{flushright}

    \columnbreak
    
    \textit{\textcolor{red}{\Vbar.} Réjouis-toi et sois dans la joie, Vierge Marie, alléluia.\\
    \textcolor{red}{\Rbar.} Parce que le Seigneur est vraiment ressuscité, alléluia.}\\
  \end{multicols}

  \begin{multicols}{2}
    \parindent=0pt
    Deus qui per resurrectiónem Fílii tui Dómini nostri Jesu Christi mundum lætificáre dignátus es :\textcolor{red}{†} præsta, quæsumus ; ut per ejus Genitrícem Vírginem Maríam \textcolor{red}{*} perpétuæ capiámus gáudia vitæ.\\ Per eúmdem Christum Dóminum nostrum.\\ 
    \textcolor{red}{\Rbar.} Amen.

    \columnbreak

    \textit{Dieu qui par la résurrection de votre Fils notre Seigneur Jésus Christ avez daigné réjouir le monde, faites, s’il vous plaît, que par sa Mère la Vierge Marie, nous goûtions les joies d’une vie éternelle.\\ Par le même Christ notre Seigneur.\\
    Amen.
    }
  \end{multicols}

  \begin{center}
    \rule{2cm}{0.4pt}
  \end{center}

  \begin{center}
    \textcolor{red}{\large{En l'honneur Du Saint Sacrement}}
  \end{center}

  \gresetinitiallines{1}
  \greillumination{\initfamily\fontsize{11mm}{11mm}\selectfont T}
  \gregorioscore{../ordinaires/hy--tantum_ergo--solesmes}

  \begin{center}
    \begin{footnotesize}
      \begin{enumerate}[label=\textcolor{red}{\emph{\arabic*}}]
        \item \textit{Devant un sacrement si grand, prosternons-nous, adorons ; et que les symboles anciens s'effacent devant le rite nouveau ; que la foi vienne suppléer à la faiblesse de nos sens.}
        \item \textit{Au Père et au Fils louanges et acclamations, gloire honneur et puissance ainsi que bénédictions. A Celui qui de tous deux procède offrons une égale louange.}
      \end{enumerate}
    \end{footnotesize}
  \end{center}

  \medskip

  \begin{multicols}{2}
    \parindent=0pt
    \textcolor{red}{\Vbar.} Panem de caelo praestitisti eis.\\
    \textcolor{red}{\Rbar.} Omne delectamentum in se habentem.\\
    
    \textit{\textcolor{red}{\Vbar.} Tu leur a donné le pain du ciel.\\
    \textcolor{red}{\Rbar.} Toute saveur se trouve en lui.}\\
    
  \end{multicols}

  \begin{center}
    \rule{2cm}{0.4pt}
  \end{center}

  \begin{center}
    \textcolor{red}{\large{Oraison}}
  \end{center}

  \begin{multicols}{2}
    \parindent=0pt
    Deus, qui nobis sub sacramento mirabili
    passionis tuæ memoriam reliquisti : \textcolor{red}{~†}
    tribue, quæsumus, ita nos Corporis et
    Sanguinis tui sacra mysteria venerari, \textcolor{red}{~*} ut
    redemptionis tuæ fructum in nobis
    jugiter sentiamus.\\
    Qui vivis et regnas
    cum Deo Patre in unitate Spiritus Sancti,
    Deus, per omnia sæcula sæculorum.
    Amen.
    \columnbreak

    \textit{
      Seigneur Jésus Christ, dans cet admirable
      sacrement tu nous a laissé le mémorial de
      ta passion ; donne-nous de vénérer d’un si
      grand amour le mystère de ton Corps et de
      ton Sang, que nous puissions recueillir
      sans cesse le fruit de ta rédemption. Toi
      qui règnes avec le Père et le Saint Esprit
      pour les siècles des siècles.
      Amen. 
    }
  \end{multicols}

  \begin{center}
    \rule{2cm}{0.4pt}
  \end{center}


  \begin{center}
    \textcolor{red}{\large{Louanges divines}}
  \end{center}


  \begin{normalsize}
    \parindent=0pt
    Dieu soit béni.\\
    Béni soit son Saint Nom.\\
    Béni soit Jésus-Christ, vrai Dieu et vrai homme.\\
    Béni soit le Nom de Jésus.\\
    Béni soit son Sacré Cœur.\\
    Béni soit son précieux Sang.\\
    Béni soit Jésus dans le très Saint Sacrement de l’autel.\\
    Béni soit l’Esprit Saint Consolateur.\\
    Bénie soit l’auguste Mère de Dieu, la très Sainte Vierge Marie.\\
    Bénie soit sa Sainte et Immaculée Conception.\\
    Bénie soit sa glorieuse Assomption.\\
    Béni soit le nom de Marie, Vierge et Mère.\\
    Béni soit Saint Joseph, son très chaste époux.\\
    Béni soit Dieu dans ses anges et dans ses saints.\\
    Seigneur, donnez-nous des prêtres.\\
    Seigneur, donnez-nous de saints prêtres.\\
    Seigneur, donnez-nous beaucoup de saints prêtres.\\
    Seigneur, donnez-nous beaucoup de saintes vocations religieuses.\\
  \end{normalsize}


  \newpage

  \begin{center}
    \textcolor{red}{\large{Déposition}}\\
    \textit{Psaume 116}
  \end{center}

  \gresetinitiallines{1}
  \greillumination{\initfamily\fontsize{11mm}{11mm}\selectfont L}
  \gregorioscore{psaumes/ps--laudate_dominum_omnes_gentes_(psalmus_116)--solesmes}
  \bigskip
  \begin{footnotesize}
    \textit{
      Louez le Seigneur, tous les
      peuples ;
      Fêtez-Le, tous les pays !
      Son Amour envers nous
      S'est montré le plus fort ;
      Eternelle est la Fidélité du
      Seigneur !
      Gloire au Père, au Fils
      Et au Saint-Esprit,
      Comme il était au
      commencement,
      Maintenant et toujours,
      Pour les siècles des siècles,
      amen.
    }
  \end{footnotesize}

  \newpage

  \begin{titlepage}\centering
    \vspace*{\fill}
    \LARGE Propre du temps.
    \vspace*{\fill}
  \end{titlepage}

  \newpage

  \subfile{propre-temps-pascal.tex}

\end{document}