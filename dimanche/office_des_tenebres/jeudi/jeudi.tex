% !TeX program = lualatex
\documentclass[12pt, a4paper]{article}
\usepackage{fullpage}
\usepackage{subfiles}
\usepackage{fontspec}
\usepackage{libertine}
\usepackage{xcolor}
\usepackage{GotIn}
\usepackage{geometry}
\usepackage{multicol}
\usepackage{multicolrule}
\usepackage{graphicx}
\usepackage{enumitem}
\usepackage[autocompile]{gregoriotex}

\geometry{top=1cm, bottom=1cm, right=1cm, left=1cm}
\pagestyle{empty}

\definecolor{red}{HTML}{C70039}
% \input GoudyIn.fd
% \newcommand*\initfamily{\usefont{U}{GoudyIn}{xl}{n}}

\input Acorn.fd
\newcommand*\initfamily{\usefont{U}{Acorn}{xl}{n}}
% cette ligne ajoute de l'espace entre les portées
% \grechangedim{baselineskip}{60pt}{scalable}

\begin{document}
\gresetlinecolor{gregoriocolor}
  % ===== DEBUT Antienne =========
  \gresetinitiallines{1}
  \greillumination{\initfamily\fontsize{11mm}{11mm}\selectfont Z}
  \gregorioscore{antiennes/an--zelus_domus_tuae--solesmes_1961}
  \begin{center}
    \footnotesize{
      \textit{
        Le zèle de votre maison m'a dévoré ; et sur moi sont tombés les opprobres de ceux qui s'attaquaient à vous.
      }
    }
  \end{center}
  % ===== FIN Antienne ===========

  % ===== DEBUT psaume ===========
  % gresetinitiallines : avec le parametre à 0, supprime l'ornement
  \begin{center}
    \large{Psaume 68.}\\
  \end{center}

  \gresetinitiallines{0}
  \gregorioscore{psaumes/psaume68-VIIIc}

  \begin{enumerate}[label=\textcolor{red}{\arabic*}]
    \setcounter{enumi}{1}
    \item Infíxus sum in limo pro\textbf{fún}di:\textcolor{red}{~*} et non \textit{est} \textit{sub}\textbf{stán}tia.

    \item Veni in altitúdinem \textbf{ma}ris:\textcolor{red}{~*} et tempés\textit{tas} \textit{de}\textbf{mér}sit me.

    \item Laborávi clamans, raucæ factæ sunt fauces \textbf{me}æ:\textcolor{red}{~*} defecérunt óculi mei, dum spero in \textit{De}\textit{um} \textbf{me}um.

    \item Multiplicáti sunt super capíllos cápitis \textbf{me}i,\textcolor{red}{~*} qui odé\textit{runt} \textit{me} \textbf{gra}tis.

    \item Confortáti sunt qui persecúti sunt me inimíci mei in\textbf{jús}te:\textcolor{red}{~*} quæ non rápui, tunc \textit{ex}\textit{sol}\textbf{vé}bam.

    \item Deus, tu scis insipiéntiam \textbf{me}am:\textcolor{red}{~*} et delícta mea a te non \textit{sunt} \textit{abs}\textbf{cón}dita.

    \item Non erubéscant in me qui exspéctant te, \textbf{Dó}mine,\textcolor{red}{~*} Dómi\textit{ne} \textit{vir}\textbf{tú}tum

    \item Non confundántur \textbf{su}per me\textcolor{red}{~*} qui quærunt te, \textit{De}\textit{us} \textbf{Is}raël.

    \item Quóniam propter te sustínui op\textbf{pró}brium:\textcolor{red}{~*} opéruit confúsio fá\textit{ci}\textit{em} \textbf{me}am.

    \item Extráneus factus sum frátribus \textbf{me}is,\textcolor{red}{~*} et peregrínus fíliis \textit{ma}\textit{tris} \textbf{me}æ.

    \item Quóniam zelus domus tuæ com\textbf{é}dit me:\textcolor{red}{~*} et oppróbria exprobrántium tibi ceci\textit{dé}\textit{runt} \textbf{su}per me.

    \item Et opérui in jejúnio ánimam \textbf{me}am:\textcolor{red}{~*} et factum est in oppró\textit{bri}\textit{um} \textbf{mi}hi.

    \item Et pósui vestiméntum meum ci\textbf{lí}cium:\textcolor{red}{~*} et factus sum illis \textit{in} \textit{pa}\textbf{rá}bolam.

    \item Advérsum me loquebántur, qui sedébant in \textbf{por}ta:\textcolor{red}{~*} et in me psallébant qui bi\textit{bé}\textit{bant} \textbf{vi}num.

    \item Ego vero oratiónem meam ad te, \textbf{Dó}mine:\textcolor{red}{~*} tempus beneplá\textit{ci}\textit{ti}, \textbf{De}us.

    \item In multitúdine misericórdiæ tuæ ex\textbf{áu}di me,\textcolor{red}{~*} in veritáte sa\textit{lú}\textit{tis} \textbf{tu}æ:

    \item Eripe me de luto, ut non in\textbf{fí}gar:\textcolor{red}{~*} líbera me ab iis, qui odérunt me, et de profún\textit{dis} \textit{a}\textbf{quá}rum.

    \item Non me demérgat tempéstas aquæ,\textcolor{red}{~†} neque absórbeat me pro\textbf{fún}dum:\textcolor{red}{~*}  neque úrgeat super me púte\textit{us} \textit{os} \textbf{su}um.

    \item Exáudi me, Dómine, quóniam benígna est misericórdia \textbf{tu}a:\textcolor{red}{~*}  secúndum multitúdinem miseratiónum tuárum ré\textit{spi}\textit{ce} \textbf{in} me.

    \item Et ne avértas fáciem tuam a púero \textbf{tu}o:\textcolor{red}{~*} quóniam tríbulor, velóci\textit{ter} \textit{ex}\textbf{áu}di me.

    \item Inténde ánimæ meæ, et líbera \textbf{e}am:\textcolor{red}{~*} propter inimícos meos \textit{é}\textit{ri}\textbf{pe} me.

    \item Tu scis impropérium meum, et confusiónem \textbf{me}am,\textcolor{red}{~*} et reverén\textit{ti}\textit{am} \textbf{me}am.

    \item In conspéctu tuo sunt omnes qui tríbu\textbf{lant} me:\textcolor{red}{~*} impropérium exspectávit cor meum, \textit{et} \textit{mi}\textbf{sé}riam.

    \item Et sustínui qui simul contristarétur, et non \textbf{fu}it:\textcolor{red}{~*} et qui consolarétur, et \textit{non} \textit{in}\textbf{vé}ni.

    \item Et dedérunt in escam \textbf{me}am fel:\textcolor{red}{~*} et in siti mea potavérunt \textit{me} \textit{a}\textbf{cé}to.

    \item Fiat mensa eórum coram ipsis in \textbf{lá}queum,\textcolor{red}{~*} et in retributiónes, \textit{et} \textit{in} \textbf{scán}dalum.

    \item Obscuréntur óculi eórum ne \textbf{ví}deant:\textcolor{red}{~*} et dorsum eórum sem\textit{per} \textit{in}\textbf{cúr}va.

    \item Effúnde super eos iram \textbf{tu}am:\textcolor{red}{~*} et furor iræ tuæ compre\textit{hén}\textit{dat} \textbf{e}os.

    \item Fiat habitátio eórum de\textbf{sér}ta:\textcolor{red}{~*} et in tabernáculis eórum non sit \textit{qui} \textit{in}\textbf{há}bitet.

    \item Quóniam quem tu percussísti, perse\textbf{cú}ti sunt:\textcolor{red}{~*} et super dolórem vúlnerum meórum \textit{ad}\textit{di}\textbf{dé}runt.

    \item Appóne iniquitátem super iniquitátem e\textbf{ó}rum:\textcolor{red}{~*} et non intrent in justí\textit{ti}\textit{am} \textbf{tu}am.

    \item Deleántur de libro vi\textbf{vén}tium:\textcolor{red}{~*} et cum justis \textit{non} \textit{scri}\textbf{bán}tur.

    \item Ego sum pauper et \textbf{do}lens:\textcolor{red}{~*} salus tua, De\textit{us}, \textit{su}\textbf{scé}pit me.

    \item Laudábo nomen Dei cum \textbf{cán}tico:\textcolor{red}{~*} et magnificábo e\textit{um} \textit{in} \textbf{lau}de:

    \item Et placébit Deo super vítulum no\textbf{vél}lum:\textcolor{red}{~*} córnua producén\textit{tem} \textit{et} \textbf{ún}gulas.

    \item Vídeant páuperes et læ\textbf{tén}tur:\textcolor{red}{~*} quǽrite Deum, et vivet á\textit{ni}\textit{ma} \textbf{ves}tra.

    \item Quóniam exaudívit páuperes \textbf{Dó}minus:\textcolor{red}{~*} et vinctos suos \textit{non} \textit{de}\textbf{spé}xit.

    \item Laudent illum cæli et \textbf{ter}ra,\textcolor{red}{~*} mare et ómnia reptíli\textit{a} \textit{in} \textbf{e}is.

    \item Quóniam Deus salvam fáciet \textbf{Si}on:\textcolor{red}{~*} et ædificabúntur civi\textit{tá}\textit{tes} \textbf{Ju}da.

    \item Et inhabitábunt \textbf{i}bi,\textcolor{red}{~*} et hereditáte ac\textit{quí}\textit{rent} \textbf{e}am.

    \item Et semen servórum ejus possidébit \textbf{e}am:\textcolor{red}{~*} et qui díligunt nomen ejus, habitá\textit{bunt} \textit{in} \textbf{e}a.
  \end{enumerate}

  \grecommentary{\textit{Reprise de l'Antienne.}}
  \gabcsnippet{(c4) Ze(j)lus(j) do(j)mus(j) tu(j_k)ae(ji/jkj') co(h)mé(j')dit(i) me,(g.) (;) et(f) op(h)pró(j')bri(i)a(g_[uh:l]h) ex(f)pro(h)brán(j')ti(k)um(i') ti(j)bi(h.) (,) ce(h)ci(g)dé(hi)runt(h') su(g)per(fg~) me.(g.) (::)}

  \medskip
  \begin{multicols}{2}
    \begin{footnotesize}
      \begin{enumerate}[label=\textcolor{red}{\emph{\arabic*}}]
        \item \textit{O mon Dieu, sauvez-moi, parce que les eaux ont pénétré jusques dans mon âme}
        \item \textit{Je suis enfoncé dans une boue profonde, où je ne
        trouve point de fermeté.}
        \item \textit{Je suis tombé dans la mer profonde ; et la tempête
        m’a submergé.}
        \item \textit{Je me suis fatigué en criant, ma gorge en a été enrouée ; mes yeux se sont fermés de faiblesse, tandis que j’espère en mon Dieu.}
        \item \textit{Ceux qui me haïssent sans sujet, se sont multipliés
        plus que les cheveux de ma tête.}
        \item \textit{Mes ennemis qui me persécutaient injustement, se
        sont fortifiés contre moi ; alors j’ai payé ce que je n’avais pas pris.}
        \item \textit{Mon Dieu, vous connaissez ma folie, et mes
        crimes ne vous sont point cachés.}
        \item \textit{Seigneur, souverain des vertus ; que ceux qui espèrent en vous ne rougissent point à cause de moi.}
        \item \textit{O Dieu d’Israël, que ceux qui vous cherchent,
        n’aient point de confusion à mon sujet.}
        \item \textit{Car c’est à cause de vous que j’ai souffert les opprobres, et que mon visage a été couvert de confusion.}
        \item \textit{Je suis devenu comme étranger à mes frères, et comme inconnu aux enfants de ma mère ;}
        \item \textit{Parce que le zèle de votre maison me dévore, et que les opprobres de ceux qui vous ont outragé, sont retombées sur moi.}
        \item \textit{Je me suis couvert d’un sac pendant mon jeûne, ce qui est devenu pour moi un sujet d’opprobre.} 
        \item \textit{J’ai pris pour vêtement un cilice, ce qui m’a encore rendu l’objet de leur railleries.}
        \item \textit{Ceux qui étaient assis à la porte, parlaient contre moi, et ceux qui buvaient du vin se moquaient de moi dans leurs chansons.}
        \item \textit{Mais pour moi, Seigneur, je vous adressais ma prière ; ô mon Dieu, voici le temps de votre bonté.}
        \item \textit{Exaucez-moi dans la multitude de votre miséricorde, et dans la vérité des promesses qui regardent mon salut.}
        \item \textit{Retirez-moi du bourbier, afin que je ne m’y enfonce pas davantage ; délivrez-moi de ceux qui me haïssent, et de la profondeur des eaux.}
        \item \textit{Que la tempête ne me submerge point, que je ne sois point enseveli dans l’abîme, et que la bouche du puits ne soit point fermée sur moi.}
        \item \textit{Seigneur, exaucez-moi, car votre miséricorde est bienfaisante et toute remplie de douceur ; regardez-moi favorablement selon la multitude de vos miséricordes.}
        \item \textit{Ne détournez pas votre visage de dessus votre serviteur ; exaucez- moi promptement, car je suis dans le trouble.}
        \item \textit{Soyez attentif sur mon âme, et délivrez-la ; sauvez-moi, à cause de mes ennemis.}
        \item \textit{Vous savez les opprobres où ils m’ont jeté ; vous voyez la confusion et l’ignominie dont je suis couvert.}
        \item \textit{Tous ceux qui me persécutent sont sous vos yeux ; mon cœur n’a envisagé que l’opprobre et la misère.}
        \item \textit{J’ai attendu que quelqu’un prit part à ma tristesse, et personne ne s’est présenté ; ou que quelqu’un me consolât, mais je n’en ai point trouvé}
        \item \textit{Ils m’ont donné du fiel pour ma nourriture, et ils m’ont présenté du vinaigre durant ma soif.}
        \item \textit{Que leur table soit devant eux comme un piège, et qu’elle leur soit une punition et une pierre de scandale.}
        \item \textit{Que leurs yeux s’obscurcissent, afin qu’ils ne voient point ; et que leurs dos soient toujours courbés.}
        \item \textit{Faites tomber sur eux votre colère, et que la fureur de votre indignation les accable.}
        \item \textit{Que leur maison devienne déserte, et que l’on ne trouve personne qui habite dans leurs tentes.}
        \item \textit{Car ils ont persécuté celui que vous avez frappé, et ils ont ajouté de nouvelles blessures à mes plaies.}
        \item \textit{Permettez qu’ils ajoutent iniquité sur iniquité, et qu’ils n’entrent point dans votre justice}
        \item \textit{Qu’ils soient effacés du livre des vivants, et qu’ils ne soient point écrits parmi les justes.}
        \item \textit{Je suis pauvre et affligé ; mais, mon Dieu, votre protection m’a soutenu.}
        \item \textit{Je louerai le nom de Dieu dans mes Cantiques, et je le glorifierai par mes louanges.}
        \item \textit{Elles seront plus agréables à Dieu que le sacrifice d’un jeune veau, dont les cornes et les ongles commencent à pousser.}
        \item \textit{Que les pauvres le voient et qu’ils se réjouissent ; cherchez Dieu, et votre âme vivra.}
        \item \textit{Car le Seigneur a exaucé les pauvres, et il n’a pas méprisé ceux qui étaient dans l’esclavage.}
        \item \textit{Que les cieux et la terre le louent, aussi bien que la mer et les animaux qu’elle renferme.}
        \item \textit{Parce que Dieu sauvera Sion, et que les villes de Juda seront bâties.}
        \item \textit{C’est là qu’ils habiteront, quand ils en seront mis
        en possession comme d’un héritage.}
        \item \textit{Et la postérité de ses serviteurs la possèdera, et ceux qui aiment son nom y feront leur demeure.}
      \end{enumerate}
    \end{footnotesize}
  \end{multicols}

  % ===== FIN psaume ===========

  \bigskip

  \par Le psalmiste continue à exprimer les plaintes du Sauveur délaissé et à prédire le châtiment réservé à ses ennemis.
  \medskip

  % ===== DEBUT Antienne =========
  \gresetinitiallines{1}
  \greillumination{\initfamily\fontsize{11mm}{11mm}\selectfont A}
  \gregorioscore{antiennes/an--avertantur_retrorsum--solesmes_1961}
  \begin{center}
    \footnotesize{
      \textit{
        Que ceux qui me veulent du mal, soient repoussés en arrière, et couverts de confusion.
      }
    }
  \end{center}
  % ===== FIN Antienne ===========

  % ===== DEBUT psaume ===========
  % gresetinitiallines : avec le parametre à 0, supprime l'ornement
  \begin{center}
    \large{Psaume 69.}\\
  \end{center}

  \gresetinitiallines{0}
  \gregorioscore{psaumes/psaume69-VIIIc}

  \begin{enumerate}[label=\textcolor{red}{\arabic*}]
    \setcounter{enumi}{1}
    \item Confundántur et revere\textbf{án}tur,\textcolor{red}{~*} qui quærunt á\textit{ni}\textit{mam} \textbf{me}am.

    \item Avertántur retrórsum, et eru\textbf{bés}cant,\textcolor{red}{~*} qui volunt \textit{mi}\textit{hi} \textbf{ma}la.

    \item Avertántur statim erube\textbf{scén}tes,\textcolor{red}{~*} qui dicunt mihi: \textit{Eu}\textit{ge}, \textbf{eu}ge.

    \item Exsúltent et læténtur in te omnes qui \textbf{quæ}runt te,\textcolor{red}{~*} et dicant semper: Magnificétur Dóminus: qui díligunt salu\textit{tá}\textit{re} \textbf{tu}um.

    \item Ego vero egénus, et \textbf{pau}per sum:\textcolor{red}{~*} Deus, \textit{ád}\textit{ju}\textbf{va} me.

    \item Adjútor meus, et liberátor meus \textbf{es} tu:\textcolor{red}{~*} Dómine, \textit{ne} \textit{mo}\textbf{ré}ris.

  \end{enumerate}

  \grecommentary{\textit{Reprise de l'Antienne.}}
  \gabcsnippet{(c4) A(j)ver(j)tán(j_k)tur(k') re(k)trór(kj~)sum,(kj__) (,) et(j) e(ji)ru(h!iwj)bé(jkj)scant,(g.) (;) qui(j) có(ikjk)gi(h)tant(f_h) mi(j_h)hi(i) ma(g.)la.(g.) (::)}

  \medskip

  \begin{multicols}{2}
    \begin{footnotesize}
      \begin{enumerate}[label=\textcolor{red}{\emph{\arabic*}}]
        \item \textit{O Dieu, venez à mon aide ; Seigneur, hâtez-vous de me secourir.}
        \item \textit{Que ceux qui en veulent à ma vie, soient confondus et couverts de honte.}
        \item \textit{Que ceux qui me veulent du mal, soient repoussés en arrière honteusement.}
        \item \textit{Que ceux qui me disent : Courage, courage, soient chassés avec confusion.}
        \item \textit{Que tous ceux qui vous cherchent, se réjouissent en vous, et soient comblés de joie ; et que ceux qui aiment le salut que vous donnez, disent toujours : Que le Seigneur soit glorifié.}
        \item \textit{Pour moi je suis pauvre et dans le besoin : ô Dieu, secourez-moi.}
        \item \textit{Vous êtes mon protecteur et mon libérateur : Seigneur, ne tardez pas davantage.}
      \end{enumerate}
    \end{footnotesize}
  \end{multicols}

  % ===== FIN psaume ===========

\bigskip


  \par Le troisième Psaume fut composé par David à l'occasion des persécutions qu'il eut a subir vers la fin de son règne. Le Messie se voit entouré d'ennemis furieux qui ne craignent pas de violer toutes les prescriptions de la loi pour obtenir sa mort ; ils disent que le Père céleste a abandonné son Fils pour le moment, et ils veulent en profiter. Pour lui, il ne cesse de poursuivre jusque sur la Croix sa mission de Docteur. La génération du peuple chrétien qui doit venir entendra sa voix, et apprendra à célébrer la puissance du bras divin, qui va paraître dans la prochaine résurection du Sauveur.
  \bigskip

  % ===== DEBUT Antienne =========
  \gresetinitiallines{1}
  \greillumination{\initfamily\fontsize{11mm}{11mm}\selectfont D}
  \gregorioscore{antiennes/an--deus_meus_eripe_me--solesmes_1961}
  \begin{center}
    \footnotesize{
      \textit{
        Mon Dieu, délivrez-moi de la main du pécheur
      }
    }
  \end{center}
  % ===== FIN Antienne ===========

  % ===== DEBUT psaume ===========
  % gresetinitiallines : avec le parametre à 0, supprime l'ornement
  \begin{center}
    \large{Psaume 70.}\\
  \end{center}

  \gresetinitiallines{0}
  \gregorioscore{psaumes/psaume70-VIIIc}

  \begin{enumerate}[label=\textcolor{red}{\arabic*}]
    \setcounter{enumi}{2}
    \item Esto mihi in Deum protectórem, et in locum mu\textbf{ní}tum:\textcolor{red}{~*} ut sal\textit{vum} \textit{me} \textbf{fá}cias.

    \item Quóniam firmaméntum \textbf{me}um,\textcolor{red}{~*} et refúgium \textit{me}\textit{um} \textbf{es} tu.

    \item Deus meus, éripe me de manu pecca\textbf{tó}ris,\textcolor{red}{~*} et de manu contra legem agéntis \textit{et} \textit{in}\textbf{í}qui:

    \item Quóniam tu es patiéntia mea, \textbf{Dó}mine:\textcolor{red}{~*} Dómine, spes mea a juven\textit{tú}\textit{te} \textbf{me}a.

    \item In te confirmátus sum ex \textbf{ú}tero:\textcolor{red}{~*} de ventre matris meæ tu es pro\textit{téc}\textit{tor} \textbf{me}us.

    \item In te cantátio mea semper:\textcolor{red}{~†} tamquam prodígium factus sum \textbf{mul}tis:\textcolor{red}{~*}  et tu ad\textit{jú}\textit{tor} \textbf{for}tis.

    \item Repleátur os meum laude, ut cantem glóriam \textbf{tu}am:\textcolor{red}{~*} tota die magnitú\textit{di}\textit{nem} \textbf{tu}am.

    \item Ne projícias me in témpore senec\textbf{tú}tis:\textcolor{red}{~*} cum defécerit virtus mea, ne \textit{de}\textit{re}\textbf{lín}quas me.

    \item Quia dixérunt inimíci mei \textbf{mi}hi:\textcolor{red}{~*} et qui custodiébant ánimam meam, consílium fecé\textit{runt} \textit{in} \textbf{u}num.

    \item Dicéntes: Deus derelíquit eum,\textcolor{red}{~†} persequímini, et comprehéndite \textbf{e}um:\textcolor{red}{~*}  quia non est \textit{qui} \textit{e}\textbf{rí}piat.

    \item Deus ne elongéris \textbf{a} me:\textcolor{red}{~*} Deus meus, in auxílium \textit{me}\textit{um} \textbf{ré}spice.

    \item Confundántur, et defíciant detrahéntes ánimæ \textbf{me}æ:\textcolor{red}{~*}  operiántur confusióne et pudóre, qui quærunt \textit{ma}\textit{la} \textbf{mi}hi.

    \item Ego autem semper spe\textbf{rá}bo:\textcolor{red}{~*} et adjíciam super omnem \textit{lau}\textit{dem} \textbf{tu}am.

    \item Os meum annuntiábit justítiam \textbf{tu}am:\textcolor{red}{~*} tota die salu\textit{tá}\textit{re} \textbf{tu}um.

    \item Quóniam non cognóvi litteratúram,\textcolor{red}{~†} introíbo in poténtias \textbf{Dó}mini:\textcolor{red}{~*}  Dómine, memorábor justítiæ tu\textit{æ} \textit{so}\textbf{lí}us.

    \item Deus, docuísti me a juventúte \textbf{me}a:\textcolor{red}{~*} et usque nunc pronuntiábo mirabí\textit{li}\textit{a} \textbf{tu}a.

    \item Et usque in senéctam et \textbf{sé}nium:\textcolor{red}{~*} Deus, ne \textit{de}\textit{re}\textbf{lín}quas me,

    \item Donec annúntiem bráchium \textbf{tu}um\textcolor{red}{~*} generatióni omni, \textit{quæ} \textit{ven}\textbf{tú}ra est:

    \item Poténtiam tuam, et justítiam tuam, Deus,\textcolor{red}{~†} usque in altíssima, quæ fecísti ma\textbf{gná}lia:\textcolor{red}{~*}  Deus, quis sí\textit{mi}\textit{lis} \textbf{ti}bi?

    \item Quantas ostendísti mihi tribulatiónes multas et malas:\textcolor{red}{~†} et convérsus vivifi\textbf{cás}ti me:\textcolor{red}{~*}  et de abýssis terræ íterum \textit{re}\textit{du}\textbf{xís}ti me:

    \item Multiplicásti magnificéntiam \textbf{tu}am:\textcolor{red}{~*} et convérsus conso\textit{lá}\textit{tus} \textbf{es} me.

    \item Nam et ego confitébor tibi in vasis psalmi veritátem \textbf{tu}am:\textcolor{red}{~*} Deus, psallam tibi in cíthara, \textit{Sanc}\textit{tus} \textbf{Is}raël.

    \item Exsultábunt lábia mea cum cantávero \textbf{ti}bi:\textcolor{red}{~*} et ánima mea, quam \textit{red}\textit{e}\textbf{mís}ti.

    \item Sed et lingua mea tota die meditábitur justítiam \textbf{tu}am:\textcolor{red}{~*}  cum confúsi et revériti fúerint, qui quærunt \textit{ma}\textit{la} \textbf{mi}hi.

  \end{enumerate}
  \grecommentary{\textit{Reprise de l'Antienne.}}
  \gabcsnippet{(c4) De(j)us(j') me(j)us,(jv_IH) é(i_[uh:l]j)ri(i)pe(g_[uh:l]h) me(g.) (,) de(j_k) ma(j)nu(ikjj) pec(h)ca(f')tó(g)ris.(g.) (::) (Z)}

  \medskip
  
  \begin{multicols}{2}
    \begin{footnotesize}
      \begin{enumerate}[label=\textcolor{red}{\emph{\arabic*}}]
        \item \textit{Seigneur, j’ai espéré en vous, je ne serai pas confondu pour jamais :
        délivrez-moi par votre justice, et sauvez-moi.}
        \item \textit{Que votre oreille soit attentive pour m’écouter, et sauvez-moi}
        \item \textit{Soyez-moi un Dieu protecteur, et une place forte
        et bien munie pour me sauver}
        \item \textit{Parce que vous êtes toute ma force, et vous êtes
        mon refuge.}
        \item \textit{Mon Dieu, délivrez-moi de la main du pécheur,
        et de la main de l’homme injuste qui agit contre
        votre loi.}
        \item \textit{Car, Seigneur, vous êtes ma patience ; Seigneur,
        vous êtes mon espérance dès ma plus tendre jeunesse.}
        \item \textit{J’ai été affermi en vous dès ma naissance ; vous
        êtes mon protecteur dès le temps que j’étais dans le
        sein de ma mère.}
        \item \textit{Je chanterai toujours vos louanges : j’ai paru
        comme un prodige à plusieurs ; mais vous êtes un
        puissant protecteur.}
        \item \textit{Que ma bouche soit remplie de louanges, pour
        chanter tout le jour votre gloire et votre grandeur.}
        \item \textit{Ne me rebutez pas dans le temps de ma vieillesse,
        et ne m’abandonnez pas lorsque ma force sera affaiblie.}
        \item \textit{Car mes ennemis m’ont décrié ; et ceux qui gardaient mon âme, ont formé ensemble des complots contre moi,}
        \item \textit{En disant : Dieu l’a abandonné, poursuivez-le,
        et saisissez-vous de lui, car personne ne peut le
        délivrer.}
        \item \textit{O Dieu, ne vous éloignez pas de moi ; mon Dieu, regardez-moi pour me secourir}
        \item \textit{Que ceux qui me décrient par leurs médisances, soient confondus, et qu’ils périssent. Que ceux qui cherchent à me faire du mal, soient couverts de honte et de confusion.}
        \item \textit{Mais pour moi, j’espèrerai toujours, et je vous donnerai de nouvelles louanges.}
        \item \textit{Ma bouche annoncera votre justice, et publiera tout le jour que vous êtes le salut.}
        \item \textit{Car je n’ai pas la connaissance des lettres ; je considèrerai la puissance du Seigneur : Seigneur, je me souviendrai seulement de votre justice.}
        \item \textit{Mon Dieu, vous m’avez instruit dès ma jeunesse ; et jusqu’à maintenant je publierai vos merveilles.}
        \item \textit{Ne m’abandonnez donc pas, ô Dieu, dans mon âge avancé, et dans ma vieillesse,}
        \item \textit{Jusqu’à ce que j’annonce la force de votre bras à toutes les races futures.}
        \item \textit{Votre puissance et votre justice, ô mon Dieu, sont dans la plus haute élévation, par les merveilles que vous avez opérées : ô Dieu, qui est semblable à vous ?}
        \item \textit{A combien d’afflictions différentes et cruelles m’avez-vous exposé ? vous vous êtes retourné vers moi, et vous m’avez redonné la vie, et encore une fois retiré des abîmes de la terre.}
        \item \textit{Vous m’avez donné plusieurs marques de votre magnificence, et vous m’avez consolé en vous tournant vers moi.}
        \item \textit{Car je louerai votre vérité sur des instruments de musique : mon Dieu, je vous chanterai des Cantiques sur la harpe, ô saint d’Israël.}
        \item \textit{Mes lèvres se réjouissent en chantant vos louanges, aussi bien que mon âme que vous avez rachetée.}
        \item \textit{Et ma langue publiera tout le jour votre justice, lorsque ceux qui cherchent à me faire du mal, seront confondus et couverts de honte.}
      \end{enumerate}
    \end{footnotesize}
  \end{multicols}

  % ===== FIN psaume ===========

  \begin{center}
    \rule{4cm}{0.4pt}
  \end{center}

  \begin{center}
    \begin{footnotesize}
      \textcolor{red}{\textit{On chante le verset debout.}}
    \end{footnotesize}
    \begin{minipage}{0.8\linewidth}
      \gresetinitiallines{0}
      \large
      \gabcsnippet{(c4)<c><v>\Vbar</v>.</c> A(h)ver(h)tán(h)tur(h) re(h)tró(h)rsum(h), et(h) e(i')ru(h)bés(g.)cant(g.) (::)(Z) <c><v>\Rbar</v>.</c> Qui(h) có(h)gi(h)tant(h) mi(i')hi(h) má(g.)la(g.) (::)}
      \bigskip
      \normalsize
      \begin{center}
        \textit{\textcolor{red}{\Vbar.} Que tous ceux qui me veulent du mal soient repoussés en arrière.}\\
        \textit{\textcolor{red}{\Rbar.} Et couverts de confusion.}
      \end{center}
    \end{minipage}
  \end{center}
  \normalsize

  \newpage

  \par Les formules préparatoires aux Leçons, telles que le \textit{Pater, l'Absolution}, ou les \textit{Bénédictions} sont omises. De même, on ne dit point le \textit{Tu autem} à la fin des Leçons.
  \par Les Leçons assignées au premier Nocturne de ces trois jours depuis la plus haute antiquité, sont tirées des Lamentations de Jérémie. Elles nous tracent le tableau saisissant du châtiment infligé à la nation déicide, et que le Roi-Prophète vient d'annoncer dans les Psaumes qui précèdent. Il est aisé aussi d'appliquer plusieurs traits de ces peintures émouvante à l'Homme-Dieu, lui-même, la fleur de tout Israël.
  \par Les mots hébreux qui se trouvent au commencement de chaque strophe sont les différentes lettres de l'alphabet hébraïque dont le poête sacré a suivi l'ordre dans le choix du mot qui commence chaque verset ; parfois chaque lettre est répétée jusqu'à trois fois : c'était là un des ornements de la poésie chez les juifs.
  \par Le chant qui accompagne les Lamentations ne semble point remonter à une haute antiquité. Il ne laisse pas toutefois de produire une impression profonde sur les âmes.
  \medskip

  \begin{center}
    \large Leçon I.\\
    \normalsize
  \end{center}
  % \grechangedim{baselineskip}{55pt}{scalable}
  \gresetinitiallines{1}
  \greillumination{\initfamily\fontsize{11mm}{11mm}\selectfont I}
  \gregorioscore{lamentations/va--incipit_lamentatio_ieremiae_prophetae--solesmes}
  % \newpage
  \begin{multicols}{2}
    \begin{footnotesize}
      \par \emph{Ici commence la Lamentation du prophète Jérémie.}
      \par \textcolor{red}{\textit{Aleph.}} Comment cette Ville si pleine de peuple, est-elle maintenant déserte ? La maîtresse des Nations est devenue comme une veuve ; la première
      des Provinces est contrainte de payer le tribut.
      \par \textcolor{red}{\textit{Beth.}} Elle a pleuré pendant la nuit ; ses larmes
      coulent sur ses joues. Nul de ses plus chers amis ne la console. Tous ses amis l’ont méprisée, et sont devenus ses ennemis.
      \par \textcolor{red}{\textit{Ghimel.}} Le peuple de Juda a changé de demeure, pour éviter l’affliction et la servitude rigoureuse. Il a habité parmi les nations, et n’a point trouvé de
      repos. Tous ses persécuteurs l’ont opprimé, et il n’a pu échapper de leurs mains.
      \par \textcolor{red}{\textit{Daleth.}} Les rues de Sion pleurent, parce que personne ne vient à la solemnité. Toutes ses portes sont détruites ; ses prêtres gémissent ; ses vierges sont languissantes, malpropres, et plongées dans l’amertume et dans la douleur.
      \par \textcolor{red}{\textit{Hé.}} Ses ennemis ont pris le dessus ; ses adversaires
      se sont enrichis de ses dépouilles ; parce que le Seigneur l’a prononcé en punition de la multitude de ses iniquités. Les plus jeunes ont été menés en captivité devant la face de ceux qui les chassaient cruellement.\\
      Jérusalem, Jérusalem, convertis-toi au Seigneur ton Dieu.
      \par \hspace{\fill}
    \end{footnotesize}
  \end{multicols}


  \greillumination{\initfamily\fontsize{11mm}{11mm}\selectfont I}
  \gregorioscore{repons/re--in_monte_oliveti--solesmes_1961}

  \smallskip

  \small
  \begin{multicols}{2}
    \par\textcolor{red}{\textit{\Rbar}.} \textit{Jésus pria son Père sur la montagne des Oliviers : Mon Père, s’il est possible, faites que ce calice s’éloigne de moi. \\ \textcolor{red}{*} L’esprit est prompt mais la chair est faible : Que votre volonté soit faite.}
    \columnbreak
    \par\textcolor{red}{\textit{\Vbar}.} \textit{Veillez, et priez, afin que vous ne tombiez point en tentation.\\
    \textcolor{red}{*} L’esprit est prompt mais la chair est faible : Que votre volonté soit faite.}
  \end{multicols}
  \normalsize

  \medskip
  % \newpage
  \begin{center}
    \large Leçon II.\\
    \normalsize
  \end{center}
  % \large
  \gresetinitiallines{1}
  \greillumination{\initfamily\fontsize{11mm}{11mm}\selectfont V}
  \gregorioscore{lamentations/va--vau_et_egressus_est--solesmes}
  \begin{multicols}{2}
    \begin{footnotesize}
      \par \textcolor{red}{\textit{Vau.}} Toute la beauté de la fille de Sion l’a quittée : ses
      Princes sont devenus comme des béliers qui ne
      trouvent point de pâturages, et ils se sont retirés
      sans force devant l’ennemi qui les poursuivait.
      \par \textcolor{red}{\textit{Zaïn.}} Jérusalem s’est ressouvenue du temps de son
      affliction, de ses prévarications, et de la perte de
      toutes les choses qu’elle affectionnait le plus, et
      dont elle jouissait de tout temps ; lorsque son
      peuple tombait entre les mains de ses ennemis,
      sans être secouru de personne. Ses ennemis l’ont
      vue, et ils se sont moqués de ses fêtes du Sabbat.
      \par \textcolor{red}{\textit{Heth.}} Jérusalem a commis un grand crime ; c’est
      pourquoi elle est devenue errante. Tous ceux qui la
      comblaient de louanges, l’ont méprisée, parce qu’ils
      ont vu son ignominie : elle a tourné la tête en arrière en gémissant.
      \par \textcolor{red}{\textit{Teth.}} Ses pieds sont souillés d’ordures, et elle ne
      s’est pas souvenue de sa fin. Elle a été extrêmement abattue, n’ayant point de consolateur.
      Voyez, Seigneur, mon affliction, parce que mon ennemi a pris le dessus.\\
      Jérusalem, Jérusalem, convertis-toi au Seigneur ton Dieu.
      % \par \hspace{\fill}
    \end{footnotesize}
  \end{multicols}

  % \newpage
  \medskip

  \greillumination{\initfamily\fontsize{11mm}{11mm}\selectfont T}
  \gregorioscore{repons/re--tristis_est--solesmes_1961}
  
  \small
  \begin{multicols}{2}
    \par\textcolor{red}{\textit{\Rbar}.} \textit{Mon âme est triste jusqu’à la mort. Demeurez
    ici, et veillez avec moi ; vous verrez la troupe de gens qui m’environnera. \\ \textcolor{red}{*} Vous prendrez la fuite, et j’irai pour être immolé pour vous.}
    \columnbreak
    \par\textcolor{red}{\textit{\Vbar}.} \textit{Voici l’heure qui s’approche, et le Fils de l’homme sera livré entre les mains des pécheurs.\\
    \textcolor{red}{*} Vous prendrez la fuite, et j’irai pour être immolé pour vous.}
  \end{multicols}
  \normalsize

  \newpage

  \begin{center}
    \large Leçon III.\\
    \normalsize
  \end{center}
  % \large
  \gresetinitiallines{1}
  \greillumination{\initfamily\fontsize{11mm}{11mm}\selectfont J}
  \gregorioscore{lamentations/va--manum_suam_misit_hostis--silos}
  \begin{multicols}{2}
    \begin{small}
      \par \textcolor{red}{\textit{Jod.}} L’ennemi s’est emparé de tout ce qu’elle avait de
      plus désirable ; car elle a vu les nations introduites dans votre Sanctuaire, quoique vous eussiez défendu de les admettre dans votre assemblées.
      \par \textcolor{red}{\textit{Caph.}} Tout son peuple gémissant cherche du pain ; ils ont donné pour vivre, ce qu’ils avaient de plus précieux, pour rétablir un peu leurs forces. Voyez, Seigneur, et considérez combien je suis devenue méprisable.
      \par \textcolor{red}{\textit{Lamed.}} O vous tous, qui passez par le chemin, considérez et voyez s’il y a une douleur semblable à la mienne ; car le Seigneur, selon sa parole, m’a dépouillée au jour de sa colère, comme une vigne vendangée.
      \par \textcolor{red}{\textit{Mem.}} Il a fait tomber d’en-haut un feu dans mes
      os, et m’a châtié. Il a tendu un filet sous mes pieds ; il m’a fait tomber en arrière ; il m’a plongée dans une tristesse qui durera tout le jour.
      \par \textcolor{red}{\textit{Nun.}} Le joug de mes iniquités m’a accablé sans relâche ; ses mains en ont fait une chaîne qui a été attachée à mon cou. Ma force s’est affaiblie ; le
      Seigneur m’a livré en des mains dont je ne pourrai jamais me relever.\\
      Jérusalem, Jérusalem, convertis-toi au Seigneur ton Dieu.
      % \par \hspace{\fill}
    \end{small}
  \end{multicols}

  \medskip

  \greillumination{\initfamily\fontsize{11mm}{11mm}\selectfont E}
  \gregorioscore{repons/re--ecce_vidimus_--solesmes_1961}\

  % \newpage
  \small
  \begin{multicols}{2}
    \par\textcolor{red}{\textit{\Rbar}.} \textit{Voici que nous l’avons vu qui n’avait plus aucune beauté ; il n’était pas reconnaissable. C’est lui qui a porté nos péchés, et il est puni pour nous. A son égard, il a été percé de plaies à cause de nos iniquités. \textcolor{red}{*} Et nous avons été guéris par ses meurtrissures.}
    \par\textcolor{red}{\textit{\Vbar}.} \textit{Il a véritablement porté nos langueurs, et il a
    ressenti nos douleurs.
    \textcolor{red}{*} Et nous avons été guéris par ses meurtrissures.}
    \columnbreak
    \par\textcolor{red}{\textit{\Rbar}.} \textit{Voici que nous l’avons vu qui n’avait plus aucune beauté ; il n’était pas reconnaissable. C’est lui qui a porté nos péchés, et il est puni pour
    nous. A son égard, il a été percé de plaies à cause de nos iniquités.\\ \textcolor{red}{*} Et nous avons été guéris par ses meurtrissures.}
    \par\hspace{\fill}
  \end{multicols}
  \normalsize

  \medskip
  \begin{center}
    \rule{4cm}{0.4pt}
  \end{center}
  \medskip

  \begin{center}
    \large AU DEUXIÈME NOCTURNE.\\
  \end{center}
  \medskip
  \par Le chant royal de l'avènement que nous avons chanté à Noël ouvre le second Nocturne de cette nuit de douleur. Ce n'est pas sans raisons que l'Église fait du même Psaume un emploi si différent. Car si nous avons salué dans l'Enfant de Bethléem notre Roi et notre Libérateur, c'est sur la Croix qu'il règne véritablement ; c'est devant cette Croix que viendront s'humilier tous les rois de la terre, parce que sur elle Jésus a sauvé les pauvres de son peuple et brisé celui qui les opprimait.

  \medskip

  % ===== DEBUT Antienne =========
  \gresetinitiallines{1}
  \greillumination{\initfamily\fontsize{11mm}{11mm}\selectfont L}
  \gregorioscore{antiennes/an--liberavit_dominus--solesmes_1961}
  \begin{center}
    \footnotesize{
      \textit{
        Le Seigneur a délivré le pauvre de la main du puissant, et soutenu l’indigent qui n’avait point de protecteur.
    }
  }
  \end{center}
  % ===== FIN Antienne ===========

  % ===== DEBUT psaume ===========
  % gresetinitiallines : avec le parametre à 0, supprime l'ornement
  \begin{center}
    \large{Psaume 71.}\\
  \end{center}

  \gresetinitiallines{0}
  \gregorioscore{psaumes/psaume71-VIIc}

  \begin{enumerate}[label=\textcolor{red}{\arabic*}]
    \setcounter{enumi}{1}
    \item Judicáre pópulum tuum \textbf{in} jus\textbf{tí}tia,\textcolor{red}{~*} et páuperes tuos \textbf{in} ju\textbf{dí}cio.

    \item Suscípiant montes \textbf{pa}cem \textbf{pó}pulo:\textcolor{red}{~*} et \textbf{col}les jus\textbf{tí}tiam.

    \item Judicábit páuperes pópuli, et salvos fáciet \textbf{fí}lios \textbf{páu}perum:\textcolor{red}{~*} et humiliábit calum\textbf{ni}a\textbf{tó}rem.

    \item Et permanébit cum sole, et \textbf{an}te \textbf{lu}nam,\textcolor{red}{~*} in generatióne et gene\textbf{ra}ti\textbf{ó}nem.

    \item Descéndet sicut plúvi\textbf{a} in \textbf{vel}lus:\textcolor{red}{~*} et sicut stillicídia stillántia \textbf{su}per \textbf{ter}ram.

    \item Oriétur in diébus ejus justítia, et abun\textbf{dán}tia \textbf{pa}cis:\textcolor{red}{~*} donec aufe\textbf{rá}tur \textbf{lu}na.

    \item Et dominábitur a mari \textbf{us}que ad \textbf{ma}re:\textcolor{red}{~*} et a flúmine usque ad términos \textbf{or}bis ter\textbf{rá}rum.

    \item Coram illo próci\textbf{dent} Æ\textbf{thí}opes:\textcolor{red}{~*} et inimíci ejus \textbf{ter}ram \textbf{lin}gent.

    \item Reges Tharsis, et ínsulæ \textbf{mú}nera \textbf{óf}ferent:\textcolor{red}{~*} reges Arabum et Saba \textbf{do}na ad\textbf{dú}cent.

    \item Et adorábunt eum omnes \textbf{re}ges \textbf{ter}ræ:\textcolor{red}{~*} omnes Gentes \textbf{sér}vient \textbf{e}i:

    \item Quia liberábit páuperem \textbf{a} pot\textbf{én}te:\textcolor{red}{~*} et páuperem, cui non \textbf{e}rat ad\textbf{jú}tor.

    \item Parcet páupe\textbf{ri} et \textbf{ín}opi:\textcolor{red}{~*} et ánimas páuperum \textbf{sal}vas \textbf{fá}ciet.

    \item Ex usúris et iniquitáte rédimet áni\textbf{mas} e\textbf{ó}rum:\textcolor{red}{~*} et honorábile nomen eórum \textbf{co}ram \textbf{il}lo.

    \item Et vivet, et dábitur ei de auro Arábiæ,\textcolor{red}{~†} et adorábunt de \textbf{ip}so \textbf{sem}per:\textcolor{red}{~*} tota die bene\textbf{dí}cent \textbf{e}i.

    \item Et erit firmaméntum in terra in summis móntium,\textcolor{red}{~†} superextollétur super Líbanum \textbf{fruc}tus \textbf{e}jus:\textcolor{red}{~*}  et florébunt de civitáte sicut \textbf{fe}num \textbf{ter}ræ.

    \item Sit nomen ejus bene\textbf{díc}tum in \textbf{sǽ}cula:\textcolor{red}{~*} ante solem pérmanet \textbf{no}men \textbf{e}jus.

    \item Et benedicéntur in ipso omnes \textbf{tri}bus \textbf{ter}ræ:\textcolor{red}{~*} omnes Gentes magnifi\textbf{cá}bunt \textbf{e}um.

    \item Benedíctus Dóminus, \textbf{De}us \textbf{Is}raël,\textcolor{red}{~*} qui facit mira\textbf{bí}lia \textbf{so}lus.

    \item Et benedíctum nomen majestátis ejus \textbf{in} æ\textbf{tér}num:\textcolor{red}{~*} et replébitur majestáte ejus omnis terra: \textbf{fi}at, \textbf{fi}at.
  \end{enumerate}

  \grecommentary{\textit{Reprise de l'Antienne.}}
  \gabcsnippet{(c3) Li(i)be(g)rá(iv_[oh:h]//jki)vit(i') Dó(i)mi(hg)nus(e.) (,) páu(f!gw!hi~)pe(h)rem(hghf~) a(g) pot(f)én(e.)te,(e.) (;) et(e) ín(e)o(d_c)pem,(d'_) (,) cu(b)i(c') non(d) e(e_[oh:h]d)rat(ef) ad(f)jú(e.)tor.(e.) (::)}

  \begin{multicols}{2}
    \begin{footnotesize}
      \begin{enumerate}[label=\textcolor{red}{\emph{\arabic*}}]
        \item \textit{O Dieu, donnez au Roi votre jugement, et votre justice au fils du Roi,}
        \item \textit{Afin qu’il juge votre peuple selon la justice, et vos
        pauvres selon l’équité de ses jugements.}
        \item \textit{Que les montagnes reçoivent la paix pour le
        peuple, et les collines la justice.}
        \item \textit{Il jugera les pauvres du peuple ; il sauvera les enfants des pauvres, et il humiliera le calomniateur.}
        \item \textit{Il subsistera autant que le soleil et la lune, dans l’étendue de toutes les générations}
        \item \textit{Il descendra comme la pluie sur une toison, et comme l’eau qui tombe goutte à goutte sur la terre.}
        \item \textit{La justice paraîtra de son temps, avec une abondance de paix, qui durera autant que la lune.}
        \item \textit{Il règnera depuis une mer jusqu’à l’autre, et depuis le fleuve jusqu’aux extrémités de la terre.}
        \item \textit{Les Ethiopiens se prosterneront devant lui, et ses ennemis baiseront la terre.}
        \item \textit{Les Rois de Tharse et les îles lui offriront des présents : Les Rois de l’Arabie et de Saba lui apporteront des dons.}
        \item \textit{Et tous les Rois de la terre l’adoreront ; toutes les nations lui seront assujetties}
        \item \textit{Parce qu’il délivrera le pauvre de la main du puissant, et l’indigent qui n’avait point de protecteur.}
        \item \textit{Il épargnera le pauvre et l’indigent ; et il sauvera les âmes des pauvres.}
        \item \textit{Et il délivrera leurs âmes des usures et de l’iniquité ; et leur nom sera honorable devant lui.}
        \item \textit{Et il vivra, et on lui donnera de l’or de l’Arabie ; ils l’adoreront sans cesse, et ils le béniront durant tout le jour.}
        \item \textit{Et l’on verra le froment semé dans la terre sur le sommet des montagnes : son fruit s’élèvera audessus des cèdres du Liban ; les habitants des villes multiplieront comme les gerbes de la terre.}
        \item \textit{Que son nom soit béni dans tous les siècles : son nom subsiste avant le soleil.}
        \item \textit{Et tous les peuples de la terre seront bénis en lui ; toutes les nations le glorifieront.}
        \item \textit{Que béni soit le Seigneur, le Dieu d’Israël, qui fait seul des œuvres merveilleuses.}
        \item \textit{Et que le nom de sa Majesté soit béni éternellement ; et toute la terre sera remplie de sa Majesté ; ainsi soit fait, ainsi soit fait.}
      \end{enumerate}
    \end{footnotesize}
  \end{multicols}

  \medskip

  \par Le Psalmiste justifie la Providence sur le sort temporel des justes et des impies. C'est là un des côtés moraux du grand mystère de la Passion. Lorsqu'il nous semble que nous souffrons sans l'avoir mérité, nous n'avons qu'à considérer la grande Victime, dont les douleurs furent sans égales, et qui cependant était l'innocence et la sainteté même.

  \medskip

  % ===== DEBUT Antienne =========
  \gresetinitiallines{1}
  \greillumination{\initfamily\fontsize{11mm}{11mm}\selectfont C}
  \gregorioscore{antiennes/an--cogitaverunt_impii--solesmes_1961}
  \begin{center}
    \footnotesize{
      \textit{
        Le Seigneur a délivré le pauvre de la main du puissant, et soutenu l’indigent qui n’avait point de protecteur.
    }
  }
  \end{center}
  % ===== FIN Antienne ===========

  % ===== DEBUT psaume ===========
  % gresetinitiallines : avec le parametre à 0, supprime l'ornement
  \begin{center}
    \large{Psaume 72.}\\
  \end{center}

  \gresetinitiallines{0}
  \gregorioscore{psaumes/psaume72-VIIIc}
  \begin{enumerate}[label=\textcolor{red}{\arabic*}]
    \setcounter{enumi}{1}
    \item Mei autem pene moti sunt \textbf{pe}des:\textcolor{red}{~*} pene effúsi sunt \textit{gres}\textit{sus} \textbf{me}i.

    \item Quia zelávi super in\textbf{í}quos,\textcolor{red}{~*} pacem pecca\textit{tó}\textit{rum} \textbf{vi}dens.

    \item Quia non est respéctus morti e\textbf{ó}rum:\textcolor{red}{~*} et firmaméntum in pla\textit{ga} \textit{e}\textbf{ó}rum.

    \item In labóre hóminum \textbf{non} sunt:\textcolor{red}{~*} et cum homínibus non fla\textit{gel}\textit{la}\textbf{bún}tur:

    \item Ideo ténuit eos su\textbf{pér}bia,\textcolor{red}{~*} opérti sunt iniquitáte et impie\textit{tá}\textit{te} \textbf{su}a.

    \item Pródiit quasi ex ádipe iníquitas e\textbf{ó}rum:\textcolor{red}{~*} transiérunt in af\textit{féc}\textit{tum} \textbf{cor}dis.

    \item Cogitavérunt, et locúti sunt ne\textbf{quí}tiam:\textcolor{red}{~*} iniquitátem in excél\textit{so} \textit{lo}\textbf{cú}ti sunt.

    \item Posuérunt in cælum os \textbf{su}um:\textcolor{red}{~*} et lingua eórum transí\textit{vit} \textit{in} \textbf{ter}ra.

    \item Ideo convertétur pópulus \textbf{me}us hic:\textcolor{red}{~*} et dies pleni invenién\textit{tur} \textit{in} \textbf{e}is.

    \item Et dixérunt: Quómodo scit \textbf{De}us,\textcolor{red}{~*} et si est sciéntia \textit{in} \textit{ex}\textbf{cél}so?

    \item Ecce ipsi peccatóres, et abundántes in \textbf{sǽ}culo,\textcolor{red}{~*} obtinué\textit{runt} \textit{di}\textbf{ví}tias.

    \item Et dixi: Ergo sine causa justificávi cor \textbf{me}um,\textcolor{red}{~*} et lavi inter innocéntes \textit{ma}\textit{nus} \textbf{me}as.

    \item Et fui flagellátus tota \textbf{di}e,\textcolor{red}{~*} et castigátio mea in \textit{ma}\textit{tu}\textbf{tí}nis.

    \item Si dicébam: Nar\textbf{rá}bo sic:\textcolor{red}{~*} ecce natiónem filiórum tuórum \textit{re}\textit{pro}\textbf{bá}vi.

    \item Existimábam ut cognósce\textbf{rem} hoc,\textcolor{red}{~*} la\textit{bor} \textit{est} \textbf{an}te me.

    \item Donec intrem in Sanctuárium \textbf{De}i:\textcolor{red}{~*} et intélligam in novíssi\textit{mis} \textit{e}\textbf{ó}rum.

    \item Verúmtamen propter dolos posuísti \textbf{e}is:\textcolor{red}{~*} dejecísti eos dum al\textit{le}\textit{va}\textbf{rén}tur.

    \item Quómodo facti sunt in desolatiónem, súbito defe\textbf{cé}runt:\textcolor{red}{~*} periérunt propter iniqui\textit{tá}\textit{tem} \textbf{su}am.

    \item Velut sómnium surgéntium, \textbf{Dó}mine,\textcolor{red}{~*} in civitáte tua imáginem ipsórum ad ní\textit{hi}\textit{lum} \textbf{réd}iges.

    \item Quia inflammátum est cor meum, et renes mei commu\textbf{tá}ti sunt:\textcolor{red}{~*} et ego ad níhilum redáctus sum, \textit{et} \textit{ne}\textbf{scí}vi.

    \item Ut juméntum factus sum \textbf{a}pud te:\textcolor{red}{~*} et ego \textit{sem}\textit{per} \textbf{te}cum.

    \item Tenuísti manum déxteram meam:\textcolor{red}{~†} et in voluntáte tua dedu\textbf{xís}ti me,\textcolor{red}{~*} et cum glória \textit{su}\textit{sce}\textbf{pís}ti me.

    \item Quid enim mihi est in \textbf{cæ}lo?\textcolor{red}{~*} et a te quid vólui \textit{su}\textit{per} \textbf{ter}ram?

    \item Defécit caro mea, et cor \textbf{me}um:\textcolor{red}{~*} Deus cordis mei, et pars mea Deus \textit{in} \textit{æ}\textbf{tér}num.

    \item Quia ecce, qui elóngant se a te, per\textbf{í}bunt:\textcolor{red}{~*} perdidísti omnes, qui forni\textit{cán}\textit{tur} \textbf{abs} te.

    \item Mihi autem adhærére Deo \textbf{bo}num est:\textcolor{red}{~*} pónere in Dómino De\textit{o} \textit{spem} \textbf{me}am:

    \item Ut annúntiem omnes prædicatiónes \textbf{tu}as,\textcolor{red}{~*} in portis fí\textit{li}\textit{æ} \textbf{Si}on.
  \end{enumerate}

  \grecommentary{\textit{Reprise de l'Antienne.}}
  \gabcsnippet{(c4) Co(j)gi(i!kjk)ta(hf)vé(g)runt(h') ím(g)pi(fg)i,(g.) (,) et(g) lo(g)cú(h')ti(g) sunt(h) ne(ji)quí(h)ti(gh)am :(h.) (;) in(g)i(g')qui(g)tá(fe)tem(d'_) (,) in(e) ex(f)cél(gf~)so(h') lo(i)cú(h')ti(g) sunt.(g.) (::)}

  \begin{multicols}{2}
    \begin{footnotesize}
      \begin{enumerate}[label=\textcolor{red}{\emph{\arabic*}}]
        \item \textit{Que Dieu est bon à Israël, et à ceux qui ont le cœur droit !}
        \item \textit{Pour moi mes pieds ont presque chancelé ; j’ai été prêt de tomber en marchant.}
        \item \textit{Parce que mon zèle m’a animé contre les méchants, en voyant la paix des pécheurs.}
        \item \textit{Parce qu’on n’est point touché de leur mort, et que leurs plaies ne sont pas de longue durée}
        \item \textit{Ils ne participent point aux peines des hommes, et ils ne sont point exposés aux mêmes fléaux que les autres hommes.}
        \item \textit{C’est pourquoi l’orgueil les possède ; ils sont couverts d’iniquité et de leur impiété.}
        \item \textit{Leur iniquité est comme sortie de leur abondance ; ils se sont livrés aux désirs de leur cœur.}
        \item \textit{Leurs pensées et leurs paroles sont remplies de malice ; ils ont fait éclater leur iniquité.}
        \item \textit{Ils ont parlé contre le ciel, et leur langue a ensuite attaqué la terre.}
        \item \textit{C’est pourquoi mon peuple fera attention à cela, en voyant leurs jours pleins et heureux.}
        \item \textit{Et ils ont dit : Comment Dieu sait-il cela ? le Très-Haut a-t-il connaissance de tout ?}
        \item \textit{Voilà que les pécheurs abondent des biens du monde ; ils ont amassé de grandes richesses}
        \item \textit{Et j’ai dit : C’est donc inutilement que j’ai purifié mon cœur, et que j’ai lavé mes mains parmi les innocents :}
        \item \textit{Car j’ai été châtié durant tout le jour ; ma punition a commencé dès le matin.}
        \item \textit{Si je disais, je parlerai ainsi : je condamnerais par-là toute l’assemblée de vos enfants.}
        \item \textit{Je croyais pouvoir connaître cela ; mais j’ai trouvé une grande difficulté}
        \item \textit{Jusqu’à ce que j’entre dans le sanctuaire de Dieu, et que j’aie la connaissance de leur fin.}
        \item \textit{Cependant c’est un piège que vous leur avez dressé ; vous les avez renversés, lorsqu’ils s’élevaient.}
        \item \textit{Comment sont-ils tombés dans la dernière désolation ? ils ont manqué tout-à-coup ; ils ont péri à cause de leur iniquité.}
        \item \textit{Seigneur, ils sont comme le songe de ceux qui s’éveillent : vous réduirez au néant dans votre cité l’image de leur bonheur.}
        \item \textit{Parce que mon cœur est enflammé, et que mes reins sont altérés ; j’ai été comme réduit au néant sans le savoir.}
        \item \textit{Je suis devenu comme une bête devant vous, et cependant toujours avec vous.}
        \item \textit{Vous m’avez tenu par la main droite ; vous m’avez conduit selon votre volonté, et vous m’avez comblé de gloire.}
        \item \textit{Car qu’y a-t-il pour moi dans le ciel ? ou que pourrais-je désirer sur la terre, sinon vous ?}
        \item \textit{Ma chair et mon cœur sont tombés en défaillance ; vous êtes le Dieu de mon cœur : mon Dieu, vous êtes mon partage dans l’éternité ;}
        \item \textit{Car ceux qui s’éloignent de vous périront ; vous avez perdu tous ceux qui vous abandonnent pour se prostituer.}
        \item \textit{Mais pour moi, c’est mon bien de m’attacher à Dieu et de mettre mon espérance dans le Seigneur mon Dieu,}
        \item \textit{Afin que j’annonce vos merveilles aux portes de la fille de Sion.}
      \end{enumerate}
    \end{footnotesize}
  \end{multicols}

  \medskip

  \par Le Psaume 73 renferme une plainte émouvante sur les mauvais traitements infligés à l'Homme-Dieu \textit{(au milieu même de la solennité pascale)} par le peuple insensé qui a méconnu le grand Prophète si longtemps promis. Pour lui, c'est ainsi qu'il opérait le salut au milieu de la terre. Mais quelques années plus tard, dans ces mêmes fêtes de Paques, les aigles romaines planeront sur les ruines du temple incendié ; et la nation coupable s'en ira errante, sans prophète, sans sacerdoce, sans culte légal, cherchant péniblement son bonheur dans la jouissance des biens terrestres qui continuent de l'aveugler.

  \medskip

  % ===== DEBUT Antienne =========
  \gresetinitiallines{1}
  \greillumination{\initfamily\fontsize{11mm}{11mm}\selectfont E}
  \gregorioscore{antiennes/an--exsurge_domine_et_judica--solesmes_1961}
  \begin{center}
    \footnotesize{
      \textit{Levez-vous, Seigneur, et jugez ma cause.}
  }
  \end{center}
  % ===== FIN Antienne ===========

  % ===== DEBUT psaume ===========
  % gresetinitiallines : avec le parametre à 0, supprime l'ornement
  \begin{center}
    \large{Psaume 73.}\\
  \end{center}

  \gresetinitiallines{0}
  \gregorioscore{psaumes/psaume73-Ig}
  \begin{enumerate}[label=\textcolor{red}{\arabic*}]
    \setcounter{enumi}{1}
    \item Memor esto congregati\textbf{ó}nis \textbf{tu}æ:\textcolor{red}{~*} quam possedísti \textit{ab} \textit{in}\textbf{í}tio.

    \item Redemísti virgam heredi\textbf{tá}tis \textbf{tu}æ:\textcolor{red}{~*} mons Sion, in quo habitás\textit{ti} \textit{in} \textbf{e}o.

    \item Leva manus tuas in supérbias e\textbf{ó}rum in \textbf{fi}nem:\textcolor{red}{~*} quanta malignátus est inimí\textit{cus} \textit{in} \textbf{sanc}to!

    \item Et gloriáti sunt \textbf{qui} o\textbf{dé}runt te:\textcolor{red}{~*} in médio solemni\textit{tá}\textit{tis} \textbf{tu}æ.

    \item Posuérunt signa \textbf{su}a, \textbf{si}gna:\textcolor{red}{~*} et non cognovérunt sicut in éxitu \textit{su}\textit{per} \textbf{sum}mum.

    \item Quasi in silva lignórum secúribus excidérunt jánuas ejus \textbf{in} id\textbf{íp}sum:\textcolor{red}{~*} in secúri et áscia deje\textit{cé}\textit{runt} \textbf{e}am.

    \item Incendérunt igni sanctu\textbf{á}rium \textbf{tu}um:\textcolor{red}{~*} in terra polluérunt tabernáculum nó\textit{mi}\textit{nis} \textbf{tu}i.

    \item Dixérunt in corde suo cognátio e\textbf{ó}rum \textbf{si}mul:\textcolor{red}{~*} Quiéscere faciámus omnes dies festos De\textit{i} \textit{a} \textbf{ter}ra.

    \item Signa nostra non vídimus, jam non \textbf{est} pro\textbf{phé}ta:\textcolor{red}{~*} et nos non co\textit{gnó}\textit{scet} \textbf{ám}plius.

    \item Usquequo, Deus, improperábit \textbf{in}i\textbf{mí}cus:\textcolor{red}{~*} irrítat adversárius nomen tu\textit{um} \textit{in} \textbf{fi}nem?

    \item Ut quid avértis manum tuam, et \textbf{déx}teram \textbf{tu}am,\textcolor{red}{~*} de médio sinu tu\textit{o} \textit{in} \textbf{fi}nem?

    \item Deus autem Rex noster \textbf{an}te \textbf{sǽ}cula:\textcolor{red}{~*} operátus est salútem in mé\textit{di}\textit{o} \textbf{ter}ræ.

    \item Tu confirmásti in virtúte \textbf{tu}a \textbf{ma}re:\textcolor{red}{~*} contribulásti cápita dracó\textit{num} \textit{in} \textbf{a}quis.

    \item Tu confregísti cápi\textbf{ta} dra\textbf{có}nis:\textcolor{red}{~*} dedísti eum escam pópu\textit{lis} \textit{Æ}\textbf{thí}opum.

    \item Tu dirupísti fontes, \textbf{et} tor\textbf{rén}tes\textcolor{red}{~*} tu siccásti flú\textit{vi}\textit{os} \textbf{E}than.

    \item Tuus est dies, et \textbf{tu}a \textbf{est} nox:\textcolor{red}{~*} tu fabricátus es auró\textit{ram} \textit{et} \textbf{so}lem.

    \item Tu fecísti omnes \textbf{tér}minos \textbf{ter}ræ:\textcolor{red}{~*} æstátem et ver tu plas\textit{más}\textit{ti} \textbf{e}a.

    \item Memor esto hujus, inimícus imprope\textbf{rá}vit \textbf{Dó}mino:\textcolor{red}{~*} et pópulus insípiens incitávit \textit{no}\textit{men} \textbf{tu}um.

    \item Ne tradas béstiis ánimas confi\textbf{tén}tes \textbf{ti}bi,\textcolor{red}{~*} et ánimas páuperum tuórum ne obliviscá\textit{ris} \textit{in} \textbf{fi}nem.

    \item Réspice in testa\textbf{mén}tum \textbf{tu}um:\textcolor{red}{~*} quia repléti sunt, qui obscuráti sunt terræ dómibus in\textit{i}\textit{qui}\textbf{tá}tum.

    \item Ne avertátur húmilis \textbf{fac}tus con\textbf{fú}sus:\textcolor{red}{~*} pauper et inops laudábunt \textit{no}\textit{men} \textbf{tu}um.

    \item Exsúrge, Deus, júdica \textbf{cau}sam \textbf{tu}am:\textcolor{red}{~*} memor esto improperiórum tuórum, eórum quæ ab insipiénte sunt \textit{to}\textit{ta} \textbf{di}e.

    \item Ne obliviscáris voces inimi\textbf{có}rum tu\textbf{ó}rum:\textcolor{red}{~*} supérbia eórum, qui te odérunt, a\textit{scén}\textit{dit} \textbf{sem}per.
  \end{enumerate}

  \grecommentary{\textit{Reprise de l'Antienne.}}
  \gabcsnippet{(c4) Ex(f)súr(h)ge,(g') Dó(h)mi(fg)ne,(g.) (,) et(d_f) jú(f)di(f)ca(e_[uh:l]g) cau(f)sam(e) me(d.)am.(d.) (::)}

  \begin{multicols}{2}
    \begin{footnotesize}
      \begin{enumerate}[label=\textcolor{red}{\emph{\arabic*}}]
        \item \textit{O Dieu, pourquoi nous rejetez-vous pour toujours ? votre fureur est allumée contre les brebis de votre troupeau.}
        \item \textit{Souvenez-vous de ce peuple que vous avez rassemblé, et qui vous appartient dès le commencement.}
        \item \textit{Vous avez vous-même racheté votre héritage, et le mont de Sion, dans lequel vous avez habité.}
        \item \textit{Levez vos mains, afin d’abattre leur orgueil pour toujours. Combien l’ennemi a-t-il commis d’impiétés dans le Sanctuaire ?}
        \item \textit{Ceux qui vous haïssent, ont fait gloire de profaner vos fêtes les plus solennelles.}
        \item \textit{Ils ont sans connaissance arboré leurs étendards comme des trophées au haut du Temple, comme aux portes.}
        \item \textit{Ils ont brisé les portes à coups de hache, comme l’on coupe le bois dans une forêt ; ils les ont abattues avec la hache et la cognée.}
        \item \textit{Ils ont mis le feu à votre saint Temple ; ils ont profané sur la terre le tabernacle consacré à votre nom.}
        \item \textit{Ils ont conspiré tous ensemble, et dit au fond du cœur : Faisons cesser par toute la terre les jours des fêtes de Dieu.}
        \item \textit{Nous ne voyons plus de signes ; il n’y a plus de Prophète, et nul ne nous connaîtra plus à l’avenir.}
        \item \textit{O Dieu, jusqu’à quand l’ennemi vous insultera-til ? Vos adversaires profaneront-ils votre nom jusqu’à la fin ?}
        \item \textit{Pourquoi détournez-vous votre main ; et pourquoi tenez-vous toujours votre main droite dans votre sein ?}
        \item \textit{Cependant Dieu est notre Roi avant tous les siècles ; il a opéré notre salut au milieu de la terre.}
        \item \textit{Vous avez affermi la mer par votre puissance ; vous avez brisé sous les eaux les têtes des dragons}
        \item \textit{Vous avez brisé la tête du dragon ; vous l’avez donné en proie au peuple d’Ethiopie.}
        \item \textit{Vous avez fait sortir d’un rocher des fontaines et des torrents : vous avez séché les grands fleuves.}
        \item \textit{Le jour et la nuit sont votre ouvrage : vous avez fait l’aurore et le soleil.}
        \item \textit{Vous avez donné des bornes à la terre : vous avez créé l’été comme le printemps.}
        \item \textit{Souvenez-vous de ceci, que votre ennemi a insulté le Seigneur par ses reproches ; et qu’un peuple insensé a offensé votre nom.}
        \item \textit{Ne livrez pas à des bêtes les âmes de ceux qui vous louent, et n’oubliez pas pour toujours les âmes de vos pauvres serviteurs.}
        \item \textit{Considérez votre alliance, parce que les personnes les plus obscures de la terre se sont emparées injustement de nos maisons.}
        \item \textit{Que l’humble ne soit pas renvoyé avec confusion : le pauvre et l’indigent loueront votre nom.}
        \item \textit{Levez-vous, mon Dieu, jugez ma cause, souvenez-vous des insultes que l’on vous a faites ; de celles qu’un peuple insensé vous fait tout le jour.}
        \item \textit{N’oubliez pas les paroles de vos ennemis ; l’orgueil de ceux qui vous haïssent augmente toujours.}
      \end{enumerate}
    \end{footnotesize}
  \end{multicols}

  \medskip

  \begin{center}
    \begin{footnotesize}
      \textcolor{red}{\textit{On chante le verset debout.}}
    \end{footnotesize}
    \begin{minipage}{0.8\linewidth}
      \gresetinitiallines{0}
      % \grecommentary[10pt]{\textcolor{red}{\textit{On se lève.}}}
      \large
      \gabcsnippet{(c4)<c><v>\Vbar</v>.</c> Dé(h)us(h) mé(h)us,(h) ér(h)ipe(h) me(h) de(h) má(h)nu(h) pe(i')cca(h)tó(g.)ris.(g.) (::) (Z)<c><v>\Rbar</v>.</c> Et(h) de(h) má(h)nu(h) con(h)tra(h) lé(h)gem(h) ag(h)én(h)tis(h) et(i') in(h)í(g.)qui.(g.) (::) (Z)}
      \bigskip
      \normalsize
      \begin{center}
        \textit{\textcolor{red}{\Vbar.} Mon Dieu, délivrez-moi des mains du pécheur.}\\
        \textit{\textcolor{red}{\Rbar.} Et des mains de celui qui agit contre la loi, et du méchant.}
      \end{center}
    \end{minipage}
  \end{center}

  \newpage


  \par Les Leçons du second Nocturne sont tirées des célèbres Narrations de Saint Augustin sur les Psaumes. L'Église en détache aujourd'hui un passage sur le Psaume 54, qui convient admirablement à la situation du Messie cherchant comme David un refuge sur le Mont des Oliviers, et comme lui ayant à endurer de l'un des siens la plus odieuse des trahisons. Le Grand Docteur nous explique la nature de lapeine qui accable le Sauveur ; il expose avec sa largeur de vue ordinaire la raison d'être des méchants sur la terre : il termine en montrant le triomphe qu'à procuré à la Croix toute la malice des juifs, sans que ceux-ci soient entièrement exclus des effets salutaires du Bois sacré.

  \medskip

  % \newpage

  \begin{center}
    \large Leçon IV.\\
    \normalsize
  \end{center}
  \medskip

  \setlength{\columnsep}{2pc}
  \def\columnseprulecolor{\color{red}}
  \setlength{\columnseprule}{0.4pt}

  \begin{multicols}{2}
    \begin{center}
      Ex Tractátu sancti Augustíni\\ Epíscopi super Psalmos.
    \end{center}

    \par Exáudi, Deus, oratiónem meam, et ne
    despéxeris deprecatiónem meam : inténde mihi, et exáudi me. Satagéntis,
    sollíciti, in tribulatióne pósiti, verba
    sunt ista. Orat multa pátiens, de malo
    liberári desíderans.
    \par Súperest ut videámus, in quo malo sit : et cum
    dícere cœperit : agnoscámus ibi nos
    esse : ut, communicáta tribulatióne,
    conjungámus oratiónem. Contristátus
    sum, inquit, in exercitatióne mea, et
    conturbátus sum. Ubi contristátus ?
    ubi conturbátus ?
    \par In exercitatióne mea, inquit. Hómines malos, quos pátitur,
    commemorátus est : eamdémque passiónem malórum hóminum exercitatiónem suam dixit. Ne putétis gratis esse malos in hoc mundo, et nihil boni de illis ágere Deum. Omnis malus aut
    ídeo vivit, ut corrigátur : aut ídeo vivit, ut per illum bonus exerceátur.
    \par \hspace{\fill}
    \columnbreak
    \begin{center}
      Du Traité de S. Augustin,\\ Evêque, sur les Psaumes.\\
      \begin{footnotesize}
        \textit{Sur le Psaume 54, verset I.}
      \end{footnotesize}
    \end{center}
    \par \textit{Exaucez ma prière, ô mon Dieu, et ne méprisez
    pas ma demande ; écoutez-moi, et exaucez-moi :
    ces paroles sont d’un homme plein de souci, vigilant, et plongé dans l’affliction. Etant accablé de plusieurs maux, il prie, souhaitant d’être délivré du mal.}
    \par \textit{Il nous reste de savoir l’espèce de son mal ; et
    quand il aura commencé de l’expliquer, nous reconnaîtrons que nous sommes dans le même état ;
    afin que nos peines communes nous engagent à
    unir nos prières. J’ai été attristé, dit-il, dans mes
    exercices, et j’ai été troublé. Où est-ce qu’il a été
    attristé ? Où est-ce qu’il a été troublé ?}
    \par \textit{C’est, dit-il, dans mes exercices. Il a désigné les
    méchants qui le font souffrir ; il donne le nom
    d’exercice à cette persécution des hommes méchants.
    Ne croyez pas que les méchants soient inutiles en
    ce monde, et que Dieu n’en retire aucun bien ; car
    tout méchant vit, ou pour se corriger, ou pour exercer la patience et la vertu des bons.}
  \end{multicols}
  \setlength\columnseprule{0pt}

  \newpage
  \vspace*{\fill}
  \gresetinitiallines{1}
  \greillumination{\initfamily\fontsize{11mm}{11mm}\selectfont A}
  \gregorioscore{repons/re--amicus_meus--solesmes_1961}
  
  \small
  \begin{multicols}{2}
    \par\textcolor{red}{\textit{\Rbar}.} \textit{Mon ami m’a trahi par le signal d’un baiser :
    Celui que je baiserai, c’est lui-même, arrêtez-le. Celui- qui par un baiser a commis un homicide, a donné un signal bien criminel.  \\ \textcolor{red}{*} Ce malheureux a perdu le prix du Sang, et à la fin s’est pendu.}
    \columnbreak
    \par\textcolor{red}{\textit{\Vbar}.} \textit{Il eût été plus avantageux à cet homme de n’être point né.\\
    \textcolor{red}{*} Ce malheureux a perdu le prix du Sang, et à la fin s’est pendu.}
  \end{multicols}
  \normalsize
  \vspace*{\fill}

  \newpage

  \begin{center}
    \large Leçon V.
    \normalsize
  \end{center}
  \medskip  

  \setlength{\columnseprule}{0.4pt}
  \begin{multicols}{2}
    \par Utinam ergo qui nos modo exércent,
    convertántur, et nobíscum exerceántur : tamen quámdiu ita sunt ut exérceant, non eos odérimus : quia in eo quod malus est quis eórum, utrum usque in finem perseveratúrus sit, ignorámus. Et plerúmque cum tibi vidéris odísse inimícum, fratrem odísti, et nescis.
    \par Diábolus, et ángeli ejus in Scriptúris sanctis manifestáti sunt nobis, quod ad ignem ætérnum sint destináti. Ipsórum tantum desperánda est corréctio, contra quos habémus occúltam luctam : ad quam luctam nos armat Apóstolus, dicens : Non est nobis colluctátio advérsus carnem et sánguinem : id est, non advérsus hómines, quos vidétis, sed advérsus príncipes, et potestátes, et rectóres mundi, tenebrárum harum.
    \par Ne forte cum dixísset, mundi, intellígeres dæmones esse rectóres cæli et terræ,
    mundi dixit, tenebrárum harum : mundi dixit, amatórum mundi : mundi dixit, impiórum et iniquórum : mundi dixit, de quo dicit Evangélium : Et mundus eum non cognóvit.
    \par \par \hspace{\fill}

    \columnbreak

    \par \textit{Plût à Dieu donc, que ceux qui nous exercent maintenant, se convertissent, et qu’ils soient aussi exercés avec nous. Cependant n’ayons point de haine pour eux, tandis qu’ils sont dans la disposition de nous molester ; car nous ne savons pas, si celui qui est méchant parmi eux, continuera de l’être jusqu’à la mort. Et souvent lorsque vous croyez haïr un ennemi, vous haïssez votre frère sans le savoir}
    \par \textit{Il nous déclare dans les saintes Ecritures, que le diable et ses anges sont destinés au feu éternel ; il n’y a que leur amendement qui soit désespéré ; nous avons à lutter sûrement contre eux. L’Apôtre nous ânime à ce combat, en disant : Ce n’est point contre la chair et le sang que nous avons à combrattre ; c’est-à-dire, ce n’est pas contre les hommes que vous voyez, c’est contre les Princes, les Puissances et les Gouverneurs de ce monde, qui règnent dans les ténèbres.}
    \par \textit{De peur qu’ayant dit de ce monde, vous ne vous imaginiez que les démons sont les Gouverneurs du ciel et de la terre ; il a dit, du monde qui règne dans les ténèbres : il a du monde, des amateurs du monde : il a dit du monde, des impies et des méchants. Il a dit du monde, duquel il est dit dans l’Evangile : et le monde ne l’a point connu.}
  \end{multicols}
  \setlength{\columnseprule}{0pt}

  \medskip

  \gresetinitiallines{1}
  \greillumination{\initfamily\fontsize{11mm}{11mm}\selectfont J}
  \gregorioscore{repons/re--judas_mercator--solesmes_1961}
  
  \small
  \begin{multicols}{2}
    \par\textcolor{red}{\textit{\Rbar}.} \textit{Judas, marchant inique, a trahi son Maître
    par un baiser : lui, comme un innocent Agneau, n’a point refusé un baiser à Judas. \\ \textcolor{red}{*} Il a livré le Christ aux Juifs pour un certain nombre de deniers.}
    \columnbreak
    \par\textcolor{red}{\textit{\Vbar}.} \textit{Il eût été plus avantageux à cet homme de n’être point né.\\
    \textcolor{red}{*} Il a livré le Christ aux Juifs pour un certain nombre de deniers.}
  \end{multicols}
  \normalsize

  \bigskip

  \begin{center}
    \large Leçon VI.
    \normalsize
  \end{center}
  \medskip  

  \setlength{\columnseprule}{0.4pt}
  \begin{multicols}{2}
    \par Quóniam vidi iniquitátem, et contradictiónem in civitáte. Atténde glóriam Crucis ipsíus. Jam in fronte regum Crux illa fixa est, cui inimíci insultavérunt.
    \par Efféctus probávit virtútem : dómuit orbem non ferro, sed ligno. Lignum crucis contuméliis dignum visum est inimícis, et ante ipsum lignum stantes caput agitábant, et dicébant : Si
    Fílius Dei est, descéndat de Cruce. 
    \par Extendébat ille manus suas ad pópulum non credéntem et contradicéntem. Si
    enim justus est, qui ex fide vivit ; iníquus est qui non habet fidem. Quod ergo hic ait, iniquitátem : perfídiam intéllige. Vidébat ergo Dóminus in civitáte iniquitátem et contradictiónem, et extendébat manus suas ad pópulum non credéntem, et contradicéntem : et
    tamen et ipsos exspéctans dicébat : Pater, ignósce illis, quia nésciunt quid fáciunt.
    \par \par \hspace{\fill}

    \columnbreak

    \par \textit{D’autant que j’ai vu l’iniquité et la contradiction qui règne dans la cité. Faites attention à la gloire de la Croix : cette Croix est maintenant attachée sur la tête des Rois, quoique les ennemis l’aient insultée.}
    \par \textit{L’effet a fait connaître sa puissance. Il a triomphé du monde, non pas avec le fer, mais avec du bois. Les ennemis ont regardé le bois de la Croix comme un objet d’ignominie ; et se tenant devant ce bois, ils branlaient la tête, et disaient : S’il est le Fils de Dieu, qu’il descende de la Croix.}
    \par \textit{Pour lui, il tendait les bras à un peuple incrédule et contredisant. Car si celui qui vit de la foi est juste, celui qui n’a pas la foi est injuste. Et sous le nom d’iniquité dont il parle ici, vous devez entendre la perfidie et l’infidélité. Le Seigneur voyait donc dans la ville, l’iniquité et la contradiction ; et il tendait les mains à un peuple incrédule et contredisant ; cependant en les attendant, il disait : Mon Père, pardonnez-leur, car ils ne savent ce qu’ils font}
  \end{multicols}
  \setlength{\columnseprule}{0pt}

  \newpage

  \gresetinitiallines{1}
  \greillumination{\initfamily\fontsize{11mm}{11mm}\selectfont U}
  \gregorioscore{repons/re--unus_ex_discipulis--solesmes_1961}
  
  \small
  \begin{multicols}{2}
    \par\textcolor{red}{\textit{\Rbar}.} \textit{L’un de mes Disciples me trahira aujourd’hui.
    Malheur à celui par lequel je serai livré.}
    \par \textcolor{red}{*} \textit{Il lui eut été plus avantageux de n’être point né.}
    \par\textcolor{red}{\textit{\Vbar}.} \textit{ Celui qui met avec moi la main au plat, me livrera entre les mains des pécheurs.}
    \par \hspace{\fill}
  \end{multicols}
  \normalsize

  \medskip
  \begin{center}
    \rule{4cm}{0.4pt}
  \end{center}
  \medskip

  \begin{center}
    \large AU TROISIÈME NOCTURNE.\\
  \end{center}
  \medskip
  \par Le septième Psaume engage les méchants à cesser de pécher, en leur annonçant les jugements terribles qui les menacent. En même temps, il proclame la gloire future des justes, au jour du triomphe du Messie.

  \medskip

  % ===== DEBUT Antienne =========
  \gresetinitiallines{1}
  \greillumination{\initfamily\fontsize{11mm}{11mm}\selectfont D}
  \gregorioscore{antiennes/an--dixi_iniquis--solesmes_1961}
  \begin{center}
    \footnotesize{
      \textit{J’ai dit aux méchants : Ne parlez point avec iniquité contre Dieu.}
  }
  \end{center}
  % ===== FIN Antienne ===========

  % ===== DEBUT psaume ===========
  % gresetinitiallines : avec le parametre à 0, supprime l'ornement
  \begin{center}
    \large{Psaume 74.}\\
  \end{center}

  \gresetinitiallines{0}
  \gregorioscore{psaumes/psaume74-VIIc}

  \begin{enumerate}[label=\textcolor{red}{\arabic*}]
    \setcounter{enumi}{1}
    \item Narrábimus mira\textbf{bí}lia \textbf{tu}a:\textcolor{red}{~*} cum accépero tempus, ego justítias \textbf{ju}di\textbf{cá}bo.

    \item Liquefácta est terra, et omnes qui hábi\textbf{tant} in \textbf{e}a:\textcolor{red}{~*} ego confirmávi co\textbf{lúm}nas \textbf{e}jus.

    \item Dixi iníquis: Nolíte in\textbf{í}que \textbf{á}gere:\textcolor{red}{~*} et delinquéntibus: Nolíte exal\textbf{tá}re \textbf{cor}nu:

    \item Nolíte extóllere in altum \textbf{cor}nu \textbf{ves}trum:\textcolor{red}{~*} nolíte loqui advérsus Deum in\textbf{i}qui\textbf{tá}tem.

    \item Quia neque ab Oriénte, neque ab Occidénte, neque a de\textbf{sér}tis \textbf{món}tibus:\textcolor{red}{~*} quóniam \textbf{De}us \textbf{ju}dex est.

    \item Hunc humíliat, et \textbf{hunc} ex\textbf{ál}tat:\textcolor{red}{~*} quia calix in manu Dómini vini meri \textbf{ple}nus \textbf{mis}to.

    \item Et inclinávit ex hoc in hoc:\textcolor{red}{~†} verúmtamen fæx ejus non est ex\textbf{i}na\textbf{ní}ta:\textcolor{red}{~*} bibent omnes pecca\textbf{tó}res \textbf{ter}ræ.

    \item Ego autem annunti\textbf{á}bo in \textbf{sǽ}culum:\textcolor{red}{~*} cantábo \textbf{De}o \textbf{Ja}cob.

    \item Et ómnia córnua pecca\textbf{tó}rum con\textbf{frín}gam:\textcolor{red}{~*} et exaltabúntur \textbf{cór}nua \textbf{jus}ti.
  \end{enumerate}

  \grecommentary{\textit{Reprise de l'Antienne.}}
  \gabcsnippet{(c3) Di(i)xi(g') in(h)í(iji)quis :(h.) (,)  No(f)lí(h')te(g) lo(e_[uh:l]f)qui(e.) (,) ad(g_[uh:l]h)vér(g)sus(f) De(e_[uh:l]f)um(c_d) in(ed)i(ef)qui(f)tá(e.)tem.(e.) (::)}

  \begin{multicols}{2}
    \begin{footnotesize}
      \begin{enumerate}[label=\textcolor{red}{\emph{\arabic*}}]
        \item \textit{Nous vous louerons, mon Dieu, nous vous louerons ; et nous invoquerons votre nom.}
        \item \textit{Nous raconterons vos merveilles. Quand j’aurai pris mon temps, je jugerai selon la justice.}
        \item \textit{La terre a été détruite avec ses habitants ; c’est moi qui ai affermi ses colonnes.}
        \item \textit{J’ai dit aux méchants : Ne faites point d’injustice, et aux pécheurs : Ne vous élevez point insolemment}
        \item \textit{Ne levez point la tête avec orgueil, et ne parlez point avec iniquité contre Dieu.}
        \item \textit{Car vous ne serez pas secourus, ni du côté d’Orient, ni d’Occident, ni des déserts des montagnes ; parce que Dieu est votre juge.}
        \item \textit{Il humilie l’un, et il élève l’autre ; car il y a toujours dans la main du Seigneur un calice de vin pur, plein d’amertume.}
        \item \textit{Et quoiqu’il en verse tantôt à l’un et tantôt à l’autre, la lie n’en est pas pourtant encore épuisée, tous les pécheurs de la terre en boiront.}
        \item \textit{Pour moi j’annoncerai ses merveilles dans tous les siècles ; je chanterai le Dieu de Jacob.}
        \item \textit{Et je briserai la force des pécheurs ; et la puissance et la gloire du juste sera élevée.} 
      \end{enumerate}
    \end{footnotesize}
  \end{multicols}

  \medskip

  \par Le huitième Psaume loue Dieu au sujet de la justice qu'il a exercé contre les puissants de la terre. Ainsi seront abattus les ennemis du Christ et de son Église, et après la mort ils se trouveront les mains vides devant leur Juge.

  \medskip

  % ===== DEBUT Antienne =========
  \gresetinitiallines{1}
  \greillumination{\initfamily\fontsize{11mm}{11mm}\selectfont T}
  \gregorioscore{antiennes/an--terra_tremuit--solesmes_1961}
  \begin{center}
    \footnotesize{
      \textit{La terre a tremblé, et elle s’est reposée ; tandis que Dieu se préparait à juger}
  }
  \end{center}
  % ===== FIN Antienne ===========
  \newpage
  % ===== DEBUT psaume ===========
  % gresetinitiallines : avec le parametre à 0, supprime l'ornement
  \begin{center}
    \large{Psaume 75.}\\
  \end{center}

  \gresetinitiallines{0}
  \gregorioscore{psaumes/psaume75-VIIIc}

  \begin{enumerate}[label=\textcolor{red}{\arabic*}]
    \setcounter{enumi}{1}
    \item Et factus est in pace locus \textbf{e}jus:\textcolor{red}{~*} et habitátio e\textit{jus} \textit{in} \textbf{Si}on.

    \item Ibi confrégit poténtias \textbf{ár}cuum:\textcolor{red}{~*} scutum, gládi\textit{um}, \textit{et} \textbf{bel}lum.

    \item Illúminans tu mirabíliter a móntibus æ\textbf{tér}nis:\textcolor{red}{~*} turbáti sunt omnes insipi\textit{én}\textit{tes} \textbf{cor}de.

    \item Dormiérunt somnum \textbf{su}um:\textcolor{red}{~*} et nihil invenérunt omnes viri divitiárum in má\textit{ni}\textit{bus} \textbf{su}is.

    \item Ab increpatióne tua, Deus \textbf{Ja}cob,\textcolor{red}{~*} dormitavérunt qui ascen\textit{dé}\textit{runt} \textbf{e}quos.

    \item Tu terríbilis es, et quis resístet \textbf{ti}bi?\textcolor{red}{~*} ex tunc \textit{i}\textit{ra} \textbf{tu}a.

    \item De cælo audítum fecísti ju\textbf{dí}cium:\textcolor{red}{~*} terra trémuit \textit{et} \textit{qui}\textbf{é}vit.

    \item Cum exsúrgeret in judícium \textbf{De}us,\textcolor{red}{~*} ut salvos fáceret omnes mansu\textit{é}\textit{tos} \textbf{ter}ræ.

    \item Quóniam cogitátio hóminis confitébitur \textbf{ti}bi:\textcolor{red}{~*} et relíquiæ cogitatiónis diem festum \textit{a}\textit{gent} \textbf{ti}bi.

    \item Vovéte, et réddite Dómino Deo \textbf{ves}tro:\textcolor{red}{~*} omnes, qui in circúitu ejus af\textit{fér}\textit{tis} \textbf{mú}nera.

    \item Terríbili et ei qui aufert spíritum \textbf{prín}cipum,\textcolor{red}{~*} terríbili apud \textit{re}\textit{ges} \textbf{ter}ræ.
  \end{enumerate}

  \grecommentary{\textit{Reprise de l'Antienne.}}
  \gabcsnippet{(c4) Ter(j')ra(j) tré(j')mu(i)it(g') et(h) qui(j)é(jkj)vit,(jij.) (;) dum(ji~) ex(h)súr(jk~)ge(k)ret(jij.) <nlba>in(h) ju(g)dí(i_[uh:l]j)ci(h)o</nlba>(h) De(g.)us.(g.) (::)}

  \begin{multicols}{2}
    \begin{footnotesize}
      \begin{enumerate}[label=\textcolor{red}{\emph{\arabic*}}]
        \item \textit{Dieu est connu dans la Judée ; son nom est grand dans Israël.}
        \item \textit{Et sa demeure est au milieu de la paix ; son tabernacle est dans Sion.}
        \item \textit{C’est-là qu’il a brisé la force des arcs, les boucliers et les épées, et détruit la guerre.}
        \item \textit{Vous brillez avec éclat sur les montagnes éternelles ; tous les insensés en ont eu le cœur troublé.}
        \item \textit{Ils se sont abandonnés au sommeil, et tous ces hommes qui abondaient en richesses, n’ont rien trouvé dans leurs mains.}
        \item \textit{O Dieu de Jacob, c’est par un effet de votre indignation, que ceux qui étaient montés sur des chevaux se sont endormis.}
        \item \textit{Vous êtes terrible ; et qui pourra vous résister, quand vous serez en colère ?}
        \item \textit{Vous avez fait entendre du ciel le jugement : la terre a tremblé, et elle s’est reposée,}
        \item \textit{Lorsque Dieu s’est levé pour rendre justice, et pour sauver tous ceux qui sont doux et paisibles sur la terre.}
        \item \textit{Parce que la pensée de l’homme vous louera ; et le souvenir de cette pensée lui donnera de la joie, comme celle d’un jour de fête.} 
        \item \textit{Faites des vœux au Seigneur votre Dieu, et accomplissez-les, vous tous qui l’entourez, et lui offrez des présents :}
        \item \textit{A ce Dieu terrible, qui ôte la prudence aux Princes, et qui est la terreur de la terre.}
      \end{enumerate}
    \end{footnotesize}
  \end{multicols}

  \medskip

  \par Au neuvième Psaume, le prophète dans l'affliction, prends confiance dans le souvenir des miséricordes que Dieu a exercées en faveur de son peuple. De même le Christ, dans les souffrances de son agonie, se console à la pensée de la rédemption du genre humain.

  \medskip

  % ===== DEBUT Antienne =========
  \gresetinitiallines{1}
  \greillumination{\initfamily\fontsize{11mm}{11mm}\selectfont I}
  \gregorioscore{antiennes/an--in_die_tribulationis--solesmes_1961}
  \begin{center}
    \footnotesize{
      \textit{Au jour de mon affliction, j’ai cherché Dieu, et j’ai tendu mes mains vers lui.}
  }
  \end{center}
  % ===== FIN Antienne ===========

  % ===== DEBUT psaume ===========
  % gresetinitiallines : avec le parametre à 0, supprime l'ornement
  \begin{center}
    \large{Psaume 76.}\\
  \end{center}

  \gresetinitiallines{0}
  \gregorioscore{psaumes/psaume76-VIIa}

  \begin{enumerate}[label=\textcolor{red}{\arabic*}]
    \setcounter{enumi}{1}
    \item In die tribulatiónis meæ Deum exquisívi,\textcolor{red}{~†} mánibus meis nocte \textbf{con}tra \textbf{e}um:\textcolor{red}{~*} et non \textbf{sum} de\textbf{cép}tus.

    \item Rénuit consolári ánima mea:\textcolor{red}{~†} memor fui Dei, et delectátus sum, et ex\textbf{er}ci\textbf{tá}tus sum:\textcolor{red}{~*} et defécit \textbf{spí}ritus \textbf{me}us.

    \item Anticipavérunt vigílias \textbf{ó}culi \textbf{me}i:\textcolor{red}{~*} turbátus sum, et non \textbf{sum} lo\textbf{cú}tus.

    \item Cogitávi \textbf{di}es an\textbf{tí}quos:\textcolor{red}{~*} et annos ætérnos in \textbf{men}te \textbf{há}bui.

    \item Et meditátus sum nocte cum \textbf{cor}de \textbf{me}o,\textcolor{red}{~*} et exercitábar, et scopébam \textbf{spí}ritum \textbf{me}um.

    \item Numquid in ætérnum pro\textbf{jí}ciet \textbf{De}us:\textcolor{red}{~*} aut non appónet ut complacíti\textbf{or} sit \textbf{ad}huc?

    \item Aut in finem misericórdiam \textbf{su}am ab\textbf{scín}det,\textcolor{red}{~*} a generatióne in gene\textbf{ra}ti\textbf{ó}nem?

    \item Aut obliviscétur mise\textbf{ré}ri \textbf{De}us:\textcolor{red}{~*} aut continébit in ira sua miseri\textbf{cór}dias \textbf{su}as?

    \item Et \textbf{di}xi: Nunc \textbf{cœ}pi:\textcolor{red}{~*} hæc mutátio déxte\textbf{ræ} Ex\textbf{cél}si.

    \item Memor fui \textbf{ó}perum \textbf{Dó}mini:\textcolor{red}{~*} quia memor ero ab inítio mirabíli\textbf{um} tu\textbf{ó}rum.

    \item Et meditábor in ómnibus o\textbf{pé}ribus \textbf{tu}is:\textcolor{red}{~*} et in adinventiónibus tuis \textbf{ex}er\textbf{cé}bor.

    \item Deus, in sancto via tua:\textcolor{red}{~†} quis Deus magnus sicut \textbf{De}us \textbf{nos}ter?\textcolor{red}{~*} tu es Deus qui facis \textbf{mi}ra\textbf{bí}lia.

    \item Notam fecísti in pópulis vir\textbf{tú}tem \textbf{tu}am:\textcolor{red}{~*} redemísti in bráchio tuo pópulum tuum fílios \textbf{Ja}cob et \textbf{Jo}seph.

    \item Vidérunt te aquæ, Deus, vi\textbf{dé}runt te \textbf{a}quæ:\textcolor{red}{~*} et timuérunt et turbátæ \textbf{sunt} a\textbf{býs}si.

    \item Multitúdo sóni\textbf{tus} a\textbf{quá}rum:\textcolor{red}{~*} vocem de\textbf{dé}runt \textbf{nu}bes.

    \item Etenim sagíttæ \textbf{tu}æ \textbf{tráns}eunt:\textcolor{red}{~*} vox tonítrui \textbf{tu}i in \textbf{ro}ta.

    \item Illuxérunt coruscatiónes tuæ \textbf{or}bi \textbf{ter}ræ:\textcolor{red}{~*} commóta est, et con\textbf{tré}muit \textbf{ter}ra.

    \item In mari via tua, et sémitæ tuæ in \textbf{a}quis \textbf{mul}tis:\textcolor{red}{~*} et vestígia tua non \textbf{co}gno\textbf{scén}tur.

    \item Deduxísti sicut oves \textbf{pó}pulum \textbf{tu}um,\textcolor{red}{~*} in manu Móy\textbf{si} et \textbf{A}aron.
  \end{enumerate}

  \grecommentary{\textit{Reprise de l'Antienne.}}
  \gabcsnippet{(c3) In(e) di(g!hwi)e(i'_[oh:h]) tri(i)bu(kj)la(k)ti(jh)ó(j_k)nis(j) me(i.)æ(i.) (;) De(i)um(hghf~) ex(g)qui(f)sí(e_[uh:l]g)vi(g.) (,) má(g_[uh:l]h)ni(f_h)bus(g) me(e.)is.(e.) (::)}

  \begin{multicols}{2}
    \begin{footnotesize}
      \begin{enumerate}[label=\textcolor{red}{\emph{\arabic*}}]
        \item \textit{J’ai crié de toute ma force au Seigneur : J’ai poussé ma voix vers Dieu, et il m’a exaucé.}
        \item \textit{Au jour de mon affliction, j’ai cherché Dieu, et j’ai tendu mes mains vers lui durant la nuit, et je n’ai pas été trompé}
        \item \textit{Mon âme a refusé les consolations ; je me suis souvenu de Dieu, et j’en ai eu de la joie : je me suis exercé, et mon esprit est tombé en défaillance.}
        \item \textit{Mes yeux ont prévenu les veilles : j’ai été rempli de trouble, sans pouvoir parler.}
        \item \textit{J’ai réfléchi sur les anciens jours ; et j’ai eu dans l’esprit les années éternelles.}
        \item \textit{Et j’ai médité dans mon cœur pendant la nuit : et je m’exerçais, et je purifiais mon esprit.}
        \item \textit{Est-ce que Dieu me rejettera pour toujours ? ou, ne pourra-t-il plus se résoudre à m’être favorable ?}
        \item \textit{Nous privera-t-il de sa miséricorde éternellement : de génération en génération ?}
        \item \textit{Dieu oubliera-t-il d’avoir compassion de nous ? Sa colère suspendra-t-elle le cours de ses miséricordes ?}
        \item \textit{Et j’ai dit : C’est maintenant que je commence : la main du Très-Haut a fait ce changement.} 
        \item \textit{Et je méditerai sur toutes vos œuvres ; et je m’exercerai sur tous vos desseins.}
        \item \textit{O Dieu, vos voies sont saintes ! Y a-t-il un Dieu aussi grand que notre Dieu ?}
        \item \textit{Vous êtes le seul Dieu qui opérez des merveilles. Vous avez fait connaître votre puissance parmi les peuples : vous avez délivré, par votre bras, votre peuple, les enfants de Jacob et de Joseph.}
        \item \textit{Les eaux vous ont vu, ô Dieu ! les eaux vous ont vu ; les abîmes ont tremblé, et ont été dans l’épouvante.}
        \item \textit{La multitude des orages ont fait grand bruit : les nuées ont fait entendre leurs voix.}
        \item \textit{Vos flèches ont paru : le bruit de votre tonnerre a renversé celui des roues}
        \item \textit{La lumière de vos éclairs a brillé sur toute l’étendue de la terre : elle a tremblé, et elle a été ébranlée.}
        \item \textit{Vous avez fait un chemin dans la mer, et une route au milieu des eaux ; et l’on n’a point connu les traces de vos pieds.}
        \item \textit{Vous avez conduit votre peuple comme des brebis, par la main de Moïse et d’Aaron.}
      \end{enumerate}
    \end{footnotesize}
  \end{multicols}

  \medskip

  \begin{center}
    \begin{footnotesize}
      \textcolor{red}{\textit{On chante le verset debout.}}
    \end{footnotesize}
    \begin{minipage}{0.8\linewidth}
      \gresetinitiallines{0}
      % \grecommentary[10pt]{\textcolor{red}{\textit{On se lève.}}}
      \large
      \gabcsnippet{(c4)<c><v>\Vbar</v>.</c> Ex(h)súr(i')ge(h), Dó(g')mi(g)ne.(g.) (::) <c><v>\Rbar</v>.</c> Et(h) jú(h)di(h)ca(h) cáu(i')sam(h) mé(g.)am(g.) (::) (Z)}
      \bigskip
      \normalsize
      \begin{center}
        \textit{\textcolor{red}{\Vbar.} Levez-vous, Seigneur,}
        \textit{\textcolor{red}{\Rbar.} Et jugez ma cause.}
      \end{center}
    \end{minipage}
  \end{center}

  \medskip
  \begin{center}
    \rule{4cm}{0.4pt}
  \end{center}

  \newpage

  \par Les Leçons du troisième Nocturne, selon l'usage ancien, sont tirées de l'Écriture Sainte. L'apôtre Saint Paul y rappelle l'institution de la Sainte Eucharistie, dont ce jour est l'anniversaire ; il reprend les fidèles de Corinthe des abus qui s'étaient introduits dans leurs assemblées religieuses, et enseigne quelles dispositions il faut apporter à la communion et quel est le malheur de celui qui la reçoit indignement.

  \medskip

  % \newpage

  \begin{center}
    \large Leçon VII.\\
    \normalsize
  \end{center}
  \medskip

  \setlength{\columnsep}{2pc}
  \def\columnseprulecolor{\color{red}}
  \setlength{\columnseprule}{0.4pt}

  \begin{multicols}{2}
    \begin{center}
      De Epístola prima beáti Pauli\\ Apóstoli ad Corínthios.
    \end{center}

    \par Hoc autem præcípio : non laudans
    quod non in mélius, sed in detérius
    convenítis. Primum quidem conveniéntibus vobis in Ecclésiam, áudio
    scissúras esse inter vos, et ex parte
    credo. Nam opórtet et hæreses esse, ut
    et qui probáti sunt, manifésti fiant in
    vobis.
    \par Conveniéntibus ergo vobis in
    unum, jam non est Domínicam cœnam manducáre. Unusquísque enim
    suam cœnam præsúmit ad manducándum. Et álius quidem ésurit,
    álius autem ébrius est.
    \par Numquid domos non habétis ad manducándum, et
    bibéndum ? aut Ecclésiam Dei contémnitis, et confúnditis eos qui non habent ? Quid dicam vobis ? laudo vos ?
    in hoc non laudo.
    \par \hspace{\fill}
    \columnbreak
    \begin{center}
      De la première Epître de Saint Paul,\\ Apôtre, aux Corinthiens.
      \begin{footnotesize}
        \textit{1 Cor 11, 17-34}
      \end{footnotesize}
    \end{center}
    \par \textit{J’ai encore à vous dire que je ne vous loue pas de ce
    que vos assemblées vous nuisent plutôt qu’elles ne
    vous servent. Premièrement, j’apprends que quand
    vous vous assemblez dans l’Eglise, il arrive parmi
    vous des divisions, et j’en crois quelque chose ; car
    il faut qu’il y ait des hérésies, afin qu’on découvre
    parmi vous ceux qui sont restés fermes.}
    \par \textit{De la manière dont vous faites ces assemblées, ce
    n’est point manger la Cène du Seigneur ; car chacun prend et mange par avance le souper qu’il
    apporte ; en sorte que les uns n’ont rien à manger,
    pendant que les autres font grande chère.}
    \par \textit{N’avez-vous pas vos maisons, pour y manger et
    pour y boire ? Ou, méprisez-vous l’Eglise de
    Dieu, et voulez-vous faire honte à ceux qui n’ont
    rien ? Que vous dirai-je : Dois-je vous louer ? Je
    ne dois pas vous louer en cela.}
  \end{multicols}
  \setlength\columnseprule{0pt}

  \medskip

  \gresetinitiallines{1}
  \greillumination{\initfamily\fontsize{11mm}{11mm}\selectfont E}
  \gregorioscore{repons/re--eram_quasi_agnus--solesmes_1961}
  
  \small
  \begin{multicols}{2}
    \par\textcolor{red}{\textit{\Rbar}.} \textit{J’étais comme un Agneau innocent ; j’ai été conduit pour être immolé sans le savoir : mes ennemis ont formé des entreprises contre moi, en disant :  \\ \textcolor{red}{*} Venez, mettons du bois dans son pain, et bannissons-le de la terre des vivants.}
    \columnbreak
    \par\textcolor{red}{\textit{\Vbar}.} \textit{ Tous mes ennemis conspiraient pour me faire
    du mal ; ils ont arrêté une chose très injuste contre moi, disant :}
    \par \textit{\textcolor{red}{*} Venez, mettons du bois dans son pain, et bannissons-le de la terre des vivants.}\\
    % \par \hspace{\fill}
  \end{multicols}
  \normalsize

  \medskip

  \begin{center}
    \large Leçon VIII.\\
    \normalsize
  \end{center}
  \medskip

  \setlength{\columnsep}{2pc}
  \def\columnseprulecolor{\color{red}}
  \setlength{\columnseprule}{0.4pt}

  \begin{multicols}{2}
    \par Ego enim accépi a Dómino quod et
    trádidi vobis, quóniam Dóminus Jesus
    in qua nocte tradebátur, accépit panem, et grátias agens fregit, et dixit :
    Accípite, et manducáte : hoc est corpus meum, quod pro vobis tradétur :
    hoc fácite in meam commemoratiónem.
    \par Simíliter et cálicem, postquam
    cœnávit, dicens : Hic calix novum testaméntum est in meo sánguine : hoc
    fácite quotiescúmque bibétis, in meam
    commemoratiónem.
    \par Quotiescúmque
    enim manducábitis panem hunc, et
    cálicem bibétis, mortem Dómini annuntiábitis donec véniat.
    % \par \hspace{\fill}
    \columnbreak
    \par \textit{Car j’ai appris du Seigneur Jésus, et je vous l’ai
    aussi enseigné, que la nuit même qu’il fut livré, il
    prit du pain, et qu’ayant fait des actions de
    grâces, il le rompit, et dit : Prenez mangez, ceci
    est mon Corps qui sera livré pour vous ; faites ceci
    en mémoire de moi.}
    \par \textit{Il prit aussi la coupe, après avoir soupé, et dit :
    Ce Calice est le nouveau Testament en mon Sang :
    faites ceci en mémoire de moi, toutes les fois que
    vous le boirez.}
    \par \textit{Car toutes les fois que vous mangerez ce Pain, et
    que vous boirez ce Calice, vous annoncerez la mort
    du Seigneur, jusqu’à ce qu’il vienne.}
  \end{multicols}
  \setlength\columnseprule{0pt}

  \medskip

  \gresetinitiallines{1}
  \greillumination{\initfamily\fontsize{11mm}{11mm}\selectfont U}
  \gregorioscore{repons/re--una_hora--solesmes_1961}
  
  \small
  \begin{multicols}{2}
    \par\textcolor{red}{\textit{\Rbar}.} \textit{Vous n’avez pu veiller une heure avec moi,
    vous qui vous encouragez à mourir pour moi.  \\ \textcolor{red}{*} Vous ne voyez pas que Judas ne dort point, mais qu’il se presse pour me livrer aux Juifs.}
    \par \hspace{\fill}
    \columnbreak
    \par\textcolor{red}{\textit{\Vbar}.} \textit{  Pourquoi dormez-vous ? Levez-vous, et priez,
    afin que vous n’entriez point en tentation.}
    \par \textit{\textcolor{red}{*} Vous ne voyez pas que Judas ne dort point, mais
    qu’il se presse pour me livrer aux Juifs.}\\
    % \par \hspace{\fill}
  \end{multicols}
  \normalsize

  \medskip
  
  \begin{center}
    \large Leçon IX.\\
    \normalsize
  \end{center}
  \medskip

  \setlength{\columnsep}{2pc}
  \def\columnseprulecolor{\color{red}}
  \setlength{\columnseprule}{0.4pt}

  \begin{multicols}{2}
    \par Itaque quicúmque manducáverit panem hunc, vel bíberit cálicem Dómini
    indígne, reus erit córporis et sánguinis
    Dómini. Probet autem seípsum homo : et sic de pane illo edat, et de
    cálice bibat. Qui enim mandúcat et bibit indígne, judícium sibi mandúcat et
    bibit, non dijúdicans corpus Dómini.
    Ideo inter vos multi infírmi et imbecílles, et dórmiunt multi. Quod si
    nosmetípsos dijudicarémus, non
    útique judicarémur. Dum judicámur
    autem, a Dómino corrípimur, ut non
    cum hoc mundo damnémur. Itaque
    fratres mei, cum convenítis ad manducándum, ínvicem exspectáte. Si quis
    ésurit, domi mandúcet, ut non in
    judícium conveniátis. Cétera autem,
    cum vénero, dispónam.
    % \par \hspace{\fill}
    \columnbreak
    \par \textit{C’est pourquoi quiconque mangera ce Pain, ou
    boira le Calice du Seigneur indignement, sera coupable de la profanation du Corps et du Sang du
    Seigneur. Que l’homme donc s’éprouve soi-même,
    et qu’il mange ainsi de ce Pain et boive de ce
    Calice ; car quiconque en mange et en boit indignement, mange et boit sa propre condamnation,
    ne discernant point le Corps du Seigneur. C’est
    pour cela qu’il y a parmi vous plusieurs infirmes
    et malades, et plusieurs qui dorment. Que si
    nous nous examinions nous-mêmes, nous ne serions pas jugés de la sorte. Mais quand nous
    sommes ainsi jugés, c’est le Seigneur qui nous châtie, afin que nous ne soyons pas condamnés avec le
    monde. C’est pourquoi, mes frères, lorsque vous
    vous assemblez pour manger, attendez-vous les uns
    les autres : que ceux qui sont pressés de manger,
    mangent chez eux, afin que vous ne vous assembliez pas pour votre condamnation. A l’égard des
    autres choses, je les règlerai quand je serai venu.}
  \end{multicols}
  \setlength\columnseprule{0pt}
  \setlength{\columnsep}{0pc}

  \newpage

  \gresetinitiallines{1}
  \greillumination{\initfamily\fontsize{11mm}{11mm}\selectfont S}
  \gregorioscore{repons/re--seniores--solesmes_1961}
  
  \small
  \begin{multicols}{2}
    \par\textcolor{red}{\textit{\Rbar}.} \textit{ Les anciens du peuple tinrent conseil, } 
    \par \textcolor{red}{*} \textit{Pour se saisir adroitement de Jésus,\\ et le faire mourir : ils furent à lui avec des épées et des bâtons, comme pour prendre un voleur.}
    \par\textcolor{red}{\textit{\Vbar}.} \textit{Les Pontifes et les Pharisiens assemblèrent le conseil.}
    \par \hspace{\fill}
  \end{multicols}
  \normalsize 
  
  \medskip
  \begin{center}
    \rule{4cm}{0.4pt}
  \end{center}
  \medskip

  \begin{center}
    \large À LAUDES.\\
  \end{center}
  \medskip
  \par Tous les jours, hormis aux fêtes et au Temps Pascal, l'Office de Laudes commence par le psaume \textit{Miserére}. Il importe en effet d'être purifié de toute souillure avant de chanter les louanges divines et d'offrir le sacrifice de justice, l'oblation digne d'être reçue par Dieu.

  \medskip

  \gresetinitiallines{1}
  \greillumination{\initfamily\fontsize{11mm}{11mm}\selectfont J}
  \gregorioscore{antiennes/an--justificeris_domine--solesmes}
  \begin{center}
    \footnotesize{
      \textit{Seigneur, soyez reconnu juste dans vos paroles, et victorieux dans vos jugements.}
  }
  \end{center}

  \begin{center}
    \large{Psaume 50.}\\
  \end{center}

  \gresetinitiallines{0}
  \gregorioscore{psaumes/psaume50-VIIIG}
  
  \begin{enumerate}[label=\textcolor{red}{\arabic*}]
    \setcounter{enumi}{1}
    \item Et secúndum multitúdinem miseratiónum tu\textbf{á}rum,\textcolor{red}{~*} dele iniqui\textit{tá}\textit{tem} \textbf{me}am.

    \item Amplius lava me ab iniquitáte \textbf{me}a:\textcolor{red}{~*} et a peccáto \textit{me}\textit{o} \textbf{mun}da me.

    \item Quóniam iniquitátem meam ego co\textbf{gnós}co:\textcolor{red}{~*} et peccátum meum contra \textit{me} \textit{est} \textbf{sem}per.

    \item Tibi soli peccávi, et malum coram te \textbf{fe}ci:\textcolor{red}{~*} ut justificéris in sermónibus tuis, et vincas cum \textit{ju}\textit{di}\textbf{cá}ris.

    \item Ecce enim in iniquitátibus con\textbf{cép}tus sum:\textcolor{red}{~*} et in peccátis concépit me \textit{ma}\textit{ter} \textbf{me}a.

    \item Ecce enim veritátem dile\textbf{xís}ti:\textcolor{red}{~*} incérta et occúlta sapiéntiæ tuæ manifes\textit{tás}\textit{ti} \textbf{mi}hi.

    \item Aspérges me hyssópo, et mun\textbf{dá}bor:\textcolor{red}{~*} lavábis me, et super nivem \textit{de}\textit{al}\textbf{bá}bor.

    \item Audítui meo dabis gáudium et læ\textbf{tí}tiam:\textcolor{red}{~*} et exsultábunt ossa hu\textit{mi}\textit{li}\textbf{á}ta.

    \item Avérte fáciem tuam a peccátis \textbf{me}is:\textcolor{red}{~*} et omnes iniquitátes \textit{me}\textit{as} \textbf{de}le.

    \item Cor mundum crea in me, \textbf{De}us:\textcolor{red}{~*} et spíritum rectum ínnova in viscé\textit{ri}\textit{bus} \textbf{me}is.

    \item Ne projícias me a fácie \textbf{tu}a:\textcolor{red}{~*} et spíritum sanctum tuum ne áu\textit{fe}\textit{ras} \textbf{a} me.

    \item Redde mihi lætítiam salutáris \textbf{tu}i:\textcolor{red}{~*} et spíritu principá\textit{li} \textit{con}\textbf{fír}ma me.

    \item Docébo iníquos vias \textbf{tu}as:\textcolor{red}{~*} et ímpii ad te \textit{con}\textit{ver}\textbf{tén}tur.

    \item Líbera me de sanguínibus, Deus, Deus salútis \textbf{me}æ:\textcolor{red}{~*} et exsultábit lingua mea justí\textit{ti}\textit{am} \textbf{tu}am.

    \item Dómine, lábia mea a\textbf{pé}ries:\textcolor{red}{~*} et os meum annuntiábit \textit{lau}\textit{dem} \textbf{tu}am.

    \item Quóniam si voluísses sacrifícium, dedíssem \textbf{ú}tique:\textcolor{red}{~*} holocáustis non \textit{de}\textit{lec}\textbf{tá}beris.

    \item Sacrifícium Deo spíritus contribu\textbf{lá}tus:\textcolor{red}{~*} cor contrítum et humiliátum, Deus, \textit{non} \textit{de}\textbf{spí}cies.

    \item Benígne fac, Dómine, in bona voluntáte tua \textbf{Si}on:\textcolor{red}{~*} ut ædificéntur mu\textit{ri} \textit{Je}\textbf{rú}salem.

    \item Tunc acceptábis sacrifícium justítiæ, oblatiónes, et holo\textbf{cáus}ta:\textcolor{red}{~*} tunc impónent super altáre \textit{tu}\textit{um} \textbf{ví}tulos.
  \end{enumerate}

  \grecommentary{\textit{Reprise de l'Antienne.}}
  \gabcsnippet{(c4) Ju(g)sti(g)fi(g')cé(h)ris,(gf) Dó(g)mi(h)ne,(g.) (,) in(f) ser(h)mó(j')ni(i)bus(h') tu(j)is,(i.) (;) et(h') vin(g)cas(h'_) cum(f) ju(gh)di(h)cá(g.)ris.(g.) (::)}

  \begin{multicols}{2}
    \begin{footnotesize}
      \begin{enumerate}[label=\textcolor{red}{\emph{\arabic*}}]
        \item \textit{Ayez pitié de moi, mon Dieu, selon votre grande miséricorde.}
        \item \textit{Et selon la multitude de vos bontés, effacez mon iniquité.}
        \item \textit{Lavez-moi de plus en plus de mon iniquité, et
        purifiez-moi de mon péché ;}
        \item \textit{Parce que je connais mon iniquité, et que mon
        péché est toujours présent devant moi.}
        \item \textit{J’ai péché contre vous seul, j’ai fait le mal en votre
        présence ; afin que vous soyez reconnu juste dans
        vos paroles, et victorieux dans vos jugements.}
        \item \textit{Car j’ai été formé dans l’iniquité, et ma mère
        m’a conçu dans le péché.}
        \item \textit{Car vous avez aimé la vérité, et vous m’avez manifesté les secrets et les mystères de votre sagesse.}
        \item \textit{Vous m’arroserez avec l’hysope, et je serai pur ;
        lavez-moi, et je serai plus blanc que la neige.}
        \item \textit{Vous me ferez entendre des paroles de joie et de
        consolation ; et mes os humiliés seront dans la joie.}
        \item \textit{Détournez votre visage de dessus mes péchés, et
        effacez toutes mes iniquités.} 
        \item \textit{Mon Dieu, créez en moi un cœur pur, et renouvelez l’esprit de droiture jusques dans mes entrailles}
        \item \textit{Ne me rejetez pas de devant votre face, et ne retirez pas de moi votre Saint-Esprit.}
        \item \textit{Rendez-moi la joie de votre salut, et rassurez-moi
        par la force de votre Esprit.}
        \item \textit{J’enseignerai vos voies aux pécheurs, et les impies
        se convertiront à vous.}
        \item \textit{O Dieu, mon Dieu, auteur de mon salut, délivrez-moi du sang que j’ai répandu, et ma langue annoncera avec joie votre justice.}
        \item \textit{Seigneur, vous ouvrirez mes lèvres, et ma bouche
        annoncera vos louanges.}
        \item \textit{Car si vous eussiez voulu un sacrifice, je vous
        l’aurai offert ; mais les holocaustes ne vous sont
        pas agréables.}
        \item \textit{Un esprit pénétré de douleur, est un sacrifice que
        Dieu agrée : mon Dieu, vous ne mépriserez pas un
        cœur contrit et humilié.}
        \item \textit{Seigneur, faites sentir à Sion les effets de votre
        bonté ; afin que les murs de Jérusalem soient bâtis.}
        \item \textit{Alors vous accepterez le sacrifice de justice, les offrandes et les holocaustes : alors on offrira des veaux sur votre autel}
      \end{enumerate}
    \end{footnotesize}
  \end{multicols}

  \medskip

  \par Le Psaume 89 est propre aux Laudes du Jeudi. C'est un cantique du matin où le prophète considère la brièveté de la vie humaine, et demande la bénédiction du Seigneur sur les travaux de la journée. Aujourd'hui on peut le considérer plus spécialement comme une prière du Sauveur à son Père en faveur de l'humanité déchue et attendant la rédemption.

  \medskip

  \gresetinitiallines{1}
  \greillumination{\initfamily\fontsize{11mm}{11mm}\selectfont J}
  \gregorioscore{antiennes/an--dominus_tamquam_ovis--solesmes_1961}
  \begin{center}
    \footnotesize{
      \textit{Le Seigneur a été conduit comme une brebis, pour servir de victime, et il n’a point ouvert la bouche.}
  }
  \end{center}

  \begin{center}
    \large{Psaume 89.}\\
  \end{center}

  \gresetinitiallines{0}
  \gregorioscore{psaumes/psaume89-IID}
  
  \begin{enumerate}[label=\textcolor{red}{\arabic*}]
    \setcounter{enumi}{1}
    \item Priúsquam montes fíerent, aut formarétur terra et \textbf{or}bis:\textcolor{red}{~*} a sǽculo et usque in sǽculum tu \textit{es}, \textbf{De}us.

    \item Ne avértas hóminem in humili\textbf{tá}tem:\textcolor{red}{~*} et dixísti: Convertímini, fíli\textit{i} \textbf{hó}minum.

    \item Quóniam mille anni ante óculos \textbf{tu}os,\textcolor{red}{~*} tamquam dies hestérna, quæ \textit{præ}\textbf{tér}iit.

    \item Et custódia in \textbf{noc}te,\textcolor{red}{~*} quæ pro níhilo habéntur, eórum an\textit{ni} \textbf{e}runt.

    \item Mane sicut herba tránseat,\textcolor{red}{~†} mane flóreat, et \textbf{tráns}eat:\textcolor{red}{~*} véspere décidat, indúret et \textit{a}\textbf{rés}cat.

    \item Quia defécimus in ira \textbf{tu}a,\textcolor{red}{~*} et in furóre tuo turbá\textit{ti} \textbf{su}mus.

    \item Posuísti iniquitátes nostras in conspéctu \textbf{tu}o:\textcolor{red}{~*} sǽculum nostrum in illuminatióne vul\textit{tus} \textbf{tu}i.

    \item Quóniam omnes dies nostri defe\textbf{cé}runt:\textcolor{red}{~*} et in ira tua \textit{de}\textbf{fé}cimus.

    \item Anni nostri sicut aránea medita\textbf{bún}tur:\textcolor{red}{~*} dies annórum nostrórum in ipsis, septuagín\textit{ta} \textbf{an}ni.

    \item Si autem in potentátibus, octogínta \textbf{an}ni:\textcolor{red}{~*} et ámplius eórum, labor \textit{et} \textbf{do}lor.

    \item Quóniam supervénit mansue\textbf{tú}do:\textcolor{red}{~*} et corri\textit{pi}\textbf{é}mur.

    \item Quis novit potestátem iræ \textbf{tu}æ:\textcolor{red}{~*} et præ timóre tuo iram tuam dinu\textit{me}\textbf{rá}re?

    \item Déxteram tuam sic \textbf{no}tam fac:\textcolor{red}{~*} et erudítos corde in sa\textit{pi}\textbf{én}tia.

    \item Convértere, Dómine, \textbf{ús}quequo?\textcolor{red}{~*} et deprecábilis esto super ser\textit{vos} \textbf{tu}os.

    \item Repléti sumus mane misericórdia \textbf{tu}a:\textcolor{red}{~*} et exsultávimus, et delectáti sumus ómnibus dié\textit{bus} \textbf{nos}tris.

    \item Lætáti sumus pro diébus, quibus nos humili\textbf{ás}ti:\textcolor{red}{~*} annis, quibus vídi\textit{mus} \textbf{ma}la.

    \item Réspice in servos tuos, et in ópera \textbf{tu}a:\textcolor{red}{~*} et dírige fílios \textit{e}\textbf{ó}rum.

    \item Et sit splendor Dómini Dei nostri super nos,\textcolor{red}{~†} et ópera mánuum nostrárum dírige \textbf{su}per nos:\textcolor{red}{~*} et opus mánuum nostrá\textit{rum} \textbf{dí}rige.
  \end{enumerate}
  \smallskip

  \grecommentary{\textit{Reprise de l'Antienne.}}
  \gabcsnippet{(f3) Do(f!gwh)mi(f)nus(f'_) (,) tam(f)quam(f) o(fg)vis(f) ad(f) ví(f_e)cti(f)mam(h') du(h)ctus(g) est,(e.) (;) et(g_[oh:h]) non(i_[oh:h]) a(hf)pé(hh)ru(g')it(h) os(ih) su(f.)um.(f.) (::)}

  \begin{multicols}{2}
    \begin{footnotesize}
      \begin{enumerate}[label=\textcolor{red}{\emph{\arabic*}}]
        \item \textit{Seigneur, vous avez été notre asile, dans la suite de toutes les générations.}
        \item \textit{Avant que les montagnes eussent été faites, ou que
        la terre eût été formée, et l’Univers ; vous êtes
        Dieu de toute éternité et dans tous les siècles.}
        \item \textit{Ne faites pas rentrer l’homme dans son néant :
        vous qui avez dit aux enfants des hommes : Convertissez-vous.}
        \item \textit{Car mille ans ne paraissent devant vos yeux, que
        comme le jour d’hier qui est passé.}
        \item \textit{Et comme une veille de la nuit : leurs années seront comparées au néant.}
        \item \textit{L’homme passe comme une herbe qui fleurit le
        matin, et qui se flétrit aussitôt ; qui tombe le soir,
        qui durcit, et qui se sèche.}
        \item \textit{Nous avons succombé sous votre colère, et nous
        avons été troublés par votre fureur.}
        \item \textit{Vous avez mis nos iniquités en votre présence, et
        le temps de notre vie exposé à la lumière de votre
        visage.}
        \item \textit{Parce que tous nos jours ont été détruits, et consumés par votre colère.}
        \item \textit{Nos années sont considérées comme la toile de
        l’araignée ; nos jours et nos années ne sont plus
        que de soixante-dix ans.} 
        \item \textit{Que si les plus robustes vont jusqu’à quatre-vingt
        ans, le surplus n’est que peine et douleur.}
        \item \textit{Et c’est par un effet de votre bonté, que vous nous
        corrigez ainsi.}
        \item \textit{Qui peut connaître et comprendre la grandeur de
        votre colère, et la craindre autant qu’elle est redoutable ?}
        \item \textit{Faites connaître le pouvoir de votre main droite,
        et instruisez notre cœur dans la sagesse.}
        \item \textit{Seigneur, tournez-vous vers nous ; jusqu’à quand
        serez-vous irrité ? Soyez favorable envers vos serviteurs.}
        \item \textit{Nous avons été comblés de votre miséricorde dès le
        matin ; nous avons eu de la joie et de la consolation pendant les jours de notre vie.}
        \item \textit{Nous nous sommes réjouis dans les jours où vous
        nous avez humiliés, et pendant les années que
        nous avons passées dans les maux.}
        \item \textit{Regardez vos serviteurs et vos ouvrages ; et conduisez leurs enfants.}
        \item \textit{Et que la lumière du Seigneur notre Dieu se répande sur nous : réglez par votre sagesse les ouvrages de nos mains, et conduisez par vous-même
        les œuvres de nos mains.}
      \end{enumerate}
    \end{footnotesize}
  \end{multicols}

  \medskip

  \par Le Psaume 35, après avoir décrit la profonde perversité du méchant, exprime surtout la confiance du chrétien en la misericorde divine, confiance qui ne peut jamais être mieux fondée qu'en ces jours de la Passion du Rédempteur.

  \medskip

  \gresetinitiallines{1}
  \greillumination{\initfamily\fontsize{11mm}{11mm}\selectfont C}
  \gregorioscore{antiennes/an--contritum_est_cor_meum--solesmes}
  \begin{center}
    \footnotesize{
      \textit{Mon cœur a été brisé au-dedans de moi, et tous mes os ont été ébranlés.}
  }
  \end{center}

  \begin{center}
    \large{Psaume 35.}\\
  \end{center}

  \gresetinitiallines{0}
  \gregorioscore{psaumes/psaume35-VIIIG}
  
  \begin{enumerate}[label=\textcolor{red}{\arabic*}]
    \setcounter{enumi}{1}
    \item Quóniam dolóse egit in conspéctu \textbf{e}jus:\textcolor{red}{~*} ut inveniátur iníquitas e\textit{jus} \textit{ad} \textbf{ó}dium.

    \item Verba oris ejus iníquitas, et \textbf{do}lus:\textcolor{red}{~*} nóluit intellígere ut \textit{be}\textit{ne} \textbf{á}geret.

    \item Iniquitátem meditátus est in cubíli \textbf{su}o:\textcolor{red}{~*} ástitit omni viæ non bonæ, malítiam autem \textit{non} \textit{o}\textbf{dí}vit.

    \item Dómine, in cælo misericórdia \textbf{tu}a:\textcolor{red}{~*} et véritas tua us\textit{que} \textit{ad} \textbf{nu}bes.

    \item Justítia tua sicut montes \textbf{De}i:\textcolor{red}{~*} judícia tua a\textit{býs}\textit{sus} \textbf{mul}ta.

    \item Hómines, et juménta salvábis, \textbf{Dó}mine:\textcolor{red}{~*} quemádmodum multiplicásti misericórdiam \textit{tu}\textit{am}, \textbf{De}us,

    \item Fílii autem \textbf{hó}minum,\textcolor{red}{~*} in tégmine alárum tuá\textit{rum} \textit{spe}\textbf{rá}bunt.

    \item Inebriabúntur ab ubertáte domus \textbf{tu}æ:\textcolor{red}{~*} et torrénte voluptátis tuæ po\textit{tá}\textit{bis} \textbf{e}os.

    \item Quóniam apud te est fons \textbf{vi}tæ:\textcolor{red}{~*} et in lúmine tuo vidé\textit{bi}\textit{mus} \textbf{lu}men.

    \item Præténde misericórdiam tuam sciénti\textbf{bus} te,\textcolor{red}{~*} et justítiam tuam his, qui rec\textit{to} \textit{sunt} \textbf{cor}de.

    \item Non véniat mihi pes su\textbf{pér}biæ:\textcolor{red}{~*} et manus peccatóris non \textit{mó}\textit{ve}\textbf{at} me.

    \item Ibi cecidérunt qui operántur iniqui\textbf{tá}tem:\textcolor{red}{~*} expúlsi sunt, nec potu\textit{é}\textit{runt} \textbf{sta}re.
  \end{enumerate}
  \smallskip
  \grecommentary{\textit{Reprise de l'Antienne.}}
  \gabcsnippet{(c4) Con(h)trí(f)tum(fg) est(g'_[oh:h]) (,) cor(g) me(h)um(g') in(g) mé(h')di(g)o(f) me(ixgig)i,(h.) (;) con(h'_)tre(j_)mu(g')é(h)runt(g'_[oh:h]) (,) ó(g)mni(fe)a(d') os(f)sa(gh) me(g.)a.(g.) (::)}

  \begin{multicols}{2}
    \begin{footnotesize}
      \begin{enumerate}[label=\textcolor{red}{\emph{\arabic*}}]
        \item \textit{Le méchant a résolu de commettre le péché ; la crainte de Dieu n'est pas devant ses yeux.}
        \item \textit{Car il se flatte sous le regard même de Dieu, que son iniquité ne sera ni connue ni châtiée.}
        \item \textit{Les paroles de sa bouche sont injustice et tromperie ; il ne veut pas acquérir la sagesse, pour faire le bien.}
        \item \textit{Sur sa couche, il médite l'iniquité, il se tient dans toute voie qui n'est pas bonne ; il n'a de répugnance pour aucun mal.}
        \item \textit{Seigneur, votre bonté atteint jusqu'aux cieux, et votre fidélité jusqu'aux nues.}
        \item \textit{Votre justice est comme les montagnes de Dieu, vos jugements comme le vaste abîme des eaux.}
        \item \textit{Seigneur, votre providence garde les hommes et les animaux. Combien grande est votre bonté, ô Dieu !}
        \item \textit{Les enfants des hommes se confient à l'ombre de vos ailes.}
        \item \textit{Ils s'enivrent de l'abondance de votre maison, et vous les abreuvez au torrent de vos délices.}
      \end{enumerate}
    \end{footnotesize}
  \end{multicols}

  \medskip

  \par Le cantique chanté par Moïse après le passage de la mer rouge fait partie de l'Office des Laudes du Jeudi. Il nous rappelle aujourd'hui que le passage des Israélites à travers la mer rouge est une figure de l'affranchissement du genre humain de la captivité du démon.

  \medskip

  \gresetinitiallines{1}
  \greillumination{\initfamily\fontsize{11mm}{11mm}\selectfont E}
  \gregorioscore{antiennes/an--exhortatus_es--solesmes_1961}
  \begin{center}
    \footnotesize{
      \textit{Seigneur, vous nous avez exhortez à nous confier en vous, et dans votre sainte nourriture.}
  }
  \end{center}

  \newpage

  \begin{center}
    \large{Cantique de Moïse,}\\
    \small\textit{Exode, 15.}
  \end{center}

  \gresetinitiallines{0}
  \gregorioscore{psaumes/cantique-moise-exode-IVA}
  
  \begin{enumerate}[label=\textcolor{red}{\arabic*}]
    \setcounter{enumi}{1}
    \item Fortitúdo mea, et laus \textit{me}\textit{a} \textbf{Dó}minus,\textcolor{red}{~*} et factus est mi\textit{hi} \textit{in} \textit{sa}\textbf{lú}tem.

    \item Iste Deus meus, et glorifi\textit{cá}\textit{bo} \textbf{e}um:\textcolor{red}{~*} Deus patris mei, et ex\textit{al}\textit{tá}\textit{bo} \textbf{e}um.

    \item Dóminus quasi vir pugnátor,\textcolor{red}{~†} Omnípotens \textit{no}\textit{men} \textbf{e}jus.\textcolor{red}{~*} Currus Pharaónis et exércitum ejus pro\textit{jé}\textit{cit} \textit{in} \textbf{ma}re.

    \item Elécti príncipes ejus submérsi sunt in \textit{Ma}\textit{ri} \textbf{Ru}bro:\textcolor{red}{~*} abýssi operuérunt eos, descendérunt in profún\textit{dum} \textit{qua}\textit{si} \textbf{la}pis.

    \item Déxtera tua, Dómine, magnificáta est in fortitúdine:\textcolor{red}{~†} déxtera tua, Dómine, percússit \textit{in}\textit{i}\textbf{mí}cum.\textcolor{red}{~*} Et in multitúdine glóriæ tuæ deposuísti adver\textit{sá}\textit{ri}\textit{os} \textbf{tu}os:

    \item Misísti iram tuam, quæ devorávit eos \textit{sic}\textit{ut} \textbf{stí}pulam.\textcolor{red}{~*} Et in spíritu furóris tui congre\textit{gá}\textit{tæ} \textit{sunt} \textbf{a}quæ:

    \item Stetit \textit{un}\textit{da} \textbf{flu}ens,\textcolor{red}{~*} congregátæ sunt abýssi in \textit{mé}\textit{di}\textit{o} \textbf{ma}ri.

    \item Dixit inimícus: Pérsequar et \textit{com}\textit{pre}\textbf{hén}dam,\textcolor{red}{~*} dívidam spólia, implébitur \textit{á}\textit{ni}\textit{ma} \textbf{me}a:

    \item Evaginábo glá\textit{di}\textit{um} \textbf{me}um,\textcolor{red}{~*} interfíciet e\textit{os} \textit{ma}\textit{nus} \textbf{me}a.

    \item Flavit spíritus tuus, et opéruit \textit{e}\textit{os} \textbf{ma}re:\textcolor{red}{~*} submérsi sunt quasi plumbum in a\textit{quis} \textit{ve}\textit{he}\textbf{mén}tibus.

    \item Quis símilis tui in fór\textit{ti}\textit{bus}, \textbf{Dó}mine?\textcolor{red}{~*} quis símilis tui, magníficus in sanctitáte, terríbilis atque laudábilis, fáci\textit{ens} \textit{mi}\textit{ra}\textbf{bí}lia?

    \item Extendísti manum tuam, et devorávit \textit{e}\textit{os} \textbf{ter}ra.\textcolor{red}{~*} Dux fuísti in misericórdia tua pópulo \textit{quem} \textit{red}\textit{e}\textbf{mís}ti:

    \item Et portásti eum in fortitú\textit{di}\textit{ne} \textbf{tu}a,\textcolor{red}{~*} ad habitácu\textit{lum} \textit{sanc}\textit{tum} \textbf{tu}um.

    \item Ascendérunt pópuli, \textit{et} \textit{i}\textbf{rá}ti sunt:\textcolor{red}{~*} dolóres obtinuérunt habita\textit{tó}\textit{res} \textit{Phi}\textbf{lís}thiim.

    \item Tunc conturbáti sunt príncipes Edom,\textcolor{red}{~†} robústos Moab obtí\textit{nu}\textit{it} \textbf{tre}mor:\textcolor{red}{~*} obriguérunt omnes habi\textit{ta}\textit{tó}\textit{res} \textbf{Chá}naan.

    \item Irruat super eos formí\textit{do} \textit{et} \textbf{pa}vor,\textcolor{red}{~*} in magnitúdine \textit{brá}\textit{chi}\textit{i} \textbf{tu}i:

    \item Fiant immóbiles quasi lapis,\textcolor{red}{~†} donec pertránseat pópulus \textit{tu}\textit{us}, \textbf{Dó}mine,\textcolor{red}{~*} donec pertránseat pópulus tuus iste, \textit{quem} \textit{pos}\textit{se}\textbf{dís}ti.

    \item Introdúces eos, et plantábis in monte heredi\textit{tá}\textit{tis} \textbf{tu}æ,\textcolor{red}{~*} firmíssimo habitáculo tuo quod ope\textit{rá}\textit{tus} \textit{es}, \textbf{Dó}mine:

    \item Sanctuárium tuum, Dómine, quod firmavérunt \textit{ma}\textit{nus} \textbf{tu}æ.\textcolor{red}{~*} Dóminus regnábit in æ\textit{tér}\textit{num} \textit{et} \textbf{ul}tra.

    \item Ingréssus est enim eques Phárao cum cúrribus et equítibus e\textit{jus} \textit{in} \textbf{ma}re:\textcolor{red}{~*} et redúxit super eos Dómi\textit{nus} \textit{a}\textit{quas} \textbf{ma}ris:

    \item Fílii autem Israël ambulavé\textit{runt} \textit{per} \textbf{sic}cum\textcolor{red}{~*} in \textit{mé}\textit{di}\textit{o} \textbf{e}jus.
  \end{enumerate}
  \smallskip
  \grecommentary{\textit{Reprise de l'Antienne.}}
  \gabcsnippet{(c3) Ex(e)hor(f')tá(h)tus(hi) es(i.) (,) in(h') vir(i)tú(j)te(ih) tu(i.)a,(i.) (;) et(i) in(f') re(i)fe(g')cti(h)ó(f_e)ne(f_e) (,) san(d')cta(e) tu(f')a,(h) Dó(f')mi(f)ne.(f.) (::)}

  \begin{multicols}{2}
    \begin{footnotesize}
      \begin{enumerate}[label=\textcolor{red}{\emph{\arabic*}}]
        \item \textit{Chantons les louanges du Seigneur, qui a fait magnifiquement éclater sa puissance : il a précipité dans la mer le cheval et le cavalier.}
        \item \textit{Le Seigneur est ma force ; mes louanges ne sont
        que pour lui, et il est devenu mon salut.}
        \item \textit{C’est lui qui est mon Dieu, et je le glorifierai : il
        est le Dieu de mon Père, et je l’exalterai.}
        \item \textit{Le Seigneur a paru comme un combattant : son
        nom est le tout-puissant : il a précipité dans la mer
        le char de Pharaon et son armée.}
        \item \textit{Les Princes qu’il avait choisis, ont été engloutis
        dans la mer rouge : les abîmes les ont couverts : ils
        ont été précipités dans le fond comme une pierre.}
        \item \textit{Votre droite, Seigneur, a fait éclater sa force et sa
        puissance : votre droite, Seigneur, a frappé
        l’ennemi, et vous avez terrassé vos adversaires
        dans la grandeur de votre gloire.}
        \item \textit{Vous avez lâché votre colère, qui les a dévorés
        comme la paille : les eaux ont été ramassées par le
        souffle de votre fureur.}
        \item \textit{L’onde coulante s’est arrêtée : les abîmes se sont
        amassés au milieu de la mer.}
        \item \textit{L’ennemi a dit : Je les poursuivrai, je les prendrai ; je partagerai leurs dépouilles, et mon âme
        sera satisfaite.}
        \item \textit{Je tirerai mon épée, et ma main les mettra à mort.}
        \item \textit{Le vent a soufflé par votre ordre, et la mer les a
        couverts : ils ont été engloutis comme du plomb
        dans les eaux rapides.}
        \item \textit{Qui d’entre les plus forts est semblable à vous,
        Seigneur ? Qui est semblable à vous ? Qui est
        plus magnifique en sainteté, plus terrible et plus
        digne de louanges, par les merveilles que vous
        faites ?}
        \item \textit{Vous avez étendu la main, et la terre les a dévorés : vous avez été par votre miséricorde le conducteur du peuple que vous avez racheté.}
        \item \textit{Et vous l’avez porté par votre force, jusqu’à votre
        sainte demeure.}
        \item \textit{Les peuples se sont attroupés pleins de colère : les
        Philistins, habitants du pays, ont été pénétrés de
        douleur.}
        \item \textit{Alors les Princes d’Edom ont été troublés ; la terreur s’est saisi des plus forts Moabites : tous les
        habitants de Chanaan sont devenus immobiles.}
        \item \textit{Faites que l’épouvante et la terreur tombe sur
        eux, par la puissance de votre bras.}
        \item \textit{Qu’ils deviennent immobiles comme une pierre,
        jusqu’à ce que votre peuple soit passé, Seigneur ;
        jusqu’à ce que votre peuple soit passé, duquel vous
        êtes toujours le maître.}
        \item \textit{Vous les introduirez, et vous les établirez sur la
        montagne de votre héritage, dans cette demeure solide qui est votre ouvrage, Seigneur}
        \item \textit{Dans votre Sanctuaire, Seigneur, que vos mains
        ont affermi : le Seigneur règnera dans toute
        l’éternité, et au-delà.}
        \item \textit{Car Pharaon est entré à cheval avec ses chars et
        sa cavalerie dans la mer ; et le Seigneur a ramené
        sur eux les eaux de la mer}
        \item \textit{Mais les enfants d’Israël ont marché par un chemin sec au milieu de la mer.}
      \end{enumerate}
    \end{footnotesize}
  \end{multicols}

  \medskip

  \par Jamais la louange ne peut être plus opportune, qu'au moment où le Christ donne aux hommes les témoignages les plus éclatants de son amour. N'est-ce pas lui qui rétablit Jérusalem, l'Église de la terre et du ciel ?

  \medskip

  \gresetinitiallines{1}
  \greillumination{\initfamily\fontsize{11mm}{11mm}\selectfont O}
  \gregorioscore{antiennes/an--oblatus_est--solesmes}
  \begin{center}
    \footnotesize{
      \textit{Il a été immolé, parce qu’il l’a voulu, et il a porté lui-même nos péchés.}
  }
  \end{center}

  \begin{center}
    \large{Psaume 146.}
  \end{center}

  \gresetinitiallines{0}
  \gregorioscore{psaumes/psaume146-IID}
  
  \begin{enumerate}[label=\textcolor{red}{\arabic*}]
    \setcounter{enumi}{1}
    \item Ædíficans Jerúsalem \textbf{Dó}minus:\textcolor{red}{~*} dispersiónes Israélis con\textit{gre}\textbf{gá}bit.

    \item Qui sanat contrítos \textbf{cor}de:\textcolor{red}{~*} et álligat contritiónes \textit{e}\textbf{ó}rum.

    \item Qui númerat multitúdinem stel\textbf{lá}rum:\textcolor{red}{~*} et ómnibus eis nómi\textit{na} \textbf{vo}cat.

    \item Magnus Dóminus noster, et magna virtus \textbf{e}jus:\textcolor{red}{~*} et sapiéntiæ ejus non \textit{est} \textbf{nú}merus.

    \item Suscípiens mansuétos \textbf{Dó}minus:\textcolor{red}{~*} humílians autem peccatóres usque \textit{ad} \textbf{ter}ram.

    \item Præcínite Dómino in confessi\textbf{ó}ne:\textcolor{red}{~*} psállite Deo nostro \textit{in} \textbf{cí}thara.

    \item Qui óperit cælum \textbf{nú}bibus:\textcolor{red}{~*} et parat ter\textit{ræ} \textbf{plú}viam.

    \item Qui prodúcit in móntibus \textbf{fe}num:\textcolor{red}{~*} et herbam servitú\textit{ti} \textbf{hó}minum.

    \item Qui dat juméntis escam ip\textbf{só}rum:\textcolor{red}{~*} et pullis corvórum invocánti\textit{bus} \textbf{e}um.

    \item Non in fortitúdine equi voluntátem ha\textbf{bé}bit:\textcolor{red}{~*} nec in tíbiis viri beneplácitum e\textit{rit} \textbf{e}i.

    \item Beneplácitum est Dómino super timéntes \textbf{e}um:\textcolor{red}{~*} et in eis, qui sperant super misericórdi\textit{a} \textbf{e}jus.
  \end{enumerate}
  \smallskip
  \grecommentary{\textit{Reprise de l'Antienne.}}
  \gabcsnippet{(f3) O(h)blá(h_)tus(hg) est,(f'_) (,) qui(f)a(e') i(f)pse(h') vó(h)lu(h_2/h_2/i_[oh:h]h)it,(h.) (;) et(hjij) pec(i)cá(hi)ta(h) no(hi)stra(h) (,) i(h_g)pse(e) por(gh)tá(f.)vit.(f.) (::)}

  \begin{multicols}{2}
    \begin{footnotesize}
      \begin{enumerate}[label=\textcolor{red}{\emph{\arabic*}}]
        \item \textit{Louez le Seigneur, car il est bon de le louer ; car il est doux, il est bienséant de le célébrer.}
        \item \textit{C'est lui qui rebâtit Jérusalem, qui rassemble les dispersés d'Israël.}
        \item \textit{Il guérit ceux qui ont le coeur brisé, et il panse leurs blessures.}
        \item \textit{Il sait le nombre des étoiles ; et il les appelle toutes par leur nom.}
        \item \textit{Il est grand, notre Dieu, et sa force est infinie ; et sa sagesse n'a point de bornes.}
        \item \textit{Il relève ceux qui sont doux : il abaisse les méchants jusqu'à terre.}
        \item \textit{Entonnez au Seigneur un cantique de reconnaissance ; louez notre Dieu sur la cithare !}
        \item \textit{C'est lui qui voile le ciel de nuages, et qui prépare la pluie pour la terre ;}
        \item \textit{C'est lui qui fait croître l'herbe sur les montagne : et les plantes pour le service de l'homme ;}
        \item \textit{Qui donne aux animaux leur pâture, aux petits des corbeaux la nourriture qu'ils lui demandent par leurs cris.}
        \item \textit{Ce n'est pas la force du coursier qui lui plaît : ni le guerrier qui se fie à la rapidité de ses pieds.}
        \item \textit{Le Seigneur se complait en ceux qui le craignent : en ceux qui espèrent sa miséricorde.}
      \end{enumerate}
    \end{footnotesize}
  \end{multicols}

  \medskip

  \begin{center}
    \begin{footnotesize}
      \textcolor{red}{\textit{On ne dit ni Capitule ni Hymne.}}
      \textcolor{red}{\textit{On chante le verset debout.}}
    \end{footnotesize}
    \begin{minipage}{1\linewidth}
      \gresetinitiallines{0}
      % \grecommentary[10pt]{\textcolor{red}{\textit{On se lève.}}}
      \gabcsnippet{(c4)<c><v>\Vbar</v>.</c> Hó(h)mo(h) pá(h)cis(h) mé(h)æ(h), in(h) quo(i') spe(h)rá(g.)vi.(g.) (::) (Z) <c><v>\Rbar</v>.</c> Qui(h) edé(h)bat(h) pá(h)nes(h) mé(h)os, amp(h)liá(h)vit(h) ad(h)vér(h)sum(h) me(h) sup(h)plan(h)ta(i')ti(h)ó(g.)nem.(g.)  (::) (Z)}
      \bigskip
      \normalsize
      \begin{center}
        \textit{\textcolor{red}{\Vbar.} L’homme qui vivait en paix avec moi, dans lequel j’ai espéré}\\
        \textit{\textcolor{red}{\Rbar.} Qui mangeait de mon pain, m’a trahi par une horrible perfidie.}
      \end{center}
    \end{minipage}
  \end{center}

  \normalsize

  \medskip
  \begin{center}
    \rule{4cm}{0.4pt}
  \end{center}
  \medskip

  \par Vient ensuite le cantique de Zacharie ; son accent de joie contraste avec les douleurs de la Passion. Cependant c'est à présent que les prophéties qui y sont contenues vont recevoir leur accomplissement ; le Seigneur rachète son peuple, le délivre de ses ennemis, illumine ceux qui sont dans les ombres de la mort, et leur apprend le chemin de la vie éternelle.

  \par \textcolor{red}{Au commencement du cantique \textit{Benedictus}, il ne reste sur le chandelier triangulaire que le seul cierge supérieur allumé. Pendant qu'on dit le cantique, on éteint un à un les six cierges placés sur l'autel (à partir du verset \textit{Ut sine timore}), de telle manière qu'au dernier verset on éteigne le dernier cierge ; on éteint aussi tous les luminaires de l'église.}

  \medskip

  \gresetinitiallines{1}
  \greillumination{\initfamily\fontsize{11mm}{11mm}\selectfont T}
  \gregorioscore{antiennes/an--traditor_autem--solesmes}
  \begin{center}
    \footnotesize{
      \textit{Le traître leur avait donné ce signal, en leur disant : Celui que je baiserai, c’est lui-même ; arrêtez-le.}
  }
  \end{center}

  \newpage

  \begin{center}
    \large{Cantique de Zacharie.}\\
    \small\textit{Luc, I, 68-79.}\\
    \normalsize
  \end{center}

  \gresetinitiallines{0}
  \gregorioscore{psaumes/zacharie-Ig}
  
  \begin{enumerate}[label=\textcolor{red}{\arabic*}]
    \setcounter{enumi}{1}
    \item Et eréxit cornu sa\textbf{lú}tis \textbf{no}bis:\textcolor{red}{~*} in domo David, pú\textit{e}\textit{ri} \textbf{su}i.

    \item Sicut locútus est per \textbf{os} sanc\textbf{tó}rum,\textcolor{red}{~*} qui a sǽculo sunt, prophe\textit{tá}\textit{rum} \textbf{e}jus:

    \item Salútem ex ini\textbf{mí}cis \textbf{nos}tris,\textcolor{red}{~*} et de manu ómnium, \textit{qui} \textit{o}\textbf{dé}runt nos.

    \item Ad faciéndam misericórdiam cum \textbf{pá}tribus \textbf{nos}tris:\textcolor{red}{~*} et memorári testaménti \textit{su}\textit{i} \textbf{sanc}ti.

    \item Jusjurándum, quod jurávit ad Abraham \textbf{pa}trem \textbf{nos}trum,\textcolor{red}{~*} datú\textit{rum} \textit{se} \textbf{no}bis:

    \item Ut sine timóre, de manu inimicórum nostrórum \textbf{li}be\textbf{rá}ti,\textcolor{red}{~*} servi\textit{á}\textit{mus} \textbf{il}li.

    \item In sanctitáte, et justítia \textbf{co}ram \textbf{ip}so,\textcolor{red}{~*} ómnibus di\textit{é}\textit{bus} \textbf{nos}tris.

    \item Et tu, puer, Prophéta Altíssi\textbf{mi} vo\textbf{cá}beris:\textcolor{red}{~*} præíbis enim ante fáciem Dómini, paráre \textit{vi}\textit{as} \textbf{e}jus:

    \item Ad dandam sciéntiam salútis \textbf{ple}bi \textbf{e}jus:\textcolor{red}{~*} in remissiónem peccató\textit{rum} \textit{e}\textbf{ó}rum:

    \item Per víscera misericórdiæ \textbf{De}i \textbf{nos}tri:\textcolor{red}{~*} in quibus visitávit nos, óri\textit{ens} \textit{ex} \textbf{al}to:

    \item Illumináre his, qui in ténebris, et in umbra \textbf{mor}tis \textbf{se}dent:\textcolor{red}{~*} ad dirigéndos pedes nostros in \textit{vi}\textit{am} \textbf{pa}cis.
  \end{enumerate}
  \smallskip
  \grecommentary{\textit{Reprise de l'Antienne.}}
  \gabcsnippet{(c4) Trá(f)di(c')tor(d) au(dh)tem(ixh_i) () de(h')dit(g) e(e')is(f) si(g)gnum,(fe~) di(d.)cens :(d.) (;) Quem(d) o(fe)scu(d)lá(e')tus(f) fú(gh)e(g)ro,(f.) (,) i(f_g)pse(f) est,(c.) (;) te(e)né(g_[oh:h]e)te(f_e) e(d.)um.(d.) (::)}

  \begin{multicols}{2}
    \begin{footnotesize}
      \begin{enumerate}[label=\textcolor{red}{\emph{\arabic*}}]
        \item \textit{Béni soit le Seigneur le Dieu d’Israël ; parce qu’il a visité et racheté son peuple.}
        \item \textit{Et qu’il a suscité un puissant Sauveur, dans la
        maison de son serviteur David,}
        \item \textit{Ainsi qu’il l’avait promis par la bouche de ses
        saints Prophètes, qui ont vécu dans les siècles passés.}
        \item \textit{De nous délivrer de nos ennemis, et des mains de
        tous ceux qui nous haïssent ;}
        \item \textit{En usant de miséricorde envers nos pères, et en se
        souvenant de sa sainte alliance :}
        \item \textit{Suivant la promesse faite avec serment à Abraham notre père, qu’il se donnerait à nous,}
        \item \textit{Afin qu’étant délivrés de la main de nos ennemis,
        nous le servions sans crainte,}
        \item \textit{Dans la sainteté et la justice, nous tenants en sa
        présence tous les jours de notre vie.}
        \item \textit{Et vous petits enfants ; vous serez appelé le Prophète du Très-Haut : vous marcherez devant la face du Seigneur, pour lui préparer ses voies ;}
        \item \textit{En donnant à son peuple la connaissance du salut, pour la rémission de leurs péchés,}
        \item \textit{Par les entrailles de la miséricorde de notre Dieu,
        qui a fait qu’un soleil levant nous a visités d’enhaut,}
        \item \textit{Pour éclairer ceux qui sont dans les ténèbres et
        dans l’ombre de la mort, et pour conduire nos
        pieds dans le chemin de la paix.}
      \end{enumerate}
    \end{footnotesize}
  \end{multicols}

  \medskip

  \begin{center}
    \begin{footnotesize}
      \textcolor{red}{\textit{Après la répétition de l'Antienne à Benedictus, on chante à genoux :}}
    \end{footnotesize}
  \end{center}

  \gresetinitiallines{1}
  \greillumination{\initfamily\fontsize{11mm}{11mm}\selectfont C}
  \gregorioscore{antiennes/an--christus_factus_est--jeudi}
  \normalsize

  \begin{center}
    \textit{Le Christ s’est fait pour nous obéissant jusqu’à la mort.}\\
  \end{center}

  \medskip

  \par \textcolor{red}{Après l'Antienne \textit{Christus factus est}, on dit le \textit{Pater noster} entièrement en silence.}\\
  \medskip
  On ajoute, sans dire \textit{Orémus}, l'oraison suivante :

  \setlength{\columnsep}{2pc}
  \def\columnseprulecolor{\color{red}}
  \setlength{\columnseprule}{0.4pt}

  \begin{multicols}{2}
    \par Réspice, quæsumus Dómine, super
    hanc famíliam tuam, pro qua Dóminus
    noster Jesus Christus non dubitávit
    mánibus tradi nocéntium, et crucis
    subíre torméntum :
    % \par \hspace{\fill}
    \columnbreak
    \par \textit{Nous vous prions, Seigneur, de regarder en pitié votre famille, pour laquelle notre Seigneur JésusChrist n’a point refusé de se livrer entre les mains des méchants, et de souffrir le supplice de la croix ;}
  \end{multicols}
  \setlength\columnseprule{0pt}
  \setlength{\columnsep}{0pc}

  \medskip
  \par On récite ensuite la conclusion : 

  \setlength{\columnsep}{2pc}
  \def\columnseprulecolor{\color{red}}
  \setlength{\columnseprule}{0.4pt}

  \begin{multicols}{2}
    \par Qui tecum vivit et regnat...
    % \par \hspace{\fill}
    \columnbreak
    \par \textit{Lui qui vit et règne avec vous...}
  \end{multicols}
  \setlength\columnseprule{0pt}
  \setlength{\columnsep}{0pc}

  \smallskip
  \par \textcolor{red}{On fait ensuite grand bruit (symbole qui figure le désordre de la nature à la mort du Sauveur, Lumière du monde.). Puis, tous se lèvent et se retirent.}

\end{document}
