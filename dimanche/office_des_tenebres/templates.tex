  % ================ TEMPLATE PSAUMES =========================

  % ===== DEBUT Antienne =========
  \gresetinitiallines{1}
  \greillumination{\initfamily\fontsize{11mm}{11mm}\selectfont Z}
  \gregorioscore{antiennes/an--zelus_domus_tuae--solesmes_1961}
  \begin{center}
    \footnotesize{
      \textit{ }
    }
  \end{center}
  % ===== FIN Antienne ===========

  % ===== DEBUT psaume ===========
  % gresetinitiallines : avec le parametre à 0, supprime l'ornement
  \begin{center}
    \large{Psaume 68.}\\
  \end{center}

  \gresetinitiallines{0}
  \gregorioscore{psaumes/psaume68-VIIIc}

  \begin{enumerate}[label=\textcolor{red}{\arabic*}]
    \setcounter{enumi}{1}
    
  \end{enumerate}
  %  Répetition de l'Antienne
  \grecommentary{\textit{Reprise de l'Antienne.}}
  \gabcsnippet{(c4) A(d)it(f') la(d)tro(dc) ad(f) la(g)tró(f_h)nem :(h'_) (,) Nos(h) qui(h')dem(j) di(h')gna(g) fa(h')ctis(g) re(ge)cí(f_g)pi(h)mus,(g.) (;) hic(h_f) au(g')tem(f) quid(e') fe(f)cit?(d.) (:) Me(d)mén(f')to(d) me(e')i,(f) Dó(g_[uh:l]h)mi(g)ne,(f'_) (,) dum(f) vé(fg)ne(f)ris(c') in(e) re(g_[oh:h]e~)gnum(f_e) tu(d.)um.(d.) (::)}

  \medskip
  \begin{multicols}{2}
    \begin{footnotesize}
      \begin{enumerate}[label=\textcolor{red}{\emph{\arabic*}}]
        \item \textit{}
        \item \textit{}
        \item \textit{}
        \item \textit{}
        \item \textit{}
        \item \textit{}
        \item \textit{}
        \item \textit{}
        \item \textit{}
        \item \textit{}
        \item \textit{}
        \item \textit{}
        \item \textit{} 
        \item \textit{}
        \item \textit{}
        \item \textit{}
        \item \textit{}
        \item \textit{}
        \item \textit{}
        \item \textit{}
        \item \textit{}
      \end{enumerate}
    \end{footnotesize}
  \end{multicols}

  \medskip
  \begin{center}
    \rule{2cm}{0.4pt}
  \end{center}
  \medskip

  \par 
  \medskip
  \begin{center}
    \rule{2cm}{0.4pt}
  \end{center}
  \medskip





  % ============ TEMPLATE LAMENTATION ==============
  \begin{center}
    \large Leçon I.\\
    \normalsize
  \end{center}
  % \grechangedim{baselineskip}{55pt}{scalable}
  \gresetinitiallines{1}
  \greillumination{\initfamily\fontsize{11mm}{11mm}\selectfont I}
  \gregorioscore{lamentations/va--incipit_lamentatio_ieremiae_prophetae--solesmes}

  \begin{multicols}{2}
    \begin{footnotesize}
      \par \emph{}
      \par \textcolor{red}{\textit{Aleph.}} 
      \par \textcolor{red}{\textit{Beth.}}
      \par \textcolor{red}{\textit{Ghimel.}} 
      \par \textcolor{red}{\textit{Daleth.}}
      \par \textcolor{red}{\textit{Hé.}} 
      \par \hspace{\fill}
    \end{footnotesize}
  \end{multicols}


  \greillumination{\initfamily\fontsize{11mm}{11mm}\selectfont I}
  \gregorioscore{repons/re--in_monte_oliveti--solesmes_1961}

  \smallskip

  \small
  \begin{multicols}{2}
    \par\textcolor{red}{\textit{\Rbar}.} \textit{ \\ \textcolor{red}{*}}
    \columnbreak
    \par\textcolor{red}{\textit{\Vbar}.} \textit{\\
    \textcolor{red}{*} }
  \end{multicols}
  \normalsize

  \medskip
  \begin{center}
    \rule{4cm}{0.4pt}
  \end{center}
  \medskip



  % ================ TEMPLATE LEÇON ==================

  \begin{center}
    \large Leçon IV.\\
    \normalsize
  \end{center}
  \medskip

  \setlength{\columnsep}{2pc}
  \def\columnseprulecolor{\color{red}}
  \setlength{\columnseprule}{0.4pt}

  \begin{multicols}{2}
    \begin{center}
      Ex Tractátu sancti Augustíni\\ Epíscopi super Psalmos.
    \end{center}

    \par 
    \par 
    \par 
    \par \hspace{\fill}
    \columnbreak

    \begin{center}
      Du Traité de S. Augustin,\\ Evêque, sur les Psaumes.\\
      \begin{footnotesize}
        \textit{}
      \end{footnotesize}
    \end{center}
    \par \textit{}
    \par \textit{}
    \par \textit{}
  \end{multicols}
  \setlength\columnseprule{0pt}

  \medskip

  \gresetinitiallines{1}
  \greillumination{\initfamily\fontsize{11mm}{11mm}\selectfont A}
  \gregorioscore{repons/re--amicus_meus--solesmes_1961}

  \small
  \begin{multicols}{2}
    \par\textcolor{red}{\textit{\Rbar}.} \textit{ \\ \textcolor{red}{*} }
    \columnbreak
    \par\textcolor{red}{\textit{\Vbar}.} \textit{\\
    \textcolor{red}{*} }
  \end{multicols}
  \normalsize

  \begin{center}
    \rule{4cm}{0.4pt}
  \end{center}




  % ======================== TEMPLATE VERSET =======================
  \begin{center}
    \rule{4cm}{0.4pt}
  \end{center}

  \begin{center}
    \begin{footnotesize}
      \textcolor{red}{\textit{On chante le verset debout.}}
    \end{footnotesize}
    \begin{minipage}{0.8\linewidth}
      \gresetinitiallines{0}
      \large
      \gabcsnippet{(c4)<c><v>\Vbar</v>.</c> A(h)ver(h)tán(h)tur(h) re(h)tró(h)rsum(h), et(h) e(i')ru(h)bés(g.)cant(g.) (::) <c><v>\Rbar</v>.</c> Qui(h) có(h)gi(h)tant(h) mi(i')hi(h) má(g.)la(g.) (::)}
      \bigskip
      \normalsize
      \begin{center}
        \textit{\textcolor{red}{\Vbar.} Que tous ceux qui me veulent du mal soient repoussés en arrière.}\\
        \textit{\textcolor{red}{\Rbar.} Et couverts de confusion.}
      \end{center}
    \end{minipage}
  \end{center}
  \normalsize
  \begin{center}
    \rule{4cm}{0.4pt}
  \end{center}



% ================== TEMPLATE CLOTURE ======================

\begin{center}
  \begin{footnotesize}
    \textcolor{red}{\textit{Après la répétition de l'Antienne à Benedictus, on chante à genoux :}}
  \end{footnotesize}
\end{center}

\gresetinitiallines{1}
\greillumination{\initfamily\fontsize{11mm}{11mm}\selectfont C}
\gregorioscore{antiennes/an--christus_factus_est--jeudi}
\normalsize

\begin{center}
  \textit{Le Christ s’est fait pour nous obéissant jusqu’à la mort.}\\
\end{center}

\medskip
\begin{center}
  \rule{4cm}{0.4pt}
\end{center}
\medskip

\par \textcolor{red}{Après l'Antienne \textit{Christus factus est}, on dit le \textit{Pater noster} entièrement en silence.}\\
\medskip
On ajoute, sans dire \textit{Orémus}, l'oraison suivante :

\setlength{\columnsep}{2pc}
\def\columnseprulecolor{\color{red}}
\setlength{\columnseprule}{0.4pt}

\begin{multicols}{2}
  \par Réspice, quæsumus Dómine, super
  hanc famíliam tuam, pro qua Dóminus
  noster Jesus Christus non dubitávit
  mánibus tradi nocéntium, et crucis
  subíre torméntum :
  % \par \hspace{\fill}
  \columnbreak
  \par \textit{Nous vous prions, Seigneur, de regarder en pitié votre famille, pour laquelle notre Seigneur JésusChrist n’a point refusé de se livrer entre les mains des méchants, et de souffrir le supplice de la croix ;}
\end{multicols}
\setlength\columnseprule{0pt}
\setlength{\columnsep}{0pc}

\medskip
\par On récite ensuite la conclusion : 

\setlength{\columnsep}{2pc}
\def\columnseprulecolor{\color{red}}
\setlength{\columnseprule}{0.4pt}

\begin{multicols}{2}
  \par Qui tecum vivit et regnat...
  % \par \hspace{\fill}
  \columnbreak
  \par \textit{Lui qui vit et règne avec vous...}
\end{multicols}
\setlength\columnseprule{0pt}
\setlength{\columnsep}{0pc}

\smallskip
\par \textcolor{red}{On fait ensuite grand bruit (symbole qui figure le désordre de la nature à la mort du Sauveur, Lumière du monde.). Puis, tous se lèvent et se retirent.}




