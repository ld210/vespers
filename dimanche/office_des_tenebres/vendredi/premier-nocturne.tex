% !TeX program = lualatex
\documentclass[12pt, a4paper]{article}
\usepackage{fullpage}
\usepackage{subfiles}
\usepackage{fontspec}
\usepackage{libertine}
\usepackage{xcolor}
\usepackage{GotIn}
\usepackage{geometry}
\usepackage{multicol}
\usepackage{multicolrule}
\usepackage{graphicx}
\usepackage{enumitem}
\usepackage[autocompile]{gregoriotex}

\geometry{top=1cm, bottom=1cm, right=1cm, left=1cm}
\pagestyle{empty}

\definecolor{red}{HTML}{C70039}
% \input GoudyIn.fd
% \newcommand*\initfamily{\usefont{U}{GoudyIn}{xl}{n}}

\input Acorn.fd
\newcommand*\initfamily{\usefont{U}{Acorn}{xl}{n}}
% cette ligne ajoute de l'espace entre les portées
% \grechangedim{baselineskip}{60pt}{scalable}

\begin{document}
  \gresetlinecolor{gregoriocolor}

  \begin{center}
    \large À MATINES.\\
  \end{center}
  \medskip

  \begin{center}
    \large AU PREMIER NOCTURNE.\\
  \end{center}
  \medskip
  \par Le premier Psaume nous montre les complots des méchants qui ont amené la mort de l'Homme-Dieu. Mais leur triomphe sera court, car le Seigneur s'apprète à établir son divin Fils roi de Sion et à lui donner toutes les nations en héritage.

  \medskip
  \begin{center}
    \rule{2cm}{0.4pt}
  \end{center}

  % ================ TEMPLATE PSAUMES =========================
  % ===== DEBUT Antienne =========
  \gresetinitiallines{1}
  \greillumination{\initfamily\fontsize{11mm}{11mm}\selectfont A}
  \gregorioscore{antiennes/an--astiterunt_reges--solesmes_1961}
  \begin{center}
    \footnotesize{
      \textit{Les rois de la terre se sont levés, les princes se sont ligués ensemble contre le Seigneur et contre son Christ.}
    }
  \end{center}
  % ===== FIN Antienne ===========

  % ===== DEBUT psaume ===========
  % gresetinitiallines : avec le parametre à 0, supprime l'ornement
  \begin{center}
    \large{Psaume 2.}\\
  \end{center}

  \gresetinitiallines{0}
  \gregorioscore{psaumes/psaume2-VIIIG}

  \begin{enumerate}[label=\textcolor{red}{\arabic*}]
    \setcounter{enumi}{1}
    \item Astitérunt reges terræ, et príncipes convenérunt in \textbf{u}num\textcolor{red}{~*} advérsus Dóminum, et advérsus \textit{Chris}\textit{tum} \textbf{e}jus.

    \item Dirumpámus víncula e\textbf{ó}rum:\textcolor{red}{~*} et projiciámus a nobis ju\textit{gum} \textit{ip}\textbf{só}rum.

    \item Qui hábitat in cælis, irridébit \textbf{e}os:\textcolor{red}{~*} et Dóminus subsan\textit{ná}\textit{bit} \textbf{e}os.

    \item Tunc loquétur ad eos in ira \textbf{su}a,\textcolor{red}{~*} et in furóre suo contur\textit{bá}\textit{bit} \textbf{e}os.

    \item Ego autem constitútus sum Rex ab eo super Sion montem sanctum \textbf{e}jus,\textcolor{red}{~*} prǽdicans præ\textit{cép}\textit{tum} \textbf{e}jus.

    \item Dóminus dixit \textbf{ad} me:\textcolor{red}{~*} Fílius meus es tu, ego hódie \textit{gé}\textit{nu}\textbf{i} te.

    \item Póstula a me, et dabo tibi Gentes hereditátem \textbf{tu}am,\textcolor{red}{~*} et possessiónem tuam tér\textit{mi}\textit{nos} \textbf{ter}ræ.

    \item Reges eos in virga \textbf{fér}rea,\textcolor{red}{~*} et tamquam vas fíguli con\textit{frín}\textit{ges} \textbf{e}os.

    \item Et nunc, reges, intel\textbf{lí}gite:\textcolor{red}{~*} erudímini, qui judi\textit{cá}\textit{tis} \textbf{ter}ram.

    \item Servíte Dómino in ti\textbf{mó}re:\textcolor{red}{~*} et exsultáte ei \textit{cum} \textit{tre}\textbf{mó}re.

    \item Apprehéndite disciplínam, nequándo irascátur \textbf{Dó}minus,\textcolor{red}{~*} et pereátis de \textit{vi}\textit{a} \textbf{jus}ta.

    \item Cum exárserit in brevi ira \textbf{e}jus:\textcolor{red}{~*} beáti omnes qui confí\textit{dunt} \textit{in} \textbf{e}o.
  \end{enumerate}

  % \gresetheadercapture{commentary}{grecommentary}{}
  % \gregorioscore{antiennes/an--astiterunt_reges--solesmes_1961}
  % \gresetheadercapture{commentary}{}{}

  \medskip
  \begin{multicols}{2}
    \begin{footnotesize}
      \begin{enumerate}[label=\textcolor{red}{\emph{\arabic*}}]
        \item \textit{Pourquoi ces nations qui remuent, ces peuples qui murmurent en vain?}
        \item \textit{Des rois de la terre s’insurgent, des princes conspirent contre Yahvé et contre son Messie}
        \item \textit{«Faisons sauter leurs entraves, débarrassons-nous de leurs liens!»}
        \item \textit{Celui qui siège dans les cieux s’en amuse, Yahvé les tourne en dérision.}
        \item \textit{Puis dans sa colère il leur parle, dans sa fureur il les épouvante}
        \item \textit{«C’est moi qui ai sacré mon roi sur Sion, ma montagne sainte.»}
        \item \textit{J’énoncerai le décret de Yahvé Il m’a dit : «Tu es mon fils, moi, aujourd’hui, je t’ai engendré.}
        \item \textit{Demande, et je te donne les nations pour héritage, pour domaine les extrémités de la terre;}
        \item \textit{tu les briseras avec un sceptre de fer, comme vases de potier tu les casseras.»}
        \item \textit{Et maintenant, rois, comprenez, corrigez-vous, juges de la terre!}
        \item \textit{Servez Yahvé avec crainte, et réjouissez vous en lui avec tremblements.}
        \item \textit{Attachez-vous à sa loi, de peur que le Seigneur ne s'irrite et que vous ne périssiez hors du droit chemin,}
        \item \textit{Lorsque s'allumera sa colère. Heureux tous ceux qui mettent en lui leur confiance !} 
      \end{enumerate}
    \end{footnotesize}
  \end{multicols}

  % ===== FIN psaume ===========

  \medskip
  \begin{center}
    \rule{2cm}{0.4pt}
  \end{center}
  \medskip

  \par Le Psaume 21 est une prophétie frappante de la Passion du Sauveur. Dans la première partie, on voit décrites les souffrances de son âme et de son corps, dans la deuxième sa résurection et la converison des peuples. Le premier verset contient une des paroles que prononça Jésus-Christ sur la croix, et il est à croire qu'il acheva à voix basse le Psaume tout entier.

  \begin{center}
    \rule{2cm}{0.4pt}
  \end{center}
  
  % ===== DEBUT Antienne =========
  \gresetinitiallines{1}
  \greillumination{\initfamily\fontsize{11mm}{11mm}\selectfont D}
  \gregorioscore{antiennes/an--diviserunt_sibi--solesmes_1961}
  \begin{center}
    \footnotesize{
      \textit{Ils se partagent mes vêtements et ils tirent au sort ma tunique.}
    }
  \end{center}
  % ===== FIN Antienne ===========

  % ===== DEBUT psaume ===========
  % gresetinitiallines : avec le parametre à 0, supprime l'ornement
  \begin{center}
    \large{Psaume 21.}\\
  \end{center}

  \gresetinitiallines{0}
  \gregorioscore{psaumes/psaume21-VIIIG}

  \begin{enumerate}[label=\textcolor{red}{\arabic*}]
    \setcounter{enumi}{1}
    \item Deus meus, clamábo per diem, et non ex\textbf{áu}dies:\textcolor{red}{~*} et nocte, et non ad insipién\textit{ti}\textit{am} \textbf{mi}hi.

    \item Tu autem in sancto \textbf{há}bitas:\textcolor{red}{~*} \textit{laus} \textbf{Is}raël.

    \item In te speravérunt patres \textbf{nos}tri:\textcolor{red}{~*} speravérunt, et libe\textit{rás}\textit{ti} \textbf{e}os.

    \item Ad te clamavérunt, et salvi \textbf{fac}ti sunt:\textcolor{red}{~*} in te speravérunt, et non \textit{sunt} \textit{con}\textbf{fú}si.

    \item Ego autem sum vermis, et non \textbf{ho}mo:\textcolor{red}{~*} oppróbrium hóminum, et abjéc\textit{ti}\textit{o} \textbf{ple}bis.

    \item Omnes vidéntes me deri\textbf{sé}runt me:\textcolor{red}{~*} locúti sunt lábiis, et mo\textit{vé}\textit{runt} \textbf{ca}put.

    \item Sperávit in Dómino, erípiat \textbf{e}um:\textcolor{red}{~*} salvum fáciat eum, quóni\textit{am} \textit{vult} \textbf{e}um.

    \item Quóniam tu es, qui extraxísti me de \textbf{ven}tre:\textcolor{red}{~*}\\ \-\hspace{2cm} spes mea ab ubéribus matris meæ. In te projéctus \textit{sum} \textit{ex} \textbf{ú}tero.

    \item De ventre matris meæ Deus meus \textbf{es} tu,\textcolor{red}{~*} ne discés\textit{se}\textit{ris} \textbf{a} me:

    \item Quóniam tribulátio próxi\textbf{ma} est:\textcolor{red}{~*} quóniam non \textit{est} \textit{qui} \textbf{ád}juvet.

    \item Circumdedérunt me vítuli \textbf{mul}ti:\textcolor{red}{~*} tauri pingues \textit{ob}\textit{se}\textbf{dé}runt me.

    \item Aperuérunt super me os \textbf{su}um:\textcolor{red}{~*} sicut leo rápi\textit{ens} \textit{et} \textbf{rú}giens.

    \item Sicut aqua ef\textbf{fú}sus sum:\textcolor{red}{~*} et dispérsa sunt ómnia \textit{os}\textit{sa} \textbf{me}a.

    \item Factum est cor meum tamquam cera li\textbf{qué}scens:\textcolor{red}{~*} in médio \textit{ven}\textit{tris} \textbf{me}i.

    \item Aruit tamquam testa virtus mea,\textcolor{red}{~†} et lingua mea adhǽsit fáucibus \textbf{me}is:\textcolor{red}{~*}\\ \-\hspace{2cm} et in púlverem mortis \textit{de}\textit{du}\textbf{xís}ti me.

    \item Quóniam circumdedérunt me canes \textbf{mul}ti:\textcolor{red}{~*} concílium malignánti\textit{um} \textit{ob}\textbf{sé}dit me.

    \item Fodérunt manus meas et pedes \textbf{me}os:\textcolor{red}{~*} dinumeravérunt ómnia \textit{os}\textit{sa} \textbf{me}a.

    \item Ipsi vero consideravérunt et inspe\textbf{xé}runt me:\textcolor{red}{~*}\\ \-\hspace{2cm} divisérunt sibi vestiménta mea, et super vestem meam mi\textit{sé}\textit{runt} \textbf{sor}tem.

    \item Tu autem, Dómine, ne elongáveris auxílium tuum \textbf{a} me:\textcolor{red}{~*} ad defensiónem \textit{me}\textit{am} \textbf{cón}spice.

    \item Erue a frámea, Deus, ánimam \textbf{me}am:\textcolor{red}{~*} et de manu canis ú\textit{ni}\textit{cam} \textbf{me}am.

    \item Salva me ex ore le\textbf{ó}nis:\textcolor{red}{~*} et a córnibus unicórnium humili\textit{tá}\textit{tem} \textbf{me}am.

    \item Narrábo nomen tuum frátribus \textbf{me}is:\textcolor{red}{~*} in médio Ecclési\textit{æ} \textit{lau}\textbf{dá}bo te.

    \item Qui timétis Dóminum, laudáte \textbf{e}um:\textcolor{red}{~*} univérsum semen Jacob, glorifi\textit{cá}\textit{te} \textbf{e}um.

    \item Tímeat eum omne semen \textbf{Is}raël:\textcolor{red}{~*} quóniam non sprevit, neque despéxit deprecati\textit{ó}\textit{nem} \textbf{páu}peris:

    \item Nec avértit fáciem suam \textbf{a} me:\textcolor{red}{~*} et cum clamárem ad eum, \textit{ex}\textit{au}\textbf{dí}vit me.

    \item Apud te laus mea in ecclésia \textbf{ma}gna:\textcolor{red}{~*} vota mea reddam in conspéctu timén\textit{ti}\textit{um} \textbf{e}um.

    \item Edent páuperes, et saturabúntur:\textcolor{red}{~†} et laudábunt Dóminum qui requírunt \textbf{e}um:\textcolor{red}{~*}\\ \-\hspace{2cm} vivent corda eórum in sǽ\textit{cu}\textit{lum} \textbf{sǽ}culi.

    \item Reminiscéntur et converténtur ad \textbf{Dó}minum\textcolor{red}{~*} univérsi \textit{fi}\textit{nes} \textbf{ter}ræ:

    \item Et adorábunt in conspéctu \textbf{e}jus\textcolor{red}{~*} univérsæ famí\textit{li}\textit{æ} \textbf{Gén}tium.

    \item Quóniam Dómini est \textbf{re}gnum:\textcolor{red}{~*} et ipse dominá\textit{bi}\textit{tur} \textbf{Gén}tium.

    \item Manducavérunt et adoravérunt omnes pingues \textbf{ter}ræ:\textcolor{red}{~*}\\ \-\hspace{2cm} in conspéctu ejus cadent omnes qui descén\textit{dunt} \textit{in} \textbf{ter}ram.

    \item Et ánima mea illi \textbf{vi}vet:\textcolor{red}{~*} et semen meum sér\textit{vi}\textit{et} \textbf{ip}si.

    \item Annuntiábitur Dómino generátio ventúra :\textcolor{red}{~†}\\ \-\hspace{2cm} et annuntiábunt cæli justítiam ejus pópulo qui na\textbf{scé}tur,\textcolor{red}{~*} quem \textit{fe}\textit{cit} \textbf{Dó}minus.
  \end{enumerate}
  %  Répetition de l'Antienne
  \gresetheadercapture{commentary}{grecommentary}{}
  \gregorioscore{antiennes/an--diviserunt_sibi--solesmes_1961}
  \gresetheadercapture{commentary}{}{}

  \medskip
  \begin{multicols}{2}
    \begin{footnotesize}
      \begin{enumerate}[label=\textcolor{red}{\emph{\arabic*}}]
        \item \textit{Mon Dieu, mon Dieu, tournez vers moi votre regard : pourquoi m'avez-vous abandonné ? :a voix de mes péchés éloigne de moi le salut.}
        \item \textit{Mon Dieu, je crie pendant le jour, et vous ne m'exaucez pas; la nuit, et je n'obtiens pas de soulagement.}
        \item \textit{Pourtant vous habitez dans votre sanctuaire, et vers vous montent les louanges d'Israël.}
        \item \textit{Nos pères ont espéré en vous ; ils ont espéré, et vous les avez délivrés.}
        \item \textit{Ils ont crié vers vous, et ils ont été sauvés ; ils ont mis en vous leur confiance, et ils n'ont pas été confondus.}
        \item \textit{Et moi, je suis un ver, et non un homme, l'opprobre des hommes et le rebus du peuple.}
        \item \textit{Tous ceux qui me voient se moquent de moi ; ils ouvrent les lèvres et branlent la tête en disant :}
        \item \textit{"Il a mis sa confiance dans le Seigneur; qu'il le sauve, puisqu'il l'aime !"}
        \item \textit{Oui, c'est vous qui m'avez tiré du sein maternel; vous étiez mon espérance lorsque j'étais encore à la mamelle. À ma naissance, j'ai été porté sur vos genoux;}
        \item \textit{Depuis le sein de ma mère, c'est vous qui êtes mon Dieu. Ne vous éloignez pas de moi,}
        \item \textit{Car l'angoisse est proche, et personne ne vient à mon secours.}
        \item \textit{Autour de moi sont des taureaux nombreux; de gras taureaux m'environnent.}
        \item \textit{Ils ouvrent contre moi leur bouche, comme un lion qui déchire et rugit.}
        \item \textit{Je suis comme l'eau qui s'écoule, et tous mes os sont disjoints;}
        \item \textit{Mon cœur est comme de la cire, il se fond dans mes entrailles.}
        \item \textit{Ma force s'est desséchée comme un tesson d'argile, et ma langue s'attache à mon palais; vous me réduisez à la poussière du tombeau.}
        \item \textit{Car des chiens nombreux m'environnent; une troupe de scélérats m'assiège;}
        \item \textit{Ils ont percé mes pieds et mes mains. On pourrait compter tous mes os;}
        \item \textit{Eux, ils m'observent et me contemplent. Ils se partagent mes vêtements, ils tirent au sort ma tunique.}
        \item \textit{Vous, Seigneur, n'éloignez pas de moi votre secours, prenez soin de ma défense.}
        \item \textit{Délivrez, Seigneur, mon âme de l'épée, ma vie du pouvoir du chien !}
        \item \textit{Sauvez-moi de la gueule du lion; sauvez ma faiblesse des cornes du buffle !}
        \item \textit{J'annocerai votre nom à mes frères; au milieu de l'assemblée je vous louerai :}
        \item \textit{"Vous qui craignez le Seigneur, louez-le ! Vous tous, postérité de Jacob, glorifiez-le "}
        \item \textit{Que toute la race d'Israël le révère ! Car il n'a pas méprisé, il n'a pas dédaigné la prière du pauvre,}
        \item \textit{Il n'a pas détourné de lui son visage, et quand j'ai crié vers lui, il m'a exaucé."}
        \item \textit{Grâce à vous, mon hymne retentira dans la grande assemblée; j'acquitterai mes voeux en présence de ceux qui vous craignent.}
        \item \textit{Les pauvres mangeront et se rassasieront; ceux qui cherchent le Seigneur chanteront ses louanges; leur âme vivra éteernellement.}
        \item \textit{Les extrémités de la terre se souviendront et reviendront au Seigneur;}
        \item \textit{Toutes les familles des nations se prosterneront en sa présence.}
        \item \textit{Car au Seigneur appartient l'empire, il domine sur les nations.}
        \item \textit{Les puissants de la terre mangeront et adoreront; devant lui tomberons à genoux tous ceux qui descendent à la poussière.}
        \item \textit{Mon âme vivra pour sa gloire, et ma postérité le servira.}
        \item \textit{La génération future sera appelée le peuple du Seigneur; les cieux annonceront sa justice à ce peuple qui doit naître et que le Seigneur prépare.}
      \end{enumerate}
    \end{footnotesize}
  \end{multicols}

  % ===== FIN psaume ===========

  \bigskip

  \begin{center}
    \rule{2cm}{0.4pt}
  \end{center}

  \par Le Psaume 26 nous montre le Christ adressant une prière pleine de confiance à son Père du milieu de ses souffrances. Malgré les ennemis qui l'entourent et les douleurs qui le consumment, il ne craint rien, il est plein de force, car comme Dieu il ne cesse point de jouir de la pleine lumière de la vision béatifique.

  \begin{center}
    \rule{2cm}{0.4pt}
  \end{center}

  % ===== DEBUT Antienne =========
  \gresetinitiallines{1}
  \greillumination{\initfamily\fontsize{11mm}{11mm}\selectfont I}
  \gregorioscore{antiennes/an--insurrexerunt_in_me--solesmes_1961}
  \begin{center}
    \footnotesize{
      \textit{Des témoins iniques se sont élevés contre moi, et l'iniquité a menti contre elle-même.}
    }
  \end{center}
  % ===== FIN Antienne ===========

  % ===== DEBUT psaume ===========
  % gresetinitiallines : avec le parametre à 0, supprime l'ornement
  \begin{center}
    \large{Psaume 26.}\\
  \end{center}

  \gresetinitiallines{0}
  \gregorioscore{psaumes/psaume26-VIIIG}

  \begin{enumerate}[label=\textcolor{red}{\arabic*}]
    \setcounter{enumi}{1}
    \item Dóminus protéctor vitæ \textbf{me}æ,\textcolor{red}{~*} a quo \textit{tre}\textit{pi}\textbf{dá}bo?

    \item Dum apprópiant super me no\textbf{cén}tes,\textcolor{red}{~*} ut edant \textit{car}\textit{nes} \textbf{me}as:

    \item Qui tríbulant me inimíci \textbf{me}i,\textcolor{red}{~*} ipsi infirmáti sunt et \textit{ce}\textit{ci}\textbf{dé}runt.

    \item Si consístant advérsum me \textbf{cas}tra,\textcolor{red}{~*} non timé\textit{bit} \textit{cor} \textbf{me}um.

    \item Si exsúrgat advérsum me \textbf{prǽ}lium,\textcolor{red}{~*} in hoc e\textit{go} \textit{spe}\textbf{rá}bo.

    \item Unam pétii a Dómino, hanc re\textbf{quí}ram,\textcolor{red}{~*} ut inhábitem in domo Dómini ómnibus diébus \textit{vi}\textit{tæ} \textbf{me}æ:

    \item Ut vídeam voluptátem \textbf{Dó}mini,\textcolor{red}{~*} et vísitem \textit{tem}\textit{plum} \textbf{e}jus.

    \item Quóniam abscóndit me in tabernáculo \textbf{su}o:\textcolor{red}{~*} in die malórum protéxit me in abscóndito taberná\textit{cu}\textit{li} \textbf{su}i.

    \item In petra exal\textbf{tá}vit me:\textcolor{red}{~*} et nunc exaltávit caput meum super ini\textit{mí}\textit{cos} \textbf{me}os.

    \item Circuívi et immolávi in tabernáculo ejus hóstiam vociferati\textbf{ó}nis:\textcolor{red}{~*} cantábo et psalmum \textit{di}\textit{cam} \textbf{Dó}mino.

    \item Exáudi, Dómine, vocem meam, qua clamávi \textbf{ad} te:\textcolor{red}{~*} miserére mei, \textit{et} \textit{ex}\textbf{áu}di me.

    \item Tibi dixit cor meum, exquisívit te fácies \textbf{me}a:\textcolor{red}{~*} fáciem tuam, Dómi\textit{ne}, \textit{re}\textbf{quí}ram.

    \item Ne avértas fáciem tuam \textbf{a} me,\textcolor{red}{~*} ne declínes in ira a \textit{ser}\textit{vo} \textbf{tu}o.

    \item Adjútor meus \textbf{es}to:\textcolor{red}{~*} ne derelínquas me, neque despícias me, Deus, salu\textit{tá}\textit{ris} \textbf{me}us.

    \item Quóniam pater meus, et mater mea dereli\textbf{qué}runt me:\textcolor{red}{~*} Dóminus au\textit{tem} \textit{as}\textbf{súmp}sit me.

    \item Legem pone mihi, Dómine, in via \textbf{tu}a:\textcolor{red}{~*} et dírige me in sémitam rectam propter ini\textit{mí}\textit{cos} \textbf{me}os.

    \item Ne tradíderis me in ánimas tribulánti\textbf{um} me:\textcolor{red}{~*} quóniam insurrexérunt in me testes iníqui, et mentíta est iní\textit{qui}\textit{tas} \textbf{si}bi.

    \item Credo vidére bona \textbf{Dó}mini\textcolor{red}{~*} in ter\textit{ra} \textit{vi}\textbf{vén}tium.

    \item Exspécta Dóminum, viríliter \textbf{a}ge:\textcolor{red}{~*} et confortétur cor tuum, et sús\textit{ti}\textit{ne} \textbf{Dó}minum.
  \end{enumerate}
  %  Répetition de l'Antienne
  \gresetheadercapture{commentary}{grecommentary}{}
  \gregorioscore{antiennes/an--insurrexerunt_in_me--solesmes_1961}
  \gresetheadercapture{commentary}{}{}

  \medskip
  \begin{multicols}{2}
    \begin{footnotesize}
      \begin{enumerate}[label=\textcolor{red}{\emph{\arabic*}}]
        \item \textit{Le Seigneur est ma lumière et mon salut : qui craindrai-je ?}
        \item \textit{Le Seigneur est le défenseur de ma vie : de qui aurai-je peur ?}
        \item \textit{Quand des méchants se sont avancés contre moi pour me dévorer,}
        \item \textit{Ces persécuteurs, ces ennemis ont chancelé et sont tombés.}
        \item \textit{Qu'une armée vienne camper contre moi, mon cœur ne craindra point;}
        \item \textit{Que contre moi s'élève le combat, alors même j'aurai confiance.}
        \item \textit{Je demande au Seigneur une chose : je la désire ardemment : je voudrai habiter dans la maison du Seigneur tous les jours de ma vie,}
        \item \textit{Pour jouir des amabilités du Seigneur et visiter son sanctuaire.}
        \item \textit{Car il m'abritera dans sa demeure; au jour de l'adversité, il me cachera dans le secret de sa tente,}
        \item \textit{Et j'y serai en sureté, comme sur un rocher inaccessible. Alors il élèvera ma tête au-dessus de mes ennemis;}
        \item \textit{J'entourerai son autel et j'offrirai un sacrifice d'actions de grâces; je chanterai et je dirai des hymnes au Seigneur.}
        \item \textit{Écoutez, Seigneur, ma voix qui vous invoque; ayez pitié de moi et exaucez-moi !}
        \item \textit{Mon cœur vous a parlé et mes yeux vous ont cherché; toujours, Seigneur, je chercherai votre visage.} 
        \item \textit{Ne détournez pas de moi votre visage; ne vous retirez pas, dans votre colère, de votre serviteur.}
        \item \textit{Soyez mon secours, ne me délaissez pas et ne me dédaignez pas, ô mon Dieu et mon Sauveur !}
        \item \textit{Car mon père et ma mère m'ont abandonné, mais le Seigneur me recueillera.}
        \item \textit{Seigneur, enseignez-moi votre voie, et dirigez-moi dans le droit sentier, à cause de mes ennemis.}
        \item \textit{Ne me livrez pas à la fureur de ceux qui me persécutent; car des témoins iniques s'élèvent contre moi; mais l'iniquité a menti contre elle-même.}
        \item \textit{Je suis assuré de voir les biens du Seigneur dans la terre des vivants.}
        \item \textit{Wspère au Seigneur ! Aie courage et que ton cœur soit ferme ! Wspère au Seigneur.}
        \item \textit{}
      \end{enumerate}
    \end{footnotesize}
  \end{multicols}

  % ===== FIN psaume ===========

  \medskip
  \begin{center}
    \rule{2cm}{0.4pt}
  \end{center}
  \medskip

  \begin{center}
    \begin{footnotesize}
      \textcolor{red}{\textit{On chante le verset debout.}}\\
    \end{footnotesize}
    \begin{minipage}{0.5\linewidth}
      \gresetinitiallines{0}
      \large
      \gabcsnippet{(c4)<c><v>\Vbar</v>.</c> Di(h)vi(h)sé(h)runt(h) sí(h)bi(h) ves(h)ti(h)mén(i')ta(h) mé(g.)a(g.) (::) (Z) <c><v>\Rbar</v>.</c> Et(h) su(h)per(h) vés(h)tem(h) mé(h)am(h) mi(h)sé(i')runt(h) sór(g.)tem(g.) (::)}
      \bigskip
      \normalsize
      \begin{center}
        \textit{\textcolor{red}{\Vbar.} Ils se partagent mes vêtements.}\\
        \textit{\textcolor{red}{\Rbar.} Et ils tirent au sort ma tunique.}
      \end{center}
    \end{minipage}
  \end{center}
  \normalsize
  \medskip
  \begin{center}
    \rule{4cm}{0.4pt}
  \end{center}
  \medskip

  \par Les deux premieres leçons de ce Nocturne décrivent la ruine de Jérusalem, comme à l'Office d'hier. Dans la troisième, le Prophète, image du Messie, décrit les malheurs qui l'ont frappés lui-même.

  \medskip
  \begin{center}
    \rule{4cm}{0.4pt}
  \end{center}
  \medskip

  \begin{center}
    \large Leçon I.\\
    \normalsize
  \end{center}
  % \grechangedim{baselineskip}{55pt}{scalable}
  \gresetinitiallines{1}
  \greillumination{\initfamily\fontsize{11mm}{11mm}\selectfont D}
  \gregorioscore{lamentations/va--de_lamentatione...heth--solesmes.gabc}

  \begin{multicols}{2}
    \begin{footnotesize}
      \par \emph{Des Lamentation du prophète Jérémie, chap. 2, 8-15 ; 3, 1-9}
      \par \textcolor{red}{\textit{Heth.}} Le Seigneur a résolu de renverser les murailles de la fille de Sion. Il a tendu le cordeau, et il n'a point retiré sa main que tout ne fût renversé. Le boulevard s'est écroulé, et la muraille a été pareillement détruite. 
      \par \textcolor{red}{\textit{Teth.}} Ses portes sont enfoncées en terre; il en a rompu et broyé les gonds; son roi et ses princes sont parmis les nations. Il n'y a plus de loi; et ses prophète ne reçoivent plus de visions du Seigneur.
      \par \textcolor{red}{\textit{Jod.}}  Les vieillards de la fille de Sion sont assis par terre en silence; ils ont couvert leur tête de cendres; ils sont vêtus de cilices; les vierges de Jérusalem inclinent leur tête vers la terre.
      \par \textcolor{red}{\textit{Caph.}} Mes yeux se consumment dans les larmes : mes entrailles sont émues : mon cœur a défailli à la vue des malheurs de la fille de mon peuple, en voyant les petits enfants et les nourrissons tomber en défaillance sur les places de la ville.
      \par Jérusalem, Jérusalem, convertis-toi au Seigneur ton Dieu. 
      \par \hspace{\fill}
    \end{footnotesize}
  \end{multicols}

  \medskip
  \begin{center}
    \rule{4cm}{0.4pt}
  \end{center}
  \medskip


  \greillumination{\initfamily\fontsize{11mm}{11mm}\selectfont O}
  \gregorioscore{repons/re--omnes_amici_mei--solesmes_1961}

  \smallskip

  \small
  \begin{multicols}{2}
    \par\textcolor{red}{\textit{\Rbar}.} \textit{Tous mes amis m'ont abandonnés; mes persécuteurs ont pris le dessus; celui j'aimais m'a trahi : \\ \textcolor{red}{*} Et les yeux chargés de haine, après m'avoir cruellement couvert de plaies, ils m'ont donné du vinaigre à boire.}
    \columnbreak
    \par\textcolor{red}{\textit{\Vbar}.} \textit{Ils m'ont mis au rang des méchants; et ils n'ont point épargné ma vie : \\ \textcolor{red}{*} Et les yeux chargés de haine, après m'avoir cruellement couvert de plaies, ils m'ont donné du vinaigre à boire.}
  \end{multicols}
  \normalsize

  \medskip
  \begin{center}
    \rule{4cm}{0.4pt}
  \end{center}
  \medskip

  \begin{center}
    \large Leçon II.\\
    \normalsize
  \end{center}
  % \grechangedim{baselineskip}{55pt}{scalable}
  \gresetinitiallines{1}
  \greillumination{\initfamily\fontsize{11mm}{11mm}\selectfont L}
  \gregorioscore{lamentations/va--lamed_matribus_suis_dixerunt--solesmes}

  \begin{multicols}{2}
    \begin{footnotesize}
      \par \textcolor{red}{\textit{Lamed.}} Ils disaient à leur mère : où y a-t-il du pain et du vin ? Et ils tombaient sur les places de la ville comme blessés à mort, et ils expiraient entre les bras de leurs mères.
      \par \textcolor{red}{\textit{Mem.}} À qui te comparer ? À qui ressembles-tu, fille de Jérusalem ? Où trouver quelque chose d'égal à tes maux ? Et comment te consoler, ô vierge fille de Sion ? Ta blessure est large comme la mer : qui te guérirait ?
      
      \par \hspace{\fill}
    \end{footnotesize}
    \columnbreak
    \begin{footnotesize}
      \par \textcolor{red}{\textit{Nun.}} Tes prophètes ont eu pour toi de vaines et folles visions, ils ne découvraient point ton iniquité, pour te porter à la pénitence; mais ils t'ont donné pour vision des oracles de mensonge et de bannissement. 
      \par \textcolor{red}{\textit{Samech.}} Tous les passants battent des mains en te voyant; ils sifflent ils branlent la tête sur la fille de Jérusalem : Est-ce donc là cette ville d'une beauté si parfaite, la joie de toute la terre ?\\
      Jérusalem, Jérusalem, convertis-toi au Seigneur ton Dieu.
    \end{footnotesize}
  \end{multicols}

  \medskip
  \begin{center}
    \rule{4cm}{0.4pt}
  \end{center}
  \medskip

  \greillumination{\initfamily\fontsize{11mm}{11mm}\selectfont V}
  \gregorioscore{repons/re--velum_templi--solesmes_1961}

  \smallskip

  \small
  \begin{multicols}{2}
    \par\textcolor{red}{\textit{\Rbar}.} \textit{Le voile du temple se déchira,\\ \textcolor{red}{*} Et toute la terre trembla; le larron en croix s'écriait : souvenez-cous de moi, Seigneur, lorsque vous serez dans votre royaume.}
    \columnbreak
    \par\textcolor{red}{\textit{\Vbar}.} Les pierres se fendirent : les tombeaux s'ouvrirent et plusieurs corps des saints, qui étaient endormis, ressuscitèrent. \textit{\\
    \textcolor{red}{*} Et toute la terre trembla; le larron en croix s'écriait : souvenez-cous de moi, Seigneur, lorsque vous serez dans votre royaume.}
  \end{multicols}
  \normalsize

  \medskip
  \begin{center}
    \rule{4cm}{0.4pt}
  \end{center}
  \medskip

  \begin{center}
    \large Leçon III.\\
    \normalsize
  \end{center}
  % \grechangedim{baselineskip}{55pt}{scalable}
  \gresetinitiallines{1}
  \greillumination{\initfamily\fontsize{11mm}{11mm}\selectfont A}
  \gregorioscore{lamentations/va--aleph_ego_vir_videns--solesmes}

  \begin{multicols}{2}
    \begin{footnotesize}
      \par \textcolor{red}{\textit{Aleph.}} Je suis un homme, voyant sa misère sous la verge de son indignation. 
      \par \textcolor{red}{\textit{Aleph.}} Il m'a conduit et amené dans les ténèbres, loin de la lumière.
      \par \textcolor{red}{\textit{Aleph.}} Il tourne et retourne sa main sur moi tout le jour.
      \par \textcolor{red}{\textit{Beth.}} Il ausé ma peau et ma chair : il a brisé mes os.
      \par \textcolor{red}{\textit{Beth.}} Il m'a entouré d'un mur, il m'a environné de fiel et de chagrin.
      \par \textcolor{red}{\textit{Beth.}} Il m'a fait habité dans les ténèbres, comme ceux qui sont mort pour toujours.
      \par \textcolor{red}{\textit{Ghimel.}} Il m'a enfermé de tous côtés, et je ne saurais sortir : il m'a chargé de fers.
      \par \textcolor{red}{\textit{Ghimel.}} En vain j'ai crié vers lui, et je l'ai supplié : il a repoussé ma prière.
      \par \textcolor{red}{\textit{Ghimel.}} Il m'a fermé le passage avec des pierres de taille : il m'a coupé le chemin.\\ Jérusalem, Jérusalem, convertis-toi au Seigneur ton Dieu.
      % \par \hspace{\fill}
    \end{footnotesize}
  \end{multicols}

  \medskip
  \begin{center}
    \rule{4cm}{0.4pt}
  \end{center}
  \medskip

  \greillumination{\initfamily\fontsize{11mm}{11mm}\selectfont V}
  \gregorioscore{repons/re--vinea_mea--solesmes_1961}

  \smallskip

  \small
  \begin{multicols}{2}
    \par\textcolor{red}{\textit{\Rbar}.} \textit{Ô ma vigne ! Je t'avais choisie et plantée moi-même : \\ \textcolor{red}{*} Comment as-tu été changée en amertume, jusqu'à me crucifier et délivrer Barabbas ?}
    \columnbreak
    \par\textcolor{red}{\textit{\Vbar}.} \textit{Je t'ai environnée d'une haie; j'en ai ôté les pierres, et j'ai bâti une tour au milieu.\\
    \textcolor{red}{*} Comment as-tu été changée en amertume, jusqu'à me crucifier et délivrer Barabbas ?}
  \end{multicols}
  \normalsize

  \medskip
  \begin{center}
    \rule{4cm}{0.4pt}
  \end{center}
  \medskip
\end{document}

