% !TeX program = lualatex
\documentclass[12pt, a4paper]{article}
\usepackage{fullpage}
\usepackage{subfiles}
\usepackage{fontspec}
\usepackage{libertine}
\usepackage{xcolor}
\usepackage{GotIn}
\usepackage{geometry}
\usepackage{multicol}
\usepackage{multicolrule}
\usepackage{graphicx}
\usepackage{enumitem}
\usepackage[autocompile]{gregoriotex}

\geometry{top=2cm, bottom=2cm}
\pagestyle{empty}

\definecolor{red}{HTML}{C70039}
% \input GoudyIn.fd
% \newcommand*\initfamily{\usefont{U}{GoudyIn}{xl}{n}}

\input Acorn.fd
\newcommand*\initfamily{\usefont{U}{Acorn}{xl}{n}}
% cette ligne ajoute de l'espace entre les portées
% \grechangedim{baselineskip}{60pt}{scalable}

\begin{document}
  \gresetlinecolor{gregoriocolor}
  \begin{center}
    \large A LAUDES.\\
  \end{center}
  \medskip
  \par Le premier Psaume des Laudes est le \textit{Miserére}. Jamais les accents de pénitence ne peuvent être plus convenables, qu'au jour où le péché a causé la mort du Fils de Dieu.

  \medskip

  % ===== DEBUT Antienne =========
  \gresetinitiallines{1}
  \greillumination{\initfamily\fontsize{11mm}{11mm}\selectfont P}
  \gregorioscore{antiennes/an--proprio_filio_suo--solesmes}
  \begin{center}
    \footnotesize{
      \textit{Pour la fin. Psaume pour David : lorsque vint chez lui le prophète Nathan ; et qu'il entra chez Bethsabée.}
    }
  \end{center}
  % ===== FIN Antienne ===========

  % ===== DEBUT psaume ===========
  % gresetinitiallines : avec le parametre à 0, supprime l'ornement
  \begin{center}
    \large{Psaume 50.}\\
  \end{center}

  \gresetinitiallines{0}
  \gregorioscore{psaumes/psaume50-VIIc}

  \begin{enumerate}[label=\textcolor{red}{\arabic*}]
    \setcounter{enumi}{1}
    \item Et secúndum multitúdinem miserati\textbf{ó}num tu\textbf{á}rum,\textcolor{red}{~*} dele iniqui\textbf{tá}tem \textbf{me}am.

    \item Amplius lava me ab iniqui\textbf{tá}te \textbf{me}a:\textcolor{red}{~*} et a peccáto \textbf{me}o \textbf{mun}da me.

    \item Quóniam iniquitátem meam \textbf{e}go co\textbf{gnós}co:\textcolor{red}{~*} et peccátum meum contra \textbf{me} est \textbf{sem}per.

    \item Tibi soli peccávi, et malum \textbf{co}ram te \textbf{fe}ci:\textcolor{red}{~*} ut justificéris in sermónibus tuis, et vincas cum \textbf{ju}di\textbf{cá}ris.

    \item Ecce enim in iniquitáti\textbf{bus} con\textbf{cép}tus sum:\textcolor{red}{~*} et in peccátis concépit me \textbf{ma}ter \textbf{me}a.

    \item Ecce enim veritátem \textbf{di}le\textbf{xís}ti:\textcolor{red}{~*} incérta et occúlta sapiéntiæ tuæ manifes\textbf{tás}ti \textbf{mi}hi.

    \item Aspérges me hyssópo, \textbf{et} mun\textbf{dá}bor:\textcolor{red}{~*} lavábis me, et super nivem \textbf{de}al\textbf{bá}bor.

    \item Audítui meo dabis gáudium \textbf{et} læ\textbf{tí}tiam:\textcolor{red}{~*} et exsultábunt ossa hu\textbf{mi}li\textbf{á}ta.

    \item Avérte fáciem tuam a pec\textbf{cá}tis \textbf{me}is:\textcolor{red}{~*} et omnes iniquitátes \textbf{me}as \textbf{de}le.

    \item Cor mundum crea \textbf{in} me, \textbf{De}us:\textcolor{red}{~*} et spíritum rectum ínnova in vi\textbf{scé}ribus \textbf{me}is.

    \item Ne projícias me a \textbf{fá}cie \textbf{tu}a:\textcolor{red}{~*} et spíritum sanctum tuum ne \textbf{áu}feras \textbf{a} me.

    \item Redde mihi lætítiam salu\textbf{tá}ris \textbf{tu}i:\textcolor{red}{~*} et spíritu princi\textbf{pá}li con\textbf{fír}ma me.

    \item Docébo iníquos \textbf{vi}as \textbf{tu}as:\textcolor{red}{~*} et ímpii ad te \textbf{con}ver\textbf{tén}tur.

    \item Líbera me de sanguínibus, Deus, Deus sa\textbf{lú}tis \textbf{me}æ:\textcolor{red}{~*} et exsultábit lingua mea jus\textbf{tí}tiam \textbf{tu}am.

    \item Dómine, lábia \textbf{me}a a\textbf{pé}ries:\textcolor{red}{~*} et os meum annuntiábit \textbf{lau}dem \textbf{tu}am.

    \item Quóniam si voluísses sacrifícium, de\textbf{dís}sem \textbf{ú}tique:\textcolor{red}{~*} holocáustis non \textbf{de}lec\textbf{tá}beris.

    \item Sacrifícium Deo spíritus con\textbf{tri}bu\textbf{lá}tus:\textcolor{red}{~*} cor contrítum et humiliátum, Deus, \textbf{non} de\textbf{spí}cies.

    \item Benígne fac, Dómine, in bona voluntáte \textbf{tu}a \textbf{Si}on:\textcolor{red}{~*} ut ædificéntur \textbf{mu}ri Je\textbf{rú}salem.

    \item Tunc acceptábis sacrifícium justítiæ, oblatiónes, et \textbf{ho}lo\textbf{cáus}ta:\textcolor{red}{~*} tunc impónent super altáre \textbf{tu}um \textbf{ví}tulos.
  \end{enumerate}
  %  Répetition de l'Antienne
  \grecommentary{\textit{Reprise de l'Antienne.}}
  \gabcsnippet{(c3) Pró(i_[oh:h])pri(g)o(i!jw!kv_JI) Fí(i')li(h)o(g') su(h)o(g_[oh:h]f_[oh:h]) (,) non(e) pe(f)pér(g_[uh:l]hf~)cit(g') De(f)us,(e.) (;) <nlba>sed(ghg___) pro(f') no(h)bis</nlba>(g'_[oh:h]) ó(f)mni(e)bus(f.) (,) trá(f'_)di(h)dit(g) il(e.)lum.(e.) (::)}

  \medskip
  \begin{multicols}{2}
    \begin{footnotesize}
      \begin{enumerate}[label=\textcolor{red}{\emph{\arabic*}}]
        \item \textit{Ayez pitié de moi, mon Dieu, selon votre grande miséricorde.}
        \item \textit{Et selon la multitude de vos bontés, effacez mon iniquité.}
        \item \textit{Lavez-moi de plus en plus de mon iniquité, et
        purifiez-moi de mon péché ;}
        \item \textit{Parce que je connais mon iniquité, et que mon
        péché est toujours présent devant moi.}
        \item \textit{J’ai péché contre vous seul, j’ai fait le mal en votre
        présence ; afin que vous soyez reconnu juste dans
        vos paroles, et victorieux dans vos jugements.}
        \item \textit{Car j’ai été formé dans l’iniquité, et ma mère
        m’a conçu dans le péché.}
        \item \textit{Car vous avez aimé la vérité, et vous m’avez manifesté les secrets et les mystères de votre sagesse.}
        \item \textit{Vous m’arroserez avec l’hysope, et je serai pur ;
        lavez-moi, et je serai plus blanc que la neige.}
        \item \textit{Vous me ferez entendre des paroles de joie et de
        consolation ; et mes os humiliés seront dans la joie.}
        \item \textit{Détournez votre visage de dessus mes péchés, et
        effacez toutes mes iniquités.} 
        \item \textit{Mon Dieu, créez en moi un cœur pur, et renouvelez l’esprit de droiture jusques dans mes entrailles}
        \item \textit{Ne me rejetez pas de devant votre face, et ne retirez pas de moi votre Saint-Esprit.}
        \item \textit{Rendez-moi la joie de votre salut, et rassurez-moi
        par la force de votre Esprit.}
        \item \textit{J’enseignerai vos voies aux pécheurs, et les impies
        se convertiront à vous.}
        \item \textit{O Dieu, mon Dieu, auteur de mon salut, délivrez-moi du sang que j’ai répandu, et ma langue annoncera avec joie votre justice.}
        \item \textit{Seigneur, vous ouvrirez mes lèvres, et ma bouche
        annoncera vos louanges.}
        \item \textit{Car si vous eussiez voulu un sacrifice, je vous
        l’aurai offert ; mais les holocaustes ne vous sont
        pas agréables.}
        \item \textit{Un esprit pénétré de douleur, est un sacrifice que
        Dieu agrée : mon Dieu, vous ne mépriserez pas un
        cœur contrit et humilié.}
        \item \textit{Seigneur, faites sentir à Sion les effets de votre
        bonté ; afin que les murs de Jérusalem soient bâtis.}
        \item \textit{Alors vous accepterez le sacrifice de justice, les offrandes et les holocaustes : alors on offrira des veaux sur votre autel}
      \end{enumerate}
    \end{footnotesize}
  \end{multicols}

  \medskip
  \begin{center}
    \rule{2cm}{0.4pt}
  \end{center}
  \medskip

  \par Le Psaume 142 est le septième des Psaumes de la Pénitence. Jésus Christ y expose à son Père la douleur que lui cause la trahison des siens, et ses souffrances sur la croix. Mais bientôt il fait entendre une prière pleine de confiance non seulement pour lui-même, mais pour tous ceux qui sont rachetés par son sang.
  \medskip
  
  % ===== DEBUT Antienne =========
  \gresetinitiallines{1}
  \greillumination{\initfamily\fontsize{11mm}{11mm}\selectfont A}
  \gregorioscore{antiennes/an--anxiatus_est--solesmes}
  \begin{center}
    \footnotesize{
      \textit{Mon esprit a été agité et saisi de tristesse ; mon cœur s’est troublé intérieurement.}
    }
  \end{center}
  % ===== FIN Antienne ===========

  % ===== DEBUT psaume ===========
  % gresetinitiallines : avec le parametre à 0, supprime l'ornement
  \begin{center}
    \large{Psaume 142.}\\
  \end{center}

  \gresetinitiallines{0}
  \gregorioscore{psaumes/psaume142-IVE}

  \begin{enumerate}[label=\textcolor{red}{\arabic*}]
    \setcounter{enumi}{1}
    \item Et non intres in judícium cum \textit{ser}\textit{vo} \textbf{tu}o:\textcolor{red}{~*} quia non justificábitur in conspéctu tu\textit{o} \textit{om}\textit{nis} \textbf{vi}vens.

    \item Quia persecútus est inimícus á\textit{ni}\textit{mam} \textbf{me}am:\textcolor{red}{~*} humiliávit in ter\textit{ra} \textit{vi}\textit{tam} \textbf{me}am.

    \item Collocávit me in obscúris sicut mórtuos sǽculi:\textcolor{red}{~†}\\ et anxiátus est super me spí\textit{ri}\textit{tus} \textbf{me}us,\textcolor{red}{~*} in me turbá\textit{tum} \textit{est} \textit{cor} \textbf{me}um.

    \item Memor fui diérum antiquórum,\textcolor{red}{~†} meditátus sum in ómnibus opé\textit{ri}\textit{bus} \textbf{tu}is:\textcolor{red}{~*}\\ in factis mánuum tuá\textit{rum} \textit{me}\textit{di}\textbf{tá}bar.

    \item Expándi manus \textit{me}\textit{as} \textbf{ad} te:\textcolor{red}{~*} ánima mea sicut terra si\textit{ne} \textit{a}\textit{qua} \textbf{ti}bi.

    \item Velóciter exáu\textit{di} \textit{me}, \textbf{Dó}mine:\textcolor{red}{~*} defécit \textit{spí}\textit{ri}\textit{tus} \textbf{me}us.

    \item Non avértas fáciem \textit{tu}\textit{am} \textbf{a} me:\textcolor{red}{~*} et símilis ero descendén\textit{ti}\textit{bus} \textit{in} \textbf{la}cum.

    \item Audítam fac mihi mane misericór\textit{di}\textit{am} \textbf{tu}am:\textcolor{red}{~*} quia \textit{in} \textit{te} \textit{spe}\textbf{rá}vi.

    \item Notam fac mihi viam, \textit{in} \textit{qua} \textbf{ám}bulem:\textcolor{red}{~*} quia ad te levávi \textit{á}\textit{ni}\textit{mam} \textbf{me}am.

    \item Eripe me de inimícis meis, Dómine, ad \textit{te} \textit{con}\textbf{fú}gi:\textcolor{red}{~*} doce me fácere voluntátem tuam, quia De\textit{us} \textit{me}\textit{us} \textbf{es} tu.

    \item Spíritus tuus bonus dedúcet me in \textit{ter}\textit{ram} \textbf{rec}tam:\textcolor{red}{~*} propter nomen tuum, Dómine, vivificábis me, in æ\textit{qui}\textit{tá}\textit{te} \textbf{tu}a.

    \item Edúces de tribulatióne á\textit{ni}\textit{mam} \textbf{me}am:\textcolor{red}{~*} et in misericórdia tua dispérdes in\textit{i}\textit{mí}\textit{cos} \textbf{me}os.

    \item Et perdes omnes, qui tríbulant á\textit{ni}\textit{mam} \textbf{me}am,\textcolor{red}{~*} quóniam e\textit{go} \textit{ser}\textit{vus} \textbf{tu}\textbf{us} sum.
  \end{enumerate}
  %  Répetition de l'Antienne
  \grecommentary{\textit{Reprise de l'Antienne.}}
  \gabcsnippet{(c4) An(f)xi(d)á(e')tus(f) est(g') in(f) me(e.) (,) spí(f_d)ri(f_g)tus(f) me(d.)us :(d.) (;) in(dh~) me(h) tur(gf~)bá(gg)tum(f') est(g) cor(h_g) me(e.)um.(e.) (::)}

  \medskip
  \begin{multicols}{2}
    \begin{footnotesize}
      \begin{enumerate}[label=\textcolor{red}{\emph{\arabic*}}]
        \item \textit{Seigneur, exaucez ma prière ; prêtez l’oreille à ma demande,
        et exaucez-moi dans votre vérité et dans votre justice.}
        \item \textit{Et n’entrez point en jugement avec votre serviteur ;
        car nul homme vivant ne sera point trouvé juste en
        votre présence.}
        \item \textit{Parce que l’ennemi a persécuté mon âme, et qu’il
        a humilié ma vie sur la terre.}
        \item \textit{Il m’a mis dans les lieux obscurs comme les morts
        du siècle ; et mon esprit a été agité et saisi de
        tristesse ; mon cœur s’est troublé intérieurement.}
        \item \textit{Je me suis souvenu des anciens jours ; j’ai réfléchi
        dans mon esprit sur tous vos ouvrages, et je méditais sur les œuvres de vos mains.}
        \item \textit{J’ai tendu les mains vers vous : mon âme est devant vous comme une terre sans eau.}
        \item \textit{Exaucez-moi promptement, Seigneur, mon esprit tombe en défaillance.}
        \item \textit{Ne détournez pas de moi votre visage ; car je serais semblable à ceux qui tombent dans l’abîme.}
        \item \textit{Faites-moi connaître dès le matin votre miséricorde, parce que j’ai espéré en vous.}
        \item \textit{Montrez-moi le chemin que je dois tenir, puisque j’ai élevé mon âme vers vous.}
        \item \textit{Seigneur, délivrez-moi de mes ennemis, j’ai recours à vous : enseignez-moi à faire votre volonté, puisque vous êtes mon Dieu.}
        \item \textit{Votre bon Esprit me conduira dans une bonne terre, et pour la gloire de votre nom, Seigneur, vous me vivifierez dans votre équité.}
        \item \textit{Vous retirerez mon âme de la tribulation ; et par
        votre miséricorede vous dissiperez mes ennemis ;} 
        \item \textit{Et vous perdrez tous ceux qui affligent mon âme,
        parce que je suis votre serviteur.}
      \end{enumerate}
    \end{footnotesize}
  \end{multicols}

  \medskip
  \begin{center}
    \rule{2cm}{0.4pt}
  \end{center}
  \medskip

  \par Dans le Psaume 84, au sens spirituel le peuple ramené de la servitude et pour qui s'ouvrent de si joyeuses perspectives, est le peuple chrétien racheté de la mort et du péché par la Passion douloureuse du Christ Sauveur.

  \medskip

  % ===== DEBUT Antienne =========
  \gresetinitiallines{1}
  \greillumination{\initfamily\fontsize{11mm}{11mm}\selectfont A}
  \gregorioscore{antiennes/an--ait_latro--solesmes}
  \begin{center}
    \footnotesize{
      \textit{Le larron dit au larron : Nous sommes traités comme nous le méritons ; mais qu’à fait celui-ci ? Souvenez-vous de moi, Seigneur, quand vous serez dans votre Royaume.}
    }
  \end{center}
  % ===== FIN Antienne ===========

  % ===== DEBUT psaume ===========
  % gresetinitiallines : avec le parametre à 0, supprime l'ornement
  \begin{center}
    \large{Psaume 84.}\\
  \end{center}

  \gresetinitiallines{0}
  \gregorioscore{psaumes/psaume84-If}

  \begin{enumerate}[label=\textcolor{red}{\arabic*}]
    \setcounter{enumi}{1}
    \item Remisísti iniquitátem \textbf{ple}bis \textbf{tu}æ:\textcolor{red}{~*} operuísti ómnia peccá\textit{ta} \textit{e}\textbf{ó}rum.

    \item Mitigásti omnem \textbf{i}ram \textbf{tu}am:\textcolor{red}{~*} avertísti ab ira indignati\textit{ó}\textit{nis} \textbf{tu}æ.

    \item Convérte nos, Deus, salu\textbf{tá}ris \textbf{nos}ter:\textcolor{red}{~*} et avérte iram tu\textit{am} \textit{a} \textbf{no}bis.

    \item Numquid in ætérnum ira\textbf{scé}ris \textbf{no}bis?\textcolor{red}{~*} aut exténdes iram tuam a generatióne in gene\textit{ra}\textit{ti}\textbf{ó}nem?

    \item Deus, tu convérsus vi\textbf{vi}fi\textbf{cá}bis nos:\textcolor{red}{~*} et plebs tua lætá\textit{bi}\textit{tur} \textbf{in} te.

    \item Osténde nobis, Dómine, miseri\textbf{cór}diam \textbf{tu}am:\textcolor{red}{~*} et salutáre tu\textit{um} \textit{da} \textbf{no}bis.

    \item Audiam quid loquátur in me \textbf{Dó}minus \textbf{De}us:\textcolor{red}{~*} quóniam loquétur pacem in \textit{ple}\textit{bem} \textbf{su}am.

    \item Et super \textbf{sanc}tos \textbf{su}os:\textcolor{red}{~*} et in eos, qui conver\textit{tún}\textit{tur} \textbf{ad} cor.

    \item Verúmtamen prope timéntes eum salu\textbf{tá}re ip\textbf{sí}us:\textcolor{red}{~*} ut inhábitet glória in \textit{ter}\textit{ra} \textbf{nos}tra.

    \item Misericórdia, et véritas obvia\textbf{vé}runt \textbf{si}bi:\textcolor{red}{~*} justítia, et pax \textit{os}\textit{cu}\textbf{lá}tæ sunt.

    \item Véritas de \textbf{ter}ra \textbf{or}ta est:\textcolor{red}{~*} et justítia de cæ\textit{lo} \textit{pro}\textbf{spé}xit.

    \item Etenim Dóminus dabit be\textbf{ni}gni\textbf{tá}tem:\textcolor{red}{~*} et terra nostra dabit \textit{fruc}\textit{tum} \textbf{su}um.

    \item Justítia ante eum \textbf{am}bu\textbf{lá}bit:\textcolor{red}{~*} et ponet in via \textit{gres}\textit{sus} \textbf{su}os.
  \end{enumerate}
  %  Répetition de l'Antienne
  \grecommentary{\textit{Reprise de l'Antienne.}}
  \gabcsnippet{(c4) A(d)it(f') la(d)tro(dc) ad(f) la(g)tró(f_h)nem :(h'_) (,) Nos(h) qui(h')dem(j) di(h')gna(g) fa(h')ctis(g) re(ge)cí(f_g)pi(h)mus,(g.) (;) hic(h_f) au(g')tem(f) quid(e') fe(f)cit?(d.) (:) Me(d)mén(f')to(d) me(e')i,(f) Dó(g_[uh:l]h)mi(g)ne,(f'_) (,) dum(f) vé(fg)ne(f)ris(c') in(e) re(g_[oh:h]e~)gnum(f_e) tu(d.)um.(d.) (::)}

  \medskip
  \begin{multicols}{2}
    \begin{footnotesize}
      \begin{enumerate}[label=\textcolor{red}{\emph{\arabic*}}]
        \item \textit{Vous avez, Seigneur, béni votre terre, vous y avez ramené les captifs de Jacob.}
        \item \textit{Vous avez pardonné l'iniquité de votre peuple, vous avez couvert tous ses péchés.}
        \item \textit{Vous avez apaisé toute votre indignation, vous êtes revenu de l'ardeur de votre colère.}
        \item \textit{Rétablissez-nous, ô Dieu, notre Sauveur ; détournez de nous votre courroux.}
        \item \textit{Serez-vous éternellement irrité contre nous ? Prolongerez-vous d'âge en âge votre ressentiment ?}
        \item \textit{Ô Dieu, vous nous ferez revenir à la vie ; afin que votre peuple se réjouisse en vous.}
        \item \textit{Seigneur, faites-nous voir votre bonté, et accordez-nous votre salut.}
        \item \textit{Je veux écouter ce que dira au dedans de moi le Seigneur Dieu ; il a des paroles de paix pour son peuple,}
        \item \textit{Pour ses fidèles et pour ceux qui rentrent au fond de leur cœur.}
        \item \textit{Oui, son salut est proche de ceux qui le craignent, et la gloire habitera de nouveau sur notre terre.}
        \item \textit{La grâce et la vérité vont se rencontrer : la justice et la paix s'embrasseront.}
        \item \textit{La vérité germera de la terre, et la justice regardera du haut du ciel.}
        \item \textit{Le Seigneur nous accordera ses faveurs, et notre terre donnera son fruit.} 
        \item \textit{La justice marchera devant lui, et tracera le chemin de ses pas.}
      \end{enumerate}
    \end{footnotesize}
  \end{multicols}

  \medskip
  \begin{center}
    \rule{2cm}{0.4pt}
  \end{center}
  \medskip

  \par 
  \medskip
  \begin{center}
    \rule{2cm}{0.4pt}
  \end{center}
  \medskip

\end{document}