% !TeX program = lualatex
\documentclass[12pt, a4paper]{article}
\usepackage{fullpage}
\usepackage{subfiles}
\usepackage{fontspec}
\usepackage{libertine}
\usepackage{xcolor}
\usepackage{GotIn}
\usepackage{geometry}
\usepackage{multicol}
\usepackage{multicolrule}
\usepackage{graphicx}
\usepackage{enumitem}
\usepackage[autocompile]{gregoriotex}

\geometry{top=2cm, bottom=2cm}
\pagestyle{empty}

\definecolor{red}{HTML}{C70039}
% \input GoudyIn.fd
% \newcommand*\initfamily{\usefont{U}{GoudyIn}{xl}{n}}

\input Acorn.fd
\newcommand*\initfamily{\usefont{U}{Acorn}{xl}{n}}
% cette ligne ajoute de l'espace entre les portées
% \grechangedim{baselineskip}{60pt}{scalable}

\begin{document}
  \gresetlinecolor{gregoriocolor}
  \begin{center}
    \large A LAUDES.\\
  \end{center}
  \medskip
  \par Le premier Psaume des Laudes est le \textit{Miserére}. Jamais les accents de pénitence ne peuvent être plus convenables, qu'au jour où le péché a causé la mort du Fils de Dieu.

  \medskip

  % ===== DEBUT Antienne =========
  \gresetinitiallines{1}
  \greillumination{\initfamily\fontsize{11mm}{11mm}\selectfont P}
  \gregorioscore{antiennes/an--proprio_filio_suo--solesmes}
  \begin{center}
    \footnotesize{
      \textit{Pour la fin. Psaume pour David : lorsque vint chez lui le prophète Nathan ; et qu'il entra chez Bethsabée.}
    }
  \end{center}
  % ===== FIN Antienne ===========

  % ===== DEBUT psaume ===========
  % gresetinitiallines : avec le parametre à 0, supprime l'ornement
  \begin{center}
    \large{Psaume 50.}\\
  \end{center}

  \gresetinitiallines{0}
  \gregorioscore{psaumes/psaume50-VIIc}

  \begin{enumerate}[label=\textcolor{red}{\arabic*}]
    \setcounter{enumi}{1}
    \item Et secúndum multitúdinem miserati\textbf{ó}num tu\textbf{á}rum,\textcolor{red}{~*} dele iniqui\textbf{tá}tem \textbf{me}am.

    \item Amplius lava me ab iniqui\textbf{tá}te \textbf{me}a:\textcolor{red}{~*} et a peccáto \textbf{me}o \textbf{mun}da me.

    \item Quóniam iniquitátem meam \textbf{e}go co\textbf{gnós}co:\textcolor{red}{~*} et peccátum meum contra \textbf{me} est \textbf{sem}per.

    \item Tibi soli peccávi, et malum \textbf{co}ram te \textbf{fe}ci:\textcolor{red}{~*} ut justificéris in sermónibus tuis, et vincas cum \textbf{ju}di\textbf{cá}ris.

    \item Ecce enim in iniquitáti\textbf{bus} con\textbf{cép}tus sum:\textcolor{red}{~*} et in peccátis concépit me \textbf{ma}ter \textbf{me}a.

    \item Ecce enim veritátem \textbf{di}le\textbf{xís}ti:\textcolor{red}{~*} incérta et occúlta sapiéntiæ tuæ manifes\textbf{tás}ti \textbf{mi}hi.

    \item Aspérges me hyssópo, \textbf{et} mun\textbf{dá}bor:\textcolor{red}{~*} lavábis me, et super nivem \textbf{de}al\textbf{bá}bor.

    \item Audítui meo dabis gáudium \textbf{et} læ\textbf{tí}tiam:\textcolor{red}{~*} et exsultábunt ossa hu\textbf{mi}li\textbf{á}ta.

    \item Avérte fáciem tuam a pec\textbf{cá}tis \textbf{me}is:\textcolor{red}{~*} et omnes iniquitátes \textbf{me}as \textbf{de}le.

    \item Cor mundum crea \textbf{in} me, \textbf{De}us:\textcolor{red}{~*} et spíritum rectum ínnova in vi\textbf{scé}ribus \textbf{me}is.

    \item Ne projícias me a \textbf{fá}cie \textbf{tu}a:\textcolor{red}{~*} et spíritum sanctum tuum ne \textbf{áu}feras \textbf{a} me.

    \item Redde mihi lætítiam salu\textbf{tá}ris \textbf{tu}i:\textcolor{red}{~*} et spíritu princi\textbf{pá}li con\textbf{fír}ma me.

    \item Docébo iníquos \textbf{vi}as \textbf{tu}as:\textcolor{red}{~*} et ímpii ad te \textbf{con}ver\textbf{tén}tur.

    \item Líbera me de sanguínibus, Deus, Deus sa\textbf{lú}tis \textbf{me}æ:\textcolor{red}{~*} et exsultábit lingua mea jus\textbf{tí}tiam \textbf{tu}am.

    \item Dómine, lábia \textbf{me}a a\textbf{pé}ries:\textcolor{red}{~*} et os meum annuntiábit \textbf{lau}dem \textbf{tu}am.

    \item Quóniam si voluísses sacrifícium, de\textbf{dís}sem \textbf{ú}tique:\textcolor{red}{~*} holocáustis non \textbf{de}lec\textbf{tá}beris.

    \item Sacrifícium Deo spíritus con\textbf{tri}bu\textbf{lá}tus:\textcolor{red}{~*} cor contrítum et humiliátum, Deus, \textbf{non} de\textbf{spí}cies.

    \item Benígne fac, Dómine, in bona voluntáte \textbf{tu}a \textbf{Si}on:\textcolor{red}{~*} ut ædificéntur \textbf{mu}ri Je\textbf{rú}salem.

    \item Tunc acceptábis sacrifícium justítiæ, oblatiónes, et \textbf{ho}lo\textbf{cáus}ta:\textcolor{red}{~*} tunc impónent super altáre \textbf{tu}um \textbf{ví}tulos.
  \end{enumerate}
  %  Répetition de l'Antienne
  \grecommentary{\textit{Reprise de l'Antienne.}}
  \gabcsnippet{(c3) Pró(i_[oh:h])pri(g)o(i!jw!kv_JI) Fí(i')li(h)o(g') su(h)o(g_[oh:h]f_[oh:h]) (,) non(e) pe(f)pér(g_[uh:l]hf~)cit(g') De(f)us,(e.) (;) <nlba>sed(ghg___) pro(f') no(h)bis</nlba>(g'_[oh:h]) ó(f)mni(e)bus(f.) (,) trá(f'_)di(h)dit(g) il(e.)lum.(e.) (::)}

  \medskip
  \begin{multicols}{2}
    \begin{footnotesize}
      \begin{enumerate}[label=\textcolor{red}{\emph{\arabic*}}]
        \item \textit{Ayez pitié de moi, mon Dieu, selon votre grande miséricorde.}
        \item \textit{Et selon la multitude de vos bontés, effacez mon iniquité.}
        \item \textit{Lavez-moi de plus en plus de mon iniquité, et
        purifiez-moi de mon péché ;}
        \item \textit{Parce que je connais mon iniquité, et que mon
        péché est toujours présent devant moi.}
        \item \textit{J’ai péché contre vous seul, j’ai fait le mal en votre
        présence ; afin que vous soyez reconnu juste dans
        vos paroles, et victorieux dans vos jugements.}
        \item \textit{Car j’ai été formé dans l’iniquité, et ma mère
        m’a conçu dans le péché.}
        \item \textit{Car vous avez aimé la vérité, et vous m’avez manifesté les secrets et les mystères de votre sagesse.}
        \item \textit{Vous m’arroserez avec l’hysope, et je serai pur ;
        lavez-moi, et je serai plus blanc que la neige.}
        \item \textit{Vous me ferez entendre des paroles de joie et de
        consolation ; et mes os humiliés seront dans la joie.}
        \item \textit{Détournez votre visage de dessus mes péchés, et
        effacez toutes mes iniquités.} 
        \item \textit{Mon Dieu, créez en moi un cœur pur, et renouvelez l’esprit de droiture jusques dans mes entrailles}
        \item \textit{Ne me rejetez pas de devant votre face, et ne retirez pas de moi votre Saint-Esprit.}
        \item \textit{Rendez-moi la joie de votre salut, et rassurez-moi
        par la force de votre Esprit.}
        \item \textit{J’enseignerai vos voies aux pécheurs, et les impies
        se convertiront à vous.}
        \item \textit{O Dieu, mon Dieu, auteur de mon salut, délivrez-moi du sang que j’ai répandu, et ma langue annoncera avec joie votre justice.}
        \item \textit{Seigneur, vous ouvrirez mes lèvres, et ma bouche
        annoncera vos louanges.}
        \item \textit{Car si vous eussiez voulu un sacrifice, je vous
        l’aurai offert ; mais les holocaustes ne vous sont
        pas agréables.}
        \item \textit{Un esprit pénétré de douleur, est un sacrifice que
        Dieu agrée : mon Dieu, vous ne mépriserez pas un
        cœur contrit et humilié.}
        \item \textit{Seigneur, faites sentir à Sion les effets de votre
        bonté ; afin que les murs de Jérusalem soient bâtis.}
        \item \textit{Alors vous accepterez le sacrifice de justice, les offrandes et les holocaustes : alors on offrira des veaux sur votre autel}
      \end{enumerate}
    \end{footnotesize}
  \end{multicols}

  \medskip

  \par Le Psaume 142 est le septième des Psaumes de la Pénitence. Jésus Christ y expose à son Père la douleur que lui cause la trahison des siens, et ses souffrances sur la croix. Mais bientôt il fait entendre une prière pleine de confiance non seulement pour lui-même, mais pour tous ceux qui sont rachetés par son sang.
  \medskip
  
  % ===== DEBUT Antienne =========
  \gresetinitiallines{1}
  \greillumination{\initfamily\fontsize{11mm}{11mm}\selectfont A}
  \gregorioscore{antiennes/an--anxiatus_est--solesmes}
  \begin{center}
    \footnotesize{
      \textit{Mon esprit a été agité et saisi de tristesse ; mon cœur s’est troublé intérieurement.}
    }
  \end{center}
  % ===== FIN Antienne ===========

  % ===== DEBUT psaume ===========
  % gresetinitiallines : avec le parametre à 0, supprime l'ornement
  \begin{center}
    \large{Psaume 142.}\\
  \end{center}

  \gresetinitiallines{0}
  \gregorioscore{psaumes/psaume142-IVE}

  \begin{enumerate}[label=\textcolor{red}{\arabic*}]
    \setcounter{enumi}{1}
    \item Et non intres in judícium cum \textit{ser}\textit{vo} \textbf{tu}o:\textcolor{red}{~*} quia non justificábitur in conspéctu tu\textit{o} \textit{om}\textit{nis} \textbf{vi}vens.

    \item Quia persecútus est inimícus á\textit{ni}\textit{mam} \textbf{me}am:\textcolor{red}{~*} humiliávit in ter\textit{ra} \textit{vi}\textit{tam} \textbf{me}am.

    \item Collocávit me in obscúris sicut mórtuos sǽculi:\textcolor{red}{~†}\\ et anxiátus est super me spí\textit{ri}\textit{tus} \textbf{me}us,\textcolor{red}{~*} in me turbá\textit{tum} \textit{est} \textit{cor} \textbf{me}um.

    \item Memor fui diérum antiquórum,\textcolor{red}{~†} meditátus sum in ómnibus opé\textit{ri}\textit{bus} \textbf{tu}is:\textcolor{red}{~*} in factis mánuum tuá\textit{rum} \textit{me}\textit{di}\textbf{tá}bar.

    \item Expándi manus \textit{me}\textit{as} \textbf{ad} te:\textcolor{red}{~*} ánima mea sicut terra si\textit{ne} \textit{a}\textit{qua} \textbf{ti}bi.

    \item Velóciter exáu\textit{di} \textit{me}, \textbf{Dó}mine:\textcolor{red}{~*} defécit \textit{spí}\textit{ri}\textit{tus} \textbf{me}us.

    \item Non avértas fáciem \textit{tu}\textit{am} \textbf{a} me:\textcolor{red}{~*} et símilis ero descendén\textit{ti}\textit{bus} \textit{in} \textbf{la}cum.

    \item Audítam fac mihi mane misericór\textit{di}\textit{am} \textbf{tu}am:\textcolor{red}{~*} quia \textit{in} \textit{te} \textit{spe}\textbf{rá}vi.

    \item Notam fac mihi viam, \textit{in} \textit{qua} \textbf{ám}bulem:\textcolor{red}{~*} quia ad te levávi \textit{á}\textit{ni}\textit{mam} \textbf{me}am.

    \item Eripe me de inimícis meis, Dómine, ad \textit{te} \textit{con}\textbf{fú}gi:\textcolor{red}{~*} doce me fácere voluntátem tuam, quia De\textit{us} \textit{me}\textit{us} \textbf{es} tu.

    \item Spíritus tuus bonus dedúcet me in \textit{ter}\textit{ram} \textbf{rec}tam:\textcolor{red}{~*} propter nomen tuum, Dómine, vivificábis me, in æ\textit{qui}\textit{tá}\textit{te} \textbf{tu}a.

    \item Edúces de tribulatióne á\textit{ni}\textit{mam} \textbf{me}am:\textcolor{red}{~*} et in misericórdia tua dispérdes in\textit{i}\textit{mí}\textit{cos} \textbf{me}os.

    \item Et perdes omnes, qui tríbulant á\textit{ni}\textit{mam} \textbf{me}am,\textcolor{red}{~*} quóniam e\textit{go} \textit{ser}\textit{vus} \textbf{tu}\textbf{us} sum.
  \end{enumerate}
  %  Répetition de l'Antienne
  \grecommentary{\textit{Reprise de l'Antienne.}}
  \gabcsnippet{(c4) An(f)xi(d)á(e')tus(f) est(g') in(f) me(e.) (,) spí(f_d)ri(f_g)tus(f) me(d.)us :(d.) (;) in(dh~) me(h) tur(gf~)bá(gg)tum(f') est(g) cor(h_g) me(e.)um.(e.) (::)}

  \medskip
  \begin{multicols}{2}
    \begin{footnotesize}
      \begin{enumerate}[label=\textcolor{red}{\emph{\arabic*}}]
        \item \textit{Seigneur, exaucez ma prière ; prêtez l’oreille à ma demande,
        et exaucez-moi dans votre vérité et dans votre justice.}
        \item \textit{Et n’entrez point en jugement avec votre serviteur ;
        car nul homme vivant ne sera point trouvé juste en
        votre présence.}
        \item \textit{Parce que l’ennemi a persécuté mon âme, et qu’il
        a humilié ma vie sur la terre.}
        \item \textit{Il m’a mis dans les lieux obscurs comme les morts
        du siècle ; et mon esprit a été agité et saisi de
        tristesse ; mon cœur s’est troublé intérieurement.}
        \item \textit{Je me suis souvenu des anciens jours ; j’ai réfléchi
        dans mon esprit sur tous vos ouvrages, et je méditais sur les œuvres de vos mains.}
        \item \textit{J’ai tendu les mains vers vous : mon âme est devant vous comme une terre sans eau.}
        \item \textit{Exaucez-moi promptement, Seigneur, mon esprit tombe en défaillance.}
        \item \textit{Ne détournez pas de moi votre visage ; car je serais semblable à ceux qui tombent dans l’abîme.}
        \item \textit{Faites-moi connaître dès le matin votre miséricorde, parce que j’ai espéré en vous.}
        \item \textit{Montrez-moi le chemin que je dois tenir, puisque j’ai élevé mon âme vers vous.}
        \item \textit{Seigneur, délivrez-moi de mes ennemis, j’ai recours à vous : enseignez-moi à faire votre volonté, puisque vous êtes mon Dieu.}
        \item \textit{Votre bon Esprit me conduira dans une bonne terre, et pour la gloire de votre nom, Seigneur, vous me vivifierez dans votre équité.}
        \item \textit{Vous retirerez mon âme de la tribulation ; et par
        votre miséricorede vous dissiperez mes ennemis ;} 
        \item \textit{Et vous perdrez tous ceux qui affligent mon âme,
        parce que je suis votre serviteur.}
      \end{enumerate}
    \end{footnotesize}
  \end{multicols}

  \newpage

  \par Dans le Psaume 84, au sens spirituel le peuple ramené de la servitude et pour qui s'ouvrent de si joyeuses perspectives, est le peuple chrétien racheté de la mort et du péché par la Passion douloureuse du Christ Sauveur.

  \medskip

  % ===== DEBUT Antienne =========
  \gresetinitiallines{1}
  \greillumination{\initfamily\fontsize{11mm}{11mm}\selectfont A}
  \gregorioscore{antiennes/an--ait_latro--solesmes}
  \begin{center}
    \footnotesize{
      \textit{Le larron dit au larron : Nous sommes traités comme nous le méritons ; mais qu’à fait celui-ci ? Souvenez-vous de moi, Seigneur, quand vous serez dans votre Royaume.}
    }
  \end{center}
  % ===== FIN Antienne ===========

  % ===== DEBUT psaume ===========
  % gresetinitiallines : avec le parametre à 0, supprime l'ornement
  \begin{center}
    \large{Psaume 84.}\\
  \end{center}

  \gresetinitiallines{0}
  \gregorioscore{psaumes/psaume84-If}

  \begin{enumerate}[label=\textcolor{red}{\arabic*}]
    \setcounter{enumi}{1}
    \item Remisísti iniquitátem \textbf{ple}bis \textbf{tu}æ:\textcolor{red}{~*} operuísti ómnia peccá\textit{ta} \textit{e}\textbf{ó}rum.

    \item Mitigásti omnem \textbf{i}ram \textbf{tu}am:\textcolor{red}{~*} avertísti ab ira indignati\textit{ó}\textit{nis} \textbf{tu}æ.

    \item Convérte nos, Deus, salu\textbf{tá}ris \textbf{nos}ter:\textcolor{red}{~*} et avérte iram tu\textit{am} \textit{a} \textbf{no}bis.

    \item Numquid in ætérnum ira\textbf{scé}ris \textbf{no}bis?\textcolor{red}{~*} aut exténdes iram tuam a generatióne in gene\textit{ra}\textit{ti}\textbf{ó}nem?

    \item Deus, tu convérsus vi\textbf{vi}fi\textbf{cá}bis nos:\textcolor{red}{~*} et plebs tua lætá\textit{bi}\textit{tur} \textbf{in} te.

    \item Osténde nobis, Dómine, miseri\textbf{cór}diam \textbf{tu}am:\textcolor{red}{~*} et salutáre tu\textit{um} \textit{da} \textbf{no}bis.

    \item Audiam quid loquátur in me \textbf{Dó}minus \textbf{De}us:\textcolor{red}{~*} quóniam loquétur pacem in \textit{ple}\textit{bem} \textbf{su}am.

    \item Et super \textbf{sanc}tos \textbf{su}os:\textcolor{red}{~*} et in eos, qui conver\textit{tún}\textit{tur} \textbf{ad} cor.

    \item Verúmtamen prope timéntes eum salu\textbf{tá}re ip\textbf{sí}us:\textcolor{red}{~*} ut inhábitet glória in \textit{ter}\textit{ra} \textbf{nos}tra.

    \item Misericórdia, et véritas obvia\textbf{vé}runt \textbf{si}bi:\textcolor{red}{~*} justítia, et pax \textit{os}\textit{cu}\textbf{lá}tæ sunt.

    \item Véritas de \textbf{ter}ra \textbf{or}ta est:\textcolor{red}{~*} et justítia de cæ\textit{lo} \textit{pro}\textbf{spé}xit.

    \item Etenim Dóminus dabit be\textbf{ni}gni\textbf{tá}tem:\textcolor{red}{~*} et terra nostra dabit \textit{fruc}\textit{tum} \textbf{su}um.

    \item Justítia ante eum \textbf{am}bu\textbf{lá}bit:\textcolor{red}{~*} et ponet in via \textit{gres}\textit{sus} \textbf{su}os.
  \end{enumerate}
  %  Répetition de l'Antienne
  \grecommentary{\textit{Reprise de l'Antienne.}}
  \gabcsnippet{(c4) A(d)it(f') la(d)tro(dc) ad(f) la(g)tró(f_h)nem :(h'_) (,) Nos(h) qui(h')dem(j) di(h')gna(g) fa(h')ctis(g) re(ge)cí(f_g)pi(h)mus,(g.) (;) hic(h_f) au(g')tem(f) quid(e') fe(f)cit?(d.) (:) Me(d)mén(f')to(d) me(e')i,(f) Dó(g_[uh:l]h)mi(g)ne,(f'_) (,) dum(f) vé(fg)ne(f)ris(c') in(e) re(g_[oh:h]e~)gnum(f_e) tu(d.)um.(d.) (::)}

  \medskip
  \begin{multicols}{2}
    \begin{footnotesize}
      \begin{enumerate}[label=\textcolor{red}{\emph{\arabic*}}]
        \item \textit{Vous avez, Seigneur, béni votre terre, vous y avez ramené les captifs de Jacob.}
        \item \textit{Vous avez pardonné l'iniquité de votre peuple, vous avez couvert tous ses péchés.}
        \item \textit{Vous avez apaisé toute votre indignation, vous êtes revenu de l'ardeur de votre colère.}
        \item \textit{Rétablissez-nous, ô Dieu, notre Sauveur ; détournez de nous votre courroux.}
        \item \textit{Serez-vous éternellement irrité contre nous ? Prolongerez-vous d'âge en âge votre ressentiment ?}
        \item \textit{Ô Dieu, vous nous ferez revenir à la vie ; afin que votre peuple se réjouisse en vous.}
        \item \textit{Seigneur, faites-nous voir votre bonté, et accordez-nous votre salut.}
        \item \textit{Je veux écouter ce que dira au dedans de moi le Seigneur Dieu ; il a des paroles de paix pour son peuple,}
        \item \textit{Pour ses fidèles et pour ceux qui rentrent au fond de leur cœur.}
        \item \textit{Oui, son salut est proche de ceux qui le craignent, et la gloire habitera de nouveau sur notre terre.}
        \item \textit{La grâce et la vérité vont se rencontrer : la justice et la paix s'embrasseront.}
        \item \textit{La vérité germera de la terre, et la justice regardera du haut du ciel.}
        \item \textit{Le Seigneur nous accordera ses faveurs, et notre terre donnera son fruit.} 
        \item \textit{La justice marchera devant lui, et tracera le chemin de ses pas.}
      \end{enumerate}
    \end{footnotesize}
  \end{multicols}

  \medskip
  \par Le Cantique du prophète Habacuc est comme un résumé allégorique et prophétique de tous les mystères de la vie de Notre-Seigneur, depuis son incarnation jusqu'à son triomphe.\\
  Il est juste, en effet, de considérer l'œvre de la rédemption toute entière, en ce jour où la partie la plus importante s'accomplit.
  \medskip
  
  % ===== DEBUT Antienne =========
  \gresetinitiallines{1}
  \greillumination{\initfamily\fontsize{11mm}{11mm}\selectfont D}
  \gregorioscore{antiennes/an--dum_conturbata--solesmes}
  \begin{center}
    \footnotesize{
      \textit{Quand mon âme sera troublée, Seigner, vous vous souviendrez de votre miséricorde.}
    }
  \end{center}
  % ===== FIN Antienne ===========

  % ===== DEBUT psaume ===========
  % gresetinitiallines : avec le parametre à 0, supprime l'ornement
  \begin{center}
    \large{Cantique d'Habacuc.}\\
    \footnotesize{
      \textit{Chap. 3, 2-19}
    }
  \end{center}

  \gresetinitiallines{0}
  \gregorioscore{psaumes/cantique-habacuc-If}

  \begin{enumerate}[label=\textcolor{red}{\arabic*}]
    \setcounter{enumi}{1}
    \item Dómine, \textbf{o}pus \textbf{tu}um,\textcolor{red}{~*} in médio annórum viví\textit{fi}\textit{ca} \textbf{il}lud:

    \item In médio annórum \textbf{no}tum \textbf{fá}cies:\textcolor{red}{~*} cum irátus fúeris, misericórdiæ \textit{re}\textit{cor}\textbf{dá}beris.

    \item Deus ab \textbf{Aus}tro \textbf{vé}niet,\textcolor{red}{~*} et sanctus de \textit{mon}\textit{te} \textbf{Pha}ran:

    \item Opéruit cælos \textbf{gló}ria \textbf{e}jus:\textcolor{red}{~*} et laudis ejus ple\textit{na} \textit{est} \textbf{ter}ra.

    \item Splendor ejus \textbf{ut} lux \textbf{e}rit:\textcolor{red}{~*} córnua in má\textit{ni}\textit{bus} \textbf{e}jus:

    \item Ibi abscóndita est forti\textbf{tú}do \textbf{e}jus:\textcolor{red}{~*} ante fáciem \textit{e}\textit{jus} \textbf{i}bit mors.

    \item Et egrediétur diábolus ante \textbf{pe}des \textbf{e}jus.\textcolor{red}{~*} Stetit, et men\textit{sus} \textit{est} \textbf{ter}ram.

    \item Aspéxit, et dis\textbf{sól}vit \textbf{gen}tes:\textcolor{red}{~*} et contríti sunt \textit{mon}\textit{tes} \textbf{sǽ}culi.

    \item Incurváti sunt \textbf{col}les \textbf{mun}di,\textcolor{red}{~*} ab itinéribus æterni\textit{tá}\textit{tis} \textbf{e}jus.

    \item Pro iniquitáte vidi tentória \textbf{Æ}thi\textbf{ó}piæ,\textcolor{red}{~*} turbabúntur pelles \textit{ter}\textit{ræ} \textbf{Má}dian.

    \item Numquid in flumínibus i\textbf{rá}tus es, \textbf{Dó}mine?\textcolor{red}{~*}  aut in flumínibus furor tuus? vel in mari indigná\textit{ti}\textit{o} \textbf{tu}a?

    \item Qui ascéndes super \textbf{e}quos \textbf{tu}os:\textcolor{red}{~*} et quadrígæ tu\textit{æ} \textit{sal}\textbf{vá}tio.

    \item Súscitans suscitábis \textbf{ar}cum \textbf{tu}um:\textcolor{red}{~*} juraménta tríbubus \textit{quæ} \textit{lo}\textbf{cú}tus es.

    \item Flúvios scindes terræ:\textcolor{red}{~†} vidérunt te, et dolu\textbf{é}runt \textbf{mon}tes:\textcolor{red}{~*} gurges a\textit{quá}\textit{rum} \textbf{tráns}iit.

    \item Dedit abýssus \textbf{vo}cem \textbf{su}am:\textcolor{red}{~*} altitúdo manus su\textit{as} \textit{le}\textbf{vá}vit.

    \item Sol, et luna stetérunt in habi\textbf{tá}culo \textbf{su}o,\textcolor{red}{~*} in luce sagittárum tuárum, ibunt in splendóre fulgurántis \textit{has}\textit{tæ} \textbf{tu}æ.

    \item In frémitu concul\textbf{cá}bis \textbf{ter}ram:\textcolor{red}{~*} et in furóre obstupefá\textit{ci}\textit{es} \textbf{gen}tes.

    \item Egréssus es in salútem \textbf{pó}puli \textbf{tu}i:\textcolor{red}{~*} in salútem cum \textit{Chris}\textit{to} \textbf{tu}o.

    \item Percussísti caput de \textbf{do}mo \textbf{ím}pii:\textcolor{red}{~*} denudásti fundaméntum ejus us\textit{que} \textit{ad} \textbf{col}lum.

    \item Maledixísti sceptris ejus,\textcolor{red}{~†} cápiti bella\textbf{tó}rum \textbf{e}jus,\textcolor{red}{~*}  veniéntibus ut turbo ad \textit{di}\textit{sper}\textbf{gén}dum me.

    \item Exsultáti\textbf{o} e\textbf{ó}rum\textcolor{red}{~*} sicut ejus, qui dévorat páuperem \textit{in} \textit{abs}\textbf{cón}dito.

    \item Viam fecísti in mari \textbf{e}quis \textbf{tu}is,\textcolor{red}{~*} in luto aquá\textit{rum} \textit{mul}\textbf{tá}rum.

    \item Audívi, et conturbátus est \textbf{ven}ter \textbf{me}us:\textcolor{red}{~*} a voce contremuérunt lá\textit{bi}\textit{a} \textbf{me}a.

    \item Ingrediátur putrédo in \textbf{ós}sibus \textbf{me}is,\textcolor{red}{~*} et sub\textit{ter} \textit{me} \textbf{scá}teat.

    \item Ut requiéscam in die tribu\textbf{la}ti\textbf{ó}nis:\textcolor{red}{~*} ut ascéndam ad pópulum ac\textit{cínc}\textit{tum} \textbf{nos}trum.

    \item Ficus enim \textbf{non} flo\textbf{ré}bit:\textcolor{red}{~*} et non erit ger\textit{men} \textit{in} \textbf{ví}neis.

    \item Mentiétur \textbf{o}pus o\textbf{lí}væ:\textcolor{red}{~*} et arva non áf\textit{fe}\textit{rent} \textbf{ci}bum.

    \item Abscindétur de o\textbf{ví}li \textbf{pe}cus:\textcolor{red}{~*} et non erit arméntum \textit{in} \textit{præ}\textbf{sé}pibus.

    \item Ego autem in Dómi\textbf{no} gau\textbf{dé}bo:\textcolor{red}{~*} et exsultábo in Deo \textit{Je}\textit{su} \textbf{me}o.

    \item Deus Dóminus forti\textbf{tú}do \textbf{me}a:\textcolor{red}{~*} et ponet pedes meos qua\textit{si} \textit{cer}\textbf{vó}rum.

    \item Et super excélsa mea de\textbf{dú}cet me \textbf{vic}tor\textcolor{red}{~*} in psal\textit{mis} \textit{ca}\textbf{nén}tem.
  \end{enumerate}
  %  Répetition de l'Antienne
  \grecommentary{\textit{Reprise de l'Antienne.}}
  \gabcsnippet{(c4) Dum(d) con(d)tur(dc~)bá(f)ta(g') fú(f)e(gh)rit(h.) (,) á(ixi')ni(h)ma(g') me(g)a,(g') Dó(h)mi(fe)ne,(d.) (;) mi(gg)se(e')ri(g)cór(h')di(g)æ(f.) (,) me(e_[oh:h]c)mor(e_[uh:l]f) e(d.)ris.(d.) (::)}

  \medskip
  \begin{multicols}{2}
    \begin{footnotesize}
      \begin{enumerate}[label=\textcolor{red}{\emph{\arabic*}}]
        \item \textit{Seigneur, j’ai ouï ce que vous m’avez fait entendre, et j’ai tremblé.}
        \item \textit{Seigneur, que votre grand ouvrage s’accomplisse au
        milieu des années.}
        \item \textit{Vous le ferez paraître au milieu des années : lorsque
        vous serez en colère, vous vous souviendrez de votre
        miséricorde.}
        \item \textit{Dieu viendra du côté du midi ; et le Saint, de la
        montagne de Pharan.}
        \item \textit{Sa gloire a couvert les cieux ; et toute la terre est
        pleine de ses louanges.}
        \item \textit{Sa splendeur sera comme une lumière : sa force sera
        dans ses mains.}
        \item \textit{C’est-là que la force est cachée ; la mort marchera
        devant sa face.}
        \item \textit{Et le diable fuira de devant ses pas : il s’est arrêté,
        et il a mesuré la terre.}
        \item \textit{Il a regardé et dissipé les nations, et réduit en
        poudre les montagnes du siècle.}
        \item \textit{Les collines du monde se sont courbées devant les
        démarches éternelles}
        \item \textit{J’ai vu les tentes de l’Ethiopie, à cause de l’iniquité :
        les pavillons de la terre de Madian sont dans le
        trouble.}
        \item \textit{Etes-vous en colère contre les fleuves ? Seigneur,
        votre fureur paraîtra-t-elle sur les rivières, ou votre
        indignation sur la mer ?}
        \item \textit{Vous qui monterez sur vos chevaux et sur vos chariots, pour le salut du peuple} 
        \item \textit{Vous préparerez, et vous disposerez vos arcs : vous
        accomplirez les serments faits aux tribus.}
        \item \textit{Vous diviserez les fleuves de la terre : les montagnes
        vous ont vu, et en ont gémi : le gouffre des eaux
        s’est écoulé.}
        \item \textit{L’abîme a fait entendre sa voix : sa profondeur a
        levé ses mains.}
        \item \textit{Le soleil et la lune se sont arrêtés dans leur demeure : ils marcheront à la lueur de vos flèches, et à
        la splendeur de votre lance foudroyante.}
        \item \textit{Vous foulerez la terre dans votre indignation ; et
        vous étonnerez les nations dans votre fureur.}
        \item \textit{Vous êtes sorti pour le salut de votre peuple, pour le
        sauver avec votre Christ.}
        \item \textit{Vous avez frappé le chef de la famille de l’impie :
        vous en avez sappé le fondement jusqu’au sommet.}
        \item \textit{Vous avez maudit son sceptre, et le chef de ses
        guerriers, qui venait comme un tourbillon pour me
        perdre.}
        \item \textit{Ils se réjouissaient, et étaient semblables à celui qui
        dévore en cachette le pauvre}
        \item \textit{Vous avez fait un passage dans la mer à vos chevax ; dans la boue des eaux abondantes.}
        \item \textit{Je l’ai entendu, et mes entrailles en ont été troublées : cette voie a fait trembler mes lèvres.}
        \item \textit{Que la pourriture pénètre dans mes os, et qu’elle
        me consume au-dedans.}
        \item \textit{Afin que je me repose au jour de la tribulation, et
        que j’aille à notre peuple disposé à marcher.}
        \item \textit{Car le figuier ne fleurira point ; et les vignes ne
        pousseront point.}
        \item \textit{Le fruit de l’olivier manquera : les campagnes
        n’apporteront point de fruit.}
        \item \textit{On enlèvera le bétail de la bergerie ; et il n’y aura
        plus de troupeaux dans les étables.}
        \item \textit{Pour moi, je me réjouirai dans le Seigneur : je serai
        pénétré de joie en Jésus mon Dieu.}
        \item \textit{Mon Dieu, mon Seigneur est ma force ; et il donnera à mes pieds la légèreté de ceux des cerfs}
        \item \textit{Et il me conduira victorieux sur les lieux élevés,
        pour y chanter des psaumes à son honneur.}
      \end{enumerate}
    \end{footnotesize}
  \end{multicols}

  \medskip

  \par Le Psaume 147 nous montre la Jérusalem nouvelle, l'Église fondée et défendue par Jésus-Christ, et nourrie de la fleur du froment, ou de l'Eucharistie, ce fruit précieux de sa Passion.
  \bigskip

  % ===== DEBUT Antienne =========
  \gresetinitiallines{1}
  \greillumination{\initfamily\fontsize{11mm}{11mm}\selectfont M}
  \gregorioscore{antiennes/an--memento_mei_domine--solesmes}
  \begin{center}
    \footnotesize{
      \textit{Souvenez-vous de moi, Seigneur, quand vous serez en votre Royaume.}
    }
  \end{center}
  % ===== FIN Antienne ===========

  % ===== DEBUT psaume ===========
  % gresetinitiallines : avec le parametre à 0, supprime l'ornement
  \begin{center}
    \large{Psaume 147.}\\
  \end{center}

  \gresetinitiallines{0}
  \gregorioscore{psaumes/psaume147-VIIIG}

  \begin{enumerate}[label=\textcolor{red}{\arabic*}]
    \setcounter{enumi}{1}
    \item Quóniam confortávit seras portárum tu\textbf{á}rum:\textcolor{red}{~*} benedíxit fíliis \textit{tu}\textit{is} \textbf{in} te.

    \item Qui pósuit fines tuos \textbf{pa}cem:\textcolor{red}{~*} et ádipe fruménti \textit{sá}\textit{ti}\textbf{at} te.

    \item Qui emíttit elóquium suum \textbf{ter}ræ:\textcolor{red}{~*} velóciter currit \textit{ser}\textit{mo} \textbf{e}jus.

    \item Qui dat nivem sicut \textbf{la}nam:\textcolor{red}{~*} nébulam sicut cí\textit{ne}\textit{rem} \textbf{spar}git.

    \item Mittit crystállum suam sicut buc\textbf{cél}las:\textcolor{red}{~*} ante fáciem frígoris ejus quis \textit{sus}\textit{ti}\textbf{né}bit?

    \item Emíttet verbum suum, et liquefáciet \textbf{e}a:\textcolor{red}{~*} flabit spíritus ejus, et \textit{flu}\textit{ent} \textbf{a}quæ.

    \item Qui annúntiat verbum suum \textbf{Ja}cob:\textcolor{red}{~*} justítias, et judícia \textit{su}\textit{a} \textbf{Is}raël.

    \item Non fecit táliter omni nati\textbf{ó}ni:\textcolor{red}{~*} et judícia sua non manifes\textit{tá}\textit{vit} \textbf{e}is.
  \end{enumerate}
  %  Répetition de l'Antienne
  \grecommentary{\textit{Reprise de l'Antienne.}}
  \gabcsnippet{(c4) Me(g')mén(h)to(f') me(g)i(g.) (,) Dó(h')mi(j)ne(i') De(j)us,(h.) (;) dum(g) vé(hi)ne(hg)ris(h'_) in(f) re(gh)gnum(h) tu(g.)um.(g.) (::)}

  \medskip
  \begin{multicols}{2}
    \begin{footnotesize}
      \begin{enumerate}[label=\textcolor{red}{\emph{\arabic*}}]
        \item \textit{Jérusalem, loue le Seigneur ; Sion, célèbre ton Dieu.}
        \item \textit{Il a consolidé les verrous de tes portes, il bénit tes fils dans tes murs.}
        \item \textit{Il assure la paix à tes frontières, il te rassasie de la fleur du froment.}
        \item \textit{Il envoie ses ordres à la terre ; sa parole court avec vitesse.}
        \item \textit{Il fait tomber la neige comme une blanche toison, il répand le givre comme de la cendre.}
        \item \textit{Il jette ses glaçons par morceaux ; qui peut tenir devant ses frimas ?}
        \item \textit{Il envoie sa parole, et il les fond ; son vent souffle, et les eaux recommencent à couler.}
        \item \textit{C'est lui qui a révélé sa parole à Jacob, ses lois et ses préceptes à Israël.}
        \item \textit{Il n'a pas agit de même pour les autres nations, il ne leur a pas fait connaitre ses préceptes.}
      \end{enumerate}
    \end{footnotesize}
  \end{multicols}

  \medskip
  \begin{center}
    \rule{4cm}{0.4pt}
  \end{center}
  \medskip

  \begin{center}
    \begin{footnotesize}
      \textcolor{red}{\textit{On ne dit ni Capitule ni Hymne.}}
      \textcolor{red}{\textit{On chante le verset debout.}}
    \end{footnotesize}
    \begin{minipage}{0.8\linewidth}
      \gresetinitiallines{0}
      % \grecommentary[10pt]{\textcolor{red}{\textit{On se lève.}}}
      \large
      \gabcsnippet{(c4)<c><v>\Vbar</v>.</c> Col(h)lo(h)cá(h)vit(h) me(h) in(i') ó(h)bscú(g.)ris(g.) (::) (Z) <c><v>\Rbar</v>.</c> Si(h)cut(h) mór(i')tu(h)os(h) sæ(g)cu(g)li.(g.)  (::) (Z)}
      \bigskip
      \normalsize
      \begin{center}
        \textit{\textcolor{red}{\Vbar.} Il m'a mis dans un lieu ténébreux.}\\
        \textit{\textcolor{red}{\Rbar.} Comme ceux qui sont morts depuis longtemps}
      \end{center}
    \end{minipage}
  \end{center}

  \normalsize

  \par Vient ensuite le cantique de Zacharie ; son accent de joie contraste avec les douleurs de la Passion. Cependant c'est à présent que les prophéties qui y sont contenues vont recevoir leur accomplissement ; le Seigneur rachète son peuple, le délivre de ses ennemis, illumine ceux qui sont dans les ombres de la mort, et leur apprend le chemin de la vie éternelle.

  \par \textcolor{red}{Au commencement du cantique \textit{Benedictus}, il ne reste sur le chandelier triangulaire que le seul cierge supérieur allumé. Pendant qu'on dit le cantique, on éteint un à un les six cierges placés sur l'autel (à partir du verset \textit{Ut sine timore}), de telle manière qu'au dernier verset on éteigne le dernier cierge ; on éteint aussi tous les luminaires de l'église.}

  \medskip

  \gresetinitiallines{1}
  \greillumination{\initfamily\fontsize{11mm}{11mm}\selectfont P}
  \gregorioscore{antiennes/an--posuerunt_super_caput--solesmes}
  \begin{center}
    \footnotesize{
      \textit{Le traître leur avait donné ce signal, en leur disant : Celui que je baiserai, c’est lui-même ; arrêtez-le.}
  }
  \end{center}

  \begin{center}
    \large{Cantique de Zacharie.}\\
    \small\textit{Luc, I, 68-79.}\\
    \normalsize
  \end{center}

  \gresetinitiallines{0}
  \gregorioscore{psaumes/zacharie-Ig}
  
  \begin{enumerate}[label=\textcolor{red}{\arabic*}]
    \setcounter{enumi}{1}
    \item Et eréxit cornu sa\textbf{lú}tis \textbf{no}bis:\textcolor{red}{~*} in domo David, pú\textit{e}\textit{ri} \textbf{su}i.

    \item Sicut locútus est per \textbf{os} sanc\textbf{tó}rum,\textcolor{red}{~*} qui a sǽculo sunt, prophe\textit{tá}\textit{rum} \textbf{e}jus:

    \item Salútem ex ini\textbf{mí}cis \textbf{nos}tris,\textcolor{red}{~*} et de manu ómnium, \textit{qui} \textit{o}\textbf{dé}runt nos.

    \item Ad faciéndam misericórdiam cum \textbf{pá}tribus \textbf{nos}tris:\textcolor{red}{~*} et memorári testaménti \textit{su}\textit{i} \textbf{sanc}ti.

    \item Jusjurándum, quod jurávit ad Abraham \textbf{pa}trem \textbf{nos}trum,\textcolor{red}{~*} datú\textit{rum} \textit{se} \textbf{no}bis:

    \item Ut sine timóre, de manu inimicórum nostrórum \textbf{li}be\textbf{rá}ti,\textcolor{red}{~*} servi\textit{á}\textit{mus} \textbf{il}li.

    \item In sanctitáte, et justítia \textbf{co}ram \textbf{ip}so,\textcolor{red}{~*} ómnibus di\textit{é}\textit{bus} \textbf{nos}tris.

    \item Et tu, puer, Prophéta Altíssi\textbf{mi} vo\textbf{cá}beris:\textcolor{red}{~*} præíbis enim ante fáciem Dómini, paráre \textit{vi}\textit{as} \textbf{e}jus:

    \item Ad dandam sciéntiam salútis \textbf{ple}bi \textbf{e}jus:\textcolor{red}{~*} in remissiónem peccató\textit{rum} \textit{e}\textbf{ó}rum:

    \item Per víscera misericórdiæ \textbf{De}i \textbf{nos}tri:\textcolor{red}{~*} in quibus visitávit nos, óri\textit{ens} \textit{ex} \textbf{al}to:

    \item Illumináre his, qui in ténebris, et in umbra \textbf{mor}tis \textbf{se}dent:\textcolor{red}{~*} ad dirigéndos pedes nostros in \textit{vi}\textit{am} \textbf{pa}cis.
  \end{enumerate}
  \smallskip
  \grecommentary{\textit{Reprise de l'Antienne.}}
  \gabcsnippet{(c4) Po(c)su(d)é(ixdh'!iv)runt(h'_) (,) su(h)per(h') ca(h)put(h') e(h)jus(ixhg/hiHG'g) (,) cau(f)sam(g) i(h')psí(g)us(fe) scrip(fg)tam:(d.) (;) Je(d)sus(dgfg) Na(h)za(g)ré(fe)nus,(f_g_F_D.) (,) Rex(e_f) Ju(g)dae(fe)ó(d.)rum.(d.) (::)}

  \begin{multicols}{2}
    \begin{footnotesize}
      \begin{enumerate}[label=\textcolor{red}{\emph{\arabic*}}]
        \item \textit{Béni soit le Seigneur le Dieu d’Israël ; parce qu’il a visité et racheté son peuple.}
        \item \textit{Et qu’il a suscité un puissant Sauveur, dans la
        maison de son serviteur David,}
        \item \textit{Ainsi qu’il l’avait promis par la bouche de ses
        saints Prophètes, qui ont vécu dans les siècles passés.}
        \item \textit{De nous délivrer de nos ennemis, et des mains de
        tous ceux qui nous haïssent ;}
        \item \textit{En usant de miséricorde envers nos pères, et en se
        souvenant de sa sainte alliance :}
        \item \textit{Suivant la promesse faite avec serment à Abraham notre père, qu’il se donnerait à nous,}
        \item \textit{Afin qu’étant délivrés de la main de nos ennemis,
        nous le servions sans crainte,}
        \item \textit{Dans la sainteté et la justice, nous tenants en sa
        présence tous les jours de notre vie.}
        \item \textit{Et vous petits enfants ; vous serez appelé le Prophète du Très-Haut : vous marcherez devant la face du Seigneur, pour lui préparer ses voies ;}
        \item \textit{En donnant à son peuple la connaissance du salut, pour la rémission de leurs péchés,}
        \item \textit{Par les entrailles de la miséricorde de notre Dieu,
        qui a fait qu’un soleil levant nous a visités d’enhaut,}
        \item \textit{Pour éclairer ceux qui sont dans les ténèbres et
        dans l’ombre de la mort, et pour conduire nos
        pieds dans le chemin de la paix.}
      \end{enumerate}
    \end{footnotesize}
  \end{multicols}

  \medskip

  \begin{center}
    \begin{footnotesize}
      \textcolor{red}{\textit{Après la répétition de l'Antienne à Benedictus, on chante à genoux :}}
    \end{footnotesize}
  \end{center}

  \gresetinitiallines{1}
  \greillumination{\initfamily\fontsize{11mm}{11mm}\selectfont C}
  \gregorioscore{antiennes/an--christus_factus_est-vendredi}
  \normalsize

  \begin{center}
    \textit{Le Christ s’est fait pour nous obéissant jusqu’à la mort.}\\
    \textit{Et la mort de la croix.}\\
  \end{center}

  \medskip

  \par \textcolor{red}{Après l'Antienne \textit{Christus factus est}, on dit le \textit{Pater noster} entièrement en silence.}\\
  \medskip
  On ajoute, sans dire \textit{Orémus}, l'oraison suivante :

  \setlength{\columnsep}{2pc}
  \def\columnseprulecolor{\color{red}}
  \setlength{\columnseprule}{0.4pt}

  \begin{multicols}{2}
    \par Réspice, quæsumus Dómine, super
    hanc famíliam tuam, pro qua Dóminus
    noster Jesus Christus non dubitávit
    mánibus tradi nocéntium, et crucis
    subíre torméntum :
    % \par \hspace{\fill}
    \columnbreak
    \par \textit{Nous vous prions, Seigneur, de regarder en pitié votre famille, pour laquelle notre Seigneur JésusChrist n’a point refusé de se livrer entre les mains des méchants, et de souffrir le supplice de la croix ;}
  \end{multicols}
  \setlength\columnseprule{0pt}
  \setlength{\columnsep}{0pc}

  \medskip
  \par On récite ensuite la conclusion : 

  \setlength{\columnsep}{2pc}
  \def\columnseprulecolor{\color{red}}
  \setlength{\columnseprule}{0.4pt}

  \begin{multicols}{2}
    \par Qui tecum vivit et regnat...
    % \par \hspace{\fill}
    \columnbreak
    \par \textit{Lui qui vit et règne avec vous...}
  \end{multicols}
  \setlength\columnseprule{0pt}
  \setlength{\columnsep}{0pc}

  \smallskip
  \par \textcolor{red}{On fait ensuite grand bruit (symbole qui figure le désordre de la nature à la mort du Sauveur, Lumière du monde.). Puis, tous se lèvent et se retirent.}

\end{document}