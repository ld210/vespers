% !TeX program = lualatex
\documentclass[12pt, a4paper]{article}
\usepackage{fullpage}
\usepackage{subfiles}
\usepackage{fontspec}
\usepackage{libertine}
\usepackage{xcolor}
\usepackage{GotIn}
\usepackage{geometry}
\usepackage{multicol}
\usepackage{multicolrule}
\usepackage{graphicx}
\usepackage{enumitem}
\usepackage{setspace}
\usepackage[autocompile]{gregoriotex}

% \geometry{top=1cm, bottom=1cm, right=1cm, left=1cm}

\definecolor{red}{HTML}{C70039}
% \input GoudyIn.fd
% \newcommand*\initfamily{\usefont{U}{GoudyIn}{xl}{n}}

\input Acorn.fd
\newcommand*\initfamily{\usefont{U}{Acorn}{xl}{n}}
% cette ligne ajoute de l'espace entre les portées
% \grechangedim{baselineskip}{60pt}{scalable}
\begin{document}

  \gresetlinecolor{gregoriocolor}

  \begin{center}
    \large AU DEUXIÈME NOCTURNE.\\
  \end{center}
  \medskip
  \par Le Psaume suivant fut composé pour accompagné l'entrée de l'arche d'alliance dans Sion. Bientôt Jésus Christ, l'arche de la nouvelle alliance, triomphera de la mort et rentrera victorieux dans son royaume, reçu par les princes de la cour céleste.

  \newpage

  % ===== DEBUT Antienne =========
  \gresetinitiallines{1}
  \greillumination{\initfamily\fontsize{11mm}{11mm}\selectfont E}
  \gregorioscore{antiennes/an--elevamini--solesmes_1961}
  \begin{center}
    \footnotesize{
      \textit{Portes éternelles, ouvrez-vous, et le Roi de gloire entrera.}
    }
  \end{center}
  % ===== FIN Antienne ===========

  % ===== DEBUT psaume ===========
  % gresetinitiallines : avec le parametre à 0, supprime l'ornement
  \begin{center}
    \large{Psaume 23.}\\
  \end{center}

  \gresetinitiallines{0}
  \gregorioscore{psaumes/psaume23-Va}

  \begin{enumerate}[label=\textcolor{red}{\arabic*}]
    \setcounter{enumi}{1}
    \item Quia ipse super mária fundávit \textbf{e}um:\textcolor{red}{~*} et super flúmina præpa\textbf{rá}vit \textbf{e}um.

    \item Quis ascéndet in montem \textbf{Dó}mini?\textcolor{red}{~*} aut quis stabit in loco \textbf{sanc}to \textbf{e}jus?

    \item Innocens mánibus et mundo corde,\textcolor{red}{~†} qui non accépit in vano ánimam \textbf{su}am,\textcolor{red}{~*} nec jurávit in dolo \textbf{pró}ximo \textbf{su}o.

    \item Hic accípiet benedictiónem a \textbf{Dó}mino:\textcolor{red}{~*} et misericórdiam a Deo, salu\textbf{tá}ri \textbf{su}o.

    \item Hæc est generátio quæréntium \textbf{e}um,\textcolor{red}{~*} quæréntium fáciem \textbf{De}i \textbf{Ja}cob.

    \item Attóllite portas, príncipes, vestras,\textcolor{red}{~†} et elevámini, portæ æter\textbf{ná}les:\textcolor{red}{~*} et intro\textbf{í}bit Rex \textbf{gló}riæ.

    \item Quis est iste Rex glóriæ?\textcolor{red}{~†} Dóminus fortis et \textbf{pot}ens:\textcolor{red}{~*} Dóminus \textbf{pot}ens in \textbf{prǽ}lio.

    \item Attóllite portas, príncipes, vestras,\textcolor{red}{~†} et elevámini, portæ æter\textbf{ná}les:\textcolor{red}{~*} et intro\textbf{í}bit Rex \textbf{gló}riæ.

    \item Quis est iste Rex \textbf{gló}riæ?\textcolor{red}{~*} Dóminus virtútum ipse \textbf{est} Rex \textbf{gló}riæ.
  \end{enumerate}
  %  Répetition de l'Antienne
  \grecommentary{\textit{Reprise de l'Antienne.}}
  \gabcsnippet{(c3) E(h)le(h)vá(h_i)mi(h)ni,(e.) (,) por(f')tæ(e) æ(f')ter(d)ná(f_h)les,(g.) (;) et(h) in(e')tro(f)í(hih)bit(hv_GE) Rex(f!gwh_f) gló(d')ri(d)æ.(d.) (::)}

  \bigskip
  % \begin{multicols}{2}
  %   \begin{footnotesize}
  %     \begin{enumerate}[label=\textcolor{red}{\emph{\arabic*}}]
  %       \item \textit{La terre, avec tout ce qu’elle renferme, appartient au Seigneur : tout l’univers, et tous ceux qui l’habitent, sont à lui.}
  %       \item \textit{Car c’est lui qui l’a fondée au-dessus des mers, et
  %       qui l’a préparée et élevée au-dessus des fleuves.}
  %       \item \textit{Qui montera sur la montagne du Seigneur, et qui
  %       s’arrêtera dans son lieu saint ?}
  %       \item \textit{Celui dont les mains sont innocentes et le cœur
  %       pur ; qui n’a pas reçu son âme en vain, et qui
  %       n’a point fait à son prochain des serments trompeurs.}
  %       \item \textit{Celui-là recevra la bénédiction du Seigneur, et il
  %       obtiendra miséricorde de Dieu son Sauveur.}
  %       \item \textit{C’est là la race de ceux qui le cherchent, et qui désirent de voir la face du Dieu de Jacob}
  %       \item \textit{Princes, ouvrez vos portes : portes éternelles, ouvrez-vous, et le Roi de gloire fera son entrée.}
  %       \item \textit{Qui est ce Roi de gloire ? C’est le Seigneur fort et puissant : le Seigneur puissant dans les combats}
  %       \item \textit{Princes, ouvrez vos portes : portes éternelles, ouvrez-vous, et le Roi de gloire fera son entrée.}
  %       \item \textit{Qui est ce Roi de gloire ? Le Seigneur des armées
  %       est lui-même ce Roi de gloire}
  %     \end{enumerate}
  %   \end{footnotesize}
  % \end{multicols}

  \bigskip

  \par Ce Psaume, qui déjà a été dit aux Matines d'hier, est répété aujourd'hui, dans un sens plus empreint de joie, car la victoire du Seigneur est proche ; il va habiter pour jamais dans la demeure du Seigneur.
  \newpage

  % ===== DEBUT Antienne =========
  \gresetinitiallines{1}
  \greillumination{\initfamily\fontsize{11mm}{11mm}\selectfont C}
  \gregorioscore{antiennes/an--credo_videre_(holy_saturday)--solesmes_1961}
  \begin{center}
    \footnotesize{
      \textit{Je crois de voir les biens du Seigneur, dans la terre des vivants.}
    }
  \end{center}
  % ===== FIN Antienne ===========

  % ===== DEBUT psaume ===========
  % gresetinitiallines : avec le parametre à 0, supprime l'ornement
  \begin{center}
    \large{Psaume 26.}\\
  \end{center}

  \gresetinitiallines{0}
  \gregorioscore{psaumes/psaume26-IVE}

  \begin{enumerate}[label=\textcolor{red}{\arabic*}]
    \setcounter{enumi}{1}
    \item Dóminus protéctor \textit{vi}\textit{tæ} \textbf{me}æ,\textcolor{red}{~*} a \textit{quo} \textit{tre}\textit{pi}\textbf{dá}bo?

    \item Dum apprópiant super \textit{me} \textit{no}\textbf{cén}tes,\textcolor{red}{~*} ut e\textit{dant} \textit{car}\textit{nes} \textbf{me}as:

    \item Qui tríbulant me ini\textit{mí}\textit{ci} \textbf{me}i,\textcolor{red}{~*} ipsi infirmáti sunt \textit{et} \textit{ce}\textit{ci}\textbf{dé}runt.

    \item Si consístant advér\textit{sum} \textit{me} \textbf{cas}tra,\textcolor{red}{~*} non ti\textit{mé}\textit{bit} \textit{cor} \textbf{me}um.

    \item Si exsúrgat advér\textit{sum} \textit{me} \textbf{prǽ}lium,\textcolor{red}{~*} in hoc \textit{e}\textit{go} \textit{spe}\textbf{rá}bo.

    \item Unam pétii a Dómino, \textit{hanc} \textit{re}\textbf{quí}ram,\textcolor{red}{~*} ut inhábitem in domo Dómini ómnibus dié\textit{bus} \textit{vi}\textit{tæ} \textbf{me}æ:

    \item Ut vídeam volup\textit{tá}\textit{tem} \textbf{Dó}mini,\textcolor{red}{~*} et vísi\textit{tem} \textit{tem}\textit{plum} \textbf{e}jus.

    \item Quóniam abscóndit me in taberná\textit{cu}\textit{lo} \textbf{su}o:\textcolor{red}{~*} in die malórum protéxit me in abscóndito taber\textit{ná}\textit{cu}\textit{li} \textbf{su}i.

    \item In petra \textit{ex}\textit{al}\textbf{tá}vit me:\textcolor{red}{~*} et nunc exaltávit caput meum super in\textit{i}\textit{mí}\textit{cos} \textbf{me}os.

    \item Circuívi et immolávi in tabernáculo ejus hóstiam vocife\textit{ra}\textit{ti}\textbf{ó}nis:\textcolor{red}{~*} cantábo et psal\textit{mum} \textit{di}\textit{cam} \textbf{Dó}\textbf{mi}no.

    \item Exáudi, Dómine, vocem meam, qua cla\textit{má}\textit{vi} \textbf{ad} te:\textcolor{red}{~*} miserére me\textit{i}, \textit{et} \textit{ex}\textbf{áu}\textbf{di} me.

    \item Tibi dixit cor meum, exquisívit te fá\textit{ci}\textit{es} \textbf{me}a:\textcolor{red}{~*} fáciem tuam, Dó\textit{mi}\textit{ne}, \textit{re}\textbf{quí}ram.

    \item Ne avértas fáciem \textit{tu}\textit{am} \textbf{a} me,\textcolor{red}{~*} ne declínes in ira \textit{a} \textit{ser}\textit{vo} \textbf{tu}o.

    \item Adjútor \textit{me}\textit{us} \textbf{es}to:\textcolor{red}{~*} ne derelínquas me, neque despícias me, Deus, sa\textit{lu}\textit{tá}\textit{ris} \textbf{me}us.

    \item Quóniam pater meus, et mater mea de\textit{re}\textit{li}\textbf{qué}runt me:\textcolor{red}{~*} Dóminus \textit{au}\textit{tem} \textit{as}\textbf{súmp}\textbf{sit} me.

    \item Legem pone mihi, Dómine, in \textit{vi}\textit{a} \textbf{tu}a:\textcolor{red}{~*} et dírige me in sémitam rectam propter in\textit{i}\textit{mí}\textit{cos} \textbf{me}os.

    \item Ne tradíderis me in ánimas tribu\textit{lán}\textit{ti}\textbf{um} me:\textcolor{red}{~*} quóniam insurrexérunt in me testes iníqui, et mentíta est in\textit{í}\textit{qui}\textit{tas} \textbf{si}bi.

    \item Credo vidére \textit{bo}\textit{na} \textbf{Dó}mini\textcolor{red}{~*} in \textit{ter}\textit{ra} \textit{vi}\textbf{vén}\textbf{ti}um.

    \item Exspécta Dóminum, virí\textit{li}\textit{ter} \textbf{a}ge:\textcolor{red}{~*} et confortétur cor tuum, et \textit{sús}\textit{ti}\textit{ne} \textbf{Dó}\textbf{mi}num.
  \end{enumerate}
  %  Répetition de l'Antienne
  \grecommentary{\textit{Reprise de l'Antienne.}}
  \gabcsnippet{(c4) Cre(e)do(f') vi(d)dé(e_[oh:h]c)re(d) bo(e')na(f) Dó(g')mi(h)ni(g'_[oh:h]) (,) in(h) ter(f)ra(g') vi(g)vén(e)ti(de)um.(e.) (::)}

  \bigskip
  % \begin{multicols}{2}
  %   \begin{footnotesize}
  %     \begin{enumerate}[label=\textcolor{red}{\emph{\arabic*}}]
  %       \item \textit{Le Seigneur est ma lumière et mon salut : qui craindrai-je ?}
  %       \item \textit{Le Seigneur est le protecteur de ma vie ; qu’ai-je à
  %       redouter ?}
  %       \item \textit{Lorsque les méchants s’approchent de moi pour
  %       dévorer ma chair ;}
  %       \item \textit{Ces ennemis qui me persécutent, sont eux-mêmes
  %       affaiblis et tombés.}
  %       \item \textit{Quand une armée m’assiègerait, mon cœur ne serait point épouvanté.}
  %       \item \textit{Quand on me livrerait une bataille, j’espèrerais
  %       encore au Seigneur.}
  %       \item \textit{J’ai demandé une chose au Seigneur, je la continuerai ; c’est de passer tous les jours de ma vie
  %       dans la maison du Seigneur ;}
  %       \item \textit{Afin que je voie les délices du Seigneur, et que je
  %       visite son temple.}
  %       \item \textit{Car il m’a mis à couvert dans son tabernacle ; il
  %       m’a protégé au temps de mon affliction, dans le
  %       lieu le plus secret de son tabernacle.}
  %       \item \textit{Il m’a élevé sur la pierre, et dès-à-présent il a élevé ma tête au-dessus de mes ennemis.}
  %       \item \textit{J’ai tourné, et j’ai immolé dans son tabernacle
  %       une hostie avec des cris de joie : je chanterai, et
  %       j’offrirai des cantiques au Seigneur.}
  %       \item \textit{Seigneur, écoutez la voix que je pousse vers vous ;
  %       ayez pitié de moi, et exaucez-moi.}
  %       \item \textit{Mon cœur vous a dit : mon visage vous a cherché ;
  %       je rechercherai, Seigneur, votre face.} 
  %       \item \textit{Ne détournez point votre face de moi : ne vous retirez pas dans votre colère de votre serviteur.}
  %       \item \textit{Soyez mon protecteur, ne m’abandonnez pas ; ne
  %       me méprisez pas, vous qui êtes mon Sauveur et
  %       mon Dieu.}
  %       \item \textit{Car mon père et ma mère m’ont abandonné ;
  %       mais le Seigneur m’a pris sous sa protection.}
  %       \item \textit{Donnez-moi une loi, Seigneur, dans vos voies ; et
  %       conduisez-moi par un chemin droit, à cause de mes
  %       ennemis.}
  %       \item \textit{Ne me livrez pas au pouvoir de ceux qui me persécutent, parce que de faux témoins se sont élevés
  %       contre moi, et que l’iniquité s’est démentie ellemême.}
  %       \item \textit{Je crois que je verrai les biens du Seigneur, dans la
  %       terre des vivants.}
  %       \item \textit{Attendez le Seigneur, agissez avec courage : que
  %       votre cœur soit ferme, et attendez le Seigneur.}
  %     \end{enumerate}
  %   \end{footnotesize}
  % \end{multicols}

  \bigskip

  \par L'Église chante alors un cantique d'actions de grâces : Le Christ et l'humanité rachetée ont également raisons de louer Dieu qui les a arrachés à la mort et qui a changé leur douleur en joie.
  \medskip

  % ===== DEBUT Antienne =========
  \gresetinitiallines{1}
  \greillumination{\initfamily\fontsize{11mm}{11mm}\selectfont D}
  \gregorioscore{antiennes/an--domine_abstraxisti--solesmes_1961.1}
  \begin{center}
    \footnotesize{
      \textit{Seigneur, vous avez arraché mon âme des enfers.}
    }
  \end{center}
  % ===== FIN Antienne ===========

  % ===== DEBUT psaume ===========
  % gresetinitiallines : avec le parametre à 0, supprime l'ornement
  \begin{center}
    \large{Psaume 68.}\\
  \end{center}

  \gresetinitiallines{0}
  \gregorioscore{psaumes/psaume29-VIIIG}

  \begin{enumerate}[label=\textcolor{red}{\arabic*}]
    \setcounter{enumi}{1}
    \item Dómine, Deus meus, clamávi \textbf{ad} te,\textcolor{red}{~*} \textit{et} \textit{sa}\textbf{nás}ti me.

    \item Dómine, eduxísti ab inférno ánimam \textbf{me}am:\textcolor{red}{~*} salvásti me a descendénti\textit{bus} \textit{in} \textbf{la}cum.

    \item Psállite Dómino, sancti \textbf{e}jus:\textcolor{red}{~*} et confitémini memóriæ sancti\textit{tá}\textit{tis} \textbf{e}jus.

    \item Quóniam ira in indignatióne \textbf{e}jus:\textcolor{red}{~*} et vita in volun\textit{tá}\textit{te} \textbf{e}jus.

    \item Ad vésperum demorábitur \textbf{fle}tus:\textcolor{red}{~*} et ad matutí\textit{num} \textit{læ}\textbf{tí}tia.

    \item Ego autem dixi in abundántia \textbf{me}a:\textcolor{red}{~*} Non movébor \textit{in} \textit{æ}\textbf{tér}num.

    \item Dómine, in voluntáte \textbf{tu}a,\textcolor{red}{~*} præstitísti decóri me\textit{o} \textit{vir}\textbf{tú}tem.

    \item Avertísti fáciem tuam \textbf{a} me,\textcolor{red}{~*} et factus sum \textit{con}\textit{tur}\textbf{bá}tus.

    \item Ad te, Dómine, cla\textbf{má}bo:\textcolor{red}{~*} et ad Deum meum \textit{de}\textit{pre}\textbf{cá}bor.

    \item Quæ utílitas in sánguine \textbf{me}o,\textcolor{red}{~*} dum descéndo in cor\textit{rup}\textit{ti}\textbf{ó}nem?

    \item Numquid confitébitur tibi \textbf{pul}vis,\textcolor{red}{~*} aut annuntiábit veri\textit{tá}\textit{tem} \textbf{tu}am?

    \item Audívit Dóminus, et misértus est \textbf{me}i:\textcolor{red}{~*} Dóminus factus est ad\textit{jú}\textit{tor} \textbf{me}us.

    \item Convertísti planctum meum in gáudium \textbf{mi}hi:\textcolor{red}{~*} conscidísti saccum meum, et circumdedísti \textit{me} \textit{læ}\textbf{tí}tia:

    \item Ut cantet tibi glória mea, et non com\textbf{pún}gar:\textcolor{red}{~*} Dómine, Deus meus, in ætérnum confi\textit{té}\textit{bor} \textbf{ti}bi.
  \end{enumerate}
  %  Répetition de l'Antienne
  \grecommentary{\textit{Reprise de l'Antienne.}}
  \gabcsnippet{(c4) Do(g)mi(g)ne,(g') ab(h)stra(h')xí(g)sti(gf) (,) ab(g_[uh:l]h) ín(ji)fe(h_j)ris(i_[oh:h]h) á(g')ni(h)mam(h) me(g.)am.(g.) (::)}

  \bigskip
  % \begin{multicols}{2}
  %   \begin{footnotesize}
  %     \begin{enumerate}[label=\textcolor{red}{\emph{\arabic*}}]
  %       \item \textit{Je vous glorifierai, Seigneur, parce que vous m’avez mis sous votre garde ; et que vous ne m’avez point rendu le jouet de mes ennemis.}
  %       \item \textit{Seigneur mon Dieu, j’ai crié vers vous, et vous
  %       m’avez guéri.}
  %       \item \textit{Seigneur, vous avez délivré mon âme de l’enfer ;
  %       vous m’avez préservé d’entre ceux qui descendent
  %       dans la fosse.}
  %       \item \textit{Chantez des cantiques au Seigneur, vous qui êtes
  %       ses saints ; et célébrez la mémoire de sa sainteté ;}
  %       \item \textit{Parce que sa colère produit son indignation, et
  %       que la vie est dans sa volonté.}
  %       \item \textit{Nous serons le soir dans les larmes, et le matin
  %       dans la joie.}
  %       \item \textit{Pour moi, j’ai dit dans mon abondance : je ne serai jamais ébranlé.}
  %       \item \textit{Seigneur, par votre volonté, je suis affermi dans
  %       mon éclat et ma splendeur.}
  %       \item \textit{Vous avez détourné de moi votre visage, et je suis
  %       tombé dans le trouble.}
  %       \item \textit{Seigneur, je crierai vers vous, et j’adresserai mes prières à mon Dieu.}
  %       \item \textit{De quelle utilité sera mon sang, si je descends dans
  %       la corruption ?}
  %       \item \textit{La poussière chantera-t-elle vos louanges, ou annoncera-t-elle vos vérités ?}
  %       \item \textit{Le Seigneur m’a entendu, et il a eu compassion
  %       de moi : le Seigneur s’est fait mon protecteur.} 
  %       \item \textit{Vous avez changé mes pleurs en joie : vous avez
  %       déchiré le sac que je portais, et vous m’avez comblé de joie.}
  %       \item \textit{Afin que je vous chante au milieu de ma gloire,
  %       et sans être atteint de douleur : Seigneur mon
  %       Dieu, je vous louerai éternellement.}
  %     \end{enumerate}
  %   \end{footnotesize}
  % \end{multicols}

  \bigskip

  \begin{center}
    \begin{footnotesize}
      \textcolor{red}{\textit{On chante le verset debout.}}
    \end{footnotesize}
    \begin{minipage}{0.8\linewidth}
      \gresetinitiallines{0}
      \gabcsnippet{(c4)<c><v>\Vbar</v>.</c> Tu(h) au(h)tem,(h) Dó(h)mi(h)ne,(h) mi(h)se(h)ré(i')re(h) mé(g.)i.(g.) (::) (Z) <c><v>\Rbar</v>.</c> Et(h) re(h)sú(h)sci(h)ta(h) me(h), et(h) re(h)trí(i')bu(h)am(h) e(g.)is.(g.) (::)}
      \bigskip
      \normalsize
      \begin{center}
        \textit{\textcolor{red}{\Vbar.} Mais vous, Seigneur, ayez pitié de moi.}\\
        \textit{\textcolor{red}{\Rbar.} Ressuscitez-moi ; et je leur rendrai selon leurs méritent.}
      \end{center}
    \end{minipage}
  \end{center}
  \normalsize

  \begin{center}
    \rule{2cm}{0.4pt}
  \end{center}


  \par Les Leçons du deuxième Nocturne sont tirées comme les jours précédents des énarrations de Saint Augustin sur les Psaumes. Le saint Docteur montre aujourd'hui combien les desseins des pécheurs sont vains ; Dieu semble s'y préter quelquefois, mais il sait les déjouer dès qu'il lui plait : Le Seigneur a pu être crucifié parce qu'il l'a voulu ; mais il est ressuscité de même au temps qu'il a déterminé.

  \begin{center}
    \vspace*{\fill}
    \greseparator{3}{30}
    \vspace*{\fill}
  \end{center}

  \newpage


  \begin{center}
    \large Leçon IV.\\
    \normalsize
  \end{center}
  \medskip

  \setlength{\columnsep}{2pc}
  \def\columnseprulecolor{\color{red}}
  \setlength{\columnseprule}{0.4pt}

  \begin{multicols}{2}
    \begin{center}
      Ex Tractátu sancti Augustíni\\ Epíscopi super Psalmos.
    \end{center}
    \begin{spacing}{1.25}
      \par Accédet homo ad cor altum, et
      exaltábitur Deus. Illi dixérunt : Quis
      nos vidébit ? Defecérunt scrutantes
      scrutatiónes, consília mala. Accéssit
      homo ad ipsa consília, passus est se
      tenéri ut homo. Non enim tenerétur
      nisi homo, aut viderétur nisi homo, aut
      cæderétur nisi homo, aut crucifigerétur, aut morerétur nisi homo. Accéssit ergo homo ad illas omnes passiónes, quæ in illo nihil valérent, nisi
      esset homo. Sed si ille non esset homo, non liberarétur homo. Accéssit
      homo ad cor altum, id est, cor secrétum, objíciens aspéctibus humánis
      hóminem, servans intus Deum : celans
      formam Dei, in qua æquális est Patri,
      et ófferens formam servi, qua minor
      est Patre.
    \end{spacing}
    
    \columnbreak

    \begin{center}
      Du Traité de S. Augustin,\\ Evêque, sur les Psaumes.\\
      \begin{footnotesize}
        \textit{Ps. 63.}
      \end{footnotesize}
    \end{center}
    \par \textit{L’homme pénètrera dans la profondeur du cœur,
    et Dieu sera exalté. Ils ont dit : Qui nous verra ?
    Ils se sont épuisés en recherchant des inventions,
    et de mauvais conseils. L’homme est entré dans
    ces conseils ; il a permis d’être saisi comme un
    homme, car il n’a pu être pris, que comme un
    homme ; on ne le verrait, que comme un homme ;
    on ne le déchirerait de coups, que comme homme ;
    on ne le crucifierait, et il ne mourrait pas, s’il
    n’était homme. C’est donc l’homme qui est entré
    dans toutes ces passions, qui n’auraient aucune
    prise sur lui, s’il n’était homme. Mais s’il n’était
    homme, l’homme ne serait point racheté. L’homme
    a pénétré dans la profondeur du cœur, en présentant à leurs yeux son humanité, et en leur cachant sa divinité ; leur cachant la forme de Dieu,
    par laquelle il est égal au Père ; et présentant la
    forme de serviteur, par laquelle il est inférieur à son Père.}
  \end{multicols}
  \setlength\columnseprule{0pt}

  % \vspace*{\fill}

  % \newpage

  % \vspace*{\fill}

  \gresetinitiallines{1}
  \greillumination{\initfamily\fontsize{11mm}{11mm}\selectfont R}
  \gregorioscore{repons/re--recessit_pastor_noster--solesmes_2019}

  % \small
  % \begin{multicols}{2}
  %   \par\textcolor{red}{\textit{\Rbar}.} \textit{Notre Pasteur s’est retiré, cette source d’eau
  %   vive a disparue : le Soleil s’est obscurci à son départ.  \\ \textcolor{red}{*} Car celui qui tenait captif le premier
  %   homme, a été fait captif lui-même. Notre Sauveur
  %   a brisé aujourd’hui les portes et les serrures de la
  %   mort. }
  %   \columnbreak
  %   \par\textcolor{red}{\textit{\Vbar}.} \textit{ Il a véritablement détruit la clôture de l’enfer, et renversé la puissance du démon.\\
  %   \textcolor{red}{*} Car celui qui tenait captif le premier homme, a
  %   été fait captif lui-même. Notre Sauveur a brisé aujourd’hui les portes et les serrures de la mort.}
  % \end{multicols}
  % \normalsize

  % \vspace*{\fill}

  % \newpage

  % \smallskip
  \begin{center}
    \large Leçon V.\\
    \normalsize
  \end{center}
  \medskip

  \setlength{\columnsep}{2pc}
  \def\columnseprulecolor{\color{red}}
  \setlength{\columnseprule}{0.4pt}

  \begin{multicols}{2}
    \begin{spacing}{1.20}
      \par Quo perduxérunt illas scrutatiónes suas, quas perscrutántes defecérunt, ut étiam mórtuo Dómino et sepúlto, custódes pónerent ad sepúlcrum ? Dixérunt enim Piláto : Sedúctor ille : hoc appellabátur nómine Dóminus Jesus Christus, ad solátium servórum suórum, quando dicúntur seductóres : ergo illi Piláto : Sedúctor ille, ínquiunt, dixit adhuc vivens : Post tres dies resúrgam. Jube ítaque custodíri sepúlcrum usque in diem tértium, ne forte véniant discípuli ejus, et furéntur eum, et dicant plebi, Surréxit a mórtuis : et erit novíssimus error pejor prióre.
      \par Ait illis Pilátus : Habétis custódiam, ite, custodíte sicut scitis. Illi autem abeúntes, muniérunt sepúlcrum, signántes lápidem cum custódibus.
    \end{spacing}
    \columnbreak

    \par \textit{Jusqu'où n'ont-ils pas porté ces machinations dans lesquels ils se sont épuisés ? Au point de placer des gardes devant le sépulcre dans lequel on avait enseveli le corps du Seigneur. Car ils dirent à Pilate : Ce séducteur. Notre Seigneur Jésus Christ fut appelé ainsi, pour la consolation de ses serviteurs, quand ce nom leur est aussi donné. Ils dirent donc à Pilate : Étant encore en vie, ce séducteur a dit : Après trois jours, je ressusciterai. Ordonnez donc que l'on garde son sépulcre jusqu'au troisième jour, de peur que ses disciples viennent l'enlever, et ne disent au peuple : Il est ressuscité d'entre les morts. Cette dernière erreur serait pire que la première. }
    \par \textit{Pilate leur répondit : Vous avez une garde, allez, et gardez-le comme vous l'entendrez. Ils s'en allèrent donc, et s'assurèrent du sépulcre, en sellant la pierre et en y mettant des gardes.}
  \end{multicols}
  \setlength\columnseprule{0pt}

  \medskip

  % \newpage

  \gresetinitiallines{1}
  \greillumination{\initfamily\fontsize{11mm}{11mm}\selectfont O}
  \gregorioscore{repons/re--o_vos_omnes--solesmes_1961}

  \begin{footnotesize}
    \begin{multicols}{2}
      \par\textcolor{red}{\textit{\Rbar}.} \textit{ O vous tous qui passez par ce chemin, considérez et voyez, \\ \textcolor{red}{*} S’il y a une douleur semblable à la mienne.}
      \columnbreak
      \par\textcolor{red}{\textit{\Vbar}.} \textit{Considérez, peuples de toute la terre, et voyez ma douleur. \\
      \textcolor{red}{*} S’il y a une douleur semblable à la mienne.}
    \end{multicols}
  \end{footnotesize}

  \newpage

  \bigskip
  \begin{center}
    \large Leçon VI.\\
    \normalsize
  \end{center}
  \medskip

  \setlength{\columnsep}{2pc}
  \def\columnseprulecolor{\color{red}}
  \setlength{\columnseprule}{0.4pt}

  \begin{multicols}{2}
    \begin{spacing}{1.25}
      \par Posuérunt custódes mílites ad sepúlcrum. Concússa terra Dóminus resurréxit : mirácula facta sunt tália circa sepúlcrum, ut et ipsi mílites, qui custódes advénerant, testes fíerent, si vellent vera nuntiáre.
      \par Sed avarítia illa, quæ captivávit discípulum cómitem Christi, captivávit et mílitem custódem sepúlcri. Damus, ínquiunt, vobis pecúniam : et dícite, quia vobis dormiéntibus venérunt discípuli ejus, et abstulérunt eum. Vere defecérunt scrutántes scrutatiónes.
      \par Quid est quod dixísti, o infélix astútia ? Tantúmne déseris lucem consílii pietátis, et in profúnda versútiæ demérgeris, ut hoc dicas : Dícite, quia vobis dormiéntibus venérunt discípuli ejus, et abstulérunt eum ? 
      \par Dormiéntes testes ádhibes : vere tu ipse obdormísti, qui scrutándo tália defecísti.
    \end{spacing}

    \columnbreak

    \par \textit{
      Ils mirent des soldats à la garde du sépulchre ; la terre trembla, et le Seigneur ressuscita. Il se passa auprès du sépulcre de si grands miracles, que les soldats qui étaient venus pour le garder en seraient les témoins, s’ils voulaient dire la vérité. 
    }
    \par \textit{Mais l’avarice qui séduisit le disciple, compagnon du Christ, séduisit aussi les soldats et les gardes du sépulchre. Nous vous donnerons, dirent-ils de l’argent, et dites que ses disciples sont venus pendant que vous dormiez, et qu’ils l’ont enlevé. Véritablement leurs lumières sont demeurées courtes.}
    \par \textit{Que dites-vous, ô malheureuse ruse ? Renoncezvous tellement aux lumières d’une pieux conseil, et vous perdez-vous tellement dans le gouffre de la malice, que de dire : Dites que ses disciples sont venus pendant que vous dormiez, et l’ont enlevé ? }
    \par \textit{Vous produisez des témoins endormis ; vous êtes bien endormis vous-mêmes, de vous être épuisés dans de si vaines recherches.}
  \end{multicols}
  \setlength\columnseprule{0pt}

  \newpage

  \gresetinitiallines{1}
  \greillumination{\initfamily\fontsize{11mm}{11mm}\selectfont E}
  \gregorioscore{repons/re--ecce_quomodo_moritur--solesmes_1961}

  \small
  \begin{multicols}{2}
    \par\textcolor{red}{\textit{\Rbar}.} \textit{Voilà comment meurt le Juste, et personne n’y pense du fond du cœur. Les gens de bien sont mis à mort, et nul n’y fait attention. Le Juste a succombé sous l’iniquité ; \\ \textcolor{red}{*} Et sa mémoire sera conservée dans la paix.}
    \par\textcolor{red}{\textit{\Vbar}.} \textit{Il s’est tû comme un agneau devant celui qui le tond, et il n’a point ouvert la bouche. Il a été délivré de l’oppression et de l’injustice ; \\
    \textcolor{red}{*} Et sa mémoire sera conservée dans la paix. }
    \par\textcolor{red}{\textit{\Rbar}.} \textit{Voilà comment meurt le Juste, et personne n’y pense du fond du cœur. Les gens de bien sont mis à mort, et nul n’y fait attention. Le Juste a succombé sous l’iniquité ; \\ \textcolor{red}{*} Et sa mémoire sera conservée dans la paix.}
  \end{multicols}
  \normalsize

\end{document}