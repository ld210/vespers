% !TeX program = lualatex
\documentclass[12pt, a4paper]{article}
\usepackage{fullpage}
\usepackage{subfiles}
\usepackage{fontspec}
\usepackage{libertine}
\usepackage{xcolor}
\usepackage{GotIn}
\usepackage{geometry}
\usepackage{multicol}
\usepackage{multicolrule}
\usepackage{graphicx}
\usepackage{enumitem}
\usepackage[autocompile]{gregoriotex}

\geometry{top=1cm, bottom=1cm, right=1cm, left=1cm}
\pagestyle{empty}

\definecolor{red}{HTML}{C70039}
% \input GoudyIn.fd
% \newcommand*\initfamily{\usefont{U}{GoudyIn}{xl}{n}}

\input Acorn.fd
\newcommand*\initfamily{\usefont{U}{Acorn}{xl}{n}}
% cette ligne ajoute de l'espace entre les portées
% \grechangedim{baselineskip}{60pt}{scalable}

\begin{document}
\gresetlinecolor{gregoriocolor}
  % ===== DEBUT Antienne =========
  \gresetinitiallines{1}
  \greillumination{\initfamily\fontsize{11mm}{11mm}\selectfont Z}
  \gregorioscore{antiennes/an--zelus_domus_tuae--solesmes_1961}
  \begin{center}
    \footnotesize{
      \textit{
        Le zèle de votre maison m'a dévoré ; et sur moi sont tombés les opprobres de ceux qui s'attaquaient à vous.
      }
    }
  \end{center}
  % ===== FIN Antienne ===========

  % ===== DEBUT psaume ===========
  % gresetinitiallines : avec le parametre à 0, supprime l'ornement
  \begin{center}
    \large{Psaume 68.}\\
  \end{center}

  \gresetinitiallines{0}
  \gregorioscore{psaumes/psaume68-VIIIc}

  \begin{enumerate}[label=\textcolor{red}{\arabic*}]
    \setcounter{enumi}{1}
    \item Infíxus sum in limo pro\textbf{fún}di:\textcolor{red}{~*} et non \textit{est} \textit{sub}\textbf{stán}tia.

    \item Veni in altitúdinem \textbf{ma}ris:\textcolor{red}{~*} et tempés\textit{tas} \textit{de}\textbf{mér}sit me.

    \item Laborávi clamans, raucæ factæ sunt fauces \textbf{me}æ:\textcolor{red}{~*} defecérunt óculi mei, dum spero in \textit{De}\textit{um} \textbf{me}um.

    \item Multiplicáti sunt super capíllos cápitis \textbf{me}i,\textcolor{red}{~*} qui odé\textit{runt} \textit{me} \textbf{gra}tis.

    \item Confortáti sunt qui persecúti sunt me inimíci mei in\textbf{jús}te:\textcolor{red}{~*} quæ non rápui, tunc \textit{ex}\textit{sol}\textbf{vé}bam.

    \item Deus, tu scis insipiéntiam \textbf{me}am:\textcolor{red}{~*} et delícta mea a te non \textit{sunt} \textit{abs}\textbf{cón}dita.

    \item Non erubéscant in me qui exspéctant te, \textbf{Dó}mine,\textcolor{red}{~*} Dómi\textit{ne} \textit{vir}\textbf{tú}tum

    \item Non confundántur \textbf{su}per me\textcolor{red}{~*} qui quærunt te, \textit{De}\textit{us} \textbf{Is}raël.

    \item Quóniam propter te sustínui op\textbf{pró}brium:\textcolor{red}{~*} opéruit confúsio fá\textit{ci}\textit{em} \textbf{me}am.

    \item Extráneus factus sum frátribus \textbf{me}is,\textcolor{red}{~*} et peregrínus fíliis \textit{ma}\textit{tris} \textbf{me}æ.

    \item Quóniam zelus domus tuæ com\textbf{é}dit me:\textcolor{red}{~*} et oppróbria exprobrántium tibi ceci\textit{dé}\textit{runt} \textbf{su}per me.

    \item Et opérui in jejúnio ánimam \textbf{me}am:\textcolor{red}{~*} et factum est in oppró\textit{bri}\textit{um} \textbf{mi}hi.

    \item Et pósui vestiméntum meum ci\textbf{lí}cium:\textcolor{red}{~*} et factus sum illis \textit{in} \textit{pa}\textbf{rá}bolam.

    \item Advérsum me loquebántur, qui sedébant in \textbf{por}ta:\textcolor{red}{~*} et in me psallébant qui bi\textit{bé}\textit{bant} \textbf{vi}num.

    \item Ego vero oratiónem meam ad te, \textbf{Dó}mine:\textcolor{red}{~*} tempus beneplá\textit{ci}\textit{ti}, \textbf{De}us.

    \item In multitúdine misericórdiæ tuæ ex\textbf{áu}di me,\textcolor{red}{~*} in veritáte sa\textit{lú}\textit{tis} \textbf{tu}æ:

    \item Eripe me de luto, ut non in\textbf{fí}gar:\textcolor{red}{~*} líbera me ab iis, qui odérunt me, et de profún\textit{dis} \textit{a}\textbf{quá}rum.

    \item Non me demérgat tempéstas aquæ,\textcolor{red}{~†} neque absórbeat me pro\textbf{fún}dum:\textcolor{red}{~*} \\ \-\hspace{2cm} neque úrgeat super me púte\textit{us} \textit{os} \textbf{su}um.

    \item Exáudi me, Dómine, quóniam benígna est misericórdia \textbf{tu}a:\textcolor{red}{~*} \\ \-\hspace{2cm} secúndum multitúdinem miseratiónum tuárum ré\textit{spi}\textit{ce} \textbf{in} me.

    \item Et ne avértas fáciem tuam a púero \textbf{tu}o:\textcolor{red}{~*} quóniam tríbulor, velóci\textit{ter} \textit{ex}\textbf{áu}di me.

    \item Inténde ánimæ meæ, et líbera \textbf{e}am:\textcolor{red}{~*} propter inimícos meos \textit{é}\textit{ri}\textbf{pe} me.

    \item Tu scis impropérium meum, et confusiónem \textbf{me}am,\textcolor{red}{~*} et reverén\textit{ti}\textit{am} \textbf{me}am.

    \item In conspéctu tuo sunt omnes qui tríbu\textbf{lant} me:\textcolor{red}{~*} impropérium exspectávit cor meum, \textit{et} \textit{mi}\textbf{sé}riam.

    \item Et sustínui qui simul contristarétur, et non \textbf{fu}it:\textcolor{red}{~*} et qui consolarétur, et \textit{non} \textit{in}\textbf{vé}ni.

    \item Et dedérunt in escam \textbf{me}am fel:\textcolor{red}{~*} et in siti mea potavérunt \textit{me} \textit{a}\textbf{cé}to.

    \item Fiat mensa eórum coram ipsis in \textbf{lá}queum,\textcolor{red}{~*} et in retributiónes, \textit{et} \textit{in} \textbf{scán}dalum.

    \item Obscuréntur óculi eórum ne \textbf{ví}deant:\textcolor{red}{~*} et dorsum eórum sem\textit{per} \textit{in}\textbf{cúr}va.

    \item Effúnde super eos iram \textbf{tu}am:\textcolor{red}{~*} et furor iræ tuæ compre\textit{hén}\textit{dat} \textbf{e}os.

    \item Fiat habitátio eórum de\textbf{sér}ta:\textcolor{red}{~*} et in tabernáculis eórum non sit \textit{qui} \textit{in}\textbf{há}bitet.

    \item Quóniam quem tu percussísti, perse\textbf{cú}ti sunt:\textcolor{red}{~*} et super dolórem vúlnerum meórum \textit{ad}\textit{di}\textbf{dé}runt.

    \item Appóne iniquitátem super iniquitátem e\textbf{ó}rum:\textcolor{red}{~*} et non intrent in justí\textit{ti}\textit{am} \textbf{tu}am.

    \item Deleántur de libro vi\textbf{vén}tium:\textcolor{red}{~*} et cum justis \textit{non} \textit{scri}\textbf{bán}tur.

    \item Ego sum pauper et \textbf{do}lens:\textcolor{red}{~*} salus tua, De\textit{us}, \textit{su}\textbf{scé}pit me.

    \item Laudábo nomen Dei cum \textbf{cán}tico:\textcolor{red}{~*} et magnificábo e\textit{um} \textit{in} \textbf{lau}de:

    \item Et placébit Deo super vítulum no\textbf{vél}lum:\textcolor{red}{~*} córnua producén\textit{tem} \textit{et} \textbf{ún}gulas.

    \item Vídeant páuperes et læ\textbf{tén}tur:\textcolor{red}{~*} quǽrite Deum, et vivet á\textit{ni}\textit{ma} \textbf{ves}tra.

    \item Quóniam exaudívit páuperes \textbf{Dó}minus:\textcolor{red}{~*} et vinctos suos \textit{non} \textit{de}\textbf{spé}xit.

    \item Laudent illum cæli et \textbf{ter}ra,\textcolor{red}{~*} mare et ómnia reptíli\textit{a} \textit{in} \textbf{e}is.

    \item Quóniam Deus salvam fáciet \textbf{Si}on:\textcolor{red}{~*} et ædificabúntur civi\textit{tá}\textit{tes} \textbf{Ju}da.

    \item Et inhabitábunt \textbf{i}bi,\textcolor{red}{~*} et hereditáte ac\textit{quí}\textit{rent} \textbf{e}am.

    \item Et semen servórum ejus possidébit \textbf{e}am:\textcolor{red}{~*} et qui díligunt nomen ejus, habitá\textit{bunt} \textit{in} \textbf{e}a.
  \end{enumerate}

  \gresetheadercapture{commentary}{grecommentary}{}
  \gregorioscore{antiennes/an--zelus_domus_tuae--solesmes_1961}
  \medskip
  \begin{multicols}{2}
    \begin{footnotesize}
      \begin{enumerate}[label=\textcolor{red}{\emph{\arabic*}}]
        \item \textit{O mon Dieu, sauvez-moi, parce que les eaux ont pénétré jusques dans mon âme}
        \item \textit{Je suis enfoncé dans une boue profonde, où je ne
        trouve point de fermeté.}
        \item \textit{Je suis tombé dans la mer profonde ; et la tempête
        m’a submergé.}
        \item \textit{Je me suis fatigué en criant, ma gorge en a été enrouée ; mes yeux se sont fermés de faiblesse, tandis que j’espère en mon Dieu.}
        \item \textit{Ceux qui me haïssent sans sujet, se sont multipliés
        plus que les cheveux de ma tête.}
        \item \textit{Mes ennemis qui me persécutaient injustement, se
        sont fortifiés contre moi ; alors j’ai payé ce que je n’avais pas pris.}
        \item \textit{Mon Dieu, vous connaissez ma folie, et mes
        crimes ne vous sont point cachés.}
        \item \textit{Seigneur, souverain des vertus ; que ceux qui espèrent en vous ne rougissent point à cause de moi.}
        \item \textit{O Dieu d’Israël, que ceux qui vous cherchent,
        n’aient point de confusion à mon sujet.}
        \item \textit{Car c’est à cause de vous que j’ai souffert les opprobres, et que mon visage a été couvert de confusion.}
        \item \textit{Je suis devenu comme étranger à mes frères, et comme inconnu aux enfants de ma mère ;}
        \item \textit{Parce que le zèle de votre maison me dévore, et que les opprobres de ceux qui vous ont outragé, sont retombées sur moi.}
        \item \textit{Je me suis couvert d’un sac pendant mon jeûne, ce qui est devenu pour moi un sujet d’opprobre.} 
        \item \textit{J’ai pris pour vêtement un cilice, ce qui m’a encore rendu l’objet de leur railleries.}
        \item \textit{Ceux qui étaient assis à la porte, parlaient contre moi, et ceux qui buvaient du vin se moquaient de moi dans leurs chansons.}
        \item \textit{Mais pour moi, Seigneur, je vous adressais ma prière ; ô mon Dieu, voici le temps de votre bonté.}
        \item \textit{Exaucez-moi dans la multitude de votre miséricorde, et dans la vérité des promesses qui regardent mon salut.}
        \item \textit{Retirez-moi du bourbier, afin que je ne m’y enfonce pas davantage ; délivrez-moi de ceux qui me haïssent, et de la profondeur des eaux.}
        \item \textit{Que la tempête ne me submerge point, que je ne sois point enseveli dans l’abîme, et que la bouche du puits ne soit point fermée sur moi.}
        \item \textit{Seigneur, exaucez-moi, car votre miséricorde est bienfaisante et toute remplie de douceur ; regardez-moi favorablement selon la multitude de vos miséricordes.}
        \item \textit{Ne détournez pas votre visage de dessus votre serviteur ; exaucez- moi promptement, car je suis dans le trouble.}
        \item \textit{Soyez attentif sur mon âme, et délivrez-la ; sauvez-moi, à cause de mes ennemis.}
        \item \textit{Vous savez les opprobres où ils m’ont jeté ; vous voyez la confusion et l’ignominie dont je suis couvert.}
        \item \textit{Tous ceux qui me persécutent sont sous vos yeux ; mon cœur n’a envisagé que l’opprobre et la misère.}
        \item \textit{J’ai attendu que quelqu’un prit part à ma tristesse, et personne ne s’est présenté ; ou que quelqu’un me consolât, mais je n’en ai point trouvé}
        \item \textit{Ils m’ont donné du fiel pour ma nourriture, et ils m’ont présenté du vinaigre durant ma soif.}
        \item \textit{Que leur table soit devant eux comme un piège, et qu’elle leur soit une punition et une pierre de scandale.}
        \item \textit{Que leurs yeux s’obscurcissent, afin qu’ils ne voient point ; et que leurs dos soient toujours courbés.}
        \item \textit{Faites tomber sur eux votre colère, et que la fureur de votre indignation les accable.}
        \item \textit{Que leur maison devienne déserte, et que l’on ne trouve personne qui habite dans leurs tentes.}
        \item \textit{Car ils ont persécuté celui que vous avez frappé, et ils ont ajouté de nouvelles blessures à mes plaies.}
        \item \textit{Permettez qu’ils ajoutent iniquité sur iniquité, et qu’ils n’entrent point dans votre justice}
        \item \textit{Qu’ils soient effacés du livre des vivants, et qu’ils ne soient point écrits parmi les justes.}
        \item \textit{Je suis pauvre et affligé ; mais, mon Dieu, votre protection m’a soutenu.}
        \item \textit{Je louerai le nom de Dieu dans mes Cantiques, et je le glorifierai par mes louanges.}
        \item \textit{Elles seront plus agréables à Dieu que le sacrifice d’un jeune veau, dont les cornes et les ongles commencent à pousser.}
        \item \textit{Que les pauvres le voient et qu’ils se réjouissent ; cherchez Dieu, et votre âme vivra.}
        \item \textit{Car le Seigneur a exaucé les pauvres, et il n’a pas méprisé ceux qui étaient dans l’esclavage.}
        \item \textit{Que les cieux et la terre le louent, aussi bien que la mer et les animaux qu’elle renferme.}
        \item \textit{Parce que Dieu sauvera Sion, et que les villes de Juda seront bâties.}
        \item \textit{C’est là qu’ils habiteront, quand ils en seront mis
        en possession comme d’un héritage.}
        \item \textit{Et la postérité de ses serviteurs la possèdera, et ceux qui aiment son nom y feront leur demeure.}
      \end{enumerate}
    \end{footnotesize}
  \end{multicols}

  % ===== FIN psaume ===========

  \bigskip

  \begin{center}
    \rule{2cm}{0.4pt}
  \end{center}

  \par Le psalmiste continue à exprimer les plaintes du Sauveur délaissé et à prédire le châtiment réservé à ses ennemis.
  \medskip

  % ===== DEBUT Antienne =========
  \gresetinitiallines{1}
  \greillumination{\initfamily\fontsize{11mm}{11mm}\selectfont A}
  \gregorioscore{antiennes/an--avertantur_retrorsum--solesmes_1961}
  \begin{center}
    \footnotesize{
      \textit{
        Que ceux qui me veulent du mal, soient repoussés en arrière, et couverts de confusion.
      }
    }
  \end{center}
  % ===== FIN Antienne ===========

  % ===== DEBUT psaume ===========
  % gresetinitiallines : avec le parametre à 0, supprime l'ornement
  \begin{center}
    \large{Psaume 69.}\\
  \end{center}

  \gresetinitiallines{0}
  \gregorioscore{psaumes/psaume69-VIIIc}

  \begin{enumerate}[label=\textcolor{red}{\arabic*}]
    \setcounter{enumi}{1}
    \item Confundántur et revere\textbf{án}tur,\textcolor{red}{~*} qui quærunt á\textit{ni}\textit{mam} \textbf{me}am.

    \item Avertántur retrórsum, et eru\textbf{bés}cant,\textcolor{red}{~*} qui volunt \textit{mi}\textit{hi} \textbf{ma}la.

    \item Avertántur statim erube\textbf{scén}tes,\textcolor{red}{~*} qui dicunt mihi: \textit{Eu}\textit{ge}, \textbf{eu}ge.

    \item Exsúltent et læténtur in te omnes qui \textbf{quæ}runt te,\textcolor{red}{~*} et dicant semper: Magnificétur Dóminus: qui díligunt salu\textit{tá}\textit{re} \textbf{tu}um.

    \item Ego vero egénus, et \textbf{pau}per sum:\textcolor{red}{~*} Deus, \textit{ád}\textit{ju}\textbf{va} me.

    \item Adjútor meus, et liberátor meus \textbf{es} tu:\textcolor{red}{~*} Dómine, \textit{ne} \textit{mo}\textbf{ré}ris.

  \end{enumerate}

  \medskip

  \begin{multicols}{2}
    \begin{footnotesize}
      \begin{enumerate}[label=\textcolor{red}{\emph{\arabic*}}]
        \item \textit{O Dieu, venez à mon aide ; Seigneur, hâtez-vous de me secourir.}
        \item \textit{Que ceux qui en veulent à ma vie, soient confondus et couverts de honte.}
        \item \textit{Que ceux qui me veulent du mal, soient repoussés en arrière honteusement.}
        \item \textit{Que ceux qui me disent : Courage, courage, soient chassés avec confusion.}
        \item \textit{Que tous ceux qui vous cherchent, se réjouissent en vous, et soient comblés de joie ; et que ceux qui aiment le salut que vous donnez, disent toujours : Que le Seigneur soit glorifié.}
        \item \textit{Pour moi je suis pauvre et dans le besoin : ô Dieu, secourez-moi.}
        \item \textit{Vous êtes mon protecteur et mon libérateur : Seigneur, ne tardez pas davantage.}
      \end{enumerate}
    \end{footnotesize}
  \end{multicols}

  % ===== FIN psaume ===========

\bigskip

  \begin{center}
    \rule{2cm}{0.4pt}
  \end{center}

  \par Le troisième Psaume fut composé par David à l'occasion des persécutions qu'il eut a subir vers la fin de son règne. Le Messie se voit entouré d'ennemis furieux qui ne craignent pas de violer toutes les prescriptions de la loi pour obtenir sa mort ; ils disent que le Père céleste a abandonné son Fils pour le moment, et ils veulent en profiter. Pour lui, il ne cesse de poursuivre jusque sur la Croix sa mission de Docteur. La génération du peuple chrétien qui doit venir entendra sa voix, et apprendra à célébrer la puissance du bras divin, qui va paraître dans la prochaine résurection du Sauveur.
  \bigskip

  % ===== DEBUT Antienne =========
  \gresetinitiallines{1}
  \greillumination{\initfamily\fontsize{11mm}{11mm}\selectfont D}
  \gregorioscore{antiennes/an--deus_meus_eripe_me--solesmes_1961}
  \begin{center}
    \footnotesize{
      \textit{
        Mon Dieu, délivrez-moi de la main du pécheur
      }
    }
  \end{center}
  % ===== FIN Antienne ===========

  % ===== DEBUT psaume ===========
  % gresetinitiallines : avec le parametre à 0, supprime l'ornement
  \begin{center}
    \large{Psaume 70.}\\
  \end{center}

  \gresetinitiallines{0}
  \gregorioscore{psaumes/psaume70-VIIIc}

  \begin{enumerate}[label=\textcolor{red}{\arabic*}]
    \setcounter{enumi}{2}
    \item Esto mihi in Deum protectórem, et in locum mu\textbf{ní}tum:\textcolor{red}{~*} ut sal\textit{vum} \textit{me} \textbf{fá}cias.

    \item Quóniam firmaméntum \textbf{me}um,\textcolor{red}{~*} et refúgium \textit{me}\textit{um} \textbf{es} tu.

    \item Deus meus, éripe me de manu pecca\textbf{tó}ris,\textcolor{red}{~*} et de manu contra legem agéntis \textit{et} \textit{in}\textbf{í}qui:

    \item Quóniam tu es patiéntia mea, \textbf{Dó}mine:\textcolor{red}{~*} Dómine, spes mea a juven\textit{tú}\textit{te} \textbf{me}a.

    \item In te confirmátus sum ex \textbf{ú}tero:\textcolor{red}{~*} de ventre matris meæ tu es pro\textit{téc}\textit{tor} \textbf{me}us.

    \item In te cantátio mea semper:\textcolor{red}{~†} tamquam prodígium factus sum \textbf{mul}tis:\textcolor{red}{~*} \\ \-\hspace{2cm} et tu ad\textit{jú}\textit{tor} \textbf{for}tis.

    \item Repleátur os meum laude, ut cantem glóriam \textbf{tu}am:\textcolor{red}{~*} tota die magnitú\textit{di}\textit{nem} \textbf{tu}am.

    \item Ne projícias me in témpore senec\textbf{tú}tis:\textcolor{red}{~*} cum defécerit virtus mea, ne \textit{de}\textit{re}\textbf{lín}quas me.

    \item Quia dixérunt inimíci mei \textbf{mi}hi:\textcolor{red}{~*} et qui custodiébant ánimam meam, consílium fecé\textit{runt} \textit{in} \textbf{u}num.

    \item Dicéntes: Deus derelíquit eum,\textcolor{red}{~†} persequímini, et comprehéndite \textbf{e}um:\textcolor{red}{~*} \\ \-\hspace{2cm} quia non est \textit{qui} \textit{e}\textbf{rí}piat.

    \item Deus ne elongéris \textbf{a} me:\textcolor{red}{~*} Deus meus, in auxílium \textit{me}\textit{um} \textbf{ré}spice.

    \item Confundántur, et defíciant detrahéntes ánimæ \textbf{me}æ:\textcolor{red}{~*} \\ \-\hspace{2cm} operiántur confusióne et pudóre, qui quærunt \textit{ma}\textit{la} \textbf{mi}hi.

    \item Ego autem semper spe\textbf{rá}bo:\textcolor{red}{~*} et adjíciam super omnem \textit{lau}\textit{dem} \textbf{tu}am.

    \item Os meum annuntiábit justítiam \textbf{tu}am:\textcolor{red}{~*} tota die salu\textit{tá}\textit{re} \textbf{tu}um.

    \item Quóniam non cognóvi litteratúram,\textcolor{red}{~†} introíbo in poténtias \textbf{Dó}mini:\textcolor{red}{~*} \\ \-\hspace{2cm} Dómine, memorábor justítiæ tu\textit{æ} \textit{so}\textbf{lí}us.

    \item Deus, docuísti me a juventúte \textbf{me}a:\textcolor{red}{~*} et usque nunc pronuntiábo mirabí\textit{li}\textit{a} \textbf{tu}a.

    \item Et usque in senéctam et \textbf{sé}nium:\textcolor{red}{~*} Deus, ne \textit{de}\textit{re}\textbf{lín}quas me,

    \item Donec annúntiem bráchium \textbf{tu}um\textcolor{red}{~*} generatióni omni, \textit{quæ} \textit{ven}\textbf{tú}ra est:

    \item Poténtiam tuam, et justítiam tuam, Deus,\textcolor{red}{~†} usque in altíssima, quæ fecísti ma\textbf{gná}lia:\textcolor{red}{~*} \\ \-\hspace{2cm} Deus, quis sí\textit{mi}\textit{lis} \textbf{ti}bi?

    \item Quantas ostendísti mihi tribulatiónes multas et malas:\textcolor{red}{~†} et convérsus vivifi\textbf{cás}ti me:\textcolor{red}{~*} \\ \-\hspace{2cm} et de abýssis terræ íterum \textit{re}\textit{du}\textbf{xís}ti me:

    \item Multiplicásti magnificéntiam \textbf{tu}am:\textcolor{red}{~*} et convérsus conso\textit{lá}\textit{tus} \textbf{es} me.

    \item Nam et ego confitébor tibi in vasis psalmi veritátem \textbf{tu}am:\textcolor{red}{~*} Deus, psallam tibi in cíthara, \textit{Sanc}\textit{tus} \textbf{Is}raël.

    \item Exsultábunt lábia mea cum cantávero \textbf{ti}bi:\textcolor{red}{~*} et ánima mea, quam \textit{red}\textit{e}\textbf{mís}ti.

    \item Sed et lingua mea tota die meditábitur justítiam \textbf{tu}am:\textcolor{red}{~*} \\ \-\hspace{2cm} cum confúsi et revériti fúerint, qui quærunt \textit{ma}\textit{la} \textbf{mi}hi.

  \end{enumerate}

  \medskip
  
  \begin{multicols}{2}
    \begin{footnotesize}
      \begin{enumerate}[label=\textcolor{red}{\emph{\arabic*}}]
        \item \textit{Seigneur, j’ai espéré en vous, je ne serai pas confondu pour jamais :
        délivrez-moi par votre justice, et sauvez-moi.}
        \item \textit{Que votre oreille soit attentive pour m’écouter, et sauvez-moi}
        \item \textit{Soyez-moi un Dieu protecteur, et une place forte
        et bien munie pour me sauver}
        \item \textit{Parce que vous êtes toute ma force, et vous êtes
        mon refuge.}
        \item \textit{Mon Dieu, délivrez-moi de la main du pécheur,
        et de la main de l’homme injuste qui agit contre
        votre loi.}
        \item \textit{Car, Seigneur, vous êtes ma patience ; Seigneur,
        vous êtes mon espérance dès ma plus tendre jeunesse.}
        \item \textit{J’ai été affermi en vous dès ma naissance ; vous
        êtes mon protecteur dès le temps que j’étais dans le
        sein de ma mère.}
        \item \textit{Je chanterai toujours vos louanges : j’ai paru
        comme un prodige à plusieurs ; mais vous êtes un
        puissant protecteur.}
        \item \textit{Que ma bouche soit remplie de louanges, pour
        chanter tout le jour votre gloire et votre grandeur.}
        \item \textit{Ne me rebutez pas dans le temps de ma vieillesse,
        et ne m’abandonnez pas lorsque ma force sera affaiblie.}
        \item \textit{Car mes ennemis m’ont décrié ; et ceux qui gardaient mon âme, ont formé ensemble des complots contre moi,}
        \item \textit{En disant : Dieu l’a abandonné, poursuivez-le,
        et saisissez-vous de lui, car personne ne peut le
        délivrer.}
        \item \textit{O Dieu, ne vous éloignez pas de moi ; mon Dieu, regardez-moi pour me secourir}
        \item \textit{Que ceux qui me décrient par leurs médisances, soient confondus, et qu’ils périssent. Que ceux qui cherchent à me faire du mal, soient couverts de honte et de confusion.}
        \item \textit{Mais pour moi, j’espèrerai toujours, et je vous donnerai de nouvelles louanges.}
        \item \textit{Ma bouche annoncera votre justice, et publiera tout le jour que vous êtes le salut.}
        \item \textit{Car je n’ai pas la connaissance des lettres ; je considèrerai la puissance du Seigneur : Seigneur, je me souviendrai seulement de votre justice.}
        \item \textit{Mon Dieu, vous m’avez instruit dès ma jeunesse ; et jusqu’à maintenant je publierai vos merveilles.}
        \item \textit{Ne m’abandonnez donc pas, ô Dieu, dans mon âge avancé, et dans ma vieillesse,}
        \item \textit{Jusqu’à ce que j’annonce la force de votre bras à toutes les races futures.}
        \item \textit{Votre puissance et votre justice, ô mon Dieu, sont dans la plus haute élévation, par les merveilles que vous avez opérées : ô Dieu, qui est semblable à vous ?}
        \item \textit{A combien d’afflictions différentes et cruelles m’avez-vous exposé ? vous vous êtes retourné vers moi, et vous m’avez redonné la vie, et encore une fois retiré des abîmes de la terre.}
        \item \textit{Vous m’avez donné plusieurs marques de votre magnificence, et vous m’avez consolé en vous tournant vers moi.}
        \item \textit{Car je louerai votre vérité sur des instruments de musique : mon Dieu, je vous chanterai des Cantiques sur la harpe, ô saint d’Israël.}
        \item \textit{Mes lèvres se réjouissent en chantant vos louanges, aussi bien que mon âme que vous avez rachetée.}
        \item \textit{Et ma langue publiera tout le jour votre justice, lorsque ceux qui cherchent à me faire du mal, seront confondus et couverts de honte.}
      \end{enumerate}
    \end{footnotesize}
  \end{multicols}

  % ===== FIN psaume ===========

  \begin{center}
    \rule{4cm}{0.4pt}
  \end{center}

  \begin{center}
    \begin{minipage}{0.8\linewidth}
      \gresetinitiallines{0}
      \large
      \gabcsnippet{(c4)<c><v>\Vbar</v>.</c> A(h)ver(h)tán(h)tur(h) re(h)tró(h)rsum(h), et(h) e(i')ru(h)bés(g.)cant(g.) (::) <c><v>\Rbar</v>.</c> Qui(h) có(h)gi(h)tant(h) mi(i')hi(h) má(g.)la(g.) (::)}
      \bigskip
      \normalsize
      \begin{center}
        \textit{\textcolor{red}{\Vbar.} Que tous ceux qui me veulent du mal soient repoussés en arrière.}\\
        \textit{\textcolor{red}{\Rbar.} Et couverts de confusion.}
      \end{center}
    \end{minipage}
  \end{center}

  \begin{center}
    \rule{4cm}{0.4pt}
  \end{center}

  \par Les formules préparatoires aux Leçons, telles que le \textit{Pater, l'Absolution}, ou les \textit{Bénédictions} sont omises. De même, on ne dit point le \textit{Tu autem} à la fin des Leçons.
  \par Les Leçons assignées au premier Nocturne de ces trois jours depuis la plus haute antiquité, sont tirées des Lamentations de Jérémie. Elles nous tracent le tableau saisissant du châtiment infligé à la nation déicide, et que le Roi-Prophète vient d'annoncer dans les Psaumes qui précèdent. Il est aisé aussi d'appliquer plusieurs traits de ces peintures émouvante à l'Homme-Dieu, lui-même, la fleur de tout Israël.
  \par Les mots hébreux qui se trouvent au commencement de chaque strophe sont les différentes lettres de l'alphabet hébraïque dont le poête sacré a suivi l'ordre dans le choix du mot qui commence chaque verset ; parfois chaque lettre est répétée jusqu'à trois fois : c'était là un des ornements de la poésie chez les juifs.
  \par Le chant qui accompagne les Lamentations ne semble point remonter à une haute antiquité. Il ne laisse pas toutefois de produire une impression profonde sur les âmes.

  \newpage

  \begin{center}
    \large Leçon I.\\
    \normalsize
  \end{center}
  \large
  % \grechangedim{baselineskip}{55pt}{scalable}
  \gresetinitiallines{1}
  \greillumination{\initfamily\fontsize{11mm}{11mm}\selectfont I}
  \gregorioscore{lamentations/va--incipit_lamentatio_ieremiae_prophetae--solesmes}
  \newpage
  \begin{multicols}{2}
    \begin{small}
      \par \emph{Ici commence la Lamentation du prophète Jérémie.}
      \par \textcolor{red}{\textit{Aleph.}} Comment cette Ville si pleine de peuple, est-elle maintenant déserte ? La maîtresse des Nations est devenue comme une veuve ; la première
      des Provinces est contrainte de payer le tribut.
      \par \textcolor{red}{\textit{Beth.}} Elle a pleuré pendant la nuit ; ses larmes
      coulent sur ses joues. Nul de ses plus chers amis ne la console. Tous ses amis l’ont méprisée, et sont devenus ses ennemis.
      \par \textcolor{red}{\textit{Ghimel.}} Le peuple de Juda a changé de demeure, pour éviter l’affliction et la servitude rigoureuse. Il a habité parmi les nations, et n’a point trouvé de
      repos. Tous ses persécuteurs l’ont opprimé, et il n’a pu échapper de leurs mains.
      \par \textcolor{red}{\textit{Daleth.}} Les rues de Sion pleurent, parce que personne ne vient à la solemnité. Toutes ses portes sont détruites ; ses prêtres gémissent ; ses vierges sont languissantes, malpropres, et plongées dans l’amertume et dans la douleur.
      \par \textcolor{red}{\textit{Hé.}} Ses ennemis ont pris le dessus ; ses adversaires
      se sont enrichis de ses dépouilles ; parce que le Seigneur l’a prononcé en punition de la multitude de ses iniquités. Les plus jeunes ont été menés en captivité devant la face de ceux qui les chassaient cruellement.\\
      Jérusalem, Jérusalem, convertis-toi au Seigneur ton Dieu.
      \par \hspace{\fill}
    \end{small}
  \end{multicols}

  \medskip
  \begin{center}
    \rule{4cm}{0.4pt}
  \end{center}
  \medskip

  \greillumination{\initfamily\fontsize{11mm}{11mm}\selectfont I}
  \gregorioscore{repons/re--in_monte_oliveti--solesmes_1961}

  \smallskip

  \small
  \begin{multicols}{2}
    \par\textcolor{red}{\textit{\Rbar}.} \textit{Jésus pria son Père sur la montagne des Oliviers : Mon Père, s’il est possible, faites que ce calice s’éloigne de moi. \\ \textcolor{red}{*} L’esprit est prompt mais la chair est faible : Que votre volonté soit faite.}
    \columnbreak
    \par\textcolor{red}{\textit{\Vbar}.} \textit{Veillez, et priez, afin que vous ne tombiez point en tentation.\\
    \textcolor{red}{*} L’esprit est prompt mais la chair est faible : Que votre volonté soit faite.}
  \end{multicols}
  \normalsize

  \medskip
  \begin{center}
    \rule{4cm}{0.4pt}
  \end{center}
  \medskip
  \newpage
  \begin{center}
    \large Leçon II.\\
    \normalsize
  \end{center}
  % \large
  \gresetinitiallines{1}
  \greillumination{\initfamily\fontsize{11mm}{11mm}\selectfont V}
  \gregorioscore{lamentations/va--vau_et_egressus_est--solesmes}
  \begin{multicols}{2}
    \begin{small}
      \par \textcolor{red}{\textit{Vau.}} Toute la beauté de la fille de Sion l’a quittée : ses
      Princes sont devenus comme des béliers qui ne
      trouvent point de pâturages, et ils se sont retirés
      sans force devant l’ennemi qui les poursuivait.
      \par \textcolor{red}{\textit{Zaïn.}} Jérusalem s’est ressouvenue du temps de son
      affliction, de ses prévarications, et de la perte de
      toutes les choses qu’elle affectionnait le plus, et
      dont elle jouissait de tout temps ; lorsque son
      peuple tombait entre les mains de ses ennemis,
      sans être secouru de personne. Ses ennemis l’ont
      vue, et ils se sont moqués de ses fêtes du Sabbat.
      \par \textcolor{red}{\textit{Heth.}} Jérusalem a commis un grand crime ; c’est
      pourquoi elle est devenue errante. Tous ceux qui la
      comblaient de louanges, l’ont méprisée, parce qu’ils
      ont vu son ignominie : elle a tourné la tête en arrière en gémissant.
      \par \textcolor{red}{\textit{Teth.}} Ses pieds sont souillés d’ordures, et elle ne
      s’est pas souvenue de sa fin. Elle a été extrêmement abattue, n’ayant point de consolateur.
      Voyez, Seigneur, mon affliction, parce que mon ennemi a pris le dessus.\\
      Jérusalem, Jérusalem, convertis-toi au Seigneur ton Dieu.
      % \par \hspace{\fill}
    \end{small}
  \end{multicols}

  \newpage

  \greillumination{\initfamily\fontsize{11mm}{11mm}\selectfont T}
  \gregorioscore{repons/re--tristis_est--solesmes_1961}
  
  \small
  \begin{multicols}{2}
    \par\textcolor{red}{\textit{\Rbar}.} \textit{Mon âme est triste jusqu’à la mort. Demeurez
    ici, et veillez avec moi ; vous verrez la troupe de gens qui m’environnera. \\ \textcolor{red}{*} Vous prendrez la fuite, et j’irai pour être immolé pour vous.}
    \columnbreak
    \par\textcolor{red}{\textit{\Vbar}.} \textit{Voici l’heure qui s’approche, et le Fils de l’homme sera livré entre les mains des pécheurs.\\
    \textcolor{red}{*} Vous prendrez la fuite, et j’irai pour être immolé pour vous.}
  \end{multicols}
  \normalsize

  \begin{center}
    \large Leçon III.\\
    \normalsize
  \end{center}
  % \large
  \gresetinitiallines{1}
  \greillumination{\initfamily\fontsize{11mm}{11mm}\selectfont J}
  \gregorioscore{lamentations/va--manum_suam_misit_hostis--silos}
  \begin{multicols}{2}
    \begin{small}
      \par \textcolor{red}{\textit{Jod.}} L’ennemi s’est emparé de tout ce qu’elle avait de
      plus désirable ; car elle a vu les nations introduites dans votre Sanctuaire, quoique vous eussiez défendu de les admettre dans votre assemblées.
      \par \textcolor{red}{\textit{Caph.}} Tout son peuple gémissant cherche du pain ; ils ont donné pour vivre, ce qu’ils avaient de plus précieux, pour rétablir un peu leurs forces. Voyez, Seigneur, et considérez combien je suis devenue méprisable.
      \par \textcolor{red}{\textit{Lamed.}} O vous tous, qui passez par le chemin, considérez et voyez s’il y a une douleur semblable à la mienne ; car le Seigneur, selon sa parole, m’a dépouillée au jour de sa colère, comme une vigne vendangée.
      \par \textcolor{red}{\textit{Mem.}} Il a fait tomber d’en-haut un feu dans mes
      os, et m’a châtié. Il a tendu un filet sous mes pieds ; il m’a fait tomber en arrière ; il m’a plongée dans une tristesse qui durera tout le jour.
      \par \textcolor{red}{\textit{Nun.}} Le joug de mes iniquités m’a accablé sans relâche ; ses mains en ont fait une chaîne qui a été attachée à mon cou. Ma force s’est affaiblie ; le
      Seigneur m’a livré en des mains dont je ne pourrai jamais me relever.\\
      Jérusalem, Jérusalem, convertis-toi au Seigneur ton Dieu.
      % \par \hspace{\fill}
    \end{small}
  \end{multicols}

  \bigskip

  \greillumination{\initfamily\fontsize{11mm}{11mm}\selectfont T}
  \gregorioscore{repons/re--ecce_vidimus_--solesmes_1961}
  \small
  \begin{multicols}{2}
    \par\textcolor{red}{\textit{\Rbar}.} \textit{Voici que nous l’avons vu qui n’avait plus aucune beauté ; il n’était pas reconnaissable. C’est lui qui a porté nos péchés, et il est puni pour nous. A son égard, il a été percé de plaies à cause de nos iniquités. \textcolor{red}{*} Et nous avons été guéris par ses meurtrissures.}
    \par\textcolor{red}{\textit{\Vbar}.} \textit{Il a véritablement porté nos langueurs, et il a
    ressenti nos douleurs.
    \textcolor{red}{*} Et nous avons été guéris par ses meurtrissures.}
    \columnbreak
    \par\textcolor{red}{\textit{\Rbar}.} \textit{Voici que nous l’avons vu qui n’avait plus aucune beauté ; il n’était pas reconnaissable. C’est lui qui a porté nos péchés, et il est puni pour
    nous. A son égard, il a été percé de plaies à cause de nos iniquités.\\ \textcolor{red}{*} Et nous avons été guéris par ses meurtrissures.}
    \par\hspace{\fill}
  \end{multicols}
  \normalsize

  \newpage

  \begin{center}
    \large AU DEUXIÈME NOCTURNE.\\
  \end{center}
  \medskip
  \par Le chant royal de l'avènement que nous avons chanté à Noël ouvre le second Nocturne de cette nuit de douleur. Ce n'est pas sans raisons que l'Église fait du même Psaume un emploi si différent. Car si nous avons salué dans l'Enfant de Bethléem notre Roi et notre Libérateur, c'est sur la Croix qu'il règne véritablement ; c'est devant cette Croix que viendront s'humilier tous les rois de la terre, parce que sur elle Jésus a sauvé les pauvres de son peuple et brisé celui qui les opprimait.

  \medskip
  \begin{center}
    \rule{4cm}{0.4pt}
  \end{center}
  \medskip

  % ===== DEBUT Antienne =========
  \gresetinitiallines{1}
  \greillumination{\initfamily\fontsize{11mm}{11mm}\selectfont L}
  \gregorioscore{antiennes/an--liberavit_dominus--solesmes_1961}
  \begin{center}
    \footnotesize{
      \textit{
        Le Seigneur a délivré le pauvre de la main du puissant, et soutenu l’indigent qui n’avait point de protecteur.
    }
  }
  \end{center}
  % ===== FIN Antienne ===========

  % ===== DEBUT psaume ===========
  % gresetinitiallines : avec le parametre à 0, supprime l'ornement
  \begin{center}
    \large{Psaume 71.}\\
  \end{center}

  \gresetinitiallines{0}
  \gregorioscore{psaumes/psaume71-VIIc}

  \begin{enumerate}[label=\textcolor{red}{\arabic*}]
    \setcounter{enumi}{2}
    \item Judicáre pópulum tuum \textbf{in} jus\textbf{tí}tia,\textcolor{red}{~*} et páuperes tuos \textbf{in} ju\textbf{dí}cio.

    \item Suscípiant montes \textbf{pa}cem \textbf{pó}pulo:\textcolor{red}{~*} et \textbf{col}les jus\textbf{tí}tiam.

    \item Judicábit páuperes pópuli, et salvos fáciet \textbf{fí}lios \textbf{páu}perum:\textcolor{red}{~*} et humiliábit calum\textbf{ni}a\textbf{tó}rem.

    \item Et permanébit cum sole, et \textbf{an}te \textbf{lu}nam,\textcolor{red}{~*} in generatióne et gene\textbf{ra}ti\textbf{ó}nem.

    \item Descéndet sicut plúvi\textbf{a} in \textbf{vel}lus:\textcolor{red}{~*} et sicut stillicídia stillántia \textbf{su}per \textbf{ter}ram.

    \item Oriétur in diébus ejus justítia, et abun\textbf{dán}tia \textbf{pa}cis:\textcolor{red}{~*} donec aufe\textbf{rá}tur \textbf{lu}na.

    \item Et dominábitur a mari \textbf{us}que ad \textbf{ma}re:\textcolor{red}{~*} et a flúmine usque ad términos \textbf{or}bis ter\textbf{rá}rum.

    \item Coram illo próci\textbf{dent} Æ\textbf{thí}opes:\textcolor{red}{~*} et inimíci ejus \textbf{ter}ram \textbf{lin}gent.

    \item Reges Tharsis, et ínsulæ \textbf{mú}nera \textbf{óf}ferent:\textcolor{red}{~*} reges Arabum et Saba \textbf{do}na ad\textbf{dú}cent.

    \item Et adorábunt eum omnes \textbf{re}ges \textbf{ter}ræ:\textcolor{red}{~*} omnes Gentes \textbf{sér}vient \textbf{e}i:

    \item Quia liberábit páuperem \textbf{a} pot\textbf{én}te:\textcolor{red}{~*} et páuperem, cui non \textbf{e}rat ad\textbf{jú}tor.

    \item Parcet páupe\textbf{ri} et \textbf{ín}opi:\textcolor{red}{~*} et ánimas páuperum \textbf{sal}vas \textbf{fá}ciet.

    \item Ex usúris et iniquitáte rédimet áni\textbf{mas} e\textbf{ó}rum:\textcolor{red}{~*} et honorábile nomen eórum \textbf{co}ram \textbf{il}lo.

    \item Et vivet, et dábitur ei de auro Arábiæ,\textcolor{red}{~†} et adorábunt de \textbf{ip}so \textbf{sem}per:\textcolor{red}{~*} tota die bene\textbf{dí}cent \textbf{e}i.

    \item Et erit firmaméntum in terra in summis móntium,\textcolor{red}{~†} superextollétur super Líbanum \textbf{fruc}tus \textbf{e}jus:\textcolor{red}{~*} \\ \-\hspace{2cm} et florébunt de civitáte sicut \textbf{fe}num \textbf{ter}ræ.

    \item Sit nomen ejus bene\textbf{díc}tum in \textbf{sǽ}cula:\textcolor{red}{~*} ante solem pérmanet \textbf{no}men \textbf{e}jus.

    \item Et benedicéntur in ipso omnes \textbf{tri}bus \textbf{ter}ræ:\textcolor{red}{~*} omnes Gentes magnifi\textbf{cá}bunt \textbf{e}um.

    \item Benedíctus Dóminus, \textbf{De}us \textbf{Is}raël,\textcolor{red}{~*} qui facit mira\textbf{bí}lia \textbf{so}lus.

    \item Et benedíctum nomen majestátis ejus \textbf{in} æ\textbf{tér}num:\textcolor{red}{~*} et replébitur majestáte ejus omnis terra: \textbf{fi}at, \textbf{fi}at.
  \end{enumerate}

  \begin{multicols}{2}
    \begin{footnotesize}
      \begin{enumerate}[label=\textcolor{red}{\emph{\arabic*}}]
        \item \textit{O Dieu, donnez au Roi votre jugement, et votre justice au fils du Roi,}
        \item \textit{Afin qu’il juge votre peuple selon la justice, et vos
        pauvres selon l’équité de ses jugements.}
        \item \textit{Que les montagnes reçoivent la paix pour le
        peuple, et les collines la justice.}
        \item \textit{Il jugera les pauvres du peuple ; il sauvera les enfants des pauvres, et il humiliera le calomniateur.}
        \item \textit{Il subsistera autant que le soleil et la lune, dans l’étendue de toutes les générations}
        \item \textit{Il descendra comme la pluie sur une toison, et comme l’eau qui tombe goutte à goutte sur la terre.}
        \item \textit{La justice paraîtra de son temps, avec une abondance de paix, qui durera autant que la lune.}
        \item \textit{Il règnera depuis une mer jusqu’à l’autre, et depuis le fleuve jusqu’aux extrémités de la terre.}
        \item \textit{Les Ethiopiens se prosterneront devant lui, et ses ennemis baiseront la terre.}
        \item \textit{Les Rois de Tharse et les îles lui offriront des présents : Les Rois de l’Arabie et de Saba lui apporteront des dons.}
        \item \textit{Et tous les Rois de la terre l’adoreront ; toutes les nations lui seront assujetties}
        \item \textit{Parce qu’il délivrera le pauvre de la main du puissant, et l’indigent qui n’avait point de protecteur.}
        \item \textit{Il épargnera le pauvre et l’indigent ; et il sauvera les âmes des pauvres.}
        \item \textit{Et il délivrera leurs âmes des usures et de l’iniquité ; et leur nom sera honorable devant lui.}
        \item \textit{Et il vivra, et on lui donnera de l’or de l’Arabie ; ils l’adoreront sans cesse, et ils le béniront durant tout le jour.}
        \item \textit{Et l’on verra le froment semé dans la terre sur le sommet des montagnes : son fruit s’élèvera audessus des cèdres du Liban ; les habitants des villes multiplieront comme les gerbes de la terre.}
        \item \textit{Que son nom soit béni dans tous les siècles : son nom subsiste avant le soleil.}
        \item \textit{Et tous les peuples de la terre seront bénis en lui ; toutes les nations le glorifieront.}
        \item \textit{Que béni soit le Seigneur, le Dieu d’Israël, qui fait seul des œuvres merveilleuses.}
        \item \textit{Et que le nom de sa Majesté soit béni éternellement ; et toute la terre sera remplie de sa Majesté ; ainsi soit fait, ainsi soit fait.}
      \end{enumerate}
    \end{footnotesize}
  \end{multicols}

\end{document}
