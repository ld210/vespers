% !TeX program = lualatex
\documentclass[12pt, a4paper]{article}
\usepackage{fullpage}
\usepackage{subfiles}
\usepackage{fontspec}
\usepackage{libertine}
\usepackage{xcolor}
\usepackage{GotIn}
\usepackage{geometry}
\usepackage{multicol}
\usepackage{multicolrule}
\usepackage{graphicx}
\usepackage{enumitem}
\usepackage[autocompile]{gregoriotex}
\usepackage[latin,french]{babel}


\geometry{top=2cm, bottom=2cm}
% \pagestyle{empty}

\definecolor{red}{HTML}{C70039}
% \input GoudyIn.fd
% \newcommand*\initfamily{\usefont{U}{GoudyIn}{xl}{n}}

\input Acorn.fd
\newcommand*\initfamily{\usefont{U}{Acorn}{xl}{n}}
% cette ligne ajoute de l'espace entre les portées
% \grechangedim{baselineskip}{60pt}{scalable}

\begin{document}
  \gresetlinecolor{gregoriocolor}

  \begin{titlepage}\centering
    \vspace*{\fill}\
    \huge Secondes Vêpres\\
    \smallskip
    \begin{Large}
      \textit{
        de la Nativité de Notre Seigneur\\
      }
    \end{Large}
    \medskip
    \large et\\
    \medskip
    \LARGE Salut du Saint-Sacrement\\
    \bigskip
    \vspace*{\fill}
    \begin{figure}[h!]
      \centering
      \includegraphics[width=7cm]{../epiphanie_septuagesime_careme/logo.png}
    \end{figure}
    \centering \normalsize Paroisse Saint Roch\\
    \bigskip
    \begin{Large}
      \centering Ne pas emporter
    \end{Large}
  \end{titlepage}

  \newpage
  \vspace*{\fill}
  \begin{center}
    \large Sommaire\\
    \begin{multicols}{2}
      \normalsize
      \begin{flushleft}
        Vêpres du jour de Noël\\
        Dimanche dans l'octave de Noël\\
      \end{flushleft}
      \columnbreak
      \begin{flushright}
        \textit{
        page 3\\
        page 12\\
      }
      \end{flushright}
    \end{multicols}
  \end{center}
  \vspace*{\fill}
  \begin{center}
    \normalsize\textit{
      Livret latin-français
    }
  \end{center}
  \newpage
  \normalsize
  \begin{center}
    \textcolor{red}{\large{Invitatoire.}}
  \end{center}

  % greillumination: remplace la première lettre, ici par une font ornementale
  \greillumination{\initfamily\fontsize{11mm}{11mm}\selectfont D}
  \gregorioscore{or--deus_in_adjutorium_(tonus_festivus)--solesmes_1961.1}

  \begin{center}
    \small{
    \emph{
        Dieu, venez à mon aide ; Seigneur, hatez-vous de me secourir.\\
        Gloire au Père, au Fils et au Saint Esprit, comme il était au commencement, maintenant et toujour et dans les siècles des siècles. Allelúia\\
      }
    }
  \end{center}

  \bigskip

  \begin{center}
    \textcolor{red}{\footnotesize\textit{
      Le célébrant entonne la première antienne. Les chantres entonnent les antiennes suivantes, ainsi que le premier verset de chaque psaumes.
    }}
  \end{center}

  \vspace*{\fill}\
  \begin{center}
    \greseparator{3}{30}
  \end{center}
  \vspace*{\fill}\

  \newpage
  \normalsize

  \begin{center}
    \begin{LARGE}
      Vêpres du jour de Noël
    \end{LARGE}
  \end{center}
  \medskip

  % ===== DEBUT Antienne =========
  \greillumination{\initfamily\fontsize{11mm}{11mm}\selectfont T}
  \gregorioscore{antiennes/an--tecum_principium--solesmes_1961}
  \begin{center}
    \footnotesize{
      \textit{
        Avec vous est le principe au jour de votre puissance, dans les splendeurs des Saints : c’est de mon sein, qu’avant que l’aurore existât, je vous ai engendré. 
      }
    }
  \end{center}
  % \medskip
  % ===== FIN Antienne ===========

  % ===== DEBUT psaume ===========
  % gresetinitiallines : avec le parametre à 0, supprime l'ornement
  \gresetinitiallines{0}

  \begin{center}
    \normalsize{Psaume 109.}
  \end{center}

  \gregorioscore{psaumes/psaume109-IG}
  \begin{enumerate}[label=\textcolor{red}{\emph{\arabic*}}]
    \setcounter{enumi}{2}
    \item Donec ponam ini\textbf{mí}cos \textbf{tu}os,\textcolor{red}{~*} scabéllum pe\textit{dum} \textit{tu}\textbf{ó}rum.

    \item Virgam virtútis tuæ emíttet Dómi\textbf{nus} ex \textbf{Si}on:\textcolor{red}{~*} domináre in médio inimicó\textit{rum} \textit{tu}\textbf{ó}rum.
    
    \item Tecum princípium in die virtútis tuæ in splendóri\textbf{bus} sanc\textbf{tó}rum:\textcolor{red}{~*} ex útero ante lucíferum \textit{gé}\textit{nu}\textbf{i} te.
    
    \item Jurávit Dóminus, et non pœni\textbf{té}bit \textbf{e}um:\textcolor{red}{~*} Tu es sacérdos in ætérnum secúndum órdi\textit{nem} \textit{Mel}\textbf{chí}sedech.
    
    \item Dóminus a \textbf{dex}tris \textbf{tu}is,\textcolor{red}{~*} confrégit in die iræ \textit{su}\textit{æ} \textbf{re}ges.
    
    \item Judicábit in natiónibus, im\textbf{plé}bit ru\textbf{í}nas:\textcolor{red}{~*} conquassábit cápita in ter\textit{ra} \textit{mul}\textbf{tó}rum.
    
    \item De torrénte in \textbf{vi}a \textbf{bi}bet:\textcolor{red}{~*} proptérea exal\textit{tá}\textit{bit} \textbf{ca}put.
    
    \item Glória \textbf{Pa}tri, et \textbf{Fí}lio,\textcolor{red}{~*} et Spirí\textit{tu}\textit{i} \textbf{Sanc}to.
    
    \item Sicut erat in princípio, et \textbf{nunc}, et \textbf{sem}per,\textcolor{red}{~*} et in sǽcula sæcu\textit{ló}\textit{rum}. \textbf{A}men.
  \end{enumerate}
  \medskip
  \grecommentary{\textit{Reprise de l'Antienne.}}
  \gabcsnippet{(c4) Te(f)cum(c') prin(d)cí(ixdh'!iv)pi(h_)um(h'_) (,) in(h_) di(h_)e(hgh) vir(f_g)tú(gh)tis(fe) tu(d)ae,(d.) (;) in(f) splen(f)dó(f_g)ri(f)bus(c') san(d_)ctó(d_)rum,(c.) (;) ex(c) ú(f')te(f_)ro(f'_) (,) an(f)te(f) lu(f)cí(g_[uh:l]h)fe(f)rum(ec~) gé(d'_)nu(f)i(e) te.(d.) (::)}

  \smallskip
  \begin{footnotesize}
    \begin{center}
      \textit{
        \textcolor{red}{1. }L'Éternel a dit à mon Seigneur: Assieds-toi à ma droite.
        \textcolor{red}{2. }Jusqu'à ce que je mette tes ennemis pour le marchepied de tes pieds.
        \textcolor{red}{3. }L'Éternel enverra de Sion la verge de ta force: Domine au milieu de tes ennemis!
        \textcolor{red}{4. }Ton peuple sera un peuple de franche volonté, au jour de ta puissance, en sainte magnificence. Du sein de l'aurore te viendra la rosée de ta jeunesse.
        \textcolor{red}{5. }L'Éternel a juré, et il ne se repentira point: Tu es sacrificateur pour toujours, selon l'ordre de Melchisédec.
        \textcolor{red}{6. }Le Seigneur, à ta droite, brisera les rois au jour de sa colère.
        \textcolor{red}{7. }Il jugera parmi les nations, il remplira tout de corps morts, il brisera le chef d'un grand pays.
        \textcolor{red}{8. }Il boira du torrent dans le chemin, c'est pourquoi il lèvera haut la tête.
        \textcolor{red}{9. }Gloire au Père, au Fils, et au Saint Esprit, 
        \textcolor{red}{10. }Comme il était au commencement, maintenant et toujours, et dans les siècles des siècles. Amen.
      }
    \end{center}
  \end{footnotesize}
  \bigskip

  % ===== DEBUT Antienne =========
  \gresetinitiallines{1}
  \greillumination{\initfamily\fontsize{11mm}{11mm}\selectfont R}
  \gregorioscore{antiennes/an--redemptionem_misit--solesmes_1961}
  \begin{center}
    \footnotesize{
      \textit{
        Il a envoyé la rédemption à son peuple : il a établi pour l’éternité son alliance. 
      }
    }
  \end{center}
  % \medskip
  % ===== FIN Antienne ===========

  % ===== DEBUT psaume ===========
  % gresetinitiallines : avec le parametre à 0, supprime l'ornement
  \gresetinitiallines{0}

  \begin{center}
    \normalsize{Psaume 110.}
  \end{center}

  \gregorioscore{psaumes/psaume110-VIIa}

  \begin{enumerate}[label=\textcolor{red}{\emph{\arabic*}}]
    \setcounter{enumi}{1}
    \item Magna \textbf{ó}pera \textbf{Dó}mini:\textcolor{red}{~*} exquisíta in omnes volun\textbf{tá}tes \textbf{e}jus.

    \item Conféssio et magnificéntia \textbf{o}pus \textbf{e}jus:\textcolor{red}{~*} et justítia ejus manet in \textbf{sǽ}culum \textbf{sǽ}culi.
    
    \item Memóriam fecit mirabílium suórum,\textcolor{red}{~†} miséricors et mise\textbf{rá}tor \textbf{Dó}minus:\textcolor{red}{~*} escam dedit ti\textbf{mén}ti\textbf{bus} se.
    
    \item Memor erit in sǽculum testa\textbf{mén}ti \textbf{su}i:\textcolor{red}{~*} virtútem óperum suórum annuntiábit \textbf{pó}pulo \textbf{su}o:
    
    \item Ut det illis heredi\textbf{tá}tem \textbf{gén}tium:\textcolor{red}{~*} ópera mánuum ejus véritas, \textbf{et} ju\textbf{dí}cium.
    
    \item Fidélia ómnia mandáta ejus:\textcolor{red}{~†} confirmáta in \textbf{sǽ}culum \textbf{sǽ}culi,\textcolor{red}{~*} facta in veritáte et \textbf{æ}qui\textbf{tá}te.
    
    \item Redemptiónem misit \textbf{pó}pulo \textbf{su}o:\textcolor{red}{~*} mandávit in ætérnum testa\textbf{mén}tum \textbf{su}um.
    
    \item Sanctum, et terríbile \textbf{no}men \textbf{e}jus:\textcolor{red}{~*} inítium sapiéntiæ \textbf{ti}mor \textbf{Dó}mini.
    
    \item Intelléctus bonus ómnibus faci\textbf{én}tibus \textbf{e}um:\textcolor{red}{~*} laudátio ejus manet in \textbf{sǽ}culum \textbf{sǽ}culi.
    
    \item Glória \textbf{Pa}tri, et \textbf{Fí}lio,\textcolor{red}{~*} et Spi\textbf{rí}tui \textbf{Sanc}to.
    
    \item Sicut erat in princípio, et \textbf{nunc}, et \textbf{sem}per,\textcolor{red}{~*} et in sǽcula sæcu\textbf{ló}rum. \textbf{A}men.
  \end{enumerate}

  \medskip
  \grecommentary{\textit{Reprise de l'Antienne.}}
  \gabcsnippet{(c3) Red(e')em(e)pti(g')ó(i)nem(h') () mi(f)sit(h) Dó(g')mi(f)nus(e'_[oh:h]) (,) pó(h)pu(h)lo(hf) su(h.)o :(g.) (;) man(i')dá(i)vit(i') in(h) ae(h)tér(iji)num(hg/hih.) (,) te(f)sta(e)mén(f!gwh_f~)tum(g_[oh:h]f~) su(e.)um.(e.) (::)}
  \smallskip
  \begin{footnotesize}
    \begin{center}
      \textit{
        \textcolor{red}{1. }Grandes sont les œuvres du Seigneur ; tous ceux qui les aiment s'en instruisent.
        \textcolor{red}{2. }De tout cœur je rendrai grâce au Seigneur dans l'assemblée, parmi les justes.
        \textcolor{red}{3. }Grandes sont les œuvres du Seigneur ; tous ceux qui les aiment s'en instruisent.
        \textcolor{red}{4. }Noblesse et beauté dans ses actions : à jamais se maintiendra sa justice.
        \textcolor{red}{5. }De ses merveilles il a laissé un mémorial ; le Seigneur est tendresse et pitié, il a donné des vivres à ses fidèles,
        \textcolor{red}{6. }Gardant toujours mémoire de son alliance, il a montré sa force à son peuple.
        \textcolor{red}{7. }Lui donnant le domaine des nations. Justesse et sûreté les œuvres de ses mains.
        \textcolor{red}{8. }Sécurité, toutes ses lois, établies pour toujours et à jamais, accomplies avec droiture et sûreté !
        \textcolor{red}{9. }Il apporte la délivrance à son peuple ; son alliance est promulguée pour toujours.
        \textcolor{red}{10. }Saint, redoutable est son nom, la sagesse commence avec la crainte du Seigneur.
        \textcolor{red}{11. }Qui accomplit sa volonté en est éclairé. A jamais se maintiendra sa louange.
        \textcolor{red}{12. }Gloire au Père, au Fils, et au Saint Esprit, 
        \textcolor{red}{13. }Comme il était au commencement, maintenant et toujours, et dans les siècles des siècles. Amen.
      }
    \end{center}
  \end{footnotesize}
  \bigskip

  % ===== DEBUT Antienne =========
  \gresetinitiallines{1}
  \greillumination{\initfamily\fontsize{11mm}{11mm}\selectfont E}
  \gregorioscore{antiennes/an--exortum_est_(christmas)--solesmes_1961}
  \begin{center}
    \footnotesize{
      \textit{
        Il s’est élevé dans les ténèbres une lumière pour les hommes droits : le Seigneur est miséricordieux, compatissant et juste. 
      }
    }
  \end{center}
  % \medskip
  % ===== FIN Antienne ===========

  % ===== DEBUT psaume ===========
  % gresetinitiallines : avec le parametre à 0, supprime l'ornement

  \gresetinitiallines{0}

  \begin{center}
    \normalsize{Psaume 111.}\\
  \end{center}
  % \smallskip
  \grechangedim{baselineskip}{50pt}{scalable}
  \gregorioscore{psaumes/psaume111-VIIb}
  \begin{enumerate}[label=\textcolor{red}{\emph{\arabic*}}]
    \setcounter{enumi}{1}
    \item Potens in terra erit \textbf{se}men \textbf{e}jus:\textcolor{red}{~*} generátio rectórum be\textbf{ne}di\textbf{cé}tur.

    \item Glória, et divítiæ in \textbf{do}mo \textbf{e}jus:\textcolor{red}{~*} et justítia ejus manet in \textbf{sǽ}culum \textbf{sǽ}culi.
    
    \item Exórtum est in ténebris \textbf{lu}men \textbf{rec}tis:\textcolor{red}{~*} miséricors, et mise\textbf{rá}tor, et \textbf{jus}tus.
    
    \item Jucúndus homo qui miserétur et cómmodat,\textcolor{red}{~†} dispónet sermónes suos \textbf{in} ju\textbf{dí}cio:\textcolor{red}{~*} quia in ætérnum non \textbf{com}mo\textbf{vé}bitur.
    
    \item In memória ætérna \textbf{e}rit \textbf{jus}tus:\textcolor{red}{~*} ab auditióne mala \textbf{non} ti\textbf{mé}bit.
    
    \item Parátum cor ejus speráre in Dómino,\textcolor{red}{~†} confirmátum \textbf{est} cor \textbf{e}jus:\textcolor{red}{~*} non commovébitur donec despíciat ini\textbf{mí}cos \textbf{su}os.
    
    \item Dispérsit, dedit paupéribus:\textcolor{red}{~†} justítia ejus manet in \textbf{sǽ}culum \textbf{sǽ}culi,\textcolor{red}{~*} cornu ejus exaltábi\textbf{tur} in \textbf{gló}ria.
    
    \item Peccátor vidébit, et irascétur,\textcolor{red}{~†} déntibus suis fremet \textbf{et} ta\textbf{bé}scet:\textcolor{red}{~*} desidérium pecca\textbf{tó}rum per\textbf{í}bit.
    
    \item Glória \textbf{Pa}tri, et \textbf{Fí}lio,\textcolor{red}{~*} et Spi\textbf{rí}tui \textbf{Sanc}to.
    
    \item Sicut erat in princípio, et \textbf{nunc}, et \textbf{sem}per,\textcolor{red}{~*} et in sǽcula sæcu\textbf{ló}rum. \textbf{A}men.
  \end{enumerate}

  \medskip
  \grecommentary{\textit{Reprise de l'Antienne.}}
  \gabcsnippet{(c3) Ex(g)ór(ig/ij)tum(i) est(i.) (,) in(i_[uh:l]j) té(i)ne(hih)bris(hgh) (,) lu(hg)men(f') re(gh'i)ctis(f') cor(e)de :(e.) (;) mi(f)sé(h')ri(h)cors(h') et(h) mi(h)se(f')rá(h)tor,(gf) (,) et(e_[uh:l]f) ju(g')stus(f) Dó(e')mi(e)nus.(e.) (::)}
  \smallskip
  \begin{footnotesize}
    \begin{center}
      \textit{
        \textcolor{red}{1. }Celui qui craint le Seigneur a une volonté ardente d’accomplir ses commandements.
        \textcolor{red}{2. }Heureux l’homme qui craint le Seigneur, qui aime entièrement sa volonté !
        \textcolor{red}{3. }Sa lignée sera puissante sur la terre ; la
        race des justes est bénie.
        \textcolor{red}{4. }Les richesses affluent dans sa maison : à jamais se maintiendra sa justice.
        \textcolor{red}{5. }Lumière des cœurs droits, il s'est levé dans les ténèbres, l’homme de justice, de tendresse et de pitié.
        \textcolor{red}{6. }L'homme de bien a pitié, il partage ; il mène ses affaires avec droiture, cet homme jamais ne tombera ;
        \textcolor{red}{7. }Toujours on fera mémoire du juste, il ne craint pas l'annonce d'un malheur :
        \textcolor{red}{8. }Le cœur ferme, il s'appuie sur le Seigneur. Son cœur est confiant, il ne
        craint pas : il verra ce que valaient ses oppresseurs.
        \textcolor{red}{9. }A pleines mains, il donne au pauvre ; à jamais se maintiendra sa justice, sa
        puissance grandira, et sa gloire !
        \textcolor{red}{10. }L'impie le voit et s'irrite ; il grince des dents et se détruit. L'ambition des impies se perdra.
        \textcolor{red}{12. }Gloire au Père, au Fils, et au Saint Esprit, 
        \textcolor{red}{13. }Comme il était au commencement, maintenant et toujours, et dans les siècles des siècles. Amen.
      }
    \end{center}
  \end{footnotesize}
  \bigskip

  % ===== DEBUT Antienne =========
  \gresetinitiallines{1}
  \greillumination{\initfamily\fontsize{11mm}{11mm}\selectfont A}
  \gregorioscore{antiennes/an--apud_dominum_misericordia--solesmes_1961}
  \begin{center}
    \footnotesize{
      \textit{
        Auprès du Seigneur est la miséricorde, et abonde chez lui la rédemption.
      }
    }
  \end{center}
  % ===== FIN Antienne ===========

  % ===== DEBUT psaume ===========
  % gresetinitiallines : avec le parametre à 0, supprime l'ornement
  \gresetinitiallines{0}

  \begin{center}
    \normalsize{Psaume 129.}\\
  \end{center}
  % \smallskip

  \gregorioscore{psaumes/psaume129-IVa}

  \begin{enumerate}[label=\textcolor{red}{\emph{\arabic*}}]
    \setcounter{enumi}{1}
    \item Fiant aures tuæ \textit{in}\textit{ten}\textbf{dén}tes:\textcolor{red}{~*} in vocem depreca\textit{ti}\textit{ó}\textit{nis} \textbf{me}æ.

    \item Si iniquitátes observá\textit{ve}\textit{ris}, \textbf{Dó}mine:\textcolor{red}{~*} Dómine, \textit{quis} \textit{sus}\textit{ti}\textbf{né}bit?
    
    \item Quia apud te propiti\textit{á}\textit{ti}\textbf{o} est:\textcolor{red}{~*} et propter legem tuam sustí\textit{nu}\textit{i} \textit{te}, \textbf{Dó}mine.
    
    \item Sustínuit ánima mea in \textit{ver}\textit{bo} \textbf{e}jus:\textcolor{red}{~*} sperávit ánima \textit{me}\textit{a} \textit{in} \textbf{Dó}mino.
    
    \item A custódia matutína us\textit{que} \textit{ad} \textbf{noc}tem:\textcolor{red}{~*} speret Is\textit{ra}\textit{ël} \textit{in} \textbf{Dó}mino.
    
    \item Quia apud Dóminum mi\textit{se}\textit{ri}\textbf{cór}dia:\textcolor{red}{~*} et copiósa apud \textit{e}\textit{um} \textit{red}\textbf{émp}tio.
    
    \item Et ipse réd\textit{i}\textit{met} \textbf{Is}raël:\textcolor{red}{~*} ex ómnibus iniqui\textit{tá}\textit{ti}\textit{bus} \textbf{e}jus.
    
    \item Glória Pa\textit{tri}, \textit{et} \textbf{Fí}lio,\textcolor{red}{~*} et Spi\textit{rí}\textit{tu}\textit{i} \textbf{Sanc}to.
    
    \item Sicut erat in princípio, et \textit{nunc}, \textit{et} \textbf{sem}per,\textcolor{red}{~*} et in sǽcula sæ\textit{cu}\textit{ló}\textit{rum}. \textbf{A}men.
  \end{enumerate}

  \medskip
  \grecommentary{\textit{Reprise de l'Antienne.}}
  \gabcsnippet{(c3) A(e')pud(f) Dó(h)mi(hi)num(i.) (,) mi(h_i)se(j)ri(ih)cór(i')di(i)a,(i.) (;) et(f_i) co(g')pi(h)ó(f_e)sa(f_e) (,) a(d')pud(e) e(gxf_g)um(h') red(h)ém(f')pti(f)o.(f.) (::) <eu>E(i) u(h) o(i) u(j) a(h) e.</eu>(f.) (::)}

  \smallskip
  \begin{footnotesize}
    \begin{center}
      \textit{
        \textcolor{red}{1. }Des profondeurs, j’ai crié vers vous, Seigneur, Seigneur écoutez ma voix. 
        \textcolor{red}{2. }Que vos oreilles soient attentives à la voix de ma prière. 
        \textcolor{red}{3. }Si vous observez nos iniquités, Seigneur, Seigneur, qui subsistera ? 
        \textcolor{red}{4. }Car près de vous est le pardon, et à cause de votre loi j’ai espéré en vous Seigneur.
        \textcolor{red}{5. }Mon âme a espéré en ses paroles, mon âme a espéré en le Seigneur.
        \textcolor{red}{6. }Depuis la garde du matin jusqu’à la nuit, qu’Israël espère en le Seigneur.
        \textcolor{red}{7. }Car auprès du Seigneur est la miséricorde, et abonde chez lui la rédemption.
        \textcolor{red}{8. }Et c’est lui qui rachètera Israël de toutes ses iniquités.
        \textcolor{red}{12. }Gloire au Père, au Fils, et au Saint Esprit, 
        \textcolor{red}{13. }Comme il était au commencement, maintenant et toujours, et dans les siècles des siècles. Amen.
      }
    \end{center}
  \end{footnotesize}
  \bigskip

  % ===== DEBUT Antienne =========
  \gresetinitiallines{1}
  \greillumination{\initfamily\fontsize{11mm}{11mm}\selectfont D}
  \gregorioscore{antiennes/an--de_fructu_ventris_tui--solesmes_1961}
  \begin{center}
    \footnotesize{
      \textit{
        Je mettrai un fils du fruit de vos entrailles sur votre trône. 
      }
    }
  \end{center}
  % ===== DEBUT psaume ===========
  % gresetinitiallines : avec le parametre à 0, supprime l'ornement
  \begin{center}
    \normalsize{Psaume 131.}
  \end{center}

  % gresetinitiallines : avec le parametre à 0, supprime l'ornement
  \gresetinitiallines{0}
  \gregorioscore{psaumes/psaume131-VIIIG}

  \begin{enumerate}[label=\textcolor{red}{\arabic*}]
    \setcounter{enumi}{1}
    \item Sicut jurávit \textbf{Dó}mino,\textcolor{red}{~*} votum vovit \textit{De}\textit{o} \textbf{Ja}cob:

    \item Si introíero in tabernáculum domus \textbf{me}æ,\textcolor{red}{~*} si ascéndero in lectum \textit{stra}\textit{ti} \textbf{me}i:
    
    \item Si dédero somnum óculis \textbf{me}is,\textcolor{red}{~*} et pálpebris meis dormi\textit{ta}\textit{ti}\textbf{ó}nem:
    
    \item Et réquiem tempóribus meis: donec invéniam locum \textbf{Dó}mino,\textcolor{red}{~*} tabernáculum \textit{De}\textit{o} \textbf{Ja}cob.
    
    \item Ecce audívimus eam in \textbf{E}phrata:\textcolor{red}{~*} invénimus eam in \textit{cam}\textit{pis} \textbf{sil}væ.
    
    \item Introíbimus in tabernáculum \textbf{e}jus:\textcolor{red}{~*} adorábimus in loco, ubi stetérunt \textit{pe}\textit{des} \textbf{e}jus.
    
    \item Surge, Dómine, in réquiem \textbf{tu}am,\textcolor{red}{~*} tu et arca sanctificati\textit{ó}\textit{nis} \textbf{tu}æ.
    
    \item Sacerdótes tui induántur jus\textbf{tí}tiam:\textcolor{red}{~*} et sancti tu\textit{i} \textit{ex}\textbf{súl}tent.
    
    \item Propter David, servum \textbf{tu}um:\textcolor{red}{~*} non avértas fáciem \textit{Chris}\textit{ti} \textbf{tu}i.
    
    \item Jurávit Dóminus David veritátem, et non frustrábitur \textbf{e}am:\textcolor{red}{~*} de fructu ventris tui ponam super \textit{se}\textit{dem} \textbf{tu}am.
    
    \item Si custodíerint fílii tui testaméntum \textbf{me}um:\textcolor{red}{~*} et testimónia mea hæc, quæ do\textit{cé}\textit{bo} \textbf{e}os.
    
    \item Et fílii eórum usque in \textbf{sǽ}culum:\textcolor{red}{~*} sedébunt super \textit{se}\textit{dem} \textbf{tu}am.
    
    \item Quóniam elégit Dóminus \textbf{Si}on:\textcolor{red}{~*} elégit eam in habitati\textit{ó}\textit{nem} \textbf{si}bi.
    
    \item Hæc réquies mea in sǽculum \textbf{sǽ}culi:\textcolor{red}{~*} hic habitábo, quóniam e\textit{lé}\textit{gi} \textbf{e}am.
    
    \item Víduam ejus benedícens bene\textbf{dí}cam:\textcolor{red}{~*} páuperes ejus satu\textit{rá}\textit{bo} \textbf{pá}nibus.
    
    \item Sacerdótes ejus índuam salu\textbf{tá}ri:\textcolor{red}{~*} et sancti ejus exsultatióne \textit{ex}\textit{sul}\textbf{tá}bunt.
    
    \item Illuc prodúcam cornu \textbf{Da}vid:\textcolor{red}{~*} parávi lucérnam \textit{Chris}\textit{to} \textbf{me}o.
    
    \item Inimícos ejus índuam confusi\textbf{ó}ne:\textcolor{red}{~*} super ipsum autem efflorébit sanctificá\textit{ti}\textit{o} \textbf{me}a.
    
    \item Glória Patri, et \textbf{Fí}lio,\textcolor{red}{~*} et Spirí\textit{tu}\textit{i} \textbf{Sanc}to.
    
    \item Sicut erat in princípio, et nunc, et \textbf{sem}per,\textcolor{red}{~*} et in sǽcula sæcu\textit{ló}\textit{rum}. \textbf{A}men.
  \end{enumerate}
  %  Répetition de l'Antienne
  \grecommentary{\textit{Reprise de l'Antienne.}}
  \gabcsnippet{(c4) De(g') fru(j)ctu(ig) <c>*</c>() ven(i_[uh:l]j)tris(h_g) tu(h_g)i(f_h) (;) po(j)nam(ig~) su(i)per(j) se(h)dem(h) tu(g.)am.(g.) (::) }

  \smallskip
  \begin{footnotesize}
    \begin{center}
      \textit{
        \textcolor{red}{1. }Souvenez-vous, Seigneur ! de David, et de toute sa douceur. 
        \textcolor{red}{2. }qu’il a juré au Seigneur, et a fait ce vœu au Dieu de Jacob : 
        \textcolor{red}{3. }Si j’entre dans le secret de ma maison ; si je monte sur le lit qui est préparé pour me coucher ;
        \textcolor{red}{4. }Si je permets à mes yeux de dormir, et à mes paupières de sommeiller,
        \textcolor{red}{5. }Et si je donne aucun repos à mes tempes, jusqu’à ce que je trouve un lieu propre pour le Seigneur, et un tabernacle pour le Dieu de Jacob.
        \textcolor{red}{6. }Nous avons entendu dire, que l’arche était autrefois dans Ephrata ; nous l’avons trouvée dans un pays plein de bois.
        \textcolor{red}{7. }Nous entrerons dans son tabernacle : nous l’adorerons dans le lieu où il a posé ses pieds.
        \textcolor{red}{8. }Levez-vous, Seigneur ! pour entrer dans votre repos, vous et l’arche où éclate votre sainteté.
        \textcolor{red}{9. }Que vos prêtres soient revêtus de justice, et que vos saints tressaillent de joie.
        \textcolor{red}{10. }En considération de David, votre serviteur, ne rejetez pas le visage de votre Christ.
        \textcolor{red}{11. }Le Seigneur a fait à David un serment trèsvéritable ; et il ne le trompera point : J’établirai sur votre trône le fruit de votre ventre.
        \textcolor{red}{12. }Et que leurs enfants les gardent aussi pour toujours ; ils seront assis sur votre trône.
        \textcolor{red}{13. }Car le Seigneur a choisi Sion ; il l’a choisie pour sa demeure.
        \textcolor{red}{14. }C’est là pour toujours le lieu de mon repos : c’est là que j’habiterai, parce que je l’ai choisie.
        \textcolor{red}{15. }Je donnerai à sa veuve une bénédiction abondante ; je rassasierai de pain ses pauvres.
        \textcolor{red}{16. }Je revêtirai ses prêtres d’une vertu salutaire ; et ses saints seront ravis de joie.
        \textcolor{red}{17. }C’est là que je ferai paraître la puissance de David : j’ai préparé une lampe à mon Christ.
        \textcolor{red}{18. }Je couvrirai de confusion ses ennemis ; mais je ferai éclater sur lui ma propre sanctification.
        \textcolor{red}{19. }Gloire au Père, au Fils, et au Saint Esprit, 
        \textcolor{red}{20. }Comme il était au commencement, maintenant et toujours, et dans les siècles des siècles. Amen.
      }
    \end{center}
  \end{footnotesize}
  \bigskip


  \par \textit{\footnotesize\textcolor{red}{On se lève pour le Capitule.}}

  \begin{center}
    \textcolor{red}{\large{Capitule}}\\
    \small\textit{
      Épître aux Philippiens. IV, 4-5
    }
  \end{center}

  \begin{multicols}{2}
    \parindent=0pt
    Multifariam, multísque modis olim Deus loquens pátribus in Prophétis : \textcolor{red}{†} novíssime diébus istis locútus est nobis in Fílio, quem constítuit herédem universórum, \textcolor{red}{*} per quem fecit et sæcula.  \\
    \textcolor{red}{\Rbar.} Deo grátias.

    \columnbreak

    \textit{ Après avoir, à plusieurs reprises et en diverses manières, parlé autrefois à nos pères par les Prophètes, Dieu, en ces jours qui sont les derniers, nous a parlé par le Fils, qu’il a établi héritier de toutes choses, et par lequel il a aussi créé le monde.  \\
    \textcolor{red}{\Rbar.} Rendons grâce à Dieu.
    }
  \end{multicols}

  \par \textit{\footnotesize\textcolor{red}{Le Célébrant entonne, ensuite, les Chantres et le Chœur alternent les versets. La Doxologie est chantée par tous.}}

  \begin{center}
    \textcolor{red}{\large{Hymne}}\\
  \end{center}
  
  \gresetinitiallines{1}
  \greillumination{\initfamily\fontsize{11mm}{11mm}\selectfont J}
  \gregorioscore{hymnes/hy--jesu_redemptor_omnium_(t._nativitatis)--solesmes_1961}
  \bigskip
  % \begin{multicols}{2}
    \begin{footnotesize}
      \begin{enumerate}[label=\textcolor{red}{\emph{\arabic*}}]
        \item \textit{Christ, Rédempteur de tous les hommes, Fils Unique engendré du Père Avant l’origine du monde, En une ineffable naissance. }
        \item \textit{Vous, lumière, vous, splendeur du Père, Notre espoir éternel à tous, Ecoutez, par tout l’univers Vos serviteurs qui vous supplient. }
        \item \textit{Souvenez-vous, ô Dieu Sauveur, Que vous avez jadis reçu, Naissant de la Vierge sans tache, L’humble livrée de notre corps. }
        \item \textit{Ce jour présent en est témoin, Que le cours de l’année ramène : Descendant du trône du Père Vous seul avez sauvé le monde. }
        \item \textit{Le ciel et la terre et la mer Et tous les êtres qui les peuplent Célèbrent dans un chant joyeux Celui qui vous a envoyé. }
        \item \textit{Et nous qui sommes rachetés Au prix de votre Sang très saint, En ce jour de votre naissance Nous chantons un hymne nouveau.}
        \item \textit{Gloire soit à jamais rendue A vous Seigneur, né de la Vierge, Avec le Père et l’Esprit-Saint Dans les siècles sans fin. Amen. }
      \end{enumerate}
    \end{footnotesize}
  % \end{multicols}

  \begin{center}
    \begin{footnotesize}
      \textcolor{red}{\textit{On chante le verset debout.}}
    \end{footnotesize}
    \begin{minipage}{0.8\linewidth}
      \gresetinitiallines{0}
      \gabcsnippet{(c3)<c><v>\Vbar</v>.</c> Nó(h)tum(h) fé(h)cit(h) Dó(hi)mi(h)nus,(h) (,) al(h)le(fe)lú(f_h){ia}.(hiH'Ghih.ghG'FE'fggf.0) (::) (Z-) 
      <c><v>\Rbar</v>.</c> Sa(h)lu(h)tá(h)re(h) sú(hi)um,(h) (,) al(h)le(fe)lú(f_h){ia}.(hiH'Ghih.ghG'FE'fggf.0) (::)}
      \bigskip
      \begin{center}
        \textit{\textcolor{red}{\Vbar.} Le Seigneur à fait connaître, alléluia. }\\
        \textit{\textcolor{red}{\Rbar.} Son salut, alléluia.}
      \end{center}
    \end{minipage}
  \end{center}
  \normalsize

  \vspace*{\fill}\
  \begin{center}
    \greseparator{3}{30}
  \end{center}
  \vspace*{\fill}\

  \newpage

  \begin{center}
    \textcolor{red}{\large{Antienne à Magnificat}}\\
  \end{center}

  \gresetinitiallines{1}
  \greillumination{\initfamily\fontsize{11mm}{11mm}\selectfont H}
  \gregorioscore{antiennes/an--hodie_christus_natus_est--solesmes_1961}
  \medskip
  \begin{center}
    \footnotesize{\textit{
      Aujourd’hui est né le Christ, aujourd’hui le Sauveur est apparu ; aujourd’hui sur la terre chantent les Anges, se réjouissent les Archanges ; aujourd’hui les justes dans les transports de leur joie, répètent : Gloire à Dieu au plus haut des cieux, alléluia.  
    }}
  \end{center}
  \medskip

  \gresetinitiallines{0}
  \gregorioscore{magnificat/magnificat-Ig2}

  \begin{enumerate}[label=\textcolor{red}{\arabic*}]
    \setcounter{enumi}{2}
    \item Quia respéxit humilitátem \textit{an}\textit{cíl}\textit{læ} \textbf{su}æ:\textcolor{red}{~*} ecce enim ex hoc beátam me dicent omnes gene\textit{ra}\textit{ti}\textbf{ó}nes.

    \item Quia fecit mihi \textit{ma}\textit{gna} \textit{qui} \textbf{pot}\textbf{ens} est:\textcolor{red}{~*} et sanctum \textit{no}\textit{men} \textbf{e}jus.
    
    \item Et misericórdia ejus a progéni\textit{e} \textit{in} \textit{pro}\textbf{gé}\textbf{ni}es\textcolor{red}{~*} timén\textit{ti}\textit{bus} \textbf{e}um.
    
    \item Fecit poténtiam in \textit{brá}\textit{chi}\textit{o} \textbf{su}o:\textcolor{red}{~*} dispérsit supérbos mente \textit{cor}\textit{dis} \textbf{su}i.
    
    \item Depósuit pot\textit{én}\textit{tes} \textit{de} \textbf{se}de,\textcolor{red}{~*} et exal\textit{tá}\textit{vit} \textbf{hú}miles.
    
    \item Esuriéntes \textit{im}\textit{plé}\textit{vit} \textbf{bo}nis:\textcolor{red}{~*} et dívites dimí\textit{sit} \textit{in}\textbf{á}nes.
    
    \item Suscépit Israël \textit{pú}\textit{e}\textit{rum} \textbf{su}um,\textcolor{red}{~*} recordátus misericór\textit{di}\textit{æ} \textbf{su}æ.
    
    \item Sicut locútus est \textit{ad} \textit{pa}\textit{tres} \textbf{nos}tros,\textcolor{red}{~*} Abraham et sémini e\textit{jus} \textit{in} \textbf{sǽ}cula.
    
    \begin{center}
      \textit{\footnotesize \textcolor{red}{On attend la fin de l'encensement avant de chanter la doxologie.}}
    \end{center}
    \item Glória \textit{Pa}\textit{tri}, \textit{et} \textbf{Fí}\textbf{li}o,\textcolor{red}{~*} et Spirí\textit{tu}\textit{i} \textbf{Sanc}to.
    
    \item Sicut erat in princípio, \textit{et} \textit{nunc}, \textit{et} \textbf{sem}per,\textcolor{red}{~*} et in sǽcula sæcu\textit{ló}\textit{rum}. \textbf{A}men.
  \end{enumerate}

  \grecommentary{\textit{Reprise de l'Antienne.}}
  \gabcsnippet{(c4) Ho(f)di(gh)e(h.) (,) Chri(ixh.g!hwi)stus(h') na(g)tus(fh) est :(h.) (;) hó(f)di(fg)e(g'_[oh:h]) (,) Sal(g)vá(g)tor(g) ap(gh)pá(h)ru(gf)it :(f.) (:) hó(f)di(gh)e(h'_) (,) in(h) ter(h)ra(hg) ca(ixi)nunt(h') An(g)ge(fh)li,(h'_) (;) lae(h)<nlba>tán(hg)tur</nlba>(ixi) Ar(h')chán(g)ge(fh)li :(h.) (:) hó(h)di(j)e(j_kJ'IH') (,) ex(h)súl(ixh.g!hwi)tant(h') ju(g)sti,(f_h) (,) dí(f)cen(h)tes :(g_[uh:l]h) (;) Gló(g_[uh:l]h)ri(fe)a(d) in(e') ex(f)cél(g_[uh:l]h)sis(g) De(fe)o,(dc) al(d)le(fe)lú(d.){ia}.(d.) (::)}

  \newpage

  \begin{center}
    \textcolor{red}{\large{Oraison}}\\
  \end{center}

  \begin{multicols}{2}
    \parindent=0pt
    \begin{flushright}
      \textcolor{red}{\Vbar.} Dominus vobiscum.\\
      \textcolor{red}{\Rbar.} Et cum spiritu tuo.\\
    \end{flushright}

    \columnbreak
    
    \textit{\textcolor{red}{\Vbar.} Le Seigneur soit avec vous.\\
    \textcolor{red}{\Rbar.} Et avec votre esprit.}\\
  \end{multicols}

  \begin{multicols}{2}
    \parindent=0pt
    Concéde, quæsumus, omnípotens Deus : \textcolor{red}{†} ut nos Unigéniti tui nova per carnem Natívitas líberet ; \textcolor{red}{*} quos sub peccáti jugo vetústa sérvitus tenet.  \\Per eúmdem Dóminum nostrum Jesum Christum Fílium tuum, qui tecum vivit et regnat in unitáte Spíritus sancti Deus : per ómnia sæcula sæculórum.
    \textcolor{red}{\Rbar.} Amen.

    \columnbreak

    \textit{ Accordez, nous vous le demandons, Dieu toutpuissant, que la nouvelle naissance de votre Fils unique en la chair nous rende libres, nous qu’une antique servitude tient sous le joug du péché. Par notre même Seigneur Jésus-Christ, votre Fils, qui avec vous vit et règne en l’unité du Saint Esprit, Dieu pour tous les siècles des siècles. 
    Amen.
    }
  \end{multicols}

  \medskip

  \begin{center}
    \textcolor{red}{\large{Conclusion de l'office}}
  \end{center}
  
  
  \begin{multicols}{2}
    \parindent=0pt
    \begin{flushright}
      \textcolor{red}{\Vbar.} Dominus vobiscum.\\
      \textcolor{red}{\Rbar.} Et cum spiritu tuo.\\
    \end{flushright}
  
    \columnbreak
    
    \textit{\textcolor{red}{\Vbar.} Le Seigneur soit avec vous.\\
    \textcolor{red}{\Rbar.} Et avec votre esprit.}\\
  \end{multicols}
  \bigskip
  \gresetinitiallines{1}
  \greillumination{\initfamily\fontsize{11mm}{11mm}\selectfont B}
  \gregorioscore{or--benedicamus_domino_(i_classis_in_ii_vesperis_mode_5)--solesmes_1961}
  \begin{center}
    \begin{footnotesize}
      \textcolor{red}{\textit{Sur un ton très grave : }}
    \end{footnotesize}
  \end{center}
  \begin{multicols}{2}
    \parindent=0pt
    \textcolor{red}{\Vbar.} Fidélium ánimæ per misericórdiam Dei requiéscant in pace.\\
    \textcolor{red}{\Rbar.} Amen.\\

    \columnbreak
    
    \textit{\textcolor{red}{\Vbar.} Que les âmes des fidèles défunts, par la
    miséricorde de Dieu, reposent en paix.\\
    \textcolor{red}{\Rbar.} Amen.}\\
  \end{multicols}

  \medskip

  \begin{center}
    \textcolor{red}{\large{Salut du Saint Sacrement}}\\
    \textit{Voir page 50.}
  \end{center}

  \vspace*{\fill}\
  \begin{center}
    \greseparator{3}{30}
  \end{center}
  \vspace*{\fill}\

  \newpage

  \begin{center}
    \begin{LARGE}
      Dimanche dans l'octave de Noël \\
    \end{LARGE}
    \textit{
      \textcolor{red}{Antiennes, psaumes et hymne de Noël, page 3}
    }
  \end{center}
  \medskip

  

  \par \textit{\footnotesize\textcolor{red}{On se lève pour le Capitule.}}

  \begin{center}
    \textcolor{red}{\large{Capitule}}\\
    \small\textit{
      Épître aux Galates. IV, 1-2
    }
  \end{center}

  \begin{multicols}{2}
    \parindent=0pt
    Fratres : Quando témpore hæres párvulus est, nihil differta servo, cum sit dóminus ómnium : \textcolor{red}{†} sed sub tutóribus et actóribus est, \textcolor{red}{*} usque ad præfinítum tempus a patre.\\
    \textcolor{red}{\Rbar.} Deo grátias.

    \columnbreak

    \textit{ Frères, aussi longtemps que l’héritier est mineur, il ne diffère en rien d’un esclave, bien qu’il soit le maître de tout ; mais il est soumis à des tuteurs et à des administrateurs, jusqu’à la date fixée par son père.\\
    \textcolor{red}{\Rbar.} Rendons grâce à Dieu.
    }
  \end{multicols}

  \begin{center}
    \begin{footnotesize}
      \textcolor{red}{\textit{On chante le verset debout.}}
    \end{footnotesize}
    \begin{minipage}{0.8\linewidth}
      \gresetinitiallines{0}
      \gabcsnippet{(c3)<c><v>\Vbar</v>.</c> Vér(h)bum(h) cá(h)ro(h) fác(hi)tum(h) est,(h) (,) al(h)le(fe)lú(f_h){ia}.(hiH'Ghih.ghG'FE'fggf.0) (::) (Z-) 
      <c><v>\Rbar</v>.</c> Et(h) ha(h)bi(h)tá(h)vit(h) in(h) nó(hi)bis,(h) (,) al(h)le(fe)lú(f_h){ia}.(hiH'Ghih.ghG'FE'fggf.0) (::)}
      \bigskip
      \begin{center}
        \textit{\textcolor{red}{\Vbar.} Le Verbe s’est fait cher, alléluia.}\\
        \textit{\textcolor{red}{\Rbar.} Et il a habité parmi nous, alléluia.}
      \end{center}
    \end{minipage}
  \end{center}
  \normalsize

  \vspace*{\fill}\
  \begin{center}
    \greseparator{3}{30}
  \end{center}
  \vspace*{\fill}\

  \newpage

  \begin{center}
    \textcolor{red}{\large{Antienne à Magnificat}}\\
  \end{center}

  \gresetinitiallines{1}
  \greillumination{\initfamily\fontsize{11mm}{11mm}\selectfont P}
  \gregorioscore{antiennes/an--puer_jesus_proficiebat--solesmes_1934}
  \medskip
  \begin{center}
    \footnotesize{\textit{
      L’Enfant Jésus croissait en âge et en sagesse devant Dieu et devant les hommes.
    }}
  \end{center}
  \medskip

  \gresetinitiallines{0}
  \gregorioscore{magnificat/magnificat-VIF}

  \begin{enumerate}[label=\textcolor{red}{\arabic*}]
    \setcounter{enumi}{2}
    \item Quia respéxit humilitátem an\textbf{cíl}læ \textbf{su}æ:\textcolor{red}{~*} ecce enim ex hoc beátam me dicent omnes gene\textit{ra}\textit{ti}\textbf{ó}nes.

    \item Quia fecit mihi \textbf{ma}gna qui \textbf{pot}ens est:\textcolor{red}{~*} et sanctum \textit{no}\textit{men} \textbf{e}jus.
    
    \item Et misericórdia ejus a progénie \textbf{in} pro\textbf{gé}nies\textcolor{red}{~*} timén\textit{ti}\textit{bus} \textbf{e}um.
    
    \item Fecit poténtiam in \textbf{brá}chio \textbf{su}o:\textcolor{red}{~*} dispérsit supérbos mente \textit{cor}\textit{dis} \textbf{su}i.
    
    \item Depósuit pot\textbf{én}tes de \textbf{se}de,\textcolor{red}{~*} et exal\textit{tá}\textit{vit} \textbf{hú}miles.
    
    \item Esuriéntes im\textbf{plé}vit \textbf{bo}nis:\textcolor{red}{~*} et dívites dimí\textit{sit} \textit{in}\textbf{á}nes.
    
    \item Suscépit Israël \textbf{pú}erum \textbf{su}um,\textcolor{red}{~*} recordátus misericór\textit{di}\textit{æ} \textbf{su}æ.
    
    \item Sicut locútus est ad \textbf{pa}tres \textbf{nos}tros,\textcolor{red}{~*} Abraham et sémini e\textit{jus} \textit{in} \textbf{sǽ}cula.

    \begin{center}
      \textit{\footnotesize \textcolor{red}{On attend la fin de l'encensement avant de chanter la doxologie.}}
    \end{center}
    
    \item Glória \textbf{Pa}tri, et \textbf{Fí}lio,\textcolor{red}{~*} et Spirí\textit{tu}\textit{i} \textbf{Sanc}to.
    
    \item Sicut erat in princípio, et \textbf{nunc}, et \textbf{sem}per,\textcolor{red}{~*} et in sǽcula sæcu\textit{ló}\textit{rum}. \textbf{A}men.
  \end{enumerate}

  \grecommentary{\textit{Reprise de l'Antienne.}}
  \gabcsnippet{(c4) Pu(f)er(c) Je(df)sus(ffg.) (,) pro(f)fi(f')ci(f)é(gh)bat(h') æ(h)tá(hj)te(h') (,) et(g) sa(hg)pi(f)én(gh)ti(gf)a(fv_2/fv_/ggf.) (;) co(f)ram(f) De(f_g)o(e_f) et(d) ho(cf~)mí(f')ni(f)bus.(f.) (::)}

  \newpage

  \begin{center}
    \textcolor{red}{\large{Oraison}}\\
  \end{center}

  \begin{multicols}{2}
    \parindent=0pt
    \begin{flushright}
      \textcolor{red}{\Vbar.} Dominus vobiscum.\\
      \textcolor{red}{\Rbar.} Et cum spiritu tuo.\\
    \end{flushright}

    \columnbreak
    
    \textit{\textcolor{red}{\Vbar.} Le Seigneur soit avec vous.\\
    \textcolor{red}{\Rbar.} Et avec votre esprit.}\\
  \end{multicols}

  \begin{multicols}{2}
    \parindent=0pt
    Omnipotens sempitérne Deus, dírige actus nostros in beneplácito tuo :\textcolor{red}{†} ut in nómine dilécti Fílii \textcolor{red}{*} tui mereámur bonis opéribus abundáre :\\ Qui tecum vivis et regnas in unitáte Spíritus Sancti Deus : per ómnia sæcula sæculórum. 
    \textcolor{red}{\Rbar.} Amen.

    \columnbreak

    \textit{Dieu tout-puissant et éternel, dans votre bienveillance dirigez nos actions, afin qu’au nom de votre Fils bien-aimé, nous méritions d’abonder en bonnes œuvres. Lui qui avec vous vis et règne en l’unité du Saint Esprit, Dieu pour tous les siècles des siècles. 
    Amen.
    }
  \end{multicols}

  \medskip

  \begin{center}
    \textcolor{red}{\large{Conclusion de l'office}}
  \end{center}
  
  
  \begin{multicols}{2}
    \parindent=0pt
    \begin{flushright}
      \textcolor{red}{\Vbar.} Dominus vobiscum.\\
      \textcolor{red}{\Rbar.} Et cum spiritu tuo.\\
    \end{flushright}
  
    \columnbreak
    
    \textit{\textcolor{red}{\Vbar.} Le Seigneur soit avec vous.\\
    \textcolor{red}{\Rbar.} Et avec votre esprit.}\\
  \end{multicols}
  \bigskip
  \gresetinitiallines{1}
  \greillumination{\initfamily\fontsize{11mm}{11mm}\selectfont B}
  \gregorioscore{or--benedicamus_domino_(i_classis_in_ii_vesperis_mode_5)--solesmes_1961}
  \begin{center}
    \begin{footnotesize}
      \textcolor{red}{\textit{Sur un ton très grave : }}
    \end{footnotesize}
  \end{center}
  \begin{multicols}{2}
    \parindent=0pt
    \textcolor{red}{\Vbar.} Fidélium ánimæ per misericórdiam Dei requiéscant in pace.\\
    \textcolor{red}{\Rbar.} Amen.\\

    \columnbreak
    
    \textit{\textcolor{red}{\Vbar.} Que les âmes des fidèles défunts, par la
    miséricorde de Dieu, reposent en paix.\\
    \textcolor{red}{\Rbar.} Amen.}\\
  \end{multicols}

  \medskip
\end{document}