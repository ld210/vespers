% !TeX program = lualatex
\documentclass[12pt, a4paper]{article}
\usepackage{fullpage}
\usepackage{subfiles}
\usepackage{fontspec}
\usepackage{libertine}
\usepackage{xcolor}
\usepackage{GotIn}
\usepackage{geometry}
\usepackage{multicol}
\usepackage{multicolrule}
\usepackage{graphicx}
\usepackage{enumitem}
\usepackage[autocompile]{gregoriotex}
\usepackage[latin,french]{babel}


\geometry{top=2cm, bottom=2cm}
% \pagestyle{empty}

\definecolor{red}{HTML}{C70039}
% \input GoudyIn.fd
% \newcommand*\initfamily{\usefont{U}{GoudyIn}{xl}{n}}

\input Acorn.fd
\newcommand*\initfamily{\usefont{U}{Acorn}{xl}{n}}
% cette ligne ajoute de l'espace entre les portées
% \grechangedim{baselineskip}{60pt}{scalable}

\begin{document}
  \gresetlinecolor{gregoriocolor}

  \begin{titlepage}\centering
    \vspace*{\fill}\
    \huge Secondes Vêpres\\
    \smallskip
    \begin{Large}
      \textit{
        de la Nativité de Notre Seigneur, jusqu'à l'Épiphanie\\
      }
    \end{Large}
    \medskip
    \large et\\
    \medskip
    \LARGE Salut du Saint-Sacrement\\
    \bigskip
    \vspace*{\fill}
    \begin{figure}[h!]
      \centering
      \includegraphics[width=7cm]{../epiphanie_septuagesime_careme/logo.png}
    \end{figure}
    \centering \normalsize Paroisse Saint Roch\\
    \bigskip
    \begin{Large}
      \centering Ne pas emporter
    \end{Large}
  \end{titlepage}

  \newpage
  \vspace*{\fill}
  \begin{center}
    \large Sommaire\\
    \begin{multicols}{2}
      \normalsize
      \begin{flushleft}
        I\textsuperscript{er} dimanche de l'Avent\\
        II\textsuperscript{er} dimanche de l'Avent\\
        III\textsuperscript{er} dimanche de l'Avent\\
        IV\textsuperscript{er} dimanche de l'Avent\\
        Les grandes Antiennes "O"\\
        Salut du Très Saint Sacrement\\
        Rorate Cæli\\
      \end{flushleft}
      \columnbreak
      \begin{flushright}
        \textit{
        page 3\\
        page 17\\
        page 26\\
        page 35\\
        page 43\\
        page 50\\
        page 54\\
      }
      \end{flushright}
    \end{multicols}
  \end{center}
  \vspace*{\fill}
  \begin{center}
    \normalsize\textit{
      Livret latin-français
    }
  \end{center}
  \newpage
  \normalsize
  \begin{center}
    \textcolor{red}{\large{Invitatoire.}}
  \end{center}

  % greillumination: remplace la première lettre, ici par une font ornementale
  \greillumination{\initfamily\fontsize{11mm}{11mm}\selectfont D}
  \gregorioscore{or--deus_in_adjutorium_(tonus_festivus)--solesmes_1961.1}

  \begin{center}
    \small{
    \emph{
        Dieu, venez à mon aide ; Seigneur, hatez-vous de me secourir.\\
        Gloire au Père, au Fils et au Saint Esprit, comme il était au commencement, maintenant et toujour et dans les siècles des siècles. Allelúia\\
      }
    }
  \end{center}

  \bigskip

  \begin{center}
    \textcolor{red}{\footnotesize\textit{
      Le célébrant entonne la première antienne. Les chantres entonnent les antiennes suivantes, ainsi que le premier verset de chaque psaumes.
    }}
  \end{center}

  \vspace*{\fill}\
  \begin{center}
    \greseparator{3}{30}
  \end{center}
  \vspace*{\fill}\

  \newpage
  \normalsize

  \begin{center}
    \begin{LARGE}
      Vêpres du jour de Noël
    \end{LARGE}
  \end{center}
  \medskip

  % ===== DEBUT Antienne =========
  \greillumination{\initfamily\fontsize{11mm}{11mm}\selectfont T}
  \gregorioscore{antiennes/an--tecum_principium--solesmes_1961}
  \begin{center}
    \footnotesize{
      \textit{
        Avec vous est le principe au jour de votre puissance, dans les splendeurs des Saints : c’est de mon sein, qu’avant que l’aurore existât, je vous ai engendré. 
      }
    }
  \end{center}
  % \medskip
  % ===== FIN Antienne ===========

  % ===== DEBUT psaume ===========
  % gresetinitiallines : avec le parametre à 0, supprime l'ornement
  \gresetinitiallines{0}

  \begin{center}
    \normalsize{Psaume 109.}
  \end{center}

  \gregorioscore{psaumes/psaume109-IG}
  \begin{enumerate}[label=\textcolor{red}{\emph{\arabic*}}]
    \setcounter{enumi}{2}
    \item Donec ponam ini\textbf{mí}cos \textbf{tu}os,~* scabéllum pe\textit{dum} \textit{tu}\textbf{ó}rum.

    \item Virgam virtútis tuæ emíttet Dómi\textbf{nus} ex \textbf{Si}on:~* domináre in médio inimicó\textit{rum} \textit{tu}\textbf{ó}rum.
    
    \item Tecum princípium in die virtútis tuæ in splendóri\textbf{bus} sanc\textbf{tó}rum:~* ex útero ante lucíferum \textit{gé}\textit{nu}\textbf{i} te.
    
    \item Jurávit Dóminus, et non pœni\textbf{té}bit \textbf{e}um:~* Tu es sacérdos in ætérnum secúndum órdi\textit{nem} \textit{Mel}\textbf{chí}sedech.
    
    \item Dóminus a \textbf{dex}tris \textbf{tu}is,~* confrégit in die iræ \textit{su}\textit{æ} \textbf{re}ges.
    
    \item Judicábit in natiónibus, im\textbf{plé}bit ru\textbf{í}nas:~* conquassábit cápita in ter\textit{ra} \textit{mul}\textbf{tó}rum.
    
    \item De torrénte in \textbf{vi}a \textbf{bi}bet:~* proptérea exal\textit{tá}\textit{bit} \textbf{ca}put.
    
    \item Glória \textbf{Pa}tri, et \textbf{Fí}lio,~* et Spirí\textit{tu}\textit{i} \textbf{Sanc}to.
    
    \item Sicut erat in princípio, et \textbf{nunc}, et \textbf{sem}per,~* et in sǽcula sæcu\textit{ló}\textit{rum}. \textbf{A}men.
  \end{enumerate}
  \medskip
  \grecommentary{\textit{Reprise de l'Antienne.}}
  \gabcsnippet{((c4) Te(f)cum(c') prin(d)cí(ixdh'!iv)pi(h_)um(h'_) (,) in(h_) di(h_)e(hgh) vir(f_g)tú(gh)tis(fe) tu(d)ae,(d.) (;) in(f) splen(f)dó(f_g)ri(f)bus(c') san(d_)ctó(d_)rum,(c.) (;) ex(c) ú(f')te(f_)ro(f'_) (,) an(f)te(f) lu(f)cí(g_[uh:l]h)fe(f)rum(ec~) gé(d'_)nu(f)i(e) te.(d.) (::)}

  \newpage
  \vspace*{\fill}\
  \begin{normalsize}
    \begin{center}
      \begin{enumerate}[label=\textcolor{red}{\emph{\arabic*}}]
        \item \textit{L'Éternel a dit à mon Seigneur: Assieds-toi à ma droite.}
        \item \textit{Jusqu'à ce que je mette tes ennemis pour le marchepied de tes pieds.}
        \item \textit{L'Éternel enverra de Sion la verge de ta force: Domine au milieu de tes ennemis!}
        \item \textit{Ton peuple sera un peuple de franche volonté, au jour de ta puissance, en sainte magnificence. Du sein de l'aurore te viendra la rosée de ta jeunesse.}
        \item \textit{L'Éternel a juré, et il ne se repentira point: Tu es sacrificateur pour toujours, selon l'ordre de Melchisédec.}
        \item \textit{Le Seigneur, à ta droite, brisera les rois au jour de sa colère.}
        \item \textit{Il jugera parmi les nations, il remplira tout de corps morts, il brisera le chef d'un grand pays.}
        \item \textit{Il boira du torrent dans le chemin, c'est pourquoi il lèvera haut la tête.}
        \item \textit{Gloire au Père, au Fils, et au Saint Esprit, }
        \item \textit{Comme il était au commencement, maintenant et toujours, et dans les siècles des siècles. Amen. }
      \end{enumerate}
    \end{center}
  \end{normalsize}
  \vspace*{\fill}\
  \newpage

  % ===== DEBUT Antienne =========
  \gresetinitiallines{1}
  \greillumination{\initfamily\fontsize{11mm}{11mm}\selectfont R}
  \gregorioscore{antiennes/an--redemptionem_misit--solesmes_1961}
  \begin{center}
    \footnotesize{
      \textit{
        Il a envoyé la rédemption à son peuple : il a établi pour l’éternité son alliance. 
      }
    }
  \end{center}
  % \medskip
  % ===== FIN Antienne ===========

  % ===== DEBUT psaume ===========
  % gresetinitiallines : avec le parametre à 0, supprime l'ornement
  \gresetinitiallines{0}

  \begin{center}
    \normalsize{Psaume 110.}
  \end{center}

  \gregorioscore{psaumes/psaume110-VIIa}

  \begin{enumerate}[label=\textcolor{red}{\emph{\arabic*}}]
    \setcounter{enumi}{1}
    \item Magna \textbf{ó}pera \textbf{Dó}mini:~* exquisíta in omnes volun\textbf{tá}tes \textbf{e}jus.

    \item Conféssio et magnificéntia \textbf{o}pus \textbf{e}jus:~* et justítia ejus manet in \textbf{sǽ}culum \textbf{sǽ}culi.
    
    \item Memóriam fecit mirabílium suórum,~† miséricors et mise\textbf{rá}tor \textbf{Dó}minus:~* escam dedit ti\textbf{mén}ti\textbf{bus} se.
    
    \item Memor erit in sǽculum testa\textbf{mén}ti \textbf{su}i:~* virtútem óperum suórum annuntiábit \textbf{pó}pulo \textbf{su}o:
    
    \item Ut det illis heredi\textbf{tá}tem \textbf{gén}tium:~* ópera mánuum ejus véritas, \textbf{et} ju\textbf{dí}cium.
    
    \item Fidélia ómnia mandáta ejus:~† confirmáta in \textbf{sǽ}culum \textbf{sǽ}culi,~* facta in veritáte et \textbf{æ}qui\textbf{tá}te.
    
    \item Redemptiónem misit \textbf{pó}pulo \textbf{su}o:~* mandávit in ætérnum testa\textbf{mén}tum \textbf{su}um.
    
    \item Sanctum, et terríbile \textbf{no}men \textbf{e}jus:~* inítium sapiéntiæ \textbf{ti}mor \textbf{Dó}mini.
    
    \item Intelléctus bonus ómnibus faci\textbf{én}tibus \textbf{e}um:~* laudátio ejus manet in \textbf{sǽ}culum \textbf{sǽ}culi.
    
    \item Glória \textbf{Pa}tri, et \textbf{Fí}lio,~* et Spi\textbf{rí}tui \textbf{Sanc}to.
    
    \item Sicut erat in princípio, et \textbf{nunc}, et \textbf{sem}per,~* et in sǽcula sæcu\textbf{ló}rum. \textbf{A}men.
  \end{enumerate}

  \medskip
  \grecommentary{\textit{Reprise de l'Antienne.}}
  \gabcsnippet{(c3) Red(e')em(e)pti(g')ó(i)nem(h') () mi(f)sit(h) Dó(g')mi(f)nus(e'_[oh:h]) (,) pó(h)pu(h)lo(hf) su(h.)o :(g.) (;) man(i')dá(i)vit(i') in(h) ae(h)tér(iji)num(hg/hih.) (,) te(f)sta(e)mén(f!gwh_f~)tum(g_[oh:h]f~) su(e.)um.(e.) (::)}

  \newpage
  \vspace*{\fill}\
  \begin{normalsize}
    \begin{center}
      \begin{enumerate}[label=\textcolor{red}{\emph{\arabic*}}]
        \item \textit{Grandes sont les œuvres du Seigneur ; tous ceux qui les aiment s'en instruisent.}
        \item \textit{De tout cœur je rendrai grâce au Seigneur dans l'assemblée, parmi les justes.}
        \item \textit{Grandes sont les œuvres du Seigneur ; tous ceux qui les aiment s'en instruisent.}
        \item \textit{Noblesse et beauté dans ses actions : à jamais se maintiendra sa justice.}
        \item \textit{De ses merveilles il a laissé un mémorial ; le Seigneur est tendresse et pitié, il a donné des vivres à ses fidèles,}
        \item \textit{Gardant toujours mémoire de son
        alliance, il a montré sa force à son peuple.}
        \item \textit{Lui donnant le domaine des nations. Justesse et sûreté les œuvres de ses mains.}
        \item \textit{Sécurité, toutes ses lois, établies pour toujours et à jamais, accomplies avec droiture et sûreté ! }
        \item \textit{Il apporte la délivrance à son peuple ; son alliance est promulguée pour toujours.}
        \item \textit{Saint, redoutable est son nom, la sagesse commence avec la crainte du Seigneur.}
        \item \textit{Qui accomplit sa volonté en est éclairé. A jamais se maintiendra sa louange.}
        \item \textit{Gloire au Père, au Fils, et au Saint Esprit, }
        \item \textit{Comme il était au commencement, maintenant et toujours, et dans les siècles des siècles. Amen. }
      \end{enumerate}
    \end{center}
  \end{normalsize}
  \vspace*{\fill}\
  \newpage

  % ===== DEBUT Antienne =========
  \gresetinitiallines{1}
  \greillumination{\initfamily\fontsize{11mm}{11mm}\selectfont E}
  \gregorioscore{antiennes/an--exortum_est_(christmas)--solesmes_1961}
  \begin{center}
    \footnotesize{
      \textit{
        Il s’est élevé dans les ténèbres une lumière pour les hommes droits : le Seigneur est miséricordieux, compatissant et juste. 
      }
    }
  \end{center}
  % \medskip
  % ===== FIN Antienne ===========

  % ===== DEBUT psaume ===========
  % gresetinitiallines : avec le parametre à 0, supprime l'ornement

  \gresetinitiallines{0}

  \begin{center}
    \normalsize{Psaume 111.}\\
  \end{center}
  % \smallskip
  \grechangedim{baselineskip}{50pt}{scalable}
  \gregorioscore{psaumes/psaume111-VIIb}
  \begin{enumerate}[label=\textcolor{red}{\emph{\arabic*}}]
    \setcounter{enumi}{1}
    \item Potens in terra erit \textbf{se}men \textbf{e}jus:~* generátio rectórum be\textbf{ne}di\textbf{cé}tur.

    \item Glória, et divítiæ in \textbf{do}mo \textbf{e}jus:~* et justítia ejus manet in \textbf{sǽ}culum \textbf{sǽ}culi.
    
    \item Exórtum est in ténebris \textbf{lu}men \textbf{rec}tis:~* miséricors, et mise\textbf{rá}tor, et \textbf{jus}tus.
    
    \item Jucúndus homo qui miserétur et cómmodat,~† dispónet sermónes suos \textbf{in} ju\textbf{dí}cio:~* quia in ætérnum non \textbf{com}mo\textbf{vé}bitur.
    
    \item In memória ætérna \textbf{e}rit \textbf{jus}tus:~* ab auditióne mala \textbf{non} ti\textbf{mé}bit.
    
    \item Parátum cor ejus speráre in Dómino,~† confirmátum \textbf{est} cor \textbf{e}jus:~* non commovébitur donec despíciat ini\textbf{mí}cos \textbf{su}os.
    
    \item Dispérsit, dedit paupéribus:~† justítia ejus manet in \textbf{sǽ}culum \textbf{sǽ}culi,~* cornu ejus exaltábi\textbf{tur} in \textbf{gló}ria.
    
    \item Peccátor vidébit, et irascétur,~† déntibus suis fremet \textbf{et} ta\textbf{bé}scet:~* desidérium pecca\textbf{tó}rum per\textbf{í}bit.
    
    \item Glória \textbf{Pa}tri, et \textbf{Fí}lio,~* et Spi\textbf{rí}tui \textbf{Sanc}to.
    
    \item Sicut erat in princípio, et \textbf{nunc}, et \textbf{sem}per,~* et in sǽcula sæcu\textbf{ló}rum. \textbf{A}men.
  \end{enumerate}

  \medskip
  \grecommentary{\textit{Reprise de l'Antienne.}}
  \gabcsnippet{(c3) Ex(g)ór(ig/ij)tum(i) est(i.) (,) in(i_[uh:l]j) té(i)ne(hih)bris(hgh) (,) lu(hg)men(f') re(gh'i)ctis(f') cor(e)de :(e.) (;) mi(f)sé(h')ri(h)cors(h') et(h) mi(h)se(f')rá(h)tor,(gf) (,) et(e_[uh:l]f) ju(g')stus(f) Dó(e')mi(e)nus.(e.) (::)}

  \newpage
  \vspace*{\fill}\
  \begin{normalsize}
    \begin{center}
      \begin{enumerate}[label=\textcolor{red}{\emph{\arabic*}}]
        \item \textit{Celui qui craint le Seigneur a une volonté ardente d’accomplir ses commandements.}
        \item \textit{Heureux l’homme qui craint le Seigneur, qui aime entièrement sa volonté !}
        \item \textit{Sa lignée sera puissante sur la terre ; la
        race des justes est bénie.}
        \item \textit{Les richesses affluent dans sa maison : à
        jamais se maintiendra sa justice.}
        \item \textit{Lumière des cœurs droits, il s'est levé
        dans les ténèbres, l’homme de justice, de
        tendresse et de pitié.}
        \item \textit{L'homme de bien a pitié, il partage ; il
        mène ses affaires avec droiture, cet
        homme jamais ne tombera ;}
        \item \textit{Toujours on fera mémoire du juste, il ne
        craint pas l'annonce d'un malheur :}
        \item \textit{Le cœur ferme, il s'appuie sur le
        Seigneur. Son cœur est confiant, il ne
        craint pas : il verra ce que valaient ses
        oppresseurs.}
        \item \textit{A pleines mains, il donne au pauvre ; à
        jamais se maintiendra sa justice, sa
        puissance grandira, et sa gloire !}
        \item \textit{L'impie le voit et s'irrite ; il grince des
        dents et se détruit. L'ambition des impies
        se perdra.}
        \item \textit{Gloire au Père, au Fils, et au Saint Esprit, }
        \item \textit{Comme il était au commencement, maintenant et toujours, et dans les siècles des siècles. Amen. }
      \end{enumerate}
    \end{center}
  \end{normalsize}
  \vspace*{\fill}\
  \newpage

  % ===== DEBUT Antienne =========
  \gresetinitiallines{1}
  \greillumination{\initfamily\fontsize{11mm}{11mm}\selectfont A}
  \gregorioscore{antiennes/an--apud_dominum_misericordia--solesmes_1961}
  \begin{center}
    \footnotesize{
      \textit{
        Auprès du Seigneur est la miséricorde, et abonde chez lui la rédemption.
      }
    }
  \end{center}
  % ===== FIN Antienne ===========

  % ===== DEBUT psaume ===========
  % gresetinitiallines : avec le parametre à 0, supprime l'ornement
  \gresetinitiallines{0}

  \begin{center}
    \normalsize{Psaume 129.}\\
  \end{center}
  % \smallskip

  \gregorioscore{psaumes/psaume129-IVa}

  \begin{enumerate}[label=\textcolor{red}{\emph{\arabic*}}]
    \setcounter{enumi}{1}
    \item Fiant aures tuæ \textit{in}\textit{ten}\textbf{dén}tes:~* in vocem depreca\textit{ti}\textit{ó}\textit{nis} \textbf{me}æ.

    \item Si iniquitátes observá\textit{ve}\textit{ris}, \textbf{Dó}mine:~* Dómine, \textit{quis} \textit{sus}\textit{ti}\textbf{né}bit?
    
    \item Quia apud te propiti\textit{á}\textit{ti}\textbf{o} est:~* et propter legem tuam sustí\textit{nu}\textit{i} \textit{te}, \textbf{Dó}mine.
    
    \item Sustínuit ánima mea in \textit{ver}\textit{bo} \textbf{e}jus:~* sperávit ánima \textit{me}\textit{a} \textit{in} \textbf{Dó}mino.
    
    \item A custódia matutína us\textit{que} \textit{ad} \textbf{noc}tem:~* speret Is\textit{ra}\textit{ël} \textit{in} \textbf{Dó}mino.
    
    \item Quia apud Dóminum mi\textit{se}\textit{ri}\textbf{cór}dia:~* et copiósa apud \textit{e}\textit{um} \textit{red}\textbf{émp}tio.
    
    \item Et ipse réd\textit{i}\textit{met} \textbf{Is}raël:~* ex ómnibus iniqui\textit{tá}\textit{ti}\textit{bus} \textbf{e}jus.
    
    \item Glória Pa\textit{tri}, \textit{et} \textbf{Fí}lio,~* et Spi\textit{rí}\textit{tu}\textit{i} \textbf{Sanc}to.
    
    \item Sicut erat in princípio, et \textit{nunc}, \textit{et} \textbf{sem}per,~* et in sǽcula sæ\textit{cu}\textit{ló}\textit{rum}. \textbf{A}men.
  \end{enumerate}

  \medskip
  \grecommentary{\textit{Reprise de l'Antienne.}}
  \gabcsnippet{(c3) A(e')pud(f) Dó(h)mi(hi)num(i.) (,) mi(h_i)se(j)ri(ih)cór(i')di(i)a,(i.) (;) et(f_i) co(g')pi(h)ó(f_e)sa(f_e) (,) a(d')pud(e) e(gxf_g)um(h') red(h)ém(f')pti(f)o.(f.) (::) <eu>E(i) u(h) o(i) u(j) a(h) e.</eu>(f.) (::)}

  \newpage
  \vspace*{\fill}\
  \begin{normalsize}
    \begin{center}
      \begin{enumerate}[label=\textcolor{red}{\emph{\arabic*}}]
        \item \textit{Des profondeurs, j’ai crié vers vous, Seigneur, Seigneur écoutez ma voix. }
        \item \textit{Que vos oreilles soient attentives à la voix de ma prière. }
        \item \textit{Si vous observez nos iniquités, Seigneur, Seigneur, qui subsistera ? }
        \item \textit{Car près de vous est le pardon, et à cause de votre loi j’ai espéré en vous Seigneur. }
        \item \textit{Mon âme a espéré en ses paroles, mon âme a espéré en le Seigneur. }
        \item \textit{Depuis la garde du matin jusqu’à la nuit, qu’Israël espère en le Seigneur. }
        \item \textit{Car auprès du Seigneur est la miséricorde, et abonde chez lui la rédemption. }
        \item \textit{ Et c’est lui qui rachètera Israël de toutes ses iniquités.}
        \item \textit{Gloire au Père, au Fils, et au Saint Esprit, }
        \item \textit{Comme il était au commencement, maintenant et toujours, et dans les siècles des siècles. Amen. }
      \end{enumerate}
    \end{center}
  \end{normalsize}
  \vspace*{\fill}\
  \newpage

  % ===== DEBUT Antienne =========
  \gresetinitiallines{1}
  \greillumination{\initfamily\fontsize{11mm}{11mm}\selectfont D}
  \gregorioscore{antiennes/an--de_fructu_ventris_tui--solesmes_1961}
  \begin{center}
    \footnotesize{
      \textit{
        Je mettrai un fils du fruit de vos entrailles sur votre trône. 
      }
    }
  \end{center}
  % ===== DEBUT psaume ===========
  % gresetinitiallines : avec le parametre à 0, supprime l'ornement
  \begin{center}
    \normalsize{Psaume 131.}
  \end{center}

  % gresetinitiallines : avec le parametre à 0, supprime l'ornement
  \gresetinitiallines{0}
  \gregorioscore{psaumes/psaume131-VIIIG}

  \begin{enumerate}[label=\textcolor{red}{\arabic*}]
    \setcounter{enumi}{1}
    \item Sicut jurávit \textbf{Dó}mino,~* votum vovit \textit{De}\textit{o} \textbf{Ja}cob:

    \item Si introíero in tabernáculum domus \textbf{me}æ,~* si ascéndero in lectum \textit{stra}\textit{ti} \textbf{me}i:
    
    \item Si dédero somnum óculis \textbf{me}is,~* et pálpebris meis dormi\textit{ta}\textit{ti}\textbf{ó}nem:
    
    \item Et réquiem tempóribus meis: donec invéniam locum \textbf{Dó}mino,~* tabernáculum \textit{De}\textit{o} \textbf{Ja}cob.
    
    \item Ecce audívimus eam in \textbf{E}phrata:~* invénimus eam in \textit{cam}\textit{pis} \textbf{sil}væ.
    
    \item Introíbimus in tabernáculum \textbf{e}jus:~* adorábimus in loco, ubi stetérunt \textit{pe}\textit{des} \textbf{e}jus.
    
    \item Surge, Dómine, in réquiem \textbf{tu}am,~* tu et arca sanctificati\textit{ó}\textit{nis} \textbf{tu}æ.
    
    \item Sacerdótes tui induántur jus\textbf{tí}tiam:~* et sancti tu\textit{i} \textit{ex}\textbf{súl}tent.
    
    \item Propter David, servum \textbf{tu}um:~* non avértas fáciem \textit{Chris}\textit{ti} \textbf{tu}i.
    
    \item Jurávit Dóminus David veritátem, et non frustrábitur \textbf{e}am:~* de fructu ventris tui ponam super \textit{se}\textit{dem} \textbf{tu}am.
    
    \item Si custodíerint fílii tui testaméntum \textbf{me}um:~* et testimónia mea hæc, quæ do\textit{cé}\textit{bo} \textbf{e}os.
    
    \item Et fílii eórum usque in \textbf{sǽ}culum:~* sedébunt super \textit{se}\textit{dem} \textbf{tu}am.
    
    \item Quóniam elégit Dóminus \textbf{Si}on:~* elégit eam in habitati\textit{ó}\textit{nem} \textbf{si}bi.
    
    \item Hæc réquies mea in sǽculum \textbf{sǽ}culi:~* hic habitábo, quóniam e\textit{lé}\textit{gi} \textbf{e}am.
    
    \item Víduam ejus benedícens bene\textbf{dí}cam:~* páuperes ejus satu\textit{rá}\textit{bo} \textbf{pá}nibus.
    
    \item Sacerdótes ejus índuam salu\textbf{tá}ri:~* et sancti ejus exsultatióne \textit{ex}\textit{sul}\textbf{tá}bunt.
    
    \item Illuc prodúcam cornu \textbf{Da}vid:~* parávi lucérnam \textit{Chris}\textit{to} \textbf{me}o.
    
    \item Inimícos ejus índuam confusi\textbf{ó}ne:~* super ipsum autem efflorébit sanctificá\textit{ti}\textit{o} \textbf{me}a.
    
    \item Glória Patri, et \textbf{Fí}lio,~* et Spirí\textit{tu}\textit{i} \textbf{Sanc}to.
    
    \item Sicut erat in princípio, et nunc, et \textbf{sem}per,~* et in sǽcula sæcu\textit{ló}\textit{rum}. \textbf{A}men.
  \end{enumerate}
  %  Répetition de l'Antienne
  \grecommentary{\textit{Reprise de l'Antienne.}}
  \gabcsnippet{(c4) De(g') fru(j)ctu(ig) <c>*</c>() ven(i_[uh:l]j)tris(h_g) tu(h_g)i(f_h) (;) po(j)nam(ig~) su(i)per(j) se(h)dem(h) tu(g.)am.(g.) (::) }

  \newpage
  \vspace*{\fill}\
  \begin{normalsize}
    \begin{center}
      \begin{enumerate}[label=\textcolor{red}{\emph{\arabic*}}]
        \item \textit{Souvenez-vous, Seigneur ! de David, et de toute sa douceur. }
        \item \textit{qu’il a juré au Seigneur, et a fait ce vœu au Dieu de Jacob : }
        \item \textit{Si j’entre dans le secret de ma maison ; si je monte sur le lit qui est préparé pour me coucher ; }
        \item \textit{Si je permets à mes yeux de dormir, et à mes paupières de sommeiller, }
        \item \textit{Et si je donne aucun repos à mes tempes, jusqu’à ce que je trouve un lieu propre pour le Seigneur, et un tabernacle pour le Dieu de Jacob. }
        \item \textit{Nous avons entendu dire, que l’arche était autrefois dans Ephrata ; nous l’avons trouvée dans un pays plein de bois. }
        \item \textit{Nous entrerons dans son tabernacle : nous l’adorerons dans le lieu où il a posé ses pieds. }
        \item \textit{Levez-vous, Seigneur ! pour entrer dans votre repos, vous et l’arche où éclate votre sainteté. }
        \item \textit{Que vos prêtres soient revêtus de justice, et que vos saints tressaillent de joie. }
        \item \textit{En considération de David, votre serviteur, ne rejetez pas le visage de votre Christ. }
        \item \textit{Le Seigneur a fait à David un serment trèsvéritable ; et il ne le trompera point : J’établirai sur votre trône le fruit de votre ventre. }
        \item \textit{Et que leurs enfants les gardent aussi pour toujours ; ils seront assis sur votre trône. }
        \item \textit{Car le Seigneur a choisi Sion ; il l’a choisie pour sa demeure. }
        \item \textit{C’est là pour toujours le lieu de mon repos : c’est là que j’habiterai, parce que je l’ai choisie. } 
        \item \textit{Je donnerai à sa veuve une bénédiction abondante ; je rassasierai de pain ses pauvres. }
        \item \textit{Je revêtirai ses prêtres d’une vertu salutaire ; et ses saints seront ravis de joie. }
        \item \textit{C’est là que je ferai paraître la puissance de David : j’ai préparé une lampe à mon Christ. }
        \item \textit{Je couvrirai de confusion ses ennemis ; mais je ferai éclater sur lui ma propre sanctification. }
        \item \textit{Gloire au Père, au Fils, et au Saint Esprit, }
        \item \textit{Comme il était au commencement, maintenant et toujours, et dans les siècles des siècles. Amen. }
      \end{enumerate}
    \end{center}
  \end{normalsize}
  \vspace*{\fill}\
  \newpage


  \par \textit{\footnotesize\textcolor{red}{On se lève pour le Capitule.}}

  \begin{center}
    \textcolor{red}{\large{Capitule}}\\
    \small\textit{
      Épître aux Philippiens. IV, 4-5
    }
  \end{center}

  \begin{multicols}{2}
    \parindent=0pt
    Multifariam, multísque modis olim Deus loquens pátribus in Prophétis : \textcolor{red}{†} novíssime diébus istis locútus est nobis in Fílio, quem constítuit herédem universórum, \textcolor{red}{*} per quem fecit et sæcula.  \\
    \textcolor{red}{\Rbar.} Deo grátias.

    \columnbreak

    \textit{ Après avoir, à plusieurs reprises et en diverses manières, parlé autrefois à nos pères par les Prophètes, Dieu, en ces jours qui sont les derniers, nous a parlé par le Fils, qu’il a établi héritier de toutes choses, et par lequel il a aussi créé le monde.  \\
    \textcolor{red}{\Rbar.} Rendons grâce à Dieu.
    }
  \end{multicols}

  \par \textit{\footnotesize\textcolor{red}{Le Célébrant entonne, ensuite, les Chantres et le Chœur alternent les versets. La Doxologie est chantée par tous.}}

  \begin{center}
    \textcolor{red}{\large{Hymne}}\\
  \end{center}
  
  \gresetinitiallines{1}
  \greillumination{\initfamily\fontsize{11mm}{11mm}\selectfont J}
  \gregorioscore{hymnes/hy--jesu_redemptor_omnium_(t._nativitatis)--solesmes_1961}
  \bigskip
  % \begin{multicols}{2}
    \begin{normalsize}
      \begin{enumerate}[label=\textcolor{red}{\emph{\arabic*}}]
        \item \textit{Christ, Rédempteur de tous les hommes, Fils Unique engendré du Père Avant l’origine du monde, En une ineffable naissance. }
        \item \textit{Vous, lumière, vous, splendeur du Père, Notre espoir éternel à tous, Ecoutez, par tout l’univers Vos serviteurs qui vous supplient. }
        \item \textit{Souvenez-vous, ô Dieu Sauveur, Que vous avez jadis reçu, Naissant de la Vierge sans tache, L’humble livrée de notre corps. }
        \item \textit{Ce jour présent en est témoin, Que le cours de l’année ramène : Descendant du trône du Père Vous seul avez sauvé le monde. }
        \item \textit{Le ciel et la terre et la mer Et tous les êtres qui les peuplent Célèbrent dans un chant joyeux Celui qui vous a envoyé. }
        \item \textit{Et nous qui sommes rachetés Au prix de votre Sang très saint, En ce jour de votre naissance Nous chantons un hymne nouveau.}
        \item \textit{Gloire soit à jamais rendue A vous Seigneur, né de la Vierge, Avec le Père et l’Esprit-Saint Dans les siècles sans fin. Amen. }
      \end{enumerate}
    \end{normalsize}
  % \end{multicols}

  \begin{center}
    \begin{footnotesize}
      \textcolor{red}{\textit{On chante le verset debout.}}
    \end{footnotesize}
    \begin{minipage}{0.8\linewidth}
      \gresetinitiallines{0}
      \gabcsnippet{(c3)<c><v>\Vbar</v>.</c> Nó(h)tum(h) fé(h)cit(h) Dó(hi)mi(h)nus,(h) (,) al(h)le(fe)lú(f_h){ia}.(hiH'Ghih.ghG'FE'fggf.0) (::) (Z-) 
      <c><v>\Rbar</v>.</c> Sa(h)lu(h)tá(h)re(h) sú(hi)um,(h) (,) al(h)le(fe)lú(f_h){ia}.(hiH'Ghih.ghG'FE'fggf.0) (::)}
      \bigskip
      \begin{center}
        \textit{\textcolor{red}{\Vbar.} Le Seigneur à fait connaître, alléluia. }\\
        \textit{\textcolor{red}{\Rbar.} Son salut, alléluia.}
      \end{center}
    \end{minipage}
  \end{center}
  \normalsize

  \vspace*{\fill}\
  \begin{center}
    \greseparator{3}{30}
  \end{center}
  \vspace*{\fill}\

  \newpage

  \begin{center}
    \textcolor{red}{\large{Antienne à Magnificat}}\\
  \end{center}

  \gresetinitiallines{1}
  \greillumination{\initfamily\fontsize{11mm}{11mm}\selectfont H}
  \gregorioscore{antiennes/an--hodie_christus_natus_est--solesmes_1961}
  \medskip
  \begin{center}
    \footnotesize{\textit{
      Aujourd’hui est né le Christ, aujourd’hui le Sauveur est apparu ; aujourd’hui sur la terre chantent les Anges, se réjouissent les Archanges ; aujourd’hui les justes dans les transports de leur joie, répètent : Gloire à Dieu au plus haut des cieux, alléluia.  
    }}
  \end{center}
  \medskip

  \gresetinitiallines{0}
  \gregorioscore{magnificat/magnificat-Ig2}

  \begin{enumerate}[label=\textcolor{red}{\arabic*}]
    \setcounter{enumi}{2}
    \item Quia respéxit humilitátem \textit{an}\textit{cíl}\textit{læ} \textbf{su}æ:~* ecce enim ex hoc beátam me dicent omnes gene\textit{ra}\textit{ti}\textbf{ó}nes.

    \item Quia fecit mihi \textit{ma}\textit{gna} \textit{qui} \textbf{pot}\textbf{ens} est:~* et sanctum \textit{no}\textit{men} \textbf{e}jus.
    
    \item Et misericórdia ejus a progéni\textit{e} \textit{in} \textit{pro}\textbf{gé}\textbf{ni}es~* timén\textit{ti}\textit{bus} \textbf{e}um.
    
    \item Fecit poténtiam in \textit{brá}\textit{chi}\textit{o} \textbf{su}o:~* dispérsit supérbos mente \textit{cor}\textit{dis} \textbf{su}i.
    
    \item Depósuit pot\textit{én}\textit{tes} \textit{de} \textbf{se}de,~* et exal\textit{tá}\textit{vit} \textbf{hú}miles.
    
    \item Esuriéntes \textit{im}\textit{plé}\textit{vit} \textbf{bo}nis:~* et dívites dimí\textit{sit} \textit{in}\textbf{á}nes.
    
    \item Suscépit Israël \textit{pú}\textit{e}\textit{rum} \textbf{su}um,~* recordátus misericór\textit{di}\textit{æ} \textbf{su}æ.
    
    \item Sicut locútus est \textit{ad} \textit{pa}\textit{tres} \textbf{nos}tros,~* Abraham et sémini e\textit{jus} \textit{in} \textbf{sǽ}cula.
    
    \begin{center}
      \textit{\footnotesize \textcolor{red}{On attend la fin de l'encensement avant de chanter la doxologie.}}
    \end{center}
    \item Glória \textit{Pa}\textit{tri}, \textit{et} \textbf{Fí}\textbf{li}o,~* et Spirí\textit{tu}\textit{i} \textbf{Sanc}to.
    
    \item Sicut erat in princípio, \textit{et} \textit{nunc}, \textit{et} \textbf{sem}per,~* et in sǽcula sæcu\textit{ló}\textit{rum}. \textbf{A}men.
  \end{enumerate}

  \grecommentary{\textit{Reprise de l'Antienne.}}
  \gabcsnippet{(c4) Ho(f)di(gh)e(h.) (,) Chri(ixh.g!hwi)stus(h') na(g)tus(fh) est :(h.) (;) hó(f)di(fg)e(g'_[oh:h]) (,) Sal(g)vá(g)tor(g) ap(gh)pá(h)ru(gf)it :(f.) (:) hó(f)di(gh)e(h'_) (,) in(h) ter(h)ra(hg) ca(ixi)nunt(h') An(g)ge(fh)li,(h'_) (;) lae(h)<nlba>tán(hg)tur</nlba>(ixi) Ar(h')chán(g)ge(fh)li :(h.) (:) hó(h)di(j)e(j_kJ'IH') (,) ex(h)súl(ixh.g!hwi)tant(h') ju(g)sti,(f_h) (,) dí(f)cen(h)tes :(g_[uh:l]h) (;) Gló(g_[uh:l]h)ri(fe)a(d) in(e') ex(f)cél(g_[uh:l]h)sis(g) De(fe)o,(dc) al(d)le(fe)lú(d.){ia}.(d.) (::)}

  \newpage

  \begin{center}
    \textcolor{red}{\large{Oraison}}\\
  \end{center}

  \begin{multicols}{2}
    \parindent=0pt
    \begin{flushright}
      \textcolor{red}{\Vbar.} Dominus vobiscum.\\
      \textcolor{red}{\Rbar.} Et cum spiritu tuo.\\
    \end{flushright}

    \columnbreak
    
    \textit{\textcolor{red}{\Vbar.} Le Seigneur soit avec vous.\\
    \textcolor{red}{\Rbar.} Et avec votre esprit.}\\
  \end{multicols}

  \begin{multicols}{2}
    \parindent=0pt
    Concéde, quæsumus, omnípotens Deus : \textcolor{red}{†} ut nos Unigéniti tui nova per carnem Natívitas líberet ; \textcolor{red}{*} quos sub peccáti jugo vetústa sérvitus tenet.  \\Per eúmdem Dóminum nostrum Jesum Christum Fílium tuum, qui tecum vivit et regnat in unitáte Spíritus sancti Deus : per ómnia sæcula sæculórum.
    \textcolor{red}{\Rbar.} Amen.

    \columnbreak

    \textit{ Accordez, nous vous le demandons, Dieu toutpuissant, que la nouvelle naissance de votre Fils unique en la chair nous rende libres, nous qu’une antique servitude tient sous le joug du péché. Par notre même Seigneur Jésus-Christ, votre Fils, qui avec vous vit et règne en l’unité du Saint Esprit, Dieu pour tous les siècles des siècles. 
    Amen.
    }
  \end{multicols}

  \medskip

  \begin{center}
    \textcolor{red}{\large{Conclusion de l'office}}
  \end{center}
  
  
  \begin{multicols}{2}
    \parindent=0pt
    \begin{flushright}
      \textcolor{red}{\Vbar.} Dominus vobiscum.\\
      \textcolor{red}{\Rbar.} Et cum spiritu tuo.\\
    \end{flushright}
  
    \columnbreak
    
    \textit{\textcolor{red}{\Vbar.} Le Seigneur soit avec vous.\\
    \textcolor{red}{\Rbar.} Et avec votre esprit.}\\
  \end{multicols}
  \bigskip
  \gresetinitiallines{1}
  \greillumination{\initfamily\fontsize{11mm}{11mm}\selectfont B}
  \gregorioscore{or--benedicamus_domino_(i_classis_in_ii_vesperis_mode_5)--solesmes_1961}
  \begin{center}
    \begin{footnotesize}
      \textcolor{red}{\textit{Sur un ton très grave : }}
    \end{footnotesize}
  \end{center}
  \begin{multicols}{2}
    \parindent=0pt
    \textcolor{red}{\Vbar.} Fidélium ánimæ per misericórdiam Dei requiéscant in pace.\\
    \textcolor{red}{\Rbar.} Amen.\\

    \columnbreak
    
    \textit{\textcolor{red}{\Vbar.} Que les âmes des fidèles défunts, par la
    miséricorde de Dieu, reposent en paix.\\
    \textcolor{red}{\Rbar.} Amen.}\\
  \end{multicols}

  \medskip

  \begin{center}
    \textcolor{red}{\large{Salut du Saint Sacrement}}\\
    \textit{Voir page 50.}
  \end{center}

  \vspace*{\fill}\
  \begin{center}
    \greseparator{3}{30}
  \end{center}
  \vspace*{\fill}\

  \newpage

  \begin{center}
    \begin{LARGE}
      Dimanche dans l'octave de Noël \\
    \end{LARGE}
    \textit{
      \textcolor{red}{Antiennes, psaumes et hymne de Noël, page 3}
    }
  \end{center}
  \medskip

  

  \par \textit{\footnotesize\textcolor{red}{On se lève pour le Capitule.}}

  \begin{center}
    \textcolor{red}{\large{Capitule}}\\
    \small\textit{
      Épître aux Galates. IV, 1-2
    }
  \end{center}

  \begin{multicols}{2}
    \parindent=0pt
    Fratres : Quando témpore hæres párvulus est, nihil differta servo, cum sit dóminus ómnium : \textcolor{red}{†} sed sub tutóribus et actóribus est, \textcolor{red}{*} usque ad præfinítum tempus a patre.\\
    \textcolor{red}{\Rbar.} Deo grátias.

    \columnbreak

    \textit{ Frères, aussi longtemps que l’héritier est mineur, il ne diffère en rien d’un esclave, bien qu’il soit le maître de tout ; mais il est soumis à des tuteurs et à des administrateurs, jusqu’à la date fixée par son père.\\
    \textcolor{red}{\Rbar.} Rendons grâce à Dieu.
    }
  \end{multicols}

  \begin{center}
    \begin{footnotesize}
      \textcolor{red}{\textit{On chante le verset debout.}}
    \end{footnotesize}
    \begin{minipage}{0.8\linewidth}
      \gresetinitiallines{0}
      \gabcsnippet{(c3)<c><v>\Vbar</v>.</c> Vér(h)bum(h) cá(h)ro(h) fác(hi)tum(h) est,(h) (,) al(h)le(fe)lú(f_h){ia}.(hiH'Ghih.ghG'FE'fggf.0) (::) (Z-) 
      <c><v>\Rbar</v>.</c> Et(h) ha(h)bi(h)tá(h)vit(h) in(h) nó(hi)bis,(h) (,) al(h)le(fe)lú(f_h){ia}.(hiH'Ghih.ghG'FE'fggf.0) (::)}
      \bigskip
      \begin{center}
        \textit{\textcolor{red}{\Vbar.} Le Verbe s’est fait cher, alléluia.}\\
        \textit{\textcolor{red}{\Rbar.} Et il a habité parmi nous, alléluia.}
      \end{center}
    \end{minipage}
  \end{center}
  \normalsize

  \vspace*{\fill}\
  \begin{center}
    \greseparator{3}{30}
  \end{center}
  \vspace*{\fill}\

  \newpage

  \begin{center}
    \textcolor{red}{\large{Antienne à Magnificat}}\\
  \end{center}

  \gresetinitiallines{1}
  \greillumination{\initfamily\fontsize{11mm}{11mm}\selectfont P}
  \gregorioscore{antiennes/an--puer_jesus_proficiebat--solesmes_1934}
  \medskip
  \begin{center}
    \footnotesize{\textit{
      L’Enfant Jésus croissait en âge et en sagesse devant Dieu et devant les hommes.
    }}
  \end{center}
  \medskip

  \gresetinitiallines{0}
  \gregorioscore{magnificat/magnificat-VIF}

  \begin{enumerate}[label=\textcolor{red}{\arabic*}]
    \setcounter{enumi}{2}
    \item Quia respéxit humilitátem an\textbf{cíl}læ \textbf{su}æ:~* ecce enim ex hoc beátam me dicent omnes gene\textit{ra}\textit{ti}\textbf{ó}nes.

    \item Quia fecit mihi \textbf{ma}gna qui \textbf{pot}ens est:~* et sanctum \textit{no}\textit{men} \textbf{e}jus.
    
    \item Et misericórdia ejus a progénie \textbf{in} pro\textbf{gé}nies~* timén\textit{ti}\textit{bus} \textbf{e}um.
    
    \item Fecit poténtiam in \textbf{brá}chio \textbf{su}o:~* dispérsit supérbos mente \textit{cor}\textit{dis} \textbf{su}i.
    
    \item Depósuit pot\textbf{én}tes de \textbf{se}de,~* et exal\textit{tá}\textit{vit} \textbf{hú}miles.
    
    \item Esuriéntes im\textbf{plé}vit \textbf{bo}nis:~* et dívites dimí\textit{sit} \textit{in}\textbf{á}nes.
    
    \item Suscépit Israël \textbf{pú}erum \textbf{su}um,~* recordátus misericór\textit{di}\textit{æ} \textbf{su}æ.
    
    \item Sicut locútus est ad \textbf{pa}tres \textbf{nos}tros,~* Abraham et sémini e\textit{jus} \textit{in} \textbf{sǽ}cula.

    \begin{center}
      \textit{\footnotesize \textcolor{red}{On attend la fin de l'encensement avant de chanter la doxologie.}}
    \end{center}
    
    \item Glória \textbf{Pa}tri, et \textbf{Fí}lio,~* et Spirí\textit{tu}\textit{i} \textbf{Sanc}to.
    
    \item Sicut erat in princípio, et \textbf{nunc}, et \textbf{sem}per,~* et in sǽcula sæcu\textit{ló}\textit{rum}. \textbf{A}men.
  \end{enumerate}

  \grecommentary{\textit{Reprise de l'Antienne.}}
  \gabcsnippet{(c4) Pu(f)er(c) Je(df)sus(ffg.) (,) pro(f)fi(f')ci(f)é(gh)bat(h') æ(h)tá(hj)te(h') (,) et(g) sa(hg)pi(f)én(gh)ti(gf)a(fv_2/fv_/ggf.) (;) co(f)ram(f) De(f_g)o(e_f) et(d) ho(cf~)mí(f')ni(f)bus.(f.) (::)}

  \newpage

  \begin{center}
    \textcolor{red}{\large{Oraison}}\\
  \end{center}

  \begin{multicols}{2}
    \parindent=0pt
    \begin{flushright}
      \textcolor{red}{\Vbar.} Dominus vobiscum.\\
      \textcolor{red}{\Rbar.} Et cum spiritu tuo.\\
    \end{flushright}

    \columnbreak
    
    \textit{\textcolor{red}{\Vbar.} Le Seigneur soit avec vous.\\
    \textcolor{red}{\Rbar.} Et avec votre esprit.}\\
  \end{multicols}

  \begin{multicols}{2}
    \parindent=0pt
    Omnipotens sempitérne Deus, dírige actus nostros in beneplácito tuo :\textcolor{red}{†} ut in nómine dilécti Fílii \textcolor{red}{*} tui mereámur bonis opéribus abundáre :\\ Qui tecum vivis et regnas in unitáte Spíritus Sancti Deus : per ómnia sæcula sæculórum. 
    \textcolor{red}{\Rbar.} Amen.

    \columnbreak

    \textit{Dieu tout-puissant et éternel, dans votre bienveillance dirigez nos actions, afin qu’au nom de votre Fils bien-aimé, nous méritions d’abonder en bonnes œuvres. Lui qui avec vous vis et règne en l’unité du Saint Esprit, Dieu pour tous les siècles des siècles. 
    Amen.
    }
  \end{multicols}

  \medskip

  \begin{center}
    \textcolor{red}{\large{Conclusion de l'office}}
  \end{center}
  
  
  \begin{multicols}{2}
    \parindent=0pt
    \begin{flushright}
      \textcolor{red}{\Vbar.} Dominus vobiscum.\\
      \textcolor{red}{\Rbar.} Et cum spiritu tuo.\\
    \end{flushright}
  
    \columnbreak
    
    \textit{\textcolor{red}{\Vbar.} Le Seigneur soit avec vous.\\
    \textcolor{red}{\Rbar.} Et avec votre esprit.}\\
  \end{multicols}
  \bigskip
  \gresetinitiallines{1}
  \greillumination{\initfamily\fontsize{11mm}{11mm}\selectfont B}
  \gregorioscore{or--benedicamus_domino_(i_classis_in_ii_vesperis_mode_5)--solesmes_1961}
  \begin{center}
    \begin{footnotesize}
      \textcolor{red}{\textit{Sur un ton très grave : }}
    \end{footnotesize}
  \end{center}
  \begin{multicols}{2}
    \parindent=0pt
    \textcolor{red}{\Vbar.} Fidélium ánimæ per misericórdiam Dei requiéscant in pace.\\
    \textcolor{red}{\Rbar.} Amen.\\

    \columnbreak
    
    \textit{\textcolor{red}{\Vbar.} Que les âmes des fidèles défunts, par la
    miséricorde de Dieu, reposent en paix.\\
    \textcolor{red}{\Rbar.} Amen.}\\
  \end{multicols}

  \medskip

  \begin{center}
    \textcolor{red}{\large{Salut du Saint Sacrement}}\\
    \textit{Voir page 50.}
  \end{center}

  \vspace*{\fill}\
  \begin{center}
    \greseparator{3}{30}
  \end{center}
  \vspace*{\fill}\

  \newpage

  \begin{center}
    \begin{LARGE}
      Fête de la Circoncision
    \end{LARGE}
  \end{center}
  \medskip

  % ===== DEBUT Antienne =========
  \greillumination{\initfamily\fontsize{11mm}{11mm}\selectfont O}
  \gregorioscore{antiennes/an--o_admirabile_commercium--solesmes_1961}
  \begin{center}
    \footnotesize{
      \textit{
        O commerce admirable ! Le Créateur du genre humain prenant un corps et une âme, a daigné naître de la Vierge, et, devenu homme sans le concours de l’homme, il nous a fait part de sa divinité. 
      }
    }
  \end{center}
  % \medskip
  % ===== FIN Antienne ===========

  % ===== DEBUT psaume ===========
  % gresetinitiallines : avec le parametre à 0, supprime l'ornement
  \gresetinitiallines{0}

  \begin{center}
    \normalsize{Psaume 109.}
  \end{center}

  \gregorioscore{psaumes/psaume109-VIF}
  \begin{enumerate}[label=\textcolor{red}{\emph{\arabic*}}]
    \setcounter{enumi}{2}
    \item Donec ponam ini\textbf{mí}cos \textbf{tu}os,~* scabéllum pe\textit{dum} \textit{tu}\textbf{ó}rum.

    \item Virgam virtútis tuæ emíttet Dómi\textbf{nus} ex \textbf{Si}on:~* domináre in médio inimicó\textit{rum} \textit{tu}\textbf{ó}rum.
    
    \item Tecum princípium in die virtútis tuæ in splendóri\textbf{bus} sanc\textbf{tó}rum:~* ex útero ante lucíferum \textit{gé}\textit{nu}\textbf{i} te.
    
    \item Jurávit Dóminus, et non pœni\textbf{té}bit \textbf{e}um:~* Tu es sacérdos in ætérnum secúndum órdi\textit{nem} \textit{Mel}\textbf{chí}sedech.
    
    \item Dóminus a \textbf{dex}tris \textbf{tu}is,~* confrégit in die iræ \textit{su}\textit{æ} \textbf{re}ges.
    
    \item Judicábit in natiónibus, im\textbf{plé}bit ru\textbf{í}nas:~* conquassábit cápita in ter\textit{ra} \textit{mul}\textbf{tó}rum.
    
    \item De torrénte in \textbf{vi}a \textbf{bi}bet:~* proptérea exal\textit{tá}\textit{bit} \textbf{ca}put.
    
    \item Glória \textbf{Pa}tri, et \textbf{Fí}lio,~* et Spirí\textit{tu}\textit{i} \textbf{Sanc}to.
    
    \item Sicut erat in princípio, et \textbf{nunc}, et \textbf{sem}per,~* et in sǽcula sæcu\textit{ló}\textit{rum}. \textbf{A}men.
  \end{enumerate}
  \medskip
  \grecommentary{\textit{Reprise de l'Antienne.}}
  \gabcsnippet{(c4) O(ffddc) ad(f')mi(f)rá(fg)bi(g)le(g') com(f)mér(gh)ci(gf)um!(f.) (;) Cre(f)á(g')tor(f) gé(g')ne(f)ris(g) hu(gh)má(f)ni,(f.) (;) a(f)ni(f)má(ixfh'_!iv_H'G)tum(g') cor(ixi)pus(h') su(g)mens,(f.) (;) de(f) Vír(g')gi(f)ne(f') na(g)sci(f') di(f)gná(ixfh'_!iv_H'G)tus(hg) est :(g.) (:) et(g) pro(g)cé(h')dens(g) ho(fg)mo(f) si(fe)ne(d!ewf) sé(f_g)mi(f)ne,(c.) (;) lar(c_[uh:l-0.8mm]d)gí(f')tus(f) est(f) no(g_[uh:l]h)bis(g') su(h)am(hj) de(hg)i(h)tá(f.)tem.(f.) (::)}

  \newpage

  % ===== DEBUT Antienne =========
  \gresetinitiallines{1}
  \greillumination{\initfamily\fontsize{11mm}{11mm}\selectfont Q}
  \gregorioscore{antiennes/an--quando_natus_es--solesmes_1961}
  \begin{center}
    \footnotesize{
      \textit{
        Quand vous naquîtes ineffablement d’une Vierge, alors s’accomplirent les Écritures. Comme la rosée sur la toison, vous descendîtes pour sauver le genre humain. Nous vous louons, ô notre Dieu !  
      }
    }
  \end{center}
  % \medskip
  % ===== FIN Antienne ===========

  % ===== DEBUT psaume ===========
  % gresetinitiallines : avec le parametre à 0, supprime l'ornement
  \gresetinitiallines{0}

  \begin{center}
    \normalsize{Psaume 112.}
  \end{center}

  \gregorioscore{psaumes/psaume112-IIIa2}

  \begin{center}
    \textit{\footnotesize \textcolor{red}{On s'incline ici, par respect pour le nom de Dieu.}}
  \end{center}

  \begin{enumerate}[label=\textcolor{red}{\emph{\arabic*}}]
    \setcounter{enumi}{1}
    \item Sit nomen Dómini \textbf{be}ne\textbf{díc}tum,~* ex hoc nunc, et us\textit{que} \textit{in} \textbf{sǽ}culum.

    \item A solis ortu usque \textbf{ad} oc\textbf{cá}sum,~* laudábile \textit{no}\textit{men} \textbf{Dó}mini.
    
    \item Excélsus super omnes \textbf{gen}tes \textbf{Dó}\textbf{mi}nus,~* et super cælos gló\textit{ri}\textit{a} \textbf{e}jus.
    
    \item Quis sicut Dóminus, Deus noster, qui in \textbf{al}tis \textbf{há}\textbf{bi}tat,~* et humília réspicit in cælo \textit{et} \textit{in} \textbf{ter}ra?
    
    \item Súscitans a \textbf{ter}ra \textbf{ín}\textbf{o}pem,~* et de stércore é\textit{ri}\textit{gens} \textbf{páu}perem:
    
    \item Ut cóllocet eum \textbf{cum} prin\textbf{cí}\textbf{pi}bus,~* cum princípibus pó\textit{pu}\textit{li} \textbf{su}i.
    
    \item Qui habitáre facit stéri\textbf{lem} in \textbf{do}mo,~* matrem filió\textit{rum} \textit{læ}\textbf{tán}tem.
    
    \item Glória \textbf{Pa}tri, et \textbf{Fí}\textbf{li}o,~* et Spirí\textit{tu}\textit{i} \textbf{Sanc}to.
    
    \item Sicut erat in princípio, et \textbf{nunc}, et \textbf{sem}per,~* et in sǽcula sæcu\textit{ló}\textit{rum}. \textbf{A}men.
  \end{enumerate}

  \medskip
  \grecommentary{\textit{Reprise de l'Antienne.}}
  \gabcsnippet{(c4) Quan(e)do(ed) na(g)tus(hj) es(j'_) (,) in(j)ef(i')fa(g)bí(i')li(j)ter(h) ex(gf) Vír(g_[uh:l]h)gi(h)ne,(g.) (;) tunc(g) im(g)plé(i)tae(j) sunt(hgh___) Scri(gh)ptú(e.)rae :(e.) (:) sic(g)ut(h) plú(j')vi(i)a(jkj) in(i') vel(j)lus(h') de(h)scen(g')dí(i)sti,(h.) (;) ut(h) sal(j')vum(i) fá(j')ce(i)res(g) ge(hv_GF)nus(g) hu(gf)má(e.)num :(e.) (:) te(h_i) lau(g)dá(gf)mus(d_e) De(g)us(ghg) no(e.)ster.(e.) (::) }

  \newpage
  \vspace*{\fill}\
  \begin{normalsize}
    \begin{center}
      \begin{enumerate}[label=\textcolor{red}{\emph{\arabic*}}]
        \item \textit{Louez le Seigneur, vous qui êtes ses serviteurs ; louez le nom du Seigneur. }
        \item \textit{Que le nom du Seigneur soit béni dès maintenant, et dans tous les siècles. }
        \item \textit{Le nom du Seigneur doit être loué depuis le lever du soleil jusqu’au couchant. }
        \item \textit{Le Seigneur est élevé au-dessus de toutes les nations, et sa gloire au-dessus des cieux. }
        \item \textit{Qui est semblable au Seigneur notre Dieu, qui habite les lieux les plus élevés, et qui regarde ce qu’il y a de plus abaissé dans le ciel et sur la terre ? }
        \item \textit{Qui tire de la poussière celui qui est dans l’indigence, et qui élève le pauvre de dessus le fumier, }
        \item \textit{Pour le placer avec les princes, avec les princes de son peuple ; }
        \item \textit{Qui donne à celle qui était stérile la joie de se voir dans sa maison la mère de plusieurs enfants. }
        \item \textit{Gloire au Père, au Fils, et au Saint Esprit, }
        \item \textit{Comme il était au commencement, maintenant et toujours, et dans les siècles des siècles. Amen. }
      \end{enumerate}
    \end{center}
  \end{normalsize}
  \vspace*{\fill}\
  \newpage

  % ===== DEBUT Antienne =========
  \gresetinitiallines{1}
  \greillumination{\initfamily\fontsize{11mm}{11mm}\selectfont R}
  \gregorioscore{antiennes/an--rubum_quem--solesmes_1961.1}
  \begin{center}
    \footnotesize{
      \textit{
        En ce buisson que vit Moïse et qui brûlait sans se consumer, nous voyons l’image de votre glorieuse virginité : Mère de Dieu, intercédez pour nous. 
      }
    }
  \end{center}
  % \medskip
  % ===== FIN Antienne ===========

  % ===== DEBUT psaume ===========
  % gresetinitiallines : avec le parametre à 0, supprime l'ornement

  \gresetinitiallines{0}

  \begin{center}
    \normalsize{Psaume 121.}\\
  \end{center}
  % \smallskip
  \grechangedim{baselineskip}{50pt}{scalable}
  \gregorioscore{psaumes/psaume121-IVE}
  \begin{enumerate}[label=\textcolor{red}{\emph{\arabic*}}]
    \setcounter{enumi}{1}
    \item Stantes erant \textit{pe}\textit{des} \textbf{nos}tri,~* in átriis \textit{tu}\textit{is}, \textit{Je}\textbf{rú}\textbf{sa}lem.

    \item Jerúsalem, quæ ædificá\textit{tur} \textit{ut} \textbf{cí}vitas:~* cujus participátio e\textit{jus} \textit{in} \textit{id}\textbf{íp}sum.
    
    \item Illuc enim ascendérunt tribus, \textit{tri}\textit{bus} \textbf{Dó}mini:~* testimónium Israël ad confiténdum \textit{nó}\textit{mi}\textit{ni} \textbf{Dó}\textbf{mi}ni.
    
    \item Quia illic sedérunt sedes \textit{in} \textit{ju}\textbf{dí}cio,~* sedes su\textit{per} \textit{do}\textit{mum} \textbf{Da}vid.
    
    \item Rogáte quæ ad pacem \textit{sunt} \textit{Je}\textbf{rú}salem:~* et abundántia di\textit{li}\textit{gén}\textit{ti}\textbf{bus} te:
    
    \item Fiat pax in vir\textit{tú}\textit{te} \textbf{tu}a:~* et abundántia in \textit{túr}\textit{ri}\textit{bus} \textbf{tu}is.
    
    \item Propter fratres meos, et pró\textit{xi}\textit{mos} \textbf{me}os,~* loqué\textit{bar} \textit{pa}\textit{cem} \textbf{de} te:
    
    \item Propter domum Dómini, \textit{De}\textit{i} \textbf{nos}tri,~* quæsí\textit{vi} \textit{bo}\textit{na} \textbf{ti}bi.
    
    \item Glória Pa\textit{tri}, \textit{et} \textbf{Fí}lio,~* et Spi\textit{rí}\textit{tu}\textit{i} \textbf{Sanc}to.
    
    \item Sicut erat in princípio, et \textit{nunc}, \textit{et} \textbf{sem}per,~* et in sǽcula sæ\textit{cu}\textit{ló}\textit{rum}. \textbf{A}men.
  \end{enumerate}

  \medskip
  \grecommentary{\textit{Reprise de l'Antienne.}}
  \gabcsnippet{(c4) Ru(d)bum(e') quem(f) ví(g')de(g)rat(g)() Mó(gvFDef)y(e)ses(e') in(e)com(f)bú(g)stum,(g.) (;) con(g)ser(g)vá(g')tam(f) a(gh~)gnó(h)vi(gh)mus(e.) (;) tu(dh)am(h) lau(h)dá(ixhi)bi(h)lem(g') vir(f)gi(g)ni(gf)tá(e)tem :(e.) (:) De(h)i(g_[oh:h]f) Gé(g)ni(fe)trix,(d'_) (,) in(f)ter(g)cé(h')de(g) pro(f') no(g)bis.(e.) (::) }

  \newpage
  \vspace*{\fill}\
  \begin{normalsize}
    \begin{center}
      \begin{enumerate}[label=\textcolor{red}{\emph{\arabic*}}]
        \item \textit{Je me suis réjoui lorsqu’on m’a dit : Nous irons en la maison du Seigneur. }
        \item \textit{Nous établirons notre demeure dans l’enceinte de tes murailles, ô Jérusalem. }
        \item \textit{Jérusalem est bâtie pour être la demeure d’un peuple, qui vive ensemble dans l’union et dans la paix. }
        \item \textit{C’est là que les tribus, que toutes les tribus du Seigneur s’assemblent pour y chanter ses louanges, selon l’ordre qu’Israël en a reçu. }
        \item \textit{C’est là qu’est établi le siège de la justice, et le trône de la maison de David. }
        \item \textit{Demandez la paix pour Jérusalem, et que ceux qui l’aiment soient dans l’abondance. }
        \item \textit{Que la paix soit dans tes forteresses : et que l’abondance règne dans tes tours. }
        \item \textit{Je dis paix sur toi, à cause de mes frères et de mes proches. }
        \item \textit{En raison de la maison du Seigneur notre Dieu qui est dans ton enceinte, je recherche ton bien.}
        \item \textit{Gloire au Père, au Fils, et au Saint Esprit, }
        \item \textit{Comme il était au commencement, maintenant et toujours, et dans les siècles des siècles. Amen. }
      \end{enumerate}
    \end{center}
  \end{normalsize}
  \vspace*{\fill}\
  \newpage

  % ===== DEBUT Antienne =========
  \gresetinitiallines{1}
  \greillumination{\initfamily\fontsize{11mm}{11mm}\selectfont G}
  \gregorioscore{antiennes/an--germinavit_radix--solesmes_1961}
  \begin{center}
    \footnotesize{\textit{
      La tige de Jessé a fleuri ; l’étoile est sortie de Jacob ; la Vierge a enfanté le Sauveur. Nous vous louons, ô notre Dieu !  
    }}
  \end{center}

  % ===== FIN Antienne ===========

  % ===== DEBUT psaume ===========
  % gresetinitiallines : avec le parametre à 0, supprime l'ornement
  \gresetinitiallines{0}

  \begin{center}
    \normalsize{Psaume 126.}\\
  \end{center}
  % \smallskip

  \gregorioscore{psaumes/psaume126-If}

  \begin{enumerate}[label=\textcolor{red}{\emph{\arabic*}}]
    \setcounter{enumi}{1}
    \item Nisi Dóminus custodíerit \textbf{ci}vi\textbf{tá}tem,~* frustra vígilat qui cus\textit{tó}\textit{dit} \textbf{e}am.

    \item Vanum est vobis ante \textbf{lu}cem \textbf{súr}gere:~* súrgite postquam sedéritis, qui manducátis pa\textit{nem} \textit{do}\textbf{ló}ris.
    
    \item Cum déderit diléctis \textbf{su}is \textbf{som}num:~* ecce heréditas Dómini fílii: merces, \textit{fruc}\textit{tus} \textbf{ven}tris.
    
    \item Sicut sagíttæ in \textbf{ma}nu pot\textbf{én}tis:~* ita fílii \textit{ex}\textit{cus}\textbf{só}rum.
    
    \item Beátus vir qui implévit desidérium \textbf{su}um ex \textbf{ip}sis:~* non confundétur cum loquétur inimícis su\textit{is} \textit{in} \textbf{por}ta.
    
    \item Glória \textbf{Pa}tri, et \textbf{Fí}lio,~* et Spirí\textit{tu}\textit{i} \textbf{Sanc}to.
    
    \item Sicut erat in princípio, et \textbf{nunc}, et \textbf{sem}per,~* et in sǽcula sæcu\textit{ló}\textit{rum}. \textbf{A}men.
  \end{enumerate}

  \medskip
  \grecommentary{\textit{Reprise de l'Antienne.}}
  \gabcsnippet{(c4) Ger(d)mi(e_[uh:l]f)ná(g)vit(ge~) () ra(gh)dix(fe) Jes(d)se,(d_fddc.) (;) or(dc~)ta(f') est(g) stel(h)la(f_g) ex(h) Ja(g)cob :(f.) (:) Vir(f')go(g) pé(h')pe(g)rit(f') Sal(g)va(f)tó(f_gF'D)rem :(d.) (;) te(g_[uh:l]h) lau(f)dá(fe)mus,(c_[uh:l-0.8mm]e) De(g_[oh:h]e)us(f_e) no(d.)ster.(d.) (::)}

  \newpage

  \begin{normalsize}
    \begin{center}
      \begin{enumerate}[label=\textcolor{red}{\emph{\arabic*}}]
        \item \textit{Si le Seigneur n’édifie la maison, en vain travaillent ceux qui la bâtissent. }
        \item \textit{Si le Seigneur ne garde la ville, en vain les sentinelles veillent à sa garde. }
        \item \textit{En vain vous vous lèverez avant le jour : levez-vous après vous êtes reposé, vous qui mangez d’un pain de douleur. }
        \item \textit{C’est Dieu qui fait reposer ses bien-aimés : les enfants sont un héritage qui vient du Seigneur : et le fruit des entrailles une récompense. }
        \item \textit{Ce que sont les flèches en la main d’un vaillant homme, les enfants le sont à leurs pères. }
        \item \textit{Heureux celui qui en a selon son désir ; il ne sera point confondu, lorsqu’il parlera à ses ennemis devant les tribunaux des juges. }
        \item \textit{Gloire au Père, au Fils, et au Saint Esprit, }
        \item \textit{Comme il était au commencement, maintenant et toujours, et dans les siècles des siècles. Amen. }
      \end{enumerate}
    \end{center}
  \end{normalsize}

  \newpage

  % ===== DEBUT Antienne =========
  \gresetinitiallines{1}
  \greillumination{\initfamily\fontsize{11mm}{11mm}\selectfont E}
  \gregorioscore{antiennes/an--ecce_maria_genuit--solesmes_1961}
  \begin{center}
    \footnotesize{
      \textit{
        Voici que Marie nous a enfanté le Sauveur, à la vue duquel Jean s’est écrié : Voici l’Agneau de Dieu, voici celui qui ôte les péchés du monde, alléluia.  
      }
    }
  \end{center}
  % ===== DEBUT psaume ===========
  % gresetinitiallines : avec le parametre à 0, supprime l'ornement
  \begin{center}
    \normalsize{Psaume 147.}
  \end{center}

  % gresetinitiallines : avec le parametre à 0, supprime l'ornement
  \gresetinitiallines{0}
  \gregorioscore{psaumes/psaume147-IID}

  \begin{enumerate}[label=\textcolor{red}{\arabic*}]
    \setcounter{enumi}{1}
    \item Quóniam confortávit seras portárum tu\textbf{á}rum:~* benedíxit fíliis tu\textit{is} \textbf{in} te.

    \item Qui pósuit fines tuos \textbf{pa}cem:~* et ádipe fruménti sá\textit{ti}\textbf{at} te.
    
    \item Qui emíttit elóquium suum \textbf{ter}ræ:~* velóciter currit ser\textit{mo} \textbf{e}jus.
    
    \item Qui dat nivem sicut \textbf{la}nam:~* nébulam sicut cíne\textit{rem} \textbf{spar}git.
    
    \item Mittit crystállum suam sicut buc\textbf{cél}las:~* ante fáciem frígoris ejus quis sus\textit{ti}\textbf{né}bit?
    
    \item Emíttet verbum suum, et liquefáciet \textbf{e}a:~* flabit spíritus ejus, et flu\textit{ent} \textbf{a}quæ.
    
    \item Qui annúntiat verbum suum \textbf{Ja}cob:~* justítias, et judícia su\textit{a} \textbf{Is}raël.
    
    \item Non fecit táliter omni nati\textbf{ó}ni:~* et judícia sua non manifestá\textit{vit} \textbf{e}is.
    
    \item Glória Patri, et \textbf{Fí}lio,~* et Spirítu\textit{i} \textbf{Sanc}to.
    
    \item Sicut erat in princípio, et nunc, et \textbf{sem}per,~* et in sǽcula sæculó\textit{rum}. \textbf{A}men.
  \end{enumerate}

  %  Répetition de l'Antienne
  \grecommentary{\textit{Reprise de l'Antienne.}}
  \gabcsnippet{(f3) Ec(g)ce(e') Ma(f)rí(f)a(f'_) (,) gé(h)nu(g)it(f') no(g)bis(fe) Sal(fg)va(g')tó(f)rem,(f.) (:) quem(e') Jo(f)án(h)nes(h') vi(g)dens(i_[uh:l]jIG') (,) ex(e)cla(f)má(h_g)vit,(hih) di(f.)cens :(f.) (:) Ec(h)ce(g') A(h)gnus(f') De(g)i,(f.) (;) ec(h)ce(hg) qui(f') tol(g)lit(f') <c>+</c>() pec(e)cá(fg)ta(g') mun(f)di,(fvEC') al(e)le(ef)lú(f.){ia}.(f.) (::)(Z) <c><i>Post Septuag.</i></c>() <c>+</c>() pec(e)cá(fg)ta(g) mun(f.)di.(f.) (::) }

  \newpage

  \begin{normalsize}
    \begin{center}
      \begin{enumerate}[label=\textcolor{red}{\emph{\arabic*}}]
        \item \textit{Jérusalem, louez le Seigneur ; Sion, chantez les louanges de votre Dieu :}
        \item \textit{Parce qu’il a fortifié les serrures de tes portes, et  qu’il a béni les enfants que tu renfermes dans ton enceinte. }
        \item \textit{Il a établi la paix jusqu’aux confins de tes Etats, et il te rassasie du meilleur froment. }
        \item \textit{Il envoie sa parole à la terre, et cette parole est portée partout avec une extrême vitesse. }
        \item \textit{Il fait que la neige tombe partout comme de la laine sur la terre ; il y répand la gelée blanche comme de la cendre.}
        \item \textit{Il envoie sa glace divisée en une infinité de parties ; qui pourra soutenir la rigueur du froid extrême de son froid ? }
        \item \textit{Mais au moment qu’il aura donné ses ordres, il fera fondre toutes ces glaces. Son vent soufflera, et les eaux couleront à l’heure même. }
        \item \textit{Il annonce sa parole à Jacob, ses jugements et ses ordonnances à Israël. }
        \item \textit{Il n’a point traité de la sorte toutes les autres nations, et il ne leur a point manifesté ses préceptes. }
        \item \textit{Gloire au Père, au Fils, et au Saint Esprit, }
        \item \textit{Comme il était au commencement, maintenant et toujours, et dans les siècles des siècles. Amen. }
      \end{enumerate}
    \end{center}
  \end{normalsize}

  \newpage

  \par \textit{\footnotesize\textcolor{red}{On se lève pour le Capitule.}}

  \begin{center}
    \textcolor{red}{\large{Capitule}}\\
    \small\textit{
      Épître à Tite. II, 11-12.
    }
  \end{center}

  \begin{multicols}{2}
    \parindent=0pt
    Apparvit grátia Dei Salvatóris nostri ómnibus homínibus, \textcolor{red}{†} erúdiens nos, ut, abnegántes impietátem et sæculária desidéria, \textcolor{red}{*} sóbrie et juste et pie vivámus in hoc sæculo.  \\
    \textcolor{red}{\Rbar.} Deo grátias.

    \columnbreak

    \textit{ Voici manifestée à tous les hommes la grâce de Dieu, notre Sauveur. Elle nous enseigne à rejeter l’impiété et les convoitises du monde, pour vivre avec mesure, justice et piété, dans le siècle d’icibas. \\
    \textcolor{red}{\Rbar.} Rendons grâce à Dieu.
    }
  \end{multicols}


  \par \textit{\footnotesize\textcolor{red}{Le Célébrant entonne, ensuite, les Chantres et le Chœur alternent les versets. La Doxologie est chantée par tous.}}

  \begin{center}
    \textcolor{red}{\large{Hymne}}\\
  \end{center}
  
  \gresetinitiallines{1}
  \greillumination{\initfamily\fontsize{11mm}{11mm}\selectfont C}
  \gregorioscore{hymnes/hy--christe_redemptor_omnium_ex_patre--solesmes_1934}
  \begin{normalsize}
    \begin{enumerate}[label=\textcolor{red}{\emph{\arabic*}}]
      \item \textit{Christ, Rédempteur de tous les hommes, Fils Unique engendré du Père Avant l’origine du monde, En une ineffable naissance. }
      \item \textit{Vous, lumière, vous, splendeur du Père, Notre espoir éternel à tous, Ecoutez, par tout l’univers Vos serviteurs qui vous supplient. }
      \item \textit{Souvenez-vous, ô Dieu Sauveur, Que vous avez jadis reçu, Naissant de la Vierge sans tache, L’humble livrée de notre corps. }
      \item \textit{Ce jour présent en est témoin, Que le cours de l’année ramène : Descendant du trône du Père Vous seul avez sauvé le monde. }
      \item \textit{Le ciel et la terre et la mer Et tous les êtres qui les peuplent Célèbrent dans un chant joyeux Celui qui vous a envoyé. }
      \item \textit{Et nous qui sommes rachetés Au prix de votre Sang très saint, En ce jour de votre naissance Nous chantons un hymne nouveau.}
      \item \textit{Gloire soit à jamais rendue A vous Seigneur, né de la Vierge, Avec le Père et l’Esprit-Saint Dans les siècles sans fin. Amen. }
    \end{enumerate}
  \end{normalsize}
  \bigskip

  \begin{center}
    \begin{footnotesize}
      \textcolor{red}{\textit{On chante le verset debout.}}
    \end{footnotesize}
    \begin{minipage}{0.8\linewidth}
      \gresetinitiallines{0}
      \gabcsnippet{(c3)<c><v>\Vbar</v>.</c> No(h)tum(h) fe(h)cit(h) Dó(hi)mi(h)nus,(h) (,) al(h)le(fe)lú(f_h){ia}.(hiH'Ghih.ghG'FE'fggf.0) (::) (Z-) 
      <c><v>\Rbar</v>.</c> Sa(h)lu(h)tá(h)re(h) su(hi)um,(h) (,) al(h)le(fe)lú(f_h){ia}.(hiH'Ghih.ghG'FE'fggf.0) (::)}
      \bigskip
      \begin{center}
        \textit{\textcolor{red}{\Vbar.} Le Seigneur a fait connaître, alléluia.}\\
        \textit{\textcolor{red}{\Rbar.} Son salut, alléluia. }
      \end{center}
    \end{minipage}
  \end{center}
  \normalsize

  \vspace*{\fill}\
  \begin{center}
    \greseparator{3}{30}
  \end{center}
  \vspace*{\fill}\

  \newpage

  \begin{center}
    \textcolor{red}{\large{Antienne à Magnificat}}\\
  \end{center}

  \gresetinitiallines{1}
  \greillumination{\initfamily\fontsize{11mm}{11mm}\selectfont M}
  \gregorioscore{antiennes/an--magnum_haereditatis--solesmes_1961}
  \begin{center}
    \footnotesize{\textit{
      O grand mystère de l’hérédité divine ! Le sein d’une vierge est devenu le temple de Dieu ; celui qui d’elle a pris chair n’a contracté aucune souillure ; toutes les nations viendront et diront : Gloire à vous, Seigneur. 
    }}
  \end{center}
  \medskip
  

  \gresetinitiallines{0}
  \gregorioscore{magnificat/magnificat-IIDsolemn}

  \begin{enumerate}[label=\textcolor{red}{\arabic*}]
    \setcounter{enumi}{2}
    \item Quia respéxit humilitátem \textit{an}\textit{cíl}\textit{læ} \textbf{su}æ:~* ecce enim ex hoc beátam me dicent omnes genera\textit{ti}\textbf{ó}nes.

    \item Quia fecit mihi \textit{ma}\textit{gna} \textit{qui} \textbf{pot}ens est:~* et sanctum no\textit{men} \textbf{e}jus.
    
    \item Et misericórdia ejus a progéni\textit{e} \textit{in} \textit{pro}\textbf{gé}nies~* timénti\textit{bus} \textbf{e}um.
    
    \item Fecit poténtiam in \textit{brá}\textit{chi}\textit{o} \textbf{su}o:~* dispérsit supérbos mente cor\textit{dis} \textbf{su}i.
    
    \item Depósuit pot\textit{én}\textit{tes} \textit{de} \textbf{se}de,~* et exaltá\textit{vit} \textbf{hú}miles.
    
    \item Esuriéntes \textit{im}\textit{plé}\textit{vit} \textbf{bo}nis:~* et dívites dimísit \textit{in}\textbf{á}nes.
    
    \item Suscépit Israël \textit{pú}\textit{e}\textit{rum} \textbf{su}um,~* recordátus misericórdi\textit{æ} \textbf{su}æ.
    
    \item Sicut locútus est \textit{ad} \textit{pa}\textit{tres} \textbf{nos}tros,~* Abraham et sémini ejus \textit{in} \textbf{sǽ}cula.
    \begin{center}
      \textit{\footnotesize \textcolor{red}{On attend la fin de l'encensement avant de chanter la doxologie.}}
    \end{center}
    \item Glória \textit{Pa}\textit{tri}, \textit{et} \textbf{Fí}lio,~* et Spirítu\textit{i} \textbf{Sanc}to.
    
    \item Sicut erat in princípio, \textit{et} \textit{nunc}, \textit{et} \textbf{sem}per,~* et in sǽcula sæculó\textit{rum}. \textbf{A}men.
  \end{enumerate}

  \grecommentary{\textit{Reprise de l'Antienne.}}
  \gabcsnippet{(c3) Ma(f_e/f!gwh!ivH'_GF)gnum(f) (,) hae(g)re(e')di(f)tá(h)tis(hg) my(f)sté(f!gwh!ivH'_GFgf)ri(ef)um :(f.) (:) tem(h)plum(i') De(i)i(h') fa(i)ctus(hg) est(f'_) (,) ú(h_)te(hg)rus(e) né(f_g)sci(fe)ens(fhg) vi(f.)rum :(f.) (:) non(f) est(f) pol(f')lú(h)tus(f') (,) ex(e) e(d_e)a(e) car(f!gwh_g~)nem(e) as(eg)sú(f.)mens :(f.) (:) o(fe/fe)mnes(df) gen(hhi)tes(h'_) (,) vé(i)ni(hg)ent,(f) di(e)cén(f_e)tes :(d.) (;) Gló(f)ri(e')a(d) ti(e')bi(f) Dó(f!gwh!ivH'_GFgf)mi(ef)ne.(f.) (::)}

  \newpage

  \begin{center}
    \textcolor{red}{\large{Oraison}}\\
  \end{center}

  \begin{multicols}{2}
    \parindent=0pt
    \begin{flushright}
      \textcolor{red}{\Vbar.} Dominus vobiscum.\\
      \textcolor{red}{\Rbar.} Et cum spiritu tuo.\\
    \end{flushright}

    \columnbreak
    
    \textit{\textcolor{red}{\Vbar.} Le Seigneur soit avec vous.\\
    \textcolor{red}{\Rbar.} Et avec votre esprit.}\\
  \end{multicols}

  \begin{multicols}{2}
    \parindent=0pt
    Deus, qui salútis ætérnæ, beátæ Maríæ virginitáte fœcúnda, humáno géneri præmia præstitísti : \textcolor{red}{†} tríbue, quæsumus ; ut ipsam pro nobis intercédere sentiámus, \textcolor{red}{*} per quam merúimus auctórem vitæ suscípere, Dóminum nostrum Jesum Christum Fílium tuum : \\ Qui tecum vivit et regnat in unitáte Spíritus Sancti Deus, per ómnia sæcula sæculórum.
    \textcolor{red}{\Rbar.} Amen.
    \columnbreak

    \textit{ Dieu, par la maternité virginale de la bienheureuse Vierge Marie, avez donné au genre humain les biens du salut éternel : nous vous en prions, faites nous ressentir aussi la puissante intercession de celle qui nous a mérité la venue de l’Auteur de la vie, Notre Seigneur Jésus-Christ, votre Fils, qui avec vous vit et règne en l’unité du Saint Esprit, Dieu pour tous les siècles des siècles. 
    Amen.
    }
  \end{multicols}

  \medskip

  \begin{center}
    \textcolor{red}{\large{Conclusion de l'office}}
  \end{center}
  
  
  \begin{multicols}{2}
    \parindent=0pt
    \begin{flushright}
      \textcolor{red}{\Vbar.} Dominus vobiscum.\\
      \textcolor{red}{\Rbar.} Et cum spiritu tuo.\\
    \end{flushright}
  
    \columnbreak
    
    \textit{\textcolor{red}{\Vbar.} Le Seigneur soit avec vous.\\
    \textcolor{red}{\Rbar.} Et avec votre esprit.}\\
  \end{multicols}
  \bigskip
  \gresetinitiallines{1}
  \greillumination{\initfamily\fontsize{11mm}{11mm}\selectfont B}
  \gregorioscore{or--benedicamus_domino_(i_classis_in_ii_vesperis_mode_5)--solesmes_1961}
  \begin{center}
    \begin{footnotesize}
      \textcolor{red}{\textit{Sur un ton très grave : }}
    \end{footnotesize}
  \end{center}
  \begin{multicols}{2}
    \parindent=0pt
    \textcolor{red}{\Vbar.} Fidélium ánimæ per misericórdiam Dei requiéscant in pace.\\
    \textcolor{red}{\Rbar.} Amen.\\

    \columnbreak
    
    \textit{\textcolor{red}{\Vbar.} Que les âmes des fidèles défunts, par la
    miséricorde de Dieu, reposent en paix.\\
    \textcolor{red}{\Rbar.} Amen.}\\
  \end{multicols}

  \medskip

  \begin{center}
    \textcolor{red}{\large{Salut du Saint Sacrement}}\\
    \textit{Voir page 50.}
  \end{center}

  \vspace*{\fill}\
  \begin{center}
    \greseparator{3}{30}
  \end{center}
  \vspace*{\fill}\
  

  \newpage

  \begin{center}
    \begin{LARGE}
      Fête de la Sainte Famille
    \end{LARGE}
  \end{center}
  \medskip

  % ===== DEBUT Antienne =========
  \greillumination{\initfamily\fontsize{11mm}{11mm}\selectfont P}
  \gregorioscore{antiennes/an--post_triduum--solesmes_1961.1}
  \begin{center}
    \footnotesize{
      \textit{
        Après trois jours, ils trouvèrent Jésus dans le temple, assis au milieu des docteurs, les écoutant et les interrogeant. 
      }
    }
  \end{center}
  % \medskip
  % ===== FIN Antienne ===========

  % ===== DEBUT psaume ===========
  % gresetinitiallines : avec le parametre à 0, supprime l'ornement
  \gresetinitiallines{0}

  \begin{center}
    \normalsize{Psaume 109.}
  \end{center}

  \gregorioscore{psaumes/psaume109-VIIIG}
  \begin{enumerate}[label=\textcolor{red}{\emph{\arabic*}}]
    \setcounter{enumi}{2}
    \item Donec ponam inimícos \textbf{tu}os,~* scabéllum pe\textit{dum} \textit{tu}\textbf{ó}rum.

    \item Virgam virtútis tuæ emíttet Dóminus ex \textbf{Si}on:~* domináre in médio inimicó\textit{rum} \textit{tu}\textbf{ó}rum.
    
    \item Tecum princípium in die virtútis tuæ in splendóribus sanc\textbf{tó}rum:~* ex útero ante lucíferum \textit{gé}\textit{nu}\textbf{i} te.
    
    \item Jurávit Dóminus, et non pœnitébit \textbf{e}um:~* Tu es sacérdos in ætérnum secúndum órdi\textit{nem} \textit{Mel}\textbf{chí}sedech.
    
    \item Dóminus a dextris \textbf{tu}is,~* confrégit in die iræ \textit{su}\textit{æ} \textbf{re}ges.
    
    \item Judicábit in natiónibus, implébit ru\textbf{í}nas:~* conquassábit cápita in ter\textit{ra} \textit{mul}\textbf{tó}rum.
    
    \item De torrénte in via \textbf{bi}bet:~* proptérea exal\textit{tá}\textit{bit} \textbf{ca}put.
    
    \item Glória Patri, et \textbf{Fí}lio,~* et Spirí\textit{tu}\textit{i} \textbf{Sanc}to.
    
    \item Sicut erat in princípio, et nunc, et \textbf{sem}per,~* et in sǽcula sæcu\textit{ló}\textit{rum}. \textbf{A}men.
  \end{enumerate}
  \medskip
  \grecommentary{\textit{Reprise de l'Antienne.}}
  \gabcsnippet{(c4) Post(h') trí(f)du(fg)um,(g') (,) in(g)ve(g')né(h)runt(gf) Je(h!iwj)sum(ji) in(hg~) tem(h)plo,(g.) (;) se(i_[uh:l]j)dén(k)tem(ji) in(h) mé(j')di(i)o(g) do(f)ctó(gh)rum,(h.) (;) au(g)di(h)én(gf~)tem(gh) il(f)los(f'_) (,) et(h) in(j)ter(ji)ro(hg)gán(hi)tem(h) e(g.)os.(g.) (::) }

  \newpage

  % ===== DEBUT Antienne =========
  \gresetinitiallines{1}
  \greillumination{\initfamily\fontsize{11mm}{11mm}\selectfont D}
  \gregorioscore{antiennes/an--dixit_mater_jesu_ad_illum--solesmes_1961}
  \begin{center}
    \footnotesize{
      \textit{
        La Mère de Jésus lui dit : Mon fils, pourquoi avez-vous agi ainsi avec nous ? Voilà que votre père et moi, fort affligés, nous vous cherchions. 
      }
    }
  \end{center}
  % \medskip
  % ===== FIN Antienne ===========

  % ===== DEBUT psaume ===========
  % gresetinitiallines : avec le parametre à 0, supprime l'ornement
  \gresetinitiallines{0}

  \begin{center}
    \normalsize{Psaume 112.}
  \end{center}

  \gregorioscore{psaumes/psaume112-IVE}

  \begin{enumerate}[label=\textcolor{red}{\emph{\arabic*}}]
    \setcounter{enumi}{1}
    \item Sit nomen Dómini \textit{be}\textit{ne}\textbf{díc}tum,~* ex hoc nunc, et \textit{us}\textit{que} \textit{in} \textbf{sǽ}\textbf{cu}lum.

    \item A solis ortu usque \textit{ad} \textit{oc}\textbf{cá}sum,~* laudábi\textit{le} \textit{no}\textit{men} \textbf{Dó}\textbf{mi}ni.
    
    \item Excélsus super omnes \textit{gen}\textit{tes} \textbf{Dó}minus,~* et super cælos \textit{gló}\textit{ri}\textit{a} \textbf{e}jus.
    
    \item Quis sicut Dóminus, Deus noster, qui in \textit{al}\textit{tis} \textbf{há}bitat,~* et humília réspicit in cæ\textit{lo} \textit{et} \textit{in} \textbf{ter}ra?
    
    \item Súscitans a \textit{ter}\textit{ra} \textbf{ín}opem,~* et de stércore \textit{é}\textit{ri}\textit{gens} \textbf{páu}\textbf{pe}rem:
    
    \item Ut cóllocet eum \textit{cum} \textit{prin}\textbf{cí}pibus,~* cum princípibus \textit{pó}\textit{pu}\textit{li} \textbf{su}i.
    
    \item Qui habitáre facit stéri\textit{lem} \textit{in} \textbf{do}mo,~* matrem fili\textit{ó}\textit{rum} \textit{læ}\textbf{tán}tem.
    
    \item Glória Pa\textit{tri}, \textit{et} \textbf{Fí}lio,~* et Spi\textit{rí}\textit{tu}\textit{i} \textbf{Sanc}to.
    
    \item Sicut erat in princípio, et \textit{nunc}, \textit{et} \textbf{sem}per,~* et in sǽcula sæ\textit{cu}\textit{ló}\textit{rum}. \textbf{A}men.
  \end{enumerate}

  \medskip
  \grecommentary{\textit{Reprise de l'Antienne.}}
  \gabcsnippet{(c4) Di(f)xit(e') Ma(f)ter(d') Je(c)su(df) ad(fe) il(de)lum :(e.) <c>*</c>(;) Fi(e!g'h)li,(h') quid(h) fe(hg)cí(hj)sti(hv_GF) no(gh)bis(gf) sic?(ef..) (;) Ec(d)ce(g') pa(h)ter(g') tu(g)us(dc) et(d) e(de)go(e'_[oh:h]) (,) do(f)lén(e.d!ewf)tes(d.) quae(e')re(f)bá(gh)mus(g) te.(e.) (::) }

  \newpage

  % ===== DEBUT Antienne =========
  \gresetinitiallines{1}
  \greillumination{\initfamily\fontsize{11mm}{11mm}\selectfont D}
  \gregorioscore{antiennes/an--descendit_jesus_cum_eis_(ant.)--solesmes_1961}
  \begin{center}
    \footnotesize{
      \textit{
        Jésus descendit avec eux, et vînt à Nazareth ; et il leur était soumis. 
      }
    }
  \end{center}
  % \medskip
  % ===== FIN Antienne ===========

  % ===== DEBUT psaume ===========
  % gresetinitiallines : avec le parametre à 0, supprime l'ornement

  \gresetinitiallines{0}

  \begin{center}
    \normalsize{Psaume 121.}\\
  \end{center}
  % \smallskip
  \grechangedim{baselineskip}{50pt}{scalable}
  \gregorioscore{psaumes/psaume121-VIIIG}
  \begin{enumerate}[label=\textcolor{red}{\emph{\arabic*}}]
    \setcounter{enumi}{1}
    \item Stantes erant pedes \textbf{nos}tri,~* in átriis tu\textit{is}, \textit{Je}\textbf{rú}salem.

    \item Jerúsalem, quæ ædificátur ut \textbf{cí}vitas:~* cujus participátio ejus \textit{in} \textit{id}\textbf{íp}sum.
    
    \item Illuc enim ascendérunt tribus, tribus \textbf{Dó}mini:~* testimónium Israël ad confiténdum nó\textit{mi}\textit{ni} \textbf{Dó}mini.
    
    \item Quia illic sedérunt sedes in ju\textbf{dí}cio,~* sedes super \textit{do}\textit{mum} \textbf{Da}vid.
    
    \item Rogáte quæ ad pacem sunt Je\textbf{rú}salem:~* et abundántia dili\textit{gén}\textit{ti}\textbf{bus} te:
    
    \item Fiat pax in virtúte \textbf{tu}a:~* et abundántia in túr\textit{ri}\textit{bus} \textbf{tu}is.
    
    \item Propter fratres meos, et próximos \textbf{me}os,~* loquébar \textit{pa}\textit{cem} \textbf{de} te:
    
    \item Propter domum Dómini, Dei \textbf{nos}tri,~* quæsívi \textit{bo}\textit{na} \textbf{ti}bi.
    
    \item Glória Patri, et \textbf{Fí}lio,~* et Spirí\textit{tu}\textit{i} \textbf{Sanc}to.
    
    \item Sicut erat in princípio, et nunc, et \textbf{sem}per,~* et in sǽcula sæcu\textit{ló}\textit{rum}. \textbf{A}men.
  \end{enumerate}

  \medskip
  \grecommentary{\textit{Reprise de l'Antienne.}}
  \gabcsnippet{(c4) De(g)scén(gf)dit(d) () Je(e_[uh:l]f)sus(gh) cum(h') e(g)is,(g'_[oh:h]) (,) et(g) ve(f')nit(h) Ná(j_k)za(j)reth,(ji/jkj.) (;) et(k) e(ji)rat(gh) súb(j_h)di(i)tus(h) il(g.)lis.(g.) (::) <i>T. P.</i> Al(hi)le(h)lú(g.){ia}.(g.) (::)}

  \newpage

  % ===== DEBUT Antienne =========
  \gresetinitiallines{1}
  \greillumination{\initfamily\fontsize{11mm}{11mm}\selectfont E}
  \gregorioscore{antiennes/an--et_jesus_proficiebat--solesmes_1961}
  \begin{center}
    \footnotesize{\textit{
      Le Seigneur va venir, allez au-devant de lui, disant : Sa puissance est grande et son règne n’aura pas de fin ; il est Dieu, Fort, Dominateur, Prince de la paix, alléluia, alléluia. 
    }}
  \end{center}

  % ===== FIN Antienne ===========

  % ===== DEBUT psaume ===========
  % gresetinitiallines : avec le parametre à 0, supprime l'ornement
  \gresetinitiallines{0}

  \begin{center}
    \normalsize{Psaume 126.}\\
  \end{center}
  % \smallskip

  \gregorioscore{psaumes/psaume126-IID}

  \begin{enumerate}[label=\textcolor{red}{\emph{\arabic*}}]
    \setcounter{enumi}{1}
    \item Nisi Dóminus custodíerit civi\textbf{tá}tem,~* frustra vígilat qui custó\textit{dit} \textbf{e}am.

    \item Vanum est vobis ante lucem \textbf{súr}gere:~* súrgite postquam sedéritis, qui manducátis panem \textit{do}\textbf{ló}ris.
    
    \item Cum déderit diléctis suis \textbf{som}num:~* ecce heréditas Dómini fílii: merces, fruc\textit{tus} \textbf{ven}tris.
    
    \item Sicut sagíttæ in manu pot\textbf{én}tis:~* ita fílii ex\textit{cus}\textbf{só}rum.
    
    \item Beátus vir qui implévit desidérium suum ex \textbf{ip}sis:~* non confundétur cum loquétur inimícis suis \textit{in} \textbf{por}ta.
    
    \item Glória Patri, et \textbf{Fí}lio,~* et Spirítu\textit{i} \textbf{Sanc}to.
    
    \item Sicut erat in princípio, et nunc, et \textbf{sem}per,~* et in sǽcula sæculó\textit{rum}. \textbf{A}men.
  \end{enumerate}

  \medskip
  \grecommentary[5px]{\textit{Reprise de l'Antienne.}}
  \gabcsnippet{(f3) Et(e) Je(f)sus(hg) () pro(ij)fi(h)ci(i)é(hv_GF)bat(f_hffe.) (,) sa(h)pi(ij)én(kxjk)ti(ih)a,(h'_) (,) et(j) ae(i')tá(i)te,(g') et(i) grá(ij)ti(hg)a(f.) (;) a(f)pud(e') De(f)um(h') et(g) hó(e')mi(g)nes.(f.) (::)}

  \newpage

  % ===== DEBUT Antienne =========
  \gresetinitiallines{1}
  \greillumination{\initfamily\fontsize{11mm}{11mm}\selectfont E}
  \gregorioscore{antiennes/an--et_dicebant--solesmes_1961}
  \begin{center}
    \footnotesize{
      \textit{
        Ils disaient : D’où viennent à celui-ci cette sagesse et ces miracles ? N’est-ce pas le fils du charpentier ? 
      }
    }
  \end{center}
  % ===== DEBUT psaume ===========
  % gresetinitiallines : avec le parametre à 0, supprime l'ornement
  \begin{center}
    \normalsize{Psaume 147.}
  \end{center}

  % gresetinitiallines : avec le parametre à 0, supprime l'ornement
  \gresetinitiallines{0}
  \gregorioscore{psaumes/psaume147-VIIIG}

  \begin{enumerate}[label=\textcolor{red}{\arabic*}]
    \setcounter{enumi}{1}
    \item Quóniam confortávit seras portárum tu\textbf{á}rum:~* benedíxit fíliis \textit{tu}\textit{is} \textbf{in} te.

    \item Qui pósuit fines tuos \textbf{pa}cem:~* et ádipe fruménti \textit{sá}\textit{ti}\textbf{at} te.
    
    \item Qui emíttit elóquium suum \textbf{ter}ræ:~* velóciter currit \textit{ser}\textit{mo} \textbf{e}jus.
    
    \item Qui dat nivem sicut \textbf{la}nam:~* nébulam sicut cí\textit{ne}\textit{rem} \textbf{spar}git.
    
    \item Mittit crystállum suam sicut buc\textbf{cél}las:~* ante fáciem frígoris ejus quis \textit{sus}\textit{ti}\textbf{né}bit?
    
    \item Emíttet verbum suum, et liquefáciet \textbf{e}a:~* flabit spíritus ejus, et \textit{flu}\textit{ent} \textbf{a}quæ.
    
    \item Qui annúntiat verbum suum \textbf{Ja}cob:~* justítias, et judícia \textit{su}\textit{a} \textbf{Is}raël.
    
    \item Non fecit táliter omni nati\textbf{ó}ni:~* et judícia sua non manifes\textit{tá}\textit{vit} \textbf{e}is.
    
    \item Glória Patri, et \textbf{Fí}lio,~* et Spirí\textit{tu}\textit{i} \textbf{Sanc}to.
    
    \item Sicut erat in princípio, et nunc, et \textbf{sem}per,~* et in sǽcula sæcu\textit{ló}\textit{rum}. \textbf{A}men.
  \end{enumerate}

  %  Répetition de l'Antienne
  \grecommentary{\textit{Reprise de l'Antienne.}}
  \gabcsnippet{(c4) Et(h) di(gf)cé(g_[uh:l]h)bant :(g'_[oh:h]) (,) Un(f)de(h') hu(g)ic(g') sa(h)pi(i')én(j)ti(h')a(g) haec,(h_g) et(e') vir(f)tú(g)tes?(g.) (;) Non(g)ne(d) hic(e_[uh:l]f) est(gf) fa(gh)bri(h) fí(g)li(g)us?(g.) (::) }

  \medskip
  \begin{center}
    \rule{2cm}{0.4pt}
  \end{center}
  \medskip

  \par \textit{\footnotesize\textcolor{red}{On se lève pour le Capitule.}}

  \begin{center}
    \textcolor{red}{\large{Capitule}}\\
    \small\textit{
      Luc. II, 51
    }
  \end{center}

  \begin{multicols}{2}
    \parindent=0pt
    Descendit Jesus cum María et Joseph, et venit Názareth, \textcolor{red}{†} Hic jam quæritur inter dispensatóres,  \textcolor{red}{*} et erat subdítus illis. \\
    \textcolor{red}{\Rbar.} Deo grátias.

    \columnbreak

    \textit{Jésus descendit avec Marie et Joseph, et il vint à Nazareth, et il leur était soumis.\\
    \textcolor{red}{\Rbar.} Rendons grâce à Dieu.
    }
  \end{multicols}


  \par \textit{\footnotesize\textcolor{red}{Le Célébrant entonne, ensuite, les Chantres et le Chœur alternent les versets. La Doxologie est chantée par tous.}}

  \begin{center}
    \textcolor{red}{\large{Hymne}}\\
  \end{center}
  
  \gresetinitiallines{1}
  \greillumination{\initfamily\fontsize{11mm}{11mm}\selectfont O}
  \gregorioscore{hymnes/hy--o_lux_beata_caelitum--solesmes_1961}
  \begin{normalsize}
    \begin{enumerate}[label=\textcolor{red}{\emph{\arabic*}}]
      \item \textit{Qu’heureuse est devenue par la famille qui l’habitait, la vénérable demeure de Nazareth, dans laquelle ont germé et se sont développés les mystérieux commencements de l’Église. }
      \item \textit{Joseph assiste son Épouse,  partageant son amour et sa sollicitude :  âmes saintes que la grâce embellit  de vertus et attache par mille nœuds. }
      \item \textit{Le soleil, dont le disque parcourt  l’étendue des continents,  n’a rien vu dans la suite des siècles  qui soit plus charmant ou plus saint. }
      \item \textit{Se chérissant l’un et l’autre,  ils concentrent leur amour en Jésus,  et Jésus donne à l’un et à l’autre  les témoignages d’une charité réciproque. }
      \item \textit{Les messagers de la cour céleste  volent vers elle en grand nombre,  ils visitent, ils visitent encore,  ils vénèrent ce sanctuaire de la vertu. }
      \item \textit{Puisse la charité nous unir  également par des liens indissolubles !  puisse-t-elle entretenir la paix dans les familles  et adoucir les amertumes de la vie ! }
      \item \textit{De quel cœur, de quelle main,  Jésus accomplit les désirs paternels !  Avec quelle joie la Vierge  se livre à ses devoirs de mère ! }
      \item \textit{O Jésus, qui avez voulu  être obéissant à vos parents,  gloire à vous toujours  ainsi qu’au Père souverain et à l’Esprit saint.  Ainsi soit-il. }
    \end{enumerate}
  \end{normalsize}
  \bigskip

  \begin{center}
    \begin{footnotesize}
      \textcolor{red}{\textit{On chante le verset debout.}}
    \end{footnotesize}
    \begin{minipage}{0.8\linewidth}
      \gresetinitiallines{0}
      \gabcsnippet{(c3)<c><v>\Vbar</v>.</c> Po(h)nam(h) uni(h)vér(h)sos(h) fí(h)li(h)os(h) tu(h)os(h) doc(h)tos(h) a(h) Dó(h)mino.(g'_) (hvGF'Efgf.) (::) (Z) <c><v>\Rbar</v>.</c> Et(h) mul(h)ti(h)tú(h)di(h)nem(h) pa(h)cis(h) fí(h)liis(h) tu(h)is. (g'_) (hvGF'Efgf.) (::)}
      \bigskip
      \begin{center}
        \textit{\textcolor{red}{\Vbar.} Je ferai que tous tes fils soient instruits par le Seigneur. }\\
        \textit{\textcolor{red}{\Rbar.} Et qu’une abondance de paix soit sur tes enfants.}
      \end{center}
    \end{minipage}
  \end{center}
  \normalsize

  \vspace*{\fill}\
  \begin{center}
    \greseparator{3}{30}
  \end{center}
  \vspace*{\fill}\

  \newpage

  \begin{center}
    \textcolor{red}{\large{Antienne à Magnificat}}\\
  \end{center}

  \gresetinitiallines{1}
  \greillumination{\initfamily\fontsize{11mm}{11mm}\selectfont M}
  \gregorioscore{antiennes/an--maria_autem--solesmes_1961}
  \begin{center}
    \footnotesize{\textit{
      Or Marie conservait toutes ces choses en son cœur.
    }}
  \end{center}
  \medskip
  

  \gresetinitiallines{0}
  \gregorioscore{magnificat/magnificat-VIIIGsolemn}

  \begin{enumerate}[label=\textcolor{red}{\arabic*}]
    \setcounter{enumi}{2}
    \item Quia respéxit humilitátem \textit{an}\textit{cíl}\textit{læ} \textbf{su}æ:~* ecce enim ex hoc beátam me dicent omnes gene\textit{ra}\textit{ti}\textbf{ó}nes.

    \item Quia fecit mihi \textit{ma}\textit{gna} \textit{qui} \textbf{pot}ens est:~* et sanctum \textit{no}\textit{men} \textbf{e}jus.
    
    \item Et misericórdia ejus a progéni\textit{e} \textit{in} \textit{pro}\textbf{gé}nies~* timén\textit{ti}\textit{bus} \textbf{e}um.
    
    \item Fecit poténtiam in \textit{brá}\textit{chi}\textit{o} \textbf{su}o:~* dispérsit supérbos mente \textit{cor}\textit{dis} \textbf{su}i.
    
    \item Depósuit pot\textit{én}\textit{tes} \textit{de} \textbf{se}de,~* et exal\textit{tá}\textit{vit} \textbf{hú}miles.
    
    \item Esuriéntes \textit{im}\textit{plé}\textit{vit} \textbf{bo}nis:~* et dívites dimí\textit{sit} \textit{in}\textbf{á}nes.
    
    \item Suscépit Israël \textit{pú}\textit{e}\textit{rum} \textbf{su}um,~* recordátus misericór\textit{di}\textit{æ} \textbf{su}æ.
    
    \item Sicut locútus est \textit{ad} \textit{pa}\textit{tres} \textbf{nos}tros,~* Abraham et sémini e\textit{jus} \textit{in} \textbf{sǽ}cula.
    \begin{center}
      \textit{\footnotesize \textcolor{red}{On attend la fin de l'encensement avant de chanter la doxologie.}}
    \end{center}
    \item Glória \textit{Pa}\textit{tri}, \textit{et} \textbf{Fí}lio,~* et Spirí\textit{tu}\textit{i} \textbf{Sanc}to.

    \item Sicut erat in princípio, \textit{et} \textit{nunc}, \textit{et} \textbf{sem}per,~* et in sǽcula sæcu\textit{ló}\textit{rum}. \textbf{A}men.
  \end{enumerate}

  \grecommentary{\textit{Reprise de l'Antienne.}}
  \gabcsnippet{(c4) Ma(g)rí(i')a(h) au(j_i~)tem(g_[uh:l]h) (,) con(f)ser(h)vá(j')bat(k) ó(i')mni(j)a(h') ver(h)ba(g) haec,(h.) (;) cón(h')fe(g)rens(e) in(f) cor(gh)de(h) su(g.)o.(g.) (::)}

  \newpage

  \begin{center}
    \textcolor{red}{\large{Oraison}}\\
  \end{center}

  \begin{multicols}{2}
    \parindent=0pt
    \begin{flushright}
      \textcolor{red}{\Vbar.} Dominus vobiscum.\\
      \textcolor{red}{\Rbar.} Et cum spiritu tuo.\\
    \end{flushright}

    \columnbreak
    
    \textit{\textcolor{red}{\Vbar.} Le Seigneur soit avec vous.\\
    \textcolor{red}{\Rbar.} Et avec votre esprit.}\\
  \end{multicols}

  \begin{multicols}{2}
    \parindent=0pt
    Dómine Jesu Christe, qui Maríæ et Joseph súbditus, domésticam vitam ineffabílibus virtútibus consecrásti : \textcolor{red}{†} fac nos, utriúsque auxílio, Famíliæ sanctæ tuæ exémplis ínstrui ;\textcolor{red}{*} et consórtium cónsequi sempitérnum :  \\
    Qui vivis et regnas cum Deo Patre, in unitáte Spíritus Sancti Deus, per ómnia sæcula sæculórum. 
    \textcolor{red}{\Rbar.} Amen.

    \columnbreak

    \textit{Seigneur Jésus-Christ, vous avez consacré la vie de famille en pratiquant d’ineffables vertus et en étant soumis à Marie et à Joseph : faites qu’avec le secours de l’un et de l’autre, nous nous instruisions des exemples de la sainte Famille et partagions un jour son éternel bonheur. Vous qui vivez et règnez avec Dieu le Père en l’unité du Saint Esprit, Dieu pour tous les siècles des siècles. 
    Amen.
    }
  \end{multicols}

  \medskip

  \begin{center}
    \textcolor{red}{\large{Conclusion de l'office}}
  \end{center}
  
  
  \begin{multicols}{2}
    \parindent=0pt
    \begin{flushright}
      \textcolor{red}{\Vbar.} Dominus vobiscum.\\
      \textcolor{red}{\Rbar.} Et cum spiritu tuo.\\
    \end{flushright}
  
    \columnbreak
    
    \textit{\textcolor{red}{\Vbar.} Le Seigneur soit avec vous.\\
    \textcolor{red}{\Rbar.} Et avec votre esprit.}\\
  \end{multicols}
  \bigskip
  \gresetinitiallines{1}
  \greillumination{\initfamily\fontsize{11mm}{11mm}\selectfont B}
  \gregorioscore{or--benedicamus_domino_(i_classis_in_ii_vesperis_mode_5)--solesmes_1961}
  \begin{center}
    \begin{footnotesize}
      \textcolor{red}{\textit{Sur un ton très grave : }}
    \end{footnotesize}
  \end{center}
  \begin{multicols}{2}
    \parindent=0pt
    \textcolor{red}{\Vbar.} Fidélium ánimæ per misericórdiam Dei requiéscant in pace.\\
    \textcolor{red}{\Rbar.} Amen.\\

    \columnbreak
    
    \textit{\textcolor{red}{\Vbar.} Que les âmes des fidèles défunts, par la
    miséricorde de Dieu, reposent en paix.\\
    \textcolor{red}{\Rbar.} Amen.}\\
  \end{multicols}

  \begin{center}
    \textcolor{red}{\large{Salut du Saint Sacrement}}\\
    \textit{Voir page 50.}
  \end{center}

  \vspace*{\fill}\
  \begin{center}
    \greseparator{3}{30}
  \end{center}
  \vspace*{\fill}\

  % \newpage

  % \begin{center}
  %   \begin{LARGE}
  %     Les grandes antiennes "O"
  %   \end{LARGE}
  % \end{center}
  % \medskip
  
  % \begin{center}
  %   \textcolor{red}{\large{Le 17 décembre}}
  % \end{center}

  % \gresetinitiallines{1}
  % \greillumination{\initfamily\fontsize{11mm}{11mm}\selectfont O}
  % \gregorioscore{antiennes/an--o_sapientia--solesmes_1961}
  % \begin{center}
  %   \footnotesize{
  %     \textit{
  %       O Sagesse, qui êtes sortie de la bouche du Très-Haut (Ecclésiastique, XXIV, 3), atteignant d’une extrémité à une autre extrémité, et disposant toutes choses avec force et douceur (Sagesse VIII, 1) : venez pour nous enseigner la voie de la prudence. 
  %     }
  %   }
  % \end{center}
  % \medskip

  % \gresetinitiallines{0}
  % \gregorioscore{magnificat/magnificat-IIDsolemn}

  % \begin{enumerate}[label=\textcolor{red}{\arabic*}]
  %   \setcounter{enumi}{2}
  %   \item Quia respéxit humilitátem \textit{an}\textit{cíl}\textit{læ} \textbf{su}æ:\textcolor{red}{~*} ecce enim ex hoc beátam me dicent omnes genera\textit{ti}\textbf{ó}nes.

  %   \item Quia fecit mihi \textit{ma}\textit{gna} \textit{qui} \textbf{pot}ens est:\textcolor{red}{~*} et sanctum no\textit{men} \textbf{e}jus.
    
  %   \item Et misericórdia ejus a progéni\textit{e} \textit{in} \textit{pro}\textbf{gé}nies\textcolor{red}{~*} timénti\textit{bus} \textbf{e}um.
    
  %   \item Fecit poténtiam in \textit{brá}\textit{chi}\textit{o} \textbf{su}o:\textcolor{red}{~*} dispérsit supérbos mente cor\textit{dis} \textbf{su}i.
    
  %   \item Depósuit pot\textit{én}\textit{tes} \textit{de} \textbf{se}de,\textcolor{red}{~*} et exaltá\textit{vit} \textbf{hú}miles.
    
  %   \item Esuriéntes \textit{im}\textit{plé}\textit{vit} \textbf{bo}nis:\textcolor{red}{~*} et dívites dimísit \textit{in}\textbf{á}nes.
    
  %   \item Suscépit Israël \textit{pú}\textit{e}\textit{rum} \textbf{su}um,\textcolor{red}{~*} recordátus misericórdi\textit{æ} \textbf{su}æ.
    
  %   \item Sicut locútus est \textit{ad} \textit{pa}\textit{tres} \textbf{nos}tros,\textcolor{red}{~*} Abraham et sémini ejus \textit{in} \textbf{sǽ}cula.
    
  %   \item Glória \textit{Pa}\textit{tri}, \textit{et} \textbf{Fí}lio,\textcolor{red}{~*} et Spirítu\textit{i} \textbf{Sanc}to.
    
  %   \item Sicut erat in princípio, \textit{et} \textit{nunc}, \textit{et} \textbf{sem}per,\textcolor{red}{~*} et in sǽcula sæculó\textit{rum}. \textbf{A}men.
  % \end{enumerate}

  % \grecommentary{\textit{Reprise de l'Antienne.}}
  % \gabcsnippet{(f3) O(ehhg) Sa(hg)pi(f)én(gf~)ti(ef)a,(f'_) (,) quae(f) ex(f) o(fg)re(f) Al(f)tís(g)si(f)mi(g_[uh:l]h) pro(fg)dí(f)sti,(e.) (;) at(f)tín(cf~)gens(f) a(f) fi(g)ne(f') us(g)que(f) ad(f) fi(g')nem,(f) fór(g')ti(h)ter(kxiji/jkj.) (;) su(j)á(j')vi(i)ter(h) dis(g)po(h)néns(ih~)que(f') ó(g)mni(fe)a :(e.) (:) ve(eh)ni(g'_[oh:h]) (,) ad(h) do(f)cén(gf~)dum(ef~) nos(fv_EC.) (,) vi(e_[uh:l]f)am(h) pru(g)dén(e)ti(g)ae.(f.) (::)}

  % \newpage

  % \begin{center}
  %   \textcolor{red}{\large{Le 18 décembre}}
  % \end{center}

  % \gresetinitiallines{1}
  % \greillumination{\initfamily\fontsize{11mm}{11mm}\selectfont O}
  % \gregorioscore{antiennes/an--o_adonai--solesmes_1961}
  % \begin{center}
  %   \footnotesize{
  %     \textit{
  %       O Adonaï, et Conducteur de la maison d’Israël (Exode VI, 2-3, 13), qui avez apparu à Moïse dans le feu du buisson ardent (Exode III, 2) et lui avez donné la loi sur le Sinaï : venez pour nous racheter par la puissance de votre bras. (Exode VI, 6) 
  %     }
  %   }
  % \end{center}
  % \medskip

  % \gresetinitiallines{0}
  % \gregorioscore{magnificat/magnificat-IIDsolemn}

  % \begin{enumerate}[label=\textcolor{red}{\arabic*}]
  %   \setcounter{enumi}{2}
  %   \item Quia respéxit humilitátem \textit{an}\textit{cíl}\textit{læ} \textbf{su}æ:\textcolor{red}{~*} ecce enim ex hoc beátam me dicent omnes genera\textit{ti}\textbf{ó}nes.

  %   \item Quia fecit mihi \textit{ma}\textit{gna} \textit{qui} \textbf{pot}ens est:\textcolor{red}{~*} et sanctum no\textit{men} \textbf{e}jus.
    
  %   \item Et misericórdia ejus a progéni\textit{e} \textit{in} \textit{pro}\textbf{gé}nies\textcolor{red}{~*} timénti\textit{bus} \textbf{e}um.
    
  %   \item Fecit poténtiam in \textit{brá}\textit{chi}\textit{o} \textbf{su}o:\textcolor{red}{~*} dispérsit supérbos mente cor\textit{dis} \textbf{su}i.
    
  %   \item Depósuit pot\textit{én}\textit{tes} \textit{de} \textbf{se}de,\textcolor{red}{~*} et exaltá\textit{vit} \textbf{hú}miles.
    
  %   \item Esuriéntes \textit{im}\textit{plé}\textit{vit} \textbf{bo}nis:\textcolor{red}{~*} et dívites dimísit \textit{in}\textbf{á}nes.
    
  %   \item Suscépit Israël \textit{pú}\textit{e}\textit{rum} \textbf{su}um,\textcolor{red}{~*} recordátus misericórdi\textit{æ} \textbf{su}æ.
    
  %   \item Sicut locútus est \textit{ad} \textit{pa}\textit{tres} \textbf{nos}tros,\textcolor{red}{~*} Abraham et sémini ejus \textit{in} \textbf{sǽ}cula.
    
  %   \item Glória \textit{Pa}\textit{tri}, \textit{et} \textbf{Fí}lio,\textcolor{red}{~*} et Spirítu\textit{i} \textbf{Sanc}to.
    
  %   \item Sicut erat in princípio, \textit{et} \textit{nunc}, \textit{et} \textbf{sem}per,\textcolor{red}{~*} et in sǽcula sæculó\textit{rum}. \textbf{A}men.
  % \end{enumerate}

  % \grecommentary{\textit{Reprise de l'Antienne.}}
  % \gabcsnippet{(f3) O(ehhg) A(h_f)do(gf)ná(ef)i,(f'_) (,) et(f) Dux(f) do(g)mus(hf) Is(g)ra(f)el,(e.) (;) qui(f) Mó(cf)y(f)si(f') (,) in(f) i(g)gne(f') flam(g)mae(f) ru(g)bi(f') (,) ap(f)pa(g')ru(f)í(g_[uh:l]h)sti,(kxiji/jkj.) (;) et(j) e(j')i(i) in(h) Si(h)na(h_g) le(h)gem(i') de(g)dí(hg)sti :(e.) (:) ve(eh)ni(g'_[oh:h]) (,) ad(h) red(hg)i(f)mén(gf~)dum(ef~) nos(f.) (,) in(ec~) brá(e)chi(f)o(h_g) ex(e)tén(gg)to.(f.)}

  % \newpage

  % \begin{center}
  %   \textcolor{red}{\large{Le 19 décembre}}
  % \end{center}

  % \gresetinitiallines{1}
  % \greillumination{\initfamily\fontsize{11mm}{11mm}\selectfont O}
  % \gregorioscore{antiennes/an--o_radix_jesse--solesmes_1961}
  % \begin{center}
  %   \footnotesize{
  %     \textit{
  %       O Racine de Jessé, qui êtes comme l’étendard des peuples, (Isaïe XI, 10 et Romains XV, 12) devant qui les rois fermeront leur bouche, (Isaïe LII, 15) et dont les Nations imploreront le secours : venez nous délivrer, maintenant ne tardez plus. (Habacuc II, 3 et Hebreux X, 37.) 
  %     }
  %   }
  % \end{center}
  % \medskip

  % \gresetinitiallines{0}
  % \gregorioscore{magnificat/magnificat-IIDsolemn}

  % \begin{enumerate}[label=\textcolor{red}{\arabic*}]
  %   \setcounter{enumi}{2}
  %   \item Quia respéxit humilitátem \textit{an}\textit{cíl}\textit{læ} \textbf{su}æ:\textcolor{red}{~*} ecce enim ex hoc beátam me dicent omnes genera\textit{ti}\textbf{ó}nes.

  %   \item Quia fecit mihi \textit{ma}\textit{gna} \textit{qui} \textbf{pot}ens est:\textcolor{red}{~*} et sanctum no\textit{men} \textbf{e}jus.
    
  %   \item Et misericórdia ejus a progéni\textit{e} \textit{in} \textit{pro}\textbf{gé}nies\textcolor{red}{~*} timénti\textit{bus} \textbf{e}um.
    
  %   \item Fecit poténtiam in \textit{brá}\textit{chi}\textit{o} \textbf{su}o:\textcolor{red}{~*} dispérsit supérbos mente cor\textit{dis} \textbf{su}i.
    
  %   \item Depósuit pot\textit{én}\textit{tes} \textit{de} \textbf{se}de,\textcolor{red}{~*} et exaltá\textit{vit} \textbf{hú}miles.
    
  %   \item Esuriéntes \textit{im}\textit{plé}\textit{vit} \textbf{bo}nis:\textcolor{red}{~*} et dívites dimísit \textit{in}\textbf{á}nes.
    
  %   \item Suscépit Israël \textit{pú}\textit{e}\textit{rum} \textbf{su}um,\textcolor{red}{~*} recordátus misericórdi\textit{æ} \textbf{su}æ.
    
  %   \item Sicut locútus est \textit{ad} \textit{pa}\textit{tres} \textbf{nos}tros,\textcolor{red}{~*} Abraham et sémini ejus \textit{in} \textbf{sǽ}cula.
    
  %   \item Glória \textit{Pa}\textit{tri}, \textit{et} \textbf{Fí}lio,\textcolor{red}{~*} et Spirítu\textit{i} \textbf{Sanc}to.
    
  %   \item Sicut erat in princípio, \textit{et} \textit{nunc}, \textit{et} \textbf{sem}per,\textcolor{red}{~*} et in sǽcula sæculó\textit{rum}. \textbf{A}men.
  % \end{enumerate}

  % \grecommentary{\textit{Reprise de l'Antienne.}}
  % \gabcsnippet{(f3) O(ehhg) Ra(h_f)dix(gf) Jes(ef)se,(f'_) (,) qui(f) stas(fg) in(f) si(g)gnum(f) po(g_[uh:l]h)pu(fg)ló(f)rum,(e.) (;) su(f)per(cf~) quem(f) con(f)ti(f)né(fg)bunt(f) re(g')ges(f) os(g_[uh:l]h) su(kxiji)um,(jkj.) (;) quem(j_i~) gen(j)tes(i) de(h)pre(hg)ca(f!gwh)bún(h)tur :(e.) (:) ve(eh)ni(g'_[oh:h]) (,) ad(h) li(hg)be(f)rán(gf~)dum(ef~) nos,(fv_EC.) (;) jam(e_[uh:l]f) no(h_g)li(e) tar(gh)dá(f.)re.(f.) (::)}

  % \newpage

  % \begin{center}
  %   \textcolor{red}{\large{Le 20 décembre}}
  % \end{center}

  % \gresetinitiallines{1}
  % \greillumination{\initfamily\fontsize{11mm}{11mm}\selectfont O}
  % \gregorioscore{antiennes/an--o_clavis_david--solesmes_1961}
  % \begin{center}
  %   \footnotesize{
  %     \textit{
  %       O Emmanuel (Isaïe VII, 14 et VIII, 8), notre Roi et notre Législateur (Isaïe XXXIII, 22), Attente des Nations (Genèse XLIX, 10) et leur Sauveur : venez nous sauver, Seigneur notre Dieu.  
  %     }
  %   }
  % \end{center}
  % \medskip

  % \gresetinitiallines{0}
  % \gregorioscore{magnificat/magnificat-IIDsolemn}

  % \begin{enumerate}[label=\textcolor{red}{\arabic*}]
  %   \setcounter{enumi}{2}
  %   \item Quia respéxit humilitátem \textit{an}\textit{cíl}\textit{læ} \textbf{su}æ:\textcolor{red}{~*} ecce enim ex hoc beátam me dicent omnes genera\textit{ti}\textbf{ó}nes.

  %   \item Quia fecit mihi \textit{ma}\textit{gna} \textit{qui} \textbf{pot}ens est:\textcolor{red}{~*} et sanctum no\textit{men} \textbf{e}jus.
    
  %   \item Et misericórdia ejus a progéni\textit{e} \textit{in} \textit{pro}\textbf{gé}nies\textcolor{red}{~*} timénti\textit{bus} \textbf{e}um.
    
  %   \item Fecit poténtiam in \textit{brá}\textit{chi}\textit{o} \textbf{su}o:\textcolor{red}{~*} dispérsit supérbos mente cor\textit{dis} \textbf{su}i.
    
  %   \item Depósuit pot\textit{én}\textit{tes} \textit{de} \textbf{se}de,\textcolor{red}{~*} et exaltá\textit{vit} \textbf{hú}miles.
    
  %   \item Esuriéntes \textit{im}\textit{plé}\textit{vit} \textbf{bo}nis:\textcolor{red}{~*} et dívites dimísit \textit{in}\textbf{á}nes.
    
  %   \item Suscépit Israël \textit{pú}\textit{e}\textit{rum} \textbf{su}um,\textcolor{red}{~*} recordátus misericórdi\textit{æ} \textbf{su}æ.
    
  %   \item Sicut locútus est \textit{ad} \textit{pa}\textit{tres} \textbf{nos}tros,\textcolor{red}{~*} Abraham et sémini ejus \textit{in} \textbf{sǽ}cula.
    
  %   \item Glória \textit{Pa}\textit{tri}, \textit{et} \textbf{Fí}lio,\textcolor{red}{~*} et Spirítu\textit{i} \textbf{Sanc}to.
    
  %   \item Sicut erat in princípio, \textit{et} \textit{nunc}, \textit{et} \textbf{sem}per,\textcolor{red}{~*} et in sǽcula sæculó\textit{rum}. \textbf{A}men.
  % \end{enumerate}

  % \grecommentary{\textit{Reprise de l'Antienne.}}
  % \gabcsnippet{(f3) O(ehhg) Ra(h_f)dix(gf) Jes(ef)se,(f'_) (,) qui(f) stas(fg) in(f) si(g)gnum(f) po(g_[uh:l]h)pu(fg)ló(f)rum,(e.) (;) su(f)per(cf~) quem(f) con(f)ti(f)né(fg)bunt(f) re(g')ges(f) os(g_[uh:l]h) su(kxiji)um,(jkj.) (;) quem(j_i~) gen(j)tes(i) de(h)pre(hg)ca(f!gwh)bún(h)tur :(e.) (:) ve(eh)ni(g'_[oh:h]) (,) ad(h) li(hg)be(f)rán(gf~)dum(ef~) nos,(fv_EC.) (;) jam(e_[uh:l]f) no(h_g)li(e) tar(gh)dá(f.)re.(f.) (::)}

  % \newpage

  % \begin{center}
  %   \textcolor{red}{\large{Le 21 décembre}}
  % \end{center}

  % \gresetinitiallines{1}
  % \greillumination{\initfamily\fontsize{11mm}{11mm}\selectfont O}
  % \gregorioscore{antiennes/an--o_oriens--solesmes_1961}
  % \begin{center}
  %   \footnotesize{
  %     \textit{
  %       O Orient (Zacharie VI, 12), splendeur de la lumière éternelle (Hébreux I, 3), et soleil de justice (Malachie IV, 2) : venez et éclairez ceux qui sont assis dans les ténèbres et dans l’ombre de la mort (Isaïe IX, 2 et Luc I, 78-79). 
  %     }
  %   }
  % \end{center}
  % \medskip

  % \gresetinitiallines{0}
  % \gregorioscore{magnificat/magnificat-IIDsolemn}

  % \begin{enumerate}[label=\textcolor{red}{\arabic*}]
  %   \setcounter{enumi}{2}
  %   \item Quia respéxit humilitátem \textit{an}\textit{cíl}\textit{læ} \textbf{su}æ:\textcolor{red}{~*} ecce enim ex hoc beátam me dicent omnes genera\textit{ti}\textbf{ó}nes.

  %   \item Quia fecit mihi \textit{ma}\textit{gna} \textit{qui} \textbf{pot}ens est:\textcolor{red}{~*} et sanctum no\textit{men} \textbf{e}jus.
    
  %   \item Et misericórdia ejus a progéni\textit{e} \textit{in} \textit{pro}\textbf{gé}nies\textcolor{red}{~*} timénti\textit{bus} \textbf{e}um.
    
  %   \item Fecit poténtiam in \textit{brá}\textit{chi}\textit{o} \textbf{su}o:\textcolor{red}{~*} dispérsit supérbos mente cor\textit{dis} \textbf{su}i.
    
  %   \item Depósuit pot\textit{én}\textit{tes} \textit{de} \textbf{se}de,\textcolor{red}{~*} et exaltá\textit{vit} \textbf{hú}miles.
    
  %   \item Esuriéntes \textit{im}\textit{plé}\textit{vit} \textbf{bo}nis:\textcolor{red}{~*} et dívites dimísit \textit{in}\textbf{á}nes.
    
  %   \item Suscépit Israël \textit{pú}\textit{e}\textit{rum} \textbf{su}um,\textcolor{red}{~*} recordátus misericórdi\textit{æ} \textbf{su}æ.
    
  %   \item Sicut locútus est \textit{ad} \textit{pa}\textit{tres} \textbf{nos}tros,\textcolor{red}{~*} Abraham et sémini ejus \textit{in} \textbf{sǽ}cula.
    
  %   \item Glória \textit{Pa}\textit{tri}, \textit{et} \textbf{Fí}lio,\textcolor{red}{~*} et Spirítu\textit{i} \textbf{Sanc}to.
    
  %   \item Sicut erat in princípio, \textit{et} \textit{nunc}, \textit{et} \textbf{sem}per,\textcolor{red}{~*} et in sǽcula sæculó\textit{rum}. \textbf{A}men.
  % \end{enumerate}

  % \grecommentary[10px]{\textit{Reprise de l'Antienne.}}
  % \gabcsnippet{(f3) O(ehhg) O(h_f/gf)ri(ef)ens,(f.) <c>*</c>(,) splen(fg)dor(f) lu(f)cis(g_[uh:l]h) ae(fg)tér(f)nae,(e.) (;) et(f) sol(cf) ju(f)stí(g')ti(h)ae :(kxiji/jkj.) (:) ve(j)ni,(i) (,) et(hg) il(fh~)lú(h)mi(f_gF'E)na(e.) (;) se(h')dén(g)tes(h'_) in(f) té(gf)ne(ef)bris(fv_EC.) (;) et(e_[uh:l]f) um(h_g~)bra(e) mor(gg)tis.(f.) (::)}

  % \newpage

  % \begin{center}
  %   \textcolor{red}{\large{Le 22 décembre}}
  % \end{center}

  % \gresetinitiallines{1}
  % \greillumination{\initfamily\fontsize{11mm}{11mm}\selectfont O}
  % \gregorioscore{antiennes/an--o_rex_gentium--solesmes_1961}
  % \begin{center}
  %   \footnotesize{
  %     \textit{
  %       O Roi des Nations, et objet de leurs désirs (Aggée II, 8), Pierre angulaire (Isaïe XXVIII, 16), qui réunissez en vous les deux peuples (Ephésiens II, 14) : venez et sauvez l’homme, que vous avez formé du limon (Genèse II, 7).  
  %     }
  %   }
  % \end{center}
  % \medskip

  % \gresetinitiallines{0}
  % \gregorioscore{magnificat/magnificat-IIDsolemn}

  % \begin{enumerate}[label=\textcolor{red}{\arabic*}]
  %   \setcounter{enumi}{2}
  %   \item Quia respéxit humilitátem \textit{an}\textit{cíl}\textit{læ} \textbf{su}æ:\textcolor{red}{~*} ecce enim ex hoc beátam me dicent omnes genera\textit{ti}\textbf{ó}nes.

  %   \item Quia fecit mihi \textit{ma}\textit{gna} \textit{qui} \textbf{pot}ens est:\textcolor{red}{~*} et sanctum no\textit{men} \textbf{e}jus.
    
  %   \item Et misericórdia ejus a progéni\textit{e} \textit{in} \textit{pro}\textbf{gé}nies\textcolor{red}{~*} timénti\textit{bus} \textbf{e}um.
    
  %   \item Fecit poténtiam in \textit{brá}\textit{chi}\textit{o} \textbf{su}o:\textcolor{red}{~*} dispérsit supérbos mente cor\textit{dis} \textbf{su}i.
    
  %   \item Depósuit pot\textit{én}\textit{tes} \textit{de} \textbf{se}de,\textcolor{red}{~*} et exaltá\textit{vit} \textbf{hú}miles.
    
  %   \item Esuriéntes \textit{im}\textit{plé}\textit{vit} \textbf{bo}nis:\textcolor{red}{~*} et dívites dimísit \textit{in}\textbf{á}nes.
    
  %   \item Suscépit Israël \textit{pú}\textit{e}\textit{rum} \textbf{su}um,\textcolor{red}{~*} recordátus misericórdi\textit{æ} \textbf{su}æ.
    
  %   \item Sicut locútus est \textit{ad} \textit{pa}\textit{tres} \textbf{nos}tros,\textcolor{red}{~*} Abraham et sémini ejus \textit{in} \textbf{sǽ}cula.
    
  %   \item Glória \textit{Pa}\textit{tri}, \textit{et} \textbf{Fí}lio,\textcolor{red}{~*} et Spirítu\textit{i} \textbf{Sanc}to.
    
  %   \item Sicut erat in princípio, \textit{et} \textit{nunc}, \textit{et} \textbf{sem}per,\textcolor{red}{~*} et in sǽcula sæculó\textit{rum}. \textbf{A}men.
  % \end{enumerate}

  % \grecommentary[10px]{\textit{Reprise de l'Antienne.}}
  % \gabcsnippet{(f3) O(ehhg) Rex(h_f) gén(gf~)ti(ef)um,(f'_) (,) et(f) de(f)si(f)de(f)rá(f)tus(g_[uh:l]h) e(fg)á(f)rum,(e.) (;) la(f)pís(cf)que(f) an(g')gu(f)lá(g_[uh:l]h)ris,(kxiji/jkj.) (;) qui(j) fa(j')cis(i) ú(h)tra(hg)que(f!gwh) u(h)num :(e.) (:) ve(eh)ni,(g'_[oh:h]) (,) et(h) sal(hg~)va(f) hó(gf)mi(ef)nem,(fv_EC.) (;) quem(e) de(f) li(h')mo(g) for(e)má(gg)sti.(f.) (::) }

  % \newpage

  % \begin{center}
  %   \textcolor{red}{\large{Le 23 décembre}}
  % \end{center}

  % \gresetinitiallines{1}
  % \greillumination{\initfamily\fontsize{11mm}{11mm}\selectfont O}
  % \gregorioscore{antiennes/an--o_emmanuel--solesmes_1961}
  % \begin{center}
  %   \footnotesize{
  %     \textit{
  %       O Emmanuel (Isaïe VII, 14 et VIII, 8), notre Roi et notre Législateur (Isaïe XXXIII, 22), Attente des Nations (Genèse XLIX, 10) et leur Sauveur : venez nous sauver, Seigneur notre Dieu.  
  %     }
  %   }
  % \end{center}
  % \medskip

  % \gresetinitiallines{0}
  % \gregorioscore{magnificat/magnificat-IIDsolemn}

  % \begin{enumerate}[label=\textcolor{red}{\arabic*}]
  %   \setcounter{enumi}{2}
  %   \item Quia respéxit humilitátem \textit{an}\textit{cíl}\textit{læ} \textbf{su}æ:\textcolor{red}{~*} ecce enim ex hoc beátam me dicent omnes genera\textit{ti}\textbf{ó}nes.

  %   \item Quia fecit mihi \textit{ma}\textit{gna} \textit{qui} \textbf{pot}ens est:\textcolor{red}{~*} et sanctum no\textit{men} \textbf{e}jus.
    
  %   \item Et misericórdia ejus a progéni\textit{e} \textit{in} \textit{pro}\textbf{gé}nies\textcolor{red}{~*} timénti\textit{bus} \textbf{e}um.
    
  %   \item Fecit poténtiam in \textit{brá}\textit{chi}\textit{o} \textbf{su}o:\textcolor{red}{~*} dispérsit supérbos mente cor\textit{dis} \textbf{su}i.
    
  %   \item Depósuit pot\textit{én}\textit{tes} \textit{de} \textbf{se}de,\textcolor{red}{~*} et exaltá\textit{vit} \textbf{hú}miles.
    
  %   \item Esuriéntes \textit{im}\textit{plé}\textit{vit} \textbf{bo}nis:\textcolor{red}{~*} et dívites dimísit \textit{in}\textbf{á}nes.
    
  %   \item Suscépit Israël \textit{pú}\textit{e}\textit{rum} \textbf{su}um,\textcolor{red}{~*} recordátus misericórdi\textit{æ} \textbf{su}æ.
    
  %   \item Sicut locútus est \textit{ad} \textit{pa}\textit{tres} \textbf{nos}tros,\textcolor{red}{~*} Abraham et sémini ejus \textit{in} \textbf{sǽ}cula.
    
  %   \item Glória \textit{Pa}\textit{tri}, \textit{et} \textbf{Fí}lio,\textcolor{red}{~*} et Spirítu\textit{i} \textbf{Sanc}to.
    
  %   \item Sicut erat in princípio, \textit{et} \textit{nunc}, \textit{et} \textbf{sem}per,\textcolor{red}{~*} et in sǽcula sæculó\textit{rum}. \textbf{A}men.
  % \end{enumerate}

  % \grecommentary[10px]{\textit{Reprise de l'Antienne.}}
  % \gabcsnippet{(f3) O(ehhg) Em(h_f)má(gf)nu(ef)el,(f'_) (,) Rex(g) et(f) lé(f)gi(g_[uh:l]h)fer(fg) no(f)ster,(e.) (;) ex(f)spe(cf)ctá(fg)ti(f)o(f) gén(g')ti(h)um,(kxiji/jkj.) (;) et(j') Sal(i)vá(h)tor(hg) e(f!gwh)á(h)rum :(e.) (:) ve(eh)ni(g'_[oh:h]) (,) ad(h) sal(f)ván(gf~)dum(ef~) nos(f.) (,) Dó(ec)mi(e)ne(f) De(h_g)us(e) no(gg)ster.(f.) (::)}

  % \newpage
  % \subfile{salut.tex}

  % \vspace*{\fill}
  % \begin{center}
  %   \normalsize\textit{
  %     Paroisse Saint Roch, 296 rue Saint-Honoré, 75001 Paris
  %   }
  % \end{center}
\end{document}