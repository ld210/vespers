% !TeX program = lualatex
\documentclass[12pt, a4paper]{article}
\usepackage{fullpage}
\usepackage{subfiles}
\usepackage{fontspec}
\usepackage{libertine}
\usepackage{xcolor}
\usepackage{GotIn}
\usepackage{geometry}
\usepackage{multicol}
\usepackage{multicolrule}
\usepackage{graphicx}
\usepackage{enumitem}
\usepackage[autocompile]{gregoriotex}
\usepackage[latin,french]{babel}


\geometry{top=2cm, bottom=2cm}
% \pagestyle{empty}

\definecolor{red}{HTML}{C70039}
% \input GoudyIn.fd
% \newcommand*\initfamily{\usefont{U}{GoudyIn}{xl}{n}}

\input Acorn.fd
\newcommand*\initfamily{\usefont{U}{Acorn}{xl}{n}}
% cette ligne ajoute de l'espace entre les portées
% \grechangedim{baselineskip}{60pt}{scalable}

\begin{document}
  \gresetlinecolor{gregoriocolor}

  \begin{titlepage}\centering
    \vspace*{\fill}\
    \huge Secondes Vêpres\\
    \smallskip
    \begin{Large}
      \textit{
        des Dimanches après la Pentecôte\\
      }
    \end{Large}
    \medskip
    \large et\\
    \medskip
    \LARGE Salut du Saint-Sacrement\\
    \bigskip
    % \begin{figure}[h!]
    %   \centering
    %   \includegraphics[width=7cm]{../epiphanie-septuagesime-careme/logo.png}
    % \end{figure}
    \vspace*{\fill}
    \begin{figure}[h!]
      \centering
      \includegraphics[width=7cm]{../epiphanie-septuagesime-careme/logo.png}
    \end{figure}
    \centering \normalsize Paroisse Saint Roch\\
    \bigskip
    \begin{Large}
      \centering Ne pas emporter
    \end{Large}
  \end{titlepage}

  \newpage
  \vspace*{\fill}
  \begin{center}
    \normalsize\textit{
      Livret latin-français
    }
  \end{center}
  \newpage

  \begin{center}
    \textcolor{red}{\large{Ouverture.}}
  \end{center}

  % greillumination: remplace la première lettre, ici par une font ornementale
  \greillumination{\initfamily\fontsize{11mm}{11mm}\selectfont D}
  \gregorioscore{../epiphanie-septuagesime-careme/vepres-deus_in_adjutorium}

  \begin{center}
    \small{
    \emph{
      Dieu, venez à mon aide ; Seigneur, hatez-vous de me secourir.\\
      Gloire au Père, au Fils et au Saint Esprit, comme il était au commencement, maintenant et toujour et dans les siècles des siècles. Allelúia\\
    }
  }
  \end{center}

  \vspace*{\fill}\
  \begin{center}
    \greseparator{3}{30}
  \end{center}
  \vspace*{\fill}\

  \newpage
  \normalsize

  % ===== DEBUT Antienne =========
  \greillumination{\initfamily\fontsize{11mm}{11mm}\selectfont D}
  \gregorioscore{antiennes/dixit-dominus-domino-meo}
  \begin{center}
    \footnotesize{
      \textit{
        L'Éternel a dit à mon Seigneur: Assieds-toi à ma droite.
      }
    }
  \end{center}
  % \medskip
  % ===== FIN Antienne ===========

  % ===== DEBUT psaume ===========
  % gresetinitiallines : avec le parametre à 0, supprime l'ornement
  \gresetinitiallines{0}

  \begin{center}
    \normalsize{Psaume 109.}\\
    \footnotesize{
      \emph{Génération éternelle du Christ, Prêtre, Roi et Juge.}
    }
  \end{center}

  \gregorioscore{psaumes/psaume109-VIIC2}
  \begin{enumerate}[label=\textcolor{red}{\emph{\arabic*}}]
    \setcounter{enumi}{2}
    \item Virgam virtútis tuæ emíttet Dómi\textbf{nus} ex \textbf{Si}on:\textcolor{red}{~*} domináre in médio inimi\textbf{có}rum tu\textbf{ó}rum.

    \item Tecum princípium in die virtútis tuæ in splendóri\textbf{bus} sanc\textbf{tó}rum:\textcolor{red}{~*}\\ \-\hspace{2cm}  ex útero ante lucíferum \textbf{gé}nu\textbf{i} te.

    \item Jurávit Dóminus, et non pœni\textbf{té}bit \textbf{e}um:\textcolor{red}{~*}\\ \-\hspace{2cm}  Tu es sacérdos in ætérnum secúndum órdi\textbf{nem} Mel\textbf{chí}sedech.

    \item Dóminus a \textbf{dex}tris \textbf{tu}is,\textcolor{red}{~*} confrégit in die iræ \textbf{su}æ \textbf{re}ges.

    \item Judicábit in natiónibus, im\textbf{plé}bit ru\textbf{í}nas:\textcolor{red}{~*} 
    conquassábit cápita in \textbf{ter}ra mul\textbf{tó}rum.

    \item De torrénte in \textbf{vi}a \textbf{bi}bet:\textcolor{red}{~*} proptérea exal\textbf{tá}bit \textbf{ca}put.

    \item Glória \textbf{Pa}tri, et \textbf{Fí}lio,\textcolor{red}{~*} et Spi\textbf{rí}tui \textbf{Sanc}to.

    \item Sicut erat in princípio, et \textbf{nunc}, et \textbf{sem}per,\textcolor{red}{~*} et in sǽcula sæcu\textbf{ló}rum. \textbf{A}men.
  \end{enumerate}
  \medskip
  \grecommentary{\textit{Reprise de l'Antienne.}}
  \gabcsnippet{(c3) Di(g)xit(h) Dó(i')mi(j)nus(i.) Dó(i_h)mi(f)no(h) me(g_f)o(e.) :() (;) Se(f)de(f_) a(d) dex(fh)tris(g) me(e.)is.(e.) (::)}

  \newpage
  \vspace*{\fill}\
  \begin{normalsize}
    \begin{center}
      \par \textit{L'Éternel a dit à mon Seigneur: Assieds-toi à ma droite.}
      \medskip
      \begin{enumerate}[label=\textcolor{red}{\emph{\arabic*}}]
        \item \textit{Jusqu'à ce que je mette tes ennemis pour le marchepied de tes pieds.}
        \item \textit{L'Éternel enverra de Sion la verge de ta force: Domine au milieu de tes ennemis!}
        \item \textit{Ton peuple sera un peuple de franche volonté, au jour de ta puissance, en sainte magnificence. Du sein de l'aurore te viendra la rosée de ta jeunesse.}
        \item \textit{L'Éternel a juré, et il ne se repentira point: Tu es sacrificateur pour toujours, selon l'ordre de Melchisédec.}
        \item \textit{Le Seigneur, à ta droite, brisera les rois au jour de sa colère.}
        \item \textit{Il jugera parmi les nations, il remplira tout de corps morts, il brisera le chef d'un grand pays.}
        \item \textit{Il boira du torrent dans le chemin, c'est pourquoi il lèvera haut la tête.}
        \item \textit{Gloire au Père, au Fils, et au Saint Esprit, }
        \item \textit{Comme il était au commencement, maintenant et toujours, et dans les siècles des siècles. Amen. }
      \end{enumerate}
    \end{center}
  \end{normalsize}
  \vspace*{\fill}\
  \newpage

  % ===== DEBUT Antienne =========
  \gresetinitiallines{1}
  \greillumination{\initfamily\fontsize{11mm}{11mm}\selectfont M}
  \gregorioscore{antiennes/an--magna_opera_domini--solesmes}
  \begin{center}
    \footnotesize{
      \textit{
        Grandes sont les œuvres du Seigneur ; tous ceux qui les aiment s'en instruisent.
      }
    }
  \end{center}
  % \medskip
  % ===== FIN Antienne ===========

  % ===== DEBUT psaume ===========
  % gresetinitiallines : avec le parametre à 0, supprime l'ornement
  \gresetinitiallines{0}

  \begin{center}
    \normalsize{Psaume 110.}\\
    \footnotesize{
      \emph{Bienfaits accordés par Dieu à son peuple.}
    }
  \end{center}

  \gregorioscore{psaumes/psaume110-IIIb}

  \begin{enumerate}[label=\textcolor{red}{\emph{\arabic*}}]
    \setcounter{enumi}{1}
    \item Magna \textbf{ó}pera \textbf{Dó}\textbf{mi}ni:\textcolor{red}{~*} exquisíta in omnes voluntá\textit{tes} \textbf{e}jus.

    \item Conféssio et magnificéntia \textbf{o}pus \textbf{e}jus:\textcolor{red}{~*} et justítia ejus manet in sǽcu\textit{lum} \textbf{sǽ}culi.

    \item Memóriam fecit mirabílium suórum,\textcolor{red}{~†} miséricors et mise\textbf{rá}tor \textbf{Dó}\textbf{mi}nus:\textcolor{red}{~*}\\ \-\hspace{2cm} escam dedit timén\textit{ti}\textbf{bus} se.

    \item Memor erit in sǽculum testa\textbf{mén}ti \textbf{su}i:\textcolor{red}{~*} virtútem óperum suórum annuntiábit pópu\textit{lo} \textbf{su}o:

    \item Ut det illis heredi\textbf{tá}tem \textbf{gén}\textbf{ti}um:\textcolor{red}{~*} ópera mánuum ejus véritas, et \textit{ju}\textbf{dí}cium.

    \item Fidélia ómnia mandáta ejus:\textcolor{red}{~†} confirmáta in \textbf{sǽ}culum \textbf{sǽ}\textbf{cu}li,\textcolor{red}{~*}\\ \-\hspace{2cm} facta in veritáte et æ\textit{qui}\textbf{tá}te.

    \item Redemptiónem misit \textbf{pó}pulo \textbf{su}o:\textcolor{red}{~*} mandávit in ætérnum testamén\textit{tum} \textbf{su}um.

    \item Sanctum, et terríbile \textbf{no}men \textbf{e}jus:\textcolor{red}{~*} inítium sapiéntiæ ti\textit{mor} \textbf{Dó}mini.

    \item Intelléctus bonus ómnibus faci\textbf{én}tibus \textbf{e}um:\textcolor{red}{~*}\\ \-\hspace{2cm} laudátio ejus manet in sǽcu\textit{lum} \textbf{sǽ}culi.

    \item Glória \textbf{Pa}tri, et \textbf{Fí}\textbf{li}o,\textcolor{red}{~*} et Spirítu\textit{i} \textbf{Sanc}to.

    \item Sicut erat in princípio, et \textbf{nunc}, et \textbf{sem}per,\textcolor{red}{~*} et in sǽcula sæculó\textit{rum}. \textbf{A}men.
  \end{enumerate}

  \medskip
  \grecommentary{\textit{Reprise de l'Antienne.}}
  \gabcsnippet{(c4) Ma(g_j)gna(ji) ó(h')pe(g)ra(h') Dó(i)mi(hg)ni,(g.) (;) ex(h)qui(g)sí(e_0!fg)ta(g') in(e) o(f')mnes(d) vo(e')lun(f)tá(g_h)tes(gf) e(e.)jus.(e.) (::)}

  \newpage
  \vspace*{\fill}\
  \begin{normalsize}
    \begin{center}
      \par \textit{Grandes sont les œuvres du Seigneur ; tous ceux qui les aiment s'en instruisent.}
      \medskip
      \begin{enumerate}[label=\textcolor{red}{\emph{\arabic*}}]
        \item \textit{De tout cœur je rendrai grâce au Seigneur dans l'assemblée, parmi les justes.}
        \item \textit{Grandes sont les œuvres du Seigneur ; tous ceux qui les aiment s'en instruisent.}
        \item \textit{Noblesse et beauté dans ses actions : à jamais se maintiendra sa justice.}
        \item \textit{De ses merveilles il a laissé un mémorial ; le Seigneur est tendresse et pitié, il a donné des vivres à ses fidèles,}
        \item \textit{Gardant toujours mémoire de son
        alliance, il a montré sa force à son peuple.}
        \item \textit{Lui donnant le domaine des nations. Justesse et sûreté les œuvres de ses mains.}
        \item \textit{Sécurité, toutes ses lois, établies pour toujours et à jamais, accomplies avec droiture et sûreté ! }
        \item \textit{Il apporte la délivrance à son peuple ; son alliance est promulguée pour toujours.}
        \item \textit{Saint, redoutable est son nom, la sagesse commence avec la crainte du Seigneur.}
        \item \textit{Qui accomplit sa volonté en est éclairé. A jamais se maintiendra sa louange.}
        \item \textit{Gloire au Père, au Fils, et au Saint Esprit, }
        \item \textit{Comme il était au commencement, maintenant et toujours, et dans les siècles des siècles. Amen. }
      \end{enumerate}
    \end{center}
  \end{normalsize}
  \vspace*{\fill}\
  \newpage

  % ===== DEBUT Antienne =========
  \gresetinitiallines{1}
  \greillumination{\initfamily\fontsize{11mm}{11mm}\selectfont Q}
  \gregorioscore{antiennes/an--qui_timet_dominum--solesmes}
  \begin{center}
    \footnotesize{
      \textit{
        Celui qui craint le Seigneur a une volonté ardente d’accomplir ses commandements.
      }
    }
  \end{center}
  % \medskip
  % ===== FIN Antienne ===========

  % ===== DEBUT psaume ===========
  % gresetinitiallines : avec le parametre à 0, supprime l'ornement

  \gresetinitiallines{0}

  \begin{center}
    \normalsize{Psaume 111.}\\
    \footnotesize{
      \emph{Portrait du juste, et tableau de son bonheur.}
    }
  \end{center}
  % \smallskip
  \grechangedim{baselineskip}{50pt}{scalable}
  \gregorioscore{psaumes/psaume111-IVG}
  \begin{enumerate}[label=\textcolor{red}{\emph{\arabic*}}]
    \setcounter{enumi}{1}
    \item Potens in terra erit \textit{se}\textit{men} \textbf{e}jus:\textcolor{red}{~*}  generátio rectórum benedi\textbf{cé}tur.

    \item Glória, et divítiæ in \textit{do}\textit{mo} \textbf{e}jus:\textcolor{red}{~*}  et justítia ejus manet in sǽculum \textbf{sǽ}culi.

    \item Exórtum est in ténebris \textit{lu}\textit{men} \textbf{rec}tis:\textcolor{red}{~*}  miséricors, et miserátor, et \textbf{jus}tus.
    
    \item Jucúndus homo qui miserétur et cómmodat,\textcolor{red}{~†} dispónet sermónes suos \textit{in} \textit{ju}\textbf{dí}cio:\textcolor{red}{~*} \\ \-\hspace{2cm}  quia in ætérnum non commo\textbf{vé}bitur.

    \item In memória ætérna \textit{e}\textit{rit} \textbf{jus}tus:\textcolor{red}{~*}  ab auditióne mala non ti\textbf{mé}bit.

    \item Parátum cor ejus speráre in Dómino,\textcolor{red}{~†} confirmátum \textit{est} \textit{cor} \textbf{e}jus:\textcolor{red}{~*} \\ \-\hspace{2cm} non commovébitur donec despíciat inimícos \textbf{su}os.

    \item Dispérsit, dedit paupéribus:\textcolor{red}{~†} justítia ejus manet in sǽ\textit{cu}\textit{lum} \textbf{sǽ}culi,\textcolor{red}{~*} \\ \-\hspace{2cm} cornu ejus exaltábitur in \textbf{gló}ria.

    \item Peccátor vidébit, et irascétur,\textcolor{red}{~†} déntibus suis fremet \textit{et} \textit{ta}\textbf{bé}scet:\textcolor{red}{~*} \\ \-\hspace{2cm} desidérium peccatórum per\textbf{í}bit.

    \item Glória Pa\textit{tri}, \textit{et} \textbf{Fí}lio,\textcolor{red}{~*}  et Spirítui \textbf{Sanc}to.

    \item Sicut erat in princípio, et \textit{nunc}, \textit{et} \textbf{sem}per,\textcolor{red}{~*}  et in sǽcula sæculórum. \textbf{A}men.
  \end{enumerate}

  \medskip
  \grecommentary{\textit{Reprise de l'Antienne.}}
  \gabcsnippet{(c4) Qui(g) ti(e')met(f) Dó(g)mi(h)num,(g'_) (,) in(g) man(gh)dá(h)tis(g') e(g)jus(e_f) cu(gf)pit(de) ni(e.)mis.(e.) (::)}

  \newpage
  \vspace*{\fill}\
  \begin{normalsize}
    \begin{center}
      \par \textit{Celui qui craint le Seigneur a une volonté ardente d’accomplir ses commandements.}
      \medskip
      \begin{enumerate}[label=\textcolor{red}{\emph{\arabic*}}]
        \item \textit{Heureux l’homme qui craint le Seigneur, qui aime entièrement sa volonté !}
        \item \textit{Sa lignée sera puissante sur la terre ; la
        race des justes est bénie.}
        \item \textit{Les richesses affluent dans sa maison : à
        jamais se maintiendra sa justice.}
        \item \textit{Lumière des cœurs droits, il s'est levé
        dans les ténèbres, l’homme de justice, de
        tendresse et de pitié.}
        \item \textit{L'homme de bien a pitié, il partage ; il
        mène ses affaires avec droiture, cet
        homme jamais ne tombera ;}
        \item \textit{Toujours on fera mémoire du juste, il ne
        craint pas l'annonce d'un malheur :}
        \item \textit{Le cœur ferme, il s'appuie sur le
        Seigneur. Son cœur est confiant, il ne
        craint pas : il verra ce que valaient ses
        oppresseurs.}
        \item \textit{A pleines mains, il donne au pauvre ; à
        jamais se maintiendra sa justice, sa
        puissance grandira, et sa gloire !}
        \item \textit{L'impie le voit et s'irrite ; il grince des
        dents et se détruit. L'ambition des impies
        se perdra.}
        \item \textit{Gloire au Père, au Fils, et au Saint Esprit, }
        \item \textit{Comme il était au commencement, maintenant et toujours, et dans les siècles des siècles. Amen. }
      \end{enumerate}
    \end{center}
  \end{normalsize}
  \vspace*{\fill}\
  \newpage

  % ===== DEBUT Antienne =========
  \gresetinitiallines{1}
  \greillumination{\initfamily\fontsize{11mm}{11mm}\selectfont S}
  \gregorioscore{antiennes/an--sit_nomen_domini--solesmes}
  \begin{center}
    \footnotesize{
      \textit{
        Que le nom du Seigneur soit béni dans tous les siècles.
      }
    }
  \end{center}
  % ===== FIN Antienne ===========

  % ===== DEBUT psaume ===========
  % gresetinitiallines : avec le parametre à 0, supprime l'ornement
  \gresetinitiallines{0}

  \begin{center}
    \normalsize{Psaume 112.}\\
    \footnotesize{
      \emph{Invitation à louer Dieu et sa Providence souveraine.}
    }
  \end{center}
  % \smallskip

  \gregorioscore{psaumes/psaume112-VIIC}

  \begin{enumerate}[label=\textcolor{red}{\emph{\arabic*}}]
    \setcounter{enumi}{1}
    \item Sit nomen Dómini \textbf{be}ne\textbf{díc}tum,\textcolor{red}{~*}  ex hoc nunc, et \textbf{us}que in \textbf{sǽ}culum.

    \item A solis ortu usque \textbf{ad} oc\textbf{cá}sum,\textcolor{red}{~*}  laudábile \textbf{no}men \textbf{Dó}mini.

    \item Excélsus super omnes \textbf{gen}tes \textbf{Dó}minus,\textcolor{red}{~*}  et super cælos \textbf{gló}ria \textbf{e}jus.

    \item Quis sicut Dóminus, Deus noster, qui in \textbf{al}tis \textbf{há}bitat,\textcolor{red}{~*} \\ \-\hspace{2cm} et humília réspicit in cælo \textbf{et} in \textbf{ter}ra?

    \item Súscitans a \textbf{ter}ra \textbf{ín}opem,\textcolor{red}{~*}  et de stércore \textbf{é}rigens \textbf{páu}perem:
    
    \item Ut cóllocet eum \textbf{cum} prin\textbf{cí}pibus,\textcolor{red}{~*}  cum princípibus \textbf{pó}puli \textbf{su}i.

    \item Qui habitáre facit stéri\textbf{lem} in \textbf{do}mo,\textcolor{red}{~*}  matrem fili\textbf{ó}rum læ\textbf{tán}tem.

    \item Glória \textbf{Pa}tri, et \textbf{Fí}lio,\textcolor{red}{~*}  et Spi\textbf{rí}tui \textbf{Sanc}to.

    \item Sicut erat in princípio, et \textbf{nunc}, et \textbf{sem}per,\textcolor{red}{~*}  et in sǽcula sæcu\textbf{ló}rum. \textbf{A}men.
  \end{enumerate}

  \medskip
  \grecommentary{\textit{Reprise de l'Antienne.}}
  \gabcsnippet{(c3) Sit(ii) no(g)men(h) Dó(ij)mi(i)ni(h.) (,) be(hg)ne(f)dí(g_[uh:l]h)ctum(ih) in(f) saé(e')cu(e)la.(e.) (::)}

  \newpage
  \vspace*{\fill}\
  \begin{normalsize}
    \begin{center}
      \par \textit{Que le nom du Seigneur soit béni dans tous les siècles.}
      \medskip
      \begin{enumerate}[label=\textcolor{red}{\emph{\arabic*}}]
        \item \textit{Louez, serviteurs du Seigneur, louez le nom du Seigneur !}
        \item \textit{Béni soit le nom du Seigneur, maintenant et
        pour les siècles des siècles !}
        \item \textit{Du levant au couchant du soleil, loué soit le
        nom du Seigneur !}
        \item \textit{Le Seigneur domine tous les peuples, sa gloire
        domine les cieux.}
        \item \textit{Qui est semblable au Seigneur notre Dieu ?
        Lui, il siège là-haut, mais il abaisse son regard vers le ciel et vers la terre.}
        \item \textit{De la poussière il relève le faible, il retire le
        pauvre de la cendre}
        \item \textit{Pour qu'il siège parmi les princes, parmi les
        princes de son peuple.}
        \item \textit{Il installe en sa maison la femme stérile,
        heureuse mère au milieu de ses fils.}
        \item \textit{Gloire au Père, au Fils, et au Saint Esprit, }
        \item \textit{Comme il était au commencement, maintenant et toujours, et dans les siècles des siècles. Amen. }
      \end{enumerate}
    \end{center}
  \end{normalsize}
  \vspace*{\fill}\
  \newpage

  % ===== DEBUT Antienne =========
  \gresetinitiallines{1}
  \greillumination{\initfamily\fontsize{11mm}{11mm}\selectfont D}
  \gregorioscore{antiennes/an--deus_autem_noster--solesmes}
  \begin{center}
    \footnotesize{
      \textit{
        Notre Dieu est dans le ciel, tout ce qu’il a voulu, il l’a fait.
      }
    }
  \end{center}
  % ===== DEBUT psaume ===========
  % gresetinitiallines : avec le parametre à 0, supprime l'ornement
  \begin{center}
    \normalsize{Psaume 113.}\\
    \footnotesize{
      \emph{Le peuple délivré d'Egypte}\\
      \emph{chante son libérateur et le proclame seul vrai Dieu.}
    }
  \end{center}

  % gresetinitiallines : avec le parametre à 0, supprime l'ornement
  \gresetinitiallines{0}
  \gregorioscore{psaumes/psaume113-tPer}

  \begin{enumerate}[label=\textcolor{red}{\arabic*}]
    \setcounter{enumi}{1}
    \item Facta est Judǽa sanctifi\textit{cá}\textit{ti}\textit{o} \textbf{e}jus,\textcolor{red}{~*}  Israël potés\textit{tas} \textbf{e}jus.

    \item Mare \textit{vi}\textit{dit}, \textit{et} \textbf{fu}git:\textcolor{red}{~*}  Jordánis convérsus est \textit{re}\textbf{trór}sum.

    \item Montes exsultavé\textit{runt} \textit{ut} \textit{a}\textbf{rí}etes,\textcolor{red}{~*}  et colles sicut a\textit{gni} \textbf{ó}vium.

    \item Quid est tibi, ma\textit{re}, \textit{quod} \textit{fu}\textbf{gís}ti:\textcolor{red}{~*}  et tu, Jordánis, quia convérsus es \textit{re}\textbf{trór}sum?

    \item Montes, exsultástis \textit{sic}\textit{ut} \textit{a}\textbf{rí}etes,\textcolor{red}{~*}  et colles, sicut a\textit{gni} \textbf{ó}vium.

    \item A fácie Dómini \textit{mo}\textit{ta} \textit{est} \textbf{ter}ra,\textcolor{red}{~*}  a fácie De\textit{i} \textbf{Ja}cob.

    \item Qui convértit petram in \textit{sta}\textit{gna} \textit{a}\textbf{quá}rum,\textcolor{red}{~*}  et rupem in fontes \textit{a}\textbf{quá}rum.

    \item Non nobis, Dó\textit{mi}\textit{ne}, \textit{non} \textbf{no}bis:\textcolor{red}{~*}  sed nómini tuo \textit{da} \textbf{gló}riam.

    \item Super misericórdia tua, et ve\textit{ri}\textit{tá}\textit{te} \textbf{tu}a:\textcolor{red}{~*} \\ \-\hspace{2cm} nequándo dicant gentes: Ubi est Deus \textit{e}\textbf{ó}rum?

    \item Deus autem \textit{nos}\textit{ter} \textit{in} \textbf{cæ}lo:\textcolor{red}{~*}  ómnia quæcúmque vólu\textit{it}, \textbf{fe}cit.

    \item Simulácra géntium ar\textit{gén}\textit{tum}, \textit{et} \textbf{au}rum,\textcolor{red}{~*}  ópera mánu\textit{um} \textbf{hó}minum.

    \item Os habent, \textit{et} \textit{non} \textit{lo}\textbf{quén}tur:\textcolor{red}{~*}  óculos habent, et non \textit{vi}\textbf{dé}bunt.

    \item Aures ha\textit{bent}, \textit{et} \textit{non} \textbf{áu}dient:\textcolor{red}{~*}  nares habent, et non o\textit{do}\textbf{rá}bunt.

    \item Manus habent, et non palpábunt:\textcolor{red}{~†} pedes habent, et \textit{non} \textit{am}\textit{bu}\textbf{lá}bunt:\textcolor{red}{~*} \\ \-\hspace{2cm} non clamábunt in gúttu\textit{re} \textbf{su}o.

    \item Símiles illis fiant qui \textit{fá}\textit{ci}\textit{unt} \textbf{e}a:\textcolor{red}{~*}  et omnes qui confídunt \textit{in} \textbf{e}is.

    \item Domus Israël spe\textit{rá}\textit{vit} \textit{in} \textbf{Dó}mino:\textcolor{red}{~*}  adjútor eórum et protéctor \textit{e}\textbf{ó}rum est,

    \item Domus Aaron spe\textit{rá}\textit{vit} \textit{in} \textbf{Dó}mino:\textcolor{red}{~*}  adjútor eórum et protéctor \textit{e}\textbf{ó}rum est,

    \item Qui timent Dóminum, spera\textit{vé}\textit{runt} \textit{in} \textbf{Dó}mino:\textcolor{red}{~*} \\ \-\hspace{2cm} adjútor eórum et protéctor \textit{e}\textbf{ó}rum est.

    \item Dóminus me\textit{mor} \textit{fu}\textit{it} \textbf{nos}tri:\textcolor{red}{~*}  et benedí\textit{xit} \textbf{no}bis:

    \item Benedíxit \textit{dó}\textit{mu}\textit{i} \textbf{Is}raël:\textcolor{red}{~*}  benedíxit dómu\textit{i} \textbf{A}aron.

    \item Benedíxit ómnibus, \textit{qui} \textit{ti}\textit{ment} \textbf{Dó}minum,\textcolor{red}{~*}  pusíllis cum \textit{ma}\textbf{jó}ribus.

    \item Adjíciat \textit{Dó}\textit{mi}\textit{nus} \textbf{su}per vos:\textcolor{red}{~*}  super vos, et super fíli\textit{os} \textbf{ves}tros.
  \end{enumerate}

  \newpage
  \vspace*{\fill}\
  \begin{normalsize}
    \begin{center}
      \par \textit{Notre Dieu est dans le ciel, tout ce qu’il a voulu, il l’a fait.}
      \medskip
      \begin{enumerate}[label=\textcolor{red}{\emph{\arabic*}}]
        \item \textit{Quand Israël sortit d'Égypte, et Jacob, de chez un peuple étranger,}
        \item \textit{Juda fut pour Dieu un sanctuaire, Israël devint son domaine.}
        \item \textit{La mer voit et s'enfuit, le Jourdain retourne en arrière.}
        \item \textit{Comme des béliers, bondissent les montagnes, et les collines, comme des agneaux.}
        \item \textit{Qu'as-tu, mer, à t'enfuir, Jourdain, à retourner en arrière ?}
        \item \textit{Montagnes, pourquoi bondir comme des béliers, collines, comme des agneaux ?}
        \item \textit{Tremble, terre, devant le Maître, devant la face du Dieu de Jacob,}
        \item \textit{Lui qui change le rocher en source et la pierre en fontaine d’eau vive.}
        \item \textit{Non pas à nous, Seigneur, non pas à nous, mais à ton nom donne la gloire.}
        \item \textit{Pour ton amour et ta vérité.}
        \item \textit{Pourquoi les païens diraient-ils : « Où donc est leur Dieu ? »}
        \item \textit{Notre Dieu, il est au ciel ; tout ce qu'il veut, il le fait.}
        \item \textit{Leurs idoles : or et argent, ouvrages de mains humaines.} 
        \item \textit{Elles ont une bouche et ne parlent pas, des yeux et ne voient pas,}
        \item \textit{Des oreilles et n'entendent pas, des narines et ne sentent pas.}
        \item \textit{Leurs mains ne peuvent toucher, leurs pieds ne peuvent marcher, pas un son ne sort de leur gosier !}
        \item \textit{Qu'ils deviennent comme elles, tous ceux qui les font, ceux qui mettent leur foi en elles.}
        \item \textit{Israël, mets ta foi dans le Seigneur : le secours, le bouclier, c'est lui !}
        \item \textit{Famille d'Aaron, mets ta foi dans le Seigneur : le secours, le bouclier, c'est lui !}
        \item \textit{Vous qui le craignez, ayez foi dans le
        Seigneur : le secours, le bouclier, c'est lui !}
        \item \textit{Le Seigneur se souvient de nous : il bénira !
        Il bénira la famille d'Israël, et la famille
        d'Aaron}
        \item \textit{Il bénira tous ceux qui craignent le Seigneur,
        du plus grand au plus petit.}
        \item \textit{Que le Seigneur multiplie ses bienfaits pour
        vous et vos enfants !}
        \item \textit{Soyez bénis par le Seigneur qui a fait le ciel
        et la terre !}
        \item \textit{Le ciel, c'est le ciel du Seigneur ; aux
        hommes, il a donné la terre.}
        \item \textit{Les morts ne louent pas le Seigneur, ni ceux
        qui descendent au silence.}
        \item \textit{Nous, les vivants, bénissons le Seigneur,
        maintenant et pour les siècles des siècles}
        \item \textit{Gloire au Père, au Fils, et au Saint Esprit, }
        \item \textit{Comme il était au commencement, maintenant et toujours, et dans les siècles des siècles. Amen. }
      \end{enumerate}
    \end{center}
  \end{normalsize}
  \vspace*{\fill}\
  \newpage

  \begin{enumerate}[label=\textcolor{red}{\arabic*}]
    \setcounter{enumi}{23}
    \item Benedíc\textit{ti} \textit{vos} \textit{a} \textbf{Dó}mino,\textcolor{red}{~*}  qui fecit cælum, \textit{et} \textbf{ter}ram.

    \item Cæ\textit{lum} \textit{cæ}\textit{li} \textbf{Dó}mino:\textcolor{red}{~*}  terram autem dedit fíli\textit{is} \textbf{hó}minum.

    \item Non mórtui lau\textit{dá}\textit{bunt} \textit{te}, \textbf{Dó}mine:\textcolor{red}{~*}  neque omnes, qui descéndunt in \textit{in}\textbf{fér}num.

    \item Sed nos qui vívimus, bene\textit{dí}\textit{ci}\textit{mus} \textbf{Dó}mino,\textcolor{red}{~*}  ex hoc nunc et usque \textit{in} \textbf{sǽ}culum.

    \item Glória \textit{Pa}\textit{tri}, \textit{et} \textbf{Fí}lio,\textcolor{red}{~*}  et Spirítu\textit{i} \textbf{Sanc}to.

    \item Sicut erat in princípio, \textit{et} \textit{nunc}, \textit{et} \textbf{sem}per,\textcolor{red}{~*}  et in sǽcula sæculó\textit{rum}. \textbf{A}men.
  \end{enumerate}
  %  Répetition de l'Antienne
  \grecommentary{\textit{Reprise de l'Antienne.}}
  \gabcsnippet{(c4) De(c')us(d) au(f')tem(e) no(fg)ster(g) in(gh) cae(fe)lo:(d.) (;) ó(f')mni(g)a(h') quae(ixi)cúm(h)que(gf) vó(g')lu(f)it,(gh) fe(g.)cit.(g.) (::)}
  \bigskip

  \begin{center}
    \textcolor{red}{\large{Capitule}}\\
    \small\textit{
      II\textsuperscript{e.} Épître aux Corinthiens. 1, 3-4
    }
  \end{center}

  \begin{multicols}{2}
    \parindent=0pt
    Benedíctus Deus, et Pater Dómini nostri Iesu Christi,  \textcolor{red}{†} Pater misericordiárum, et Deus totíus consolatiónis, \textcolor{red}{*}  qui consolátur nos in omni tribulatióne nostra. \\
    \textcolor{red}{\Rbar.} Deo grátias.

    \columnbreak

    \textit{ Béni soit le Dieu et Père de notre Seigneur JésusChrist, le Père des miséricordes, et le Dieu de toute consolation ; qui nous console dans toutes nos tribulations.\\
    \textcolor{red}{\Rbar.} Rendons grâce à Dieu.
    }
  \end{multicols}

  \bigskip

  \begin{center}
    \textcolor{red}{\large{Hymne}}\\
  \end{center}
  
  \gresetinitiallines{1}
  \greillumination{\initfamily\fontsize{11mm}{11mm}\selectfont L}
  \gregorioscore{hymnes/hy--lucis_creator_optime--solesmes}
  \newpage
  % \begin{multicols}{2}
    \begin{normalsize}
      \begin{enumerate}[label=\textcolor{red}{\emph{\arabic*}}]
        \item \textit{Ô Dieu souverainement bon, qui avez créé la lumière, qui la faites luire tous les jours, et qui avez commencé par elle la création du monde. }
        \item \textit{Ô Dieu qui avez voulu qu’on donnât le nom de jour à cet espace de temps qui s’écoule depuis le matin jusqu’au soir : pendant que les ténèbres de la nuit s’approchent, écoutez nos prières et nos larmes. }
        \item \textit{Ne permettez pas que notre âme chargée de crimes, se prive de la vie de votre grâce, en oubliant les biens éternels, et en se liant sans cesse par de nouveaux péchés. }
        \item \textit{Faites au contraire qu’elle soit sans cesse tournée vers le ciel, et qu’elle obtienne cette vie qui ne finira jamais : Faites que nous évitions toutes les fautes où nous pouvons tomber, et que nous lavions toutes celles que nous avons commises. }
        \item \textit{Accorde-nous cette grâce, ô Père de miséricorde, et vous Fils unique, égal au Père, qui avec lui et l’Esprit consolateur, régnez dans tous les siècles. Amen. }
      \end{enumerate}
    \end{normalsize}
  % \end{multicols}
  \newpage

  \begin{center}
    \begin{footnotesize}
      \textcolor{red}{\textit{On chante le verset debout.}}
    \end{footnotesize}
    \begin{minipage}{0.8\linewidth}
      \gresetinitiallines{0}
      \gabcsnippet{(c3)<c><v>\Vbar</v>.</c> Di(h)ri(h)gá(h)tur(h) Dó(h)mi(h)ne(h) o(h)rá(h)ti(h)o(h) mé(h)a.(g'_) (hvGF'Efgf.) (::) (Z) <c><v>\Rbar</v>.</c> Si(h)cut(h) in(h)cén(h)sum(h) in(h) cons(h)pé(h)ctu(h) tú(h)o.(g'_) (hvGF'Efgf.) (::)}
      \bigskip
      \begin{center}
        \textit{\textcolor{red}{\Vbar.} Que ma prière s’élève,}\\
        \textit{\textcolor{red}{\Rbar.} Seigneur, comme l’encens devant votre face.}
      \end{center}
    \end{minipage}
  \end{center}
  \normalsize

  % \vspace*{\fill}\
  % \begin{center}
  %   \greseparator{3}{30}
  % \end{center}
  % \vspace*{\fill}\

  % \newpage

  \begin{center}
    \textcolor{red}{\large{Antienne à Magnificat}}\\
    \footnotesize{
      \emph{Au propre du jour (p.21)}
    }
  \end{center}


  \medskip
  \begin{center}
    \rule{2cm}{0.4pt}
  \end{center}
  \medskip

  \begin{center}
    \textcolor{red}{\large{Oraison}}\\
    \footnotesize{
      \emph{Au propre du jour (p.21)}
    }
  \end{center}

  \medskip
  \begin{center}
    \rule{2cm}{0.4pt}
  \end{center}
  \medskip

  
  \begin{center}
    \textcolor{red}{\large{Conclusion de l'office}}
  \end{center}
  
  
  \begin{multicols}{2}
    \parindent=0pt
    \begin{flushright}
      \textcolor{red}{\Vbar.} Dominus vobiscum.\\
      \textcolor{red}{\Rbar.} Et cum spiritu tuo.\\
    \end{flushright}
  
    \columnbreak
    
    \textit{\textcolor{red}{\Vbar.} Le Seigneur soit avec vous.\\
    \textcolor{red}{\Rbar.} Et avec votre esprit.}\\
  \end{multicols}
  \bigskip
  \gresetinitiallines{1}
  \greillumination{\initfamily\fontsize{11mm}{11mm}\selectfont B}
  \gregorioscore{benedicamus/ky--benedicamus_xi--solesmes}
  \begin{center}
    \begin{footnotesize}
      \textcolor{red}{\textit{Sur un ton très grave : }}
    \end{footnotesize}
  \end{center}
  \begin{multicols}{2}
    \parindent=0pt
    \textcolor{red}{\Vbar.} Fidélium ánimæ per misericórdiam Dei requiéscant in pace.\\
    \textcolor{red}{\Rbar.} Amen.\\

    \columnbreak
    
    \textit{\textcolor{red}{\Vbar.} Que les âmes des fidèles défunts, par la
    miséricorde de Dieu, reposent en paix.\\
    \textcolor{red}{\Rbar.} Amen.}\\
  \end{multicols}

  \vspace*{\fill}\
  \begin{center}
    \greseparator{3}{30}
  \end{center}
  \vspace*{\fill}\

  \newpage

  \subfile{salut.tex}

  \newpage
  \begin{center}
    \vspace*{\fill}
    \LARGE PROPRE DU TEMPS.
    \vspace*{\fill}
  \end{center}
  \newpage

  \subfile{propre.tex}

  \newpage
  \vspace*{\fill}
  \begin{center}
    \normalsize\textit{
      Paroisse Saint Roch, 296 rue Saint-Honoré, 75001 Paris
    }
  \end{center}
  \newpage

\end{document}